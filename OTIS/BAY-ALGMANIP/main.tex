\documentclass[11pt]{scrartcl}
\usepackage[usenames,dvipsnames,svgnames]{xcolor}
\usepackage[shortlabels]{enumitem}
\usepackage[framemethod=TikZ]{mdframed}
\usepackage{amsmath,amssymb,amsthm}
\usepackage{epigraph}
\usepackage[colorlinks]{hyperref}
\usepackage{microtype}
\usepackage{mathtools}
\usepackage[headsepline]{scrlayer-scrpage}
\usepackage{thmtools}
\usepackage{listings}
\usepackage{derivative}
\renewcommand{\epigraphsize}{\scriptsize}
\renewcommand{\epigraphwidth}{60ex}
\addtolength{\textheight}{3.14cm}
\ihead{\footnotesize\textbf{BAY-algmanip}}
\ohead{\footnotesize Updated Thu 15 Feb 2024 02:34:56 UTC}
\providecommand{\clubs}[1]{$#1\clubsuit$}
\providecommand{\clubg}[1]{\bgroup\color{green!40!black}[$#1\clubsuit$]\egroup}

\providecommand{\ol}{\overline}
\providecommand{\eps}{\varepsilon}
\providecommand{\half}{\frac{1}{2}}
\providecommand{\dang}{\measuredangle} %% Directed angle
\providecommand{\CC}{\mathbb C}
\providecommand{\FF}{\mathbb F}
\providecommand{\NN}{\mathbb N}
\providecommand{\QQ}{\mathbb Q}
\providecommand{\RR}{\mathbb R}
\providecommand{\ZZ}{\mathbb Z}
\providecommand{\dg}{^\circ}
\providecommand{\ii}{\item}
\providecommand{\alert}{\textbf}
\providecommand{\opname}{\operatorname}
\providecommand{\ts}{\textsuperscript}
% hacks for arc
\providecommand{\tarc}{\mbox{\large$\frown$}}
\providecommand{\arc}[1]{\stackrel{\tarc}{#1}}
\reversemarginpar
\providecommand{\printpuid}[1]{\marginpar{\href{https://otis.evanchen.cc/arch/#1}{\ttfamily\footnotesize\color{green!40!black}#1}}}

\mdfdefinestyle{mdbluebox}{roundcorner=10pt,innerbottommargin=9pt,
    linecolor=blue,backgroundcolor=TealBlue!5,}
\declaretheoremstyle[headfont=\sffamily\bfseries\color{MidnightBlue},
    mdframed={style=mdbluebox},]{thmbluebox}
\mdfdefinestyle{mdredbox}{frametitlefont=\bfseries,innerbottommargin=8pt,
    nobreak=true,backgroundcolor=Salmon!5,linecolor=RawSienna,}
\declaretheoremstyle[headfont=\bfseries\color{RawSienna},
    mdframed={style=mdredbox},headpunct={\\[3pt]},postheadspace=0pt,]{thmredbox}
\mdfdefinestyle{mdgreenbox}{linecolor=ForestGreen,backgroundcolor=ForestGreen!5,
    linewidth=2pt,rightline=false,leftline=true,topline=false,bottomline=false,}
\declaretheoremstyle[headfont=\bfseries\sffamily\color{ForestGreen!70!black},
    mdframed={style=mdgreenbox},headpunct={ --- },]{thmgreenbox}
\mdfdefinestyle{mdblackbox}{linecolor=black,backgroundcolor=RedViolet!5!gray!5,
    linewidth=3pt,rightline=false,leftline=true,topline=false,bottomline=false,}
\declaretheoremstyle[mdframed={style=mdblackbox}]{thmblackbox}
\declaretheorem[style=thmredbox,name=Problem]{problem}
\declaretheorem[style=thmredbox,name=Required Problem,sibling=problem]{reqproblem}
\declaretheorem[style=thmbluebox,name=Theorem,numberwithin=problem]{theorem}
\declaretheorem[style=thmbluebox,name=Lemma,sibling=theorem]{lemma}
\declaretheorem[style=thmbluebox,name=Theorem,numbered=no]{theorem*}
\declaretheorem[style=thmbluebox,name=Lemma,numbered=no]{lemma*}
\declaretheorem[style=thmgreenbox,name=Claim,sibling=theorem]{claim}
\declaretheorem[style=thmgreenbox,name=Claim,numbered=no]{claim*}
\declaretheorem[style=thmblackbox,name=Remark,sibling=theorem]{remark}
\declaretheorem[style=thmblackbox,name=Remark,numbered=no]{remark*}
\declaretheorem[style=thmgreenbox,name=Definition,sibling=theorem]{definition}
\declaretheorem[style=thmgreenbox,name=Definition,numbered=no]{definition*}
\declaretheorem[style=thmblackbox,name=Example,sibling=theorem]{example}
\declaretheorem[style=thmblackbox,name=Example,numbered=no]{example*}

\newenvironment{walkthrough}{\noindent\textbf{\color{green!40!black}Walkthrough.}}{}
\newlist{walk}{enumerate}{3}
\setlist[walk]{label=\bfseries (\alph*)}

\usepackage{asymptote}
\begin{asydef}
size(8cm); // set a reasonable default
usepackage("amsmath");
usepackage("amssymb");
settings.tex="pdflatex";
settings.outformat="pdf";
import geometry;
void filldraw(picture pic = currentpicture, conic g, pen fillpen=defaultpen, pen drawpen=defaultpen) { filldraw(pic, (path) g, fillpen, drawpen); }
void fill(picture pic = currentpicture, conic g, pen p=defaultpen) { filldraw(pic, (path) g, p); }
pair foot(pair P, pair A, pair B) { return foot(triangle(A,B,P).VC); }
pair centroid(pair A, pair B, pair C) { return (A+B+C)/3; }
\end{asydef}

\newcommand{\goals}[2]{\bgroup
\sffamily\small \emph{Instructions}: Solve \clubg{#1}.
If you have time, solve \clubg{#2}.\egroup\par}

%% 426c616e6b204c615465587e
\begin{document}
\title{Submission for BAY-ALGMANIP}
\subtitle{OTIS (internal use)}
\author{Michael Middlezong}
\date{\today}
\maketitle

\begin{example*}[$0\clubsuit$]
  Solve over real numbers the system of equations
  \begin{align*}
  a+2 &= b^2 \\
  b+2 &= c^2 \\
  c+2 &= a^2.
  \end{align*}
\end{example*} \printpuid{ZC329504}

\begin{walkthrough}
This is the archetypal trig problem.
\begin{walk}
  \ii Optionally, if you don't know how $\cos z$ is defined for $z \in \CC$,
  first prove that $|a| \le 2$.

  \ii Thus, we may let $a = 2\cos x$,
  $b = 2\cos y$, $c = 2 \cos z$
  where $x$, $y$, $z$ are real numbers if you did part (a),
  or complex numbers if you skipped part (a).
  Show that $\cos 2y = \cos x$, etc.

  \ii There's a lemma that whenever $\cos \theta = \cos \theta'$
  we have $\cos 2\theta = \cos 2\theta'$.
  Prove this lemma if you have not seen it; it's easy
  (the simplest proof is by using the double angle formula).

  \ii Show that $\cos x = \cos 8x$.

  \ii Use this to find eight solutions to the system of equations.

  \ii Conversely, show there are at most eight possible values of $a$,
  and hence at most eight solutions.
\end{walk}
\end{walkthrough}
%% You're not expected to write up walkthroughs (unless you really want to).
%% The source is just for your reference.

\begin{example*}[Czech Polish Slovak 2005/1, $0\clubsuit$]
  Let $n$ be a positive integer.
  Solve the system of equations
  \begin{align*}
  x_1 + x_2^2 + x_3^3 + \dotsb + x_n^n &= n \\
  x_1 + 2x_2 + 3x_3 + \dotsb + nx_n &= \frac{n(n+1)}{2}
  \end{align*}
  over nonnegative real numbers.
\end{example*} \printpuid{05CPS1}

\begin{walkthrough}
It shouldn't take too much to convince you
$x_1 = x_2 = \dots = x_n = 1$ is the only solution.
But since this has $n$ variables and $2$ equations,
the only way there could be only one solution
is if some inequality was taking place.
\begin{walk}
  \ii Prove that if $\sum_k x_k^k = n$
  then $\sum_k kx_k \le \half n (n+1)$.
  \ii Finish by extracting the equality case.
\end{walk}
\end{walkthrough}
%% You're not expected to write up walkthroughs (unless you really want to).
%% The source is just for your reference.

\begin{example*}[\href{https://aops.com/community/p13962758}{CMIMC 2020 A7}, $0\clubsuit$]
  Solve over $\RR$ the equation
  \[ (x-1)(x-4)(x-2)(x-8)(x-5)(x-7) = -48\sqrt3. \]
\end{example*} \printpuid{20CMIMCA7}

\begin{walkthrough}
The solution proceeds with just suitable substitutions.
\begin{walk}
  \ii Let $y = x^2-9x+14$.
  Rewrite everything in terms of $y$.

  \ii Let $z = y/\sqrt3$.
  Rewrite everything in terms of $z$.
  What motivated this?

  \ii You should have a cubic in $z$.
  Solve it; you should find it has integer solutions.

  \ii Use this to extract the answer for
  $x = \frac{9 \pm \sqrt{25+8\sqrt3}}{2}$.
\end{walk}
\end{walkthrough}
%% You're not expected to write up walkthroughs (unless you really want to).
%% The source is just for your reference.

\newpage

% ========================================
\section*{Practice problems}
\goals{35}{50}

\begin{problem}[IIT JEE, $2\clubsuit$]
  Find all real numbers $x$ such $4^x + 6^x = 9^x$.
\end{problem} \printpuid{IITJEE}

%% Type your solution to IIT JEE (\href{https://otis.evanchen.cc/arch/IITJEE/}{IITJEE}) here ...
Dividing both sides by $6^x$ and setting $u = \left(\frac  23 \right)^x$, we get
\[ u + 1 = \frac 1u \implies u = \frac{1 + \sqrt 5}{2}, \]
taking only the positive solution as $u$ must be positive.
This means $x = \log _{2/3} \left(\frac{1+ \sqrt 5}{2}\right)$ is the only solution,
and we can check that our steps are reversible.
%% --------------------------------------------------

\begin{problem}[CMIMC 2018 A5, $2\clubsuit$]
  Suppose $a$, $b$, $c$ are nonzero real numbers satisfying
  \[
  bc + \frac 1a
  = ca + \frac 2b
  = ab + \frac 7c
  = \frac{1}{a+b+c}.
  \]
  Find $a+b+c$.
\end{problem} \printpuid{18CMIMCA5}

%% Type your solution to CMIMC 2018 A5 (\href{https://otis.evanchen.cc/arch/18CMIMCA5/}{18CMIMCA5}), proposed by David Altizio here ...
Let the common value be $x$. Then
\[ ax + bx + cx = abc + 1 + abc + 2 + abc + 7 = (a+b+c)\frac{1}{a+b+c} = 1. \]
This means
\[ abc = -3. \]
The rest is simple. The final answer is $-\frac{\sqrt[3]{3}}{2}$.
%% --------------------------------------------------

\begin{problem}[\href{https://aops.com/community/p12141493}{EGMO 2019/1}, $3\clubsuit$]
  Find all triples $(a, b, c)$ of real numbers such that $ab + bc + ca = 1$ and
  \[ a^2b + c = b^2c + a = c^2a + b. \]
\end{problem} \printpuid{19EGMO1}

%% Type your solution to EGMO 2019/1 (\href{https://otis.evanchen.cc/arch/19EGMO1/}{19EGMO1}) here ...
Homogenize to get
\[ a^2b + bc^2 + c^2a = a^2b + a^2c + b^2c = ac^2 + ab^2 + b^2c. \]
Taking pairs of equations at a time, we get
\begin{align*}
  c^2(a+b) &= c(a^2+b^2), \\
  a^2(b+c) &= a(b^2+c^2), \\
  b^2(a+c) &= b(a^2+c^2).
\end{align*}
Assume one of the variables is zero, and WLOG $a = 0$.
Then, the condition that $ab + bc + ca = 1$ implies $b,c \ne 0$.
Since $c^2b = cb^2$, we must have $b = c$.

If all variables are nonzero, then we have
\begin{align*}
  c(a+b) &= a^2+b^2, \\
  a(b+c) &= b^2+c^2, \\
  b(a+c) &= a^2+c^2.
\end{align*}
Adding them up, we get the equality case of repeated AM-GM,
from which we can conclude that $a=b=c$.

Putting everything together, the solutions are
$a=b=c=\frac{1}{\sqrt3}$, $a=b=c=-\frac{1}{\sqrt3}$,
and permutations of $(1,1,0)$ and $(-1,-1,0)$.
%% --------------------------------------------------

\begin{reqproblem}[\href{https://aops.com/community/p3342960}{Vietnam 2014/1}, $3\clubsuit$]
  Let $(x_n)_{n \ge 1}$ and $(y_n)_{n \ge 1}$ be two sequences of positive real numbers
  with $x_1=1$ and $y_1 = \sqrt3$, satisfying the recursions
  \begin{align*}
  x_{n+1} y_{n+1} - x_n &= 0 \\
  x_{n+1}^2 + y_n &= 2.
  \end{align*}
  Show that $\lim_{n \to \infty} x_n$ and $\lim_{n \to \infty} y_n$ exist
  and determine their values.
\end{reqproblem} \printpuid{14VNM1}

%% Type your solution to Vietnam 2014/1 (\href{https://otis.evanchen.cc/arch/14VNM1/}{14VNM1}) here ...
We claim that $x_n = 2\sin \left(\frac{30\dg}{n}\right)$ and $y_n = 2\cos \left(\frac{30\dg}{n}\right)$.
We proceed by induction. The base case $n=1$ clearly works,
so assuming $n$ works, we conclude
\[ x_{n+1}^2 = 2 - y_n = 2 - 2\cos \left(\frac{30\dg}{n}\right). \]
Using either the half-angle or double-angle formulas, it follows that
\[ x_{n+1} = 2\sin \left(\frac{30\dg}{n+1}\right). \]
Also,
\begin{align*}
  x_{n+1}^2 + y_{n+1}^2 &= x_{n+1}^2 + \left(\frac{x_n}{x_{n+1}}\right)^2 \\
  &= \frac{x_n^2 + (2 - y_n)^2}{x_{n+1}^2} \\
  &= \frac{8 - 4y_n}{2 - y_n} = 4,
\end{align*}
so we must have $y_{n+1} = 2\cos \left(\frac{30\dg}{n+1}\right)$ as well.
Finally it is now clear that
$\lim_{n\to\infty} x_n = 0$
and
$\lim_{n\to\infty} y_n = 2$.

%% --------------------------------------------------

\begin{problem}[\href{https://aops.com/community/p3542095}{IMO 2014/1}, $3\clubsuit$]
  Let $a_0 < a_1 < a_2 < \dotsb$ be an infinite sequence of positive integers.
  Prove that there exists a unique integer $n\geq 1$ such that
  \[ a_n < \frac{a_0+a_1+a_2+\dotsb+a_n}{n} \le a_{n+1}. \]
\end{problem} \printpuid{14IMO1}

%% Type your solution to IMO 2014/1 (\href{https://otis.evanchen.cc/arch/14IMO1/}{14IMO1}), proposed by Gerhard Woeginger (AUT) here ...

%% --------------------------------------------------

\begin{problem}[\href{https://aops.com/community/p3431241}{AIME 2014/14}, $3\clubsuit$]
  Find the largest real number $x$ satisfying
  \[ \frac{3}{x-3} + \frac{5}{x-5} + \frac{17}{x-17}
  + \frac{19}{x-19} = x^2-11x-4.  \]
\end{problem} \printpuid{14AIME14}

%% Type your solution to AIME 2014/14 (\href{https://otis.evanchen.cc/arch/14AIME14/}{14AIME14}) here ...
First, add $4$ to both sides and use it to make the numerators on the left side all equal to $x$.
Then, substitute $u = x-11$ for symmetry purposes and the rest is easy.
The answer is $x = 11 + \sqrt{52 + 10\sqrt 2}$.
%% --------------------------------------------------

\begin{problem}[\href{https://aops.com/community/p4728593}{EGMO 2015/4}, $5\clubsuit$]
  A sequence $a_1$, $a_2$, $a_3$, \dots, $a_N$ of positive integers
  (where $N \ge 3$) satisfies the equality
  \[ a_{n+2} = a_{n+1} + \sqrt{a_{n+1}+a_{n}} \]
  for every $1 \le n \le N-2$.
  Determine the largest possible value of $N$,
  or prove that no such maximum exists.
\end{problem} \printpuid{15EGMO4}

%% Type your solution to EGMO 2015/4 (\href{https://otis.evanchen.cc/arch/15EGMO4/}{15EGMO4}) here ...

%% --------------------------------------------------

\begin{problem}[\href{https://aops.com/community/p29651019}{OTIS Mock AIME 2024, by Joshua Liu and Ashvin Sinha}, $3\clubsuit$]
  For each real number $k > 0$,
  let $S(k)$ denote the set of real numbers $x$ satisfying
  \[ \left\lfloor x \right\rfloor \cdot \left( x - \left\lfloor x \right\rfloor \right) = kx. \]
  The set of positive real numbers $k$ such that $S(k)$ has exactly $24$ elements
  is a half-open interval of length $\ell$. Compute $1/\ell$.
\end{problem} \printpuid{24OIME6}

%% Type your solution to OTIS Mock AIME 2024, by Joshua Liu and Ashvin Sinha (\href{https://otis.evanchen.cc/arch/24OIME6/}{24OIME6}), proposed by Joshua Liu, Ashvin Sinha here ...
Graph the left hand side, it's just a union of line segments.
We see that if $k \le 1$, then we get infinitely many solutions.
And if $k > 1$, we get no solutions where $x > 0$.
So, we count the number of lines in the third quadrant
that we have to intersect, and eventually get an answer of
\[ \frac{1}{\frac{23}{22} - \frac{24}{23}} = \boxed{506}. \]
%% --------------------------------------------------

\begin{problem}[\href{https://aops.com/community/p464691}{AIME II 2006/15}, $3\clubsuit$]
  Solve over real numbers the system of equations
  \begin{align*}
  x &= \sqrt{y^2-\frac{1}{16}}+\sqrt{z^2-\frac{1}{16}} \\
  y &= \sqrt{z^2-\frac{1}{25}}+\sqrt{x^2-\frac{1}{25}} \\
  z &= \sqrt{x^2-\frac{1}{36}}+\sqrt{y^2-\frac{1}{36}}.
  \end{align*}
\end{problem} \printpuid{06AIMEII15}

%% Type your solution to AIME II 2006/15 (\href{https://otis.evanchen.cc/arch/06AIMEII15/}{06AIMEII15}) here ...

%% --------------------------------------------------

\begin{problem}[\href{https://aops.com/community/p18895568}{Baltic Way 2020, added by Pedro Rosalba}, $3\clubsuit$]
  Find all real numbers $x$, $y$, $z$ so that
  \begin{align*}
  x^2 y + y^2 z + z^2 &= 0 \\
  z^3 + z^2 y + z y^3 + x^2 y &= \frac{1}{4}(x^4 + y^4).
  \end{align*}
\end{problem} \printpuid{20BWAY5}

%% Type your solution to Baltic Way 2020, added by Pedro Rosalba (\href{https://otis.evanchen.cc/arch/20BWAY5/}{20BWAY5}) here ...

%% --------------------------------------------------

\begin{problem}[\href{https://aops.com/community/p29854155}{AIME II 2024/11}, $2\clubsuit$]
  Compute the number of triples of nonnegative integers $(a,b,c)$
  satisfying $a + b + c = 300$ and
  \[ a^2b + a^2c + b^2a + b^2c + c^2a + c^2b = 6000000.\]
\end{problem} \printpuid{24AIMEII11}

%% Type your solution to AIME II 2024/11 (\href{https://otis.evanchen.cc/arch/24AIMEII11/}{24AIMEII11}), proposed by Ankan Bhattacharya here ...

%% --------------------------------------------------

\begin{reqproblem}[\href{https://aops.com/community/p27339536}{Mathematical Reflections J479}, $3\clubsuit$]
  Let $a$, $b$, $c$ be nonzero real numbers,
  not all equal, such that
  \[ \left( \frac{a^2}{bc}-1 \right)^3 + \left( \frac{b^2}{ca}-1 \right)^3
  + \left( \frac{c^2}{ab}-1 \right)^3
  = 3\left( \frac{a^2}{bc} + \frac{b^2}{ca} + \frac{c^2}{ab} - \frac{bc}{a^2} -
  \frac{ca}{b^2} - \frac{ab}{c^2} \right). \]
  Prove that $a+b+c=0$.
\end{reqproblem} \printpuid{MRJ479}

%% Type your solution to Mathematical Reflections J479 (\href{https://otis.evanchen.cc/arch/MRJ479/}{MRJ479}), proposed by Titu Andreescu here ...
Let $x = \frac{a^2}{bc} - 1$, $y = \frac{b^2}{ca} - 1$, and $z = \frac{c^2}{ab} - 1$.
Then, the equation simplifies to
\[ x^3 + y^3 + z^3 - 3xyz = 0, \]
Since $x^3 + y^3 + z^3 - 3xyz$ factors as $(x+y+z)(x^2+y^2+z^2-xy-yz-zx)$,
and $x^2+y^2+z^2-xy-yz-zx \ne 0$ (because otherwise $a=b=c$),
we have $x+y+z = 0$.

Also, we have
\begin{align*}
  x + y + z &= \frac{a^2-bc}{bc} + \frac{b^2-ac}{ac} + \frac{c^2 - ab}{ab} \\
  &= \frac{(a+b+c)(a^2+b^2+c^2-ab-bc-ca)}{abc},
\end{align*}
so we must have $a+b+c = 0$ as desired.

%% --------------------------------------------------

\begin{problem}[\href{https://aops.com/community/p3003353}{AIME II 2013/15}, $3\clubsuit$]
  In obtuse triangle $ABC$ with $\angle B > 90\dg$ we have
  \begin{align*}
  \cos^2 A + \cos^2 B + 2 \sin A \sin B \cos C &= \frac{15}{8} \\
  \cos^2 B + \cos^2 C + 2 \sin B \sin C \cos A &= \frac{14}{9}.
  \end{align*}
  Compute \[ \cos^2 C + \cos^2 A + 2 \sin C \sin A \cos B. \]
\end{problem} \printpuid{13AIMEII15}

%% Type your solution to AIME II 2013/15 (\href{https://otis.evanchen.cc/arch/13AIMEII15/}{13AIMEII15}) here ...

%% --------------------------------------------------

\begin{problem}[\href{https://aops.com/community/p29520456}{Canadian training camp, added by Haozhe Yang}, $3\clubsuit$]
  The sequences ${a_n}$ and ${b_n}$ are such that, for every positive integer $n$,
  \[a_n > 0, \qquad b_n > 0, \qquad a_{n+1} = a_n+\frac{1}{b_n}, \qquad b_{n+1}=b_n+\frac{1}{a_n}\]
  Prove that $a_{50}+b_{50} > 20$.
\end{problem} \printpuid{ZCB390FF}

%% Type your solution to Canadian training camp, added by Haozhe Yang (\href{https://otis.evanchen.cc/arch/ZCB390FF/}{ZCB390FF}) here ...

%% --------------------------------------------------

\begin{problem}[Germany 2008, added by Joel Gerlach, $2\clubsuit$]
  Solve over real numbers:
  \begin{align*}
  (x+y)(x^2-y^2)&=675\\
  (x-y)(x^2+y^2)&=351.
  \end{align*}
\end{problem} \printpuid{08GER33}

%% Type your solution to Germany 2008, added by Joel Gerlach (\href{https://otis.evanchen.cc/arch/08GER33/}{08GER33}) here ...
Expanding, we get
\begin{align*}
    x^3 + x^2y - xy^2 - y^3 &= 675 \\
    x^3 - x^2y + xy^2 - y^3 &= 351.
\end{align*}
Subtract the first equation from twice the second to get
\[ x^3 - 3x^2y + 3xy^2 - y^3 = 27 \implies (x-y)^3 = 27. \]
This means $x-y = 3$. Thus, the original second equation yields $x^2 + y^2 = 117$.
We can solve this system of equations using substitution, and
the only possible solutions are
\[ \{(9,6),(-6,-9)\}, \]
which can be checked to work.
%% --------------------------------------------------

\begin{problem}[\href{https://aops.com/community/p21660235}{ARML Local 2021, added by Qiao Zhang}, $2\clubsuit$]
  A sequence $a_1$, $a_2$, \dots\ of real numbers satisfies
  \[ a_n = na_{n-1} + (n-1)(n!(n-1)!-1)\] for integers $n \geq 2$.
  Given that $a_{2021}=(2021!+1)^2+2020!$, compute $a_1$.
\end{problem} \printpuid{21ARMLOCI10}

%% Type your solution to ARML Local 2021, added by Qiao Zhang (\href{https://otis.evanchen.cc/arch/21ARMLOCI10/}{21ARMLOCI10}) here ...
We find that
\[ a_n = (n! + 1)^2 + \frac{n!}{2021} \]
satisfies the recurrence and the given value for $a_{2021}$,
so the answer is $4 + \frac{1}{2021}$.
%% --------------------------------------------------

\begin{problem}[\href{https://aops.com/community/p16682015}{Summer Mock AIME 2020/14}, $5\clubsuit$]
  Let $P(x) = x^3 - 3x^2 + 3$.
  For how many positive integers $n < 1000$
  does there not exist a pair $(a, b)$ of positive integers
  such that the equation
  \[ \underbrace{P(P(\dots P}_{a \text{ times}}(x)\dots))
  = \underbrace{P(P(\dots P}_{b \text{ times}}(x)\dots)) \]
  has exactly $n$ distinct real solutions?
\end{problem} \printpuid{20SIME14}

%% Type your solution to Summer Mock AIME 2020/14 (\href{https://otis.evanchen.cc/arch/20SIME14/}{20SIME14}) here ...

%% --------------------------------------------------

\begin{reqproblem}[\href{https://aops.com/community/p10626524}{IMO 2018/2}, $9\clubsuit$]
  Find all integers $n \geq 3$ for which
  there exist real numbers $a_1, a_2, \dots, a_n$ satisfying
  \[ a_i a_{i+1} +1 = a_{i+2} \]
  for $i=1,2, \dots, n$, where indices are taken modulo $n$.
\end{reqproblem} \printpuid{18IMO2}

%% Type your solution to IMO 2018/2 (\href{https://otis.evanchen.cc/arch/18IMO2/}{18IMO2}), proposed by Patrik Bak (SVK) here ...
If $3 \mid n$, then the repeating sequence $(2,-1,-1,2,-1,-1,\dots)$ works.
Otherwise, multiply the given equation by $a_{i+2}$ and rearrange to get
\[ a_ia_{i+1}a_{i+2} = a_{i+2}^2 - a_{i+2}. \]
Since $a_{i+3} = a_{i+1}a_{i+2} + 1$, we can rewrite the equation as
\[ a_ia_{i+3} - a_i = a_{i+2}^2 - a_{i+2}. \]
Summing over all $i$, the degree $1$ terms cancel out and we are left with
\[ a_1a_4 + a_2a_5 + \dots = a_1^2 + a_2^2 + \dots . \]
Since $n$ is not divisible by $3$, none of the terms on the left side repeat.
Thus, a repeated application of AM-GM yields
\[ a_1 = a_2 = \dots = a_n, \]
and at this point, it is obvious that no solution can exist.
%% --------------------------------------------------

\begin{problem}[\href{https://aops.com/community/p14780288}{EGMO 2020/2}, $9\clubsuit$]
  Find all lists $(x_1, x_2, \dots, x_{2020})$ of non-negative real numbers
  such that the following three conditions are all satisfied:
  \begin{itemize}
  \ii $x_1 \le x_2 \le \dots \le x_{2020}$;
  \ii $x_{2020} \le x_1  + 1$;
  \ii there is a permutation $(y_1, y_2, \dots, y_{2020})$
  of $(x_1, x_2, \dots, x_{2020})$ such that
  \[ \sum_{i = 1}^{2020} ((x_i + 1)(y_i + 1))^2 = 8 \sum_{i = 1}^{2020} x_i^3.  \]
  \end{itemize}
\end{problem} \printpuid{20EGMO2}

%% Type your solution to EGMO 2020/2 (\href{https://otis.evanchen.cc/arch/20EGMO2/}{20EGMO2}), proposed by Patrik Bak (SVK) here ...

%% --------------------------------------------------

\begin{reqproblem}[\href{https://aops.com/community/p23439842}{Iberoamerican 2021/4}, $5\clubsuit$]
  Let $a$, $b$, $c$, $x$, $y$, $z$ be real numbers such that
  \begin{align*}
  a^2+x^2 &= b^2+y^2 = c^2+z^2 = (a+b)^2+(x+y)^2 \\
  &= (b+c)^2+(y+z)^2=(c+a)^2+(z+x)^2
  \end{align*}
  Show that $a^2+b^2+c^2=x^2+y^2+z^2$.
\end{reqproblem} \printpuid{21IBERO4}

%% Type your solution to Iberoamerican 2021/4 (\href{https://otis.evanchen.cc/arch/21IBERO4/}{21IBERO4}) here ...
Let $u = a+xi$, $v = b+yi$, and $w + c+zi$.
Then, we have
\[ |u| = |v| = |w| = |u+v| = |v+w| = |w+u|. \]
Using the law of cosines, this means that $u$, $v$, and $w$ must lie on a circle
centered at the origin and form an equilateral triangle.
Thus, the complex numbers $u^2$, $v^2$, and $w^2$ also form an equilateral triangle,
so $u^2 + v^2 + w^2 = 0$.
Expanding and taking the real part of both sides yields the desired conclusion.
%% --------------------------------------------------

\begin{problem}[\href{https://aops.com/community/p28295670}{IMC 2023/2}, $9\clubsuit$]
  Let $A$, $B$ and $C$ be $n \times n$ matrices with complex entries satisfying
  \[ A^2=B^2=C^2 \text{ and } B^3 = ABC + 2\opname{id}. \]
  Prove that $A^6 = \opname{id}$.
\end{problem} \printpuid{23IMC2}

%% Type your solution to IMC 2023/2 (\href{https://otis.evanchen.cc/arch/23IMC2/}{23IMC2}) here ...

%% --------------------------------------------------

\end{document}
