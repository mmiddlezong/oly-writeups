\documentclass{scrartcl}
\usepackage{graphicx} % Required for inserting images
\usepackage{amsmath}
\usepackage{listings}
\usepackage{evan}

\title{OTIS Application Problems}
\author{Michael Middlezong}

\begin{document}

\lstset{
    language=Python,
    basicstyle=\ttfamily\small,
    keywordstyle=\color{blue},
    commentstyle=\color{green},
    stringstyle=\color{red},
    showstringspaces=false,
    numbers=left,
    numberstyle=\tiny\color{gray},
    stepnumber=1,
    numbersep=5pt,
    frame=single,
    breaklines=true,
    backgroundcolor=\color{white},
    tabsize=4,
    captionpos=b,
}

\maketitle

\section*{Geometry Problems}
\subsection*{Problem A.1}
The angle conditions imply $\overline{AB}$ is tangent to $(PSR)$ and $\overline{AC}$ is tangent to $(QRS)$.
Assume $(PSR)$ and $(QRS)$ are not the same circle.
Then, since $A$ has the same power with respect to both circles, $A$ must lie on the radical axis, $\overline{SR}$, which is just $\overline{BC}$, so that is impossible.
Hence, they are the same circle, and we are done.

\subsection*{Problem A.2}
It suffices to show that $P$ and $Q$ have the same power with respect to $(ABC)$.
We claim triangles $APQ$ and $MLK$ are similar.
We have
\[ \angle MKL = \angle PML = \angle QML = \angle QPC = \angle QPA, \]
and likewise for the other angle.

Finally,
\[ QB \cdot QA = 2MK \cdot QA = 2PA \cdot ML = PA \cdot PC, \]
so $P$ and $Q$ have the same power with respect to $(ABC)$.

\subsection*{Problem A.3}
Let $P$, $Q$, $R$, and $S$ be the feet of the perpendiculars from $E$ to $AB$, $BC$, $CD$, and $DA$, respectively.
By a homothety with scale factor $2$ at $E$, it suffices to prove $PQRS$ is cyclic.

Notice that $DSER$, $ASEP$, $BPEQ$, and $CQER$ are cyclic.
Then,
\begin{align*}
    \angle RSP &= \angle RSE + \angle ESP \\
    &= \angle CDE + \angle EAB \\
    &= 180 - (\angle ECD + \angle ABE) \\
    &= \angle DCE + \angle EBA \\
    &= \angle RQE + \angle EQP \\
    &= \angle RQP,
\end{align*}

so we are done.

\section*{Inequalities Problems}
\subsection*{Problem B.1}
We claim the answer is $\sqrt3$.
This answer is achieved when $a=b=c$.

First, we homogenize the inequality:
\[ \frac{ab}{c} + \frac{bc}{a} + \frac{ca}{b} \geq \sqrt{3(a^2+b^2+c^2)}. \]
Clearing denominators, we get
\[ a^2b^2 + b^2c^2 + a^2c^2 \geq abc\sqrt{3(a^2 + b^2 + c^2)}. \]
For convenience, perform the substitutions $x = a^2$, $y = b^2$, and $z = c^2$.
Then, after squaring both sides, we have
\[ (xy + yz + xz)^2 \geq 3(x+y+z)xyz. \]
This reduces to
\[ x^2y^2 + y^2z^2 + z^2x^2 \geq x^2yz + xy^2z + xyz^2. \]
Using Muirhead with $(2, 2, 0) \succ (2, 1, 1)$ gives us the result.

\subsection*{Problem B.3}
First, we homogenize the inequality by multiplying the RHS by $(a+c)(b+d) = ab + bc + cd + da = 1$:
\[ \sum_{cyc} \frac{a^3}{b+c+d} \geq \frac13 (ab + bc + cd + da)\]
Then, Holder gives us
\[ \left( \sum_{cyc} \frac{a^3}{b + c + d} \right)\left( \sum_{cyc} a(b+c+d) \right) \geq (a^2 + b^2 + c^2 + d^2)^2. \]
Noting that $\sum_{cyc} a(b + c + d) = 2(ab + ac + ad + bc + bd + cd)$, it suffices to prove
\[ 3(a^2 + b^2 + c^2 + d^2)^2 \geq 2(ab + bc + cd +da)(ab+ac+ad+bc+bd+cd). \]
It is not hard to show by AM-GM that
\[ a^2 + b^2 + c^2 + d^2 \geq ab + bc + cd + da. \]
Also, using Muirhead with $(2,0,0,0) \succ (1,1,0,0)$ yields
\[ 6(a^2 + b^2 + c^2 + d^2) \geq 4(ab + ac + ad + bc + bd + cd). \]
Putting these together yields the desired result.

\section*{Additional Problems}
\subsection*{Problem C.1}
The answer is 1270.

Python source code:
\begin{lstlisting}
    # Find all primes up to 10^5
    sieve = [0] * 100000
    for i in range(2, 50001):
      if sieve[i - 1] != 0:
        continue
      for j in range(2 * i, 100001, i):
        sieve[j - 1] = 1
    primes = []
    for i in range(2, 100001):
      if sieve[i - 1] == 0:
        primes.append(i)
    
    # Loop through primes^2
    count = 0
    for i in range(len(primes)):
      p = primes[i]
      for j in range(len(primes)):
        q = primes[j]
        N = p**2 + q**3
        if len(str(N)) > 10:
          break
        if len(str(N)) < 10:
          continue
        fails = False
        for c in range(10):
          fails |= not str(c) in str(N)
        if not fails:
          count += 1
    print(count)
\end{lstlisting}

\subsection*{Problem C.2}
Let $P(x,y)$ denote the assertion.
First, we show $f(0) = 0$.
Plugging in $y = -f(x)^2$ for any $x$, we see that
\[ f(xf(x) + f(-f(x)^2))) = 0. \]
For simplicity, let $u$ be any value satisfying $f(u) = 0$.
Then, $P(0,u)$ gives
\[ f(0) = f(0)^2 + u, \]
and $P(u,u)$ gives
\[ f(0) = u. \]
Thus, $u = u^2 + u \implies u = 0$.

Then, $P(0,x)$ gives
\[ f(f(x)) = x, \]
and thus $f$ is an involution, and therefore bijective.
Next, $P(f(t),0)$ gives
\[ f(tf(t)) = t^2, \]
and $P(t,0)$ gives
\[ f(tf(t)) = f(t)^2. \]
Thus, $f(t) \in \{-t, t\}$ for all $t$.

Assume for the sake of contradiction that $f(a) = a$ and $f(b) = -b$ for nonzero $a,b$.
Then, $P(a,b)$ yields
\[ f(a^2 - b) = a^2 + b, \]
which implies either $a$ or $b$ is zero.
Therefore, the possible functions for $f(x)$ are $f(x) = x$ and $f(x) = -x$.
These can be easily checked to work.

\subsection*{Problem C.3}
The answer is $16$.
This is achievable; simply take $P(x) = (x+1)^4$.
Now, we show it is the minimum.

We have
\begin{align*}
    \prod_{j=1}^4 (x_j + 1) &= \prod_{j=1}^4 (x_j - i)(x_j + i) \\
    &= \prod_{j=1}^4 (x_j - i) \prod_{j=1}^4 (x_j + i) \\
    &= [(d-b+1)+(c-a)i][(d-b+1)-(c-a)i] \\
    &= (d-b+1)^2 + (c-a)^2.
\end{align*}

Since $\abs{d-b+1} + \abs{-1} \geq \abs{d-b} \geq 5$ by the triangle inequality, we know $\abs{d-b+1} \geq 4$.
Thus, the minimum value of the desired expression is $4^2 + 0^2 = 16$.

\subsection*{Problem C.4}
Given a word, define a streak to be a maximal contiguous subsequence with all characters equal.
We claim that all words with a streak of length $1$ work.
Indeed, if Banana chooses $k$, Ana can simply supply the word created by replacing any streak of length $1$ in Ana's original word with $k$ copies of the streak.
Clearly, this word has exactly $k$ subsequences matching Ana's original word.
So, it suffices to show all other words do not work.

Let Ana's original word be $w = \{w_i\}_{1 \leq i \leq L}$, where $L$ is the length of her word.
If $w$ does not have a streak of length $1$, then for all $1 \leq i \leq L$, either $w_{i-1} = w_i$ or $w_i = w_{i+1}$.
We claim that it is impossible for Ana to supply a word with exactly $2$ subsequences equal to $w$.

Assume FTSOC that Ana can supply a word $\{x_i\}_{1 \leq i \leq M}$ of length $M$ with exactly two subsequences equal to $w$.
We will get a contradiction by constructing another subsequence equal to $w$.

Denote the two distinct subsequences by $\{x_{a_i}\}$ and $\{x_{b_i}\}$, where $\{a_i\}_{1 \leq i \leq L}$ and $\{b_i\}_{1 \leq i \leq L}$ are indexing sequences (i.e., increasing and with terms in $[1,M]$).

First, we claim that $\{a_i\}$ and $\{b_i\}$ differ by exactly one element.
If not, then let $n > m$ be any two indices in which they differ.

Assume WLOG that $a_n > b_n$.
Then, we claim the sequence
\[ b_1, b_2, \dots, b_{n-1}, a_n, a_{n+1}, \dots, a_L \]
is an indexing sequence for a subsequence of $\{x_i\}$ equal to $w$.
It suffices to show $a_n > b_{n-1}$, but this is true because $a_n > b_n > b_{n-1}$.
Furthermore, this sequence differs from $\{a_i\}$ at index $m$ and from $\{b_i\}$ at index $n$, so this is a third valid subsequence, giving us a contradiction.

Next, let $m$ be the index at which the indexing subsequences differ.
Assume WLOG that $a_m > b_m$.
Then, we have two cases.

\textbf{Case 1: $w_{m-1} = w_m$.} In this case, we replace $a_{m-1}$ with $b_m$ to get another indexing sequence.
Specifically, we have $x_{b_m} = w_m = w_{m-1}$ and $a_m > b_m > b_{m-2} = a_{m-2}$, so
\[ a_1, a_2, \dots, a_{m-2}, b_m, a_m, a_{m+1}, a_{m+2}, a_{m+3}, \dots, a_L \]
is an indexing sequence for a third valid subsequence.

\textbf{Case 2: $w_{m+1} = w_m$.} In this case, we replace $a_m$ with $b_m$ and $a_{m+1}$ with $a_m$ to get another indexing sequence.
Specifically, we have $x_{a_m} = w_m = w_{m+1}$ and $a_m > b_m > b_{m-1} = a_{m-1}$, so
\[ a_1, a_2, \dots, a_{m-2}, a_{m-1}, b_m, a_m, a_{m+2}, a_{m+3}, \dots, a_L \]
is an indexing sequence for a third valid subsequence.

Thus, we have a contradiction in either case, and we are done.


\end{document}