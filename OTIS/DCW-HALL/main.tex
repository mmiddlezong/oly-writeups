\documentclass[11pt]{scrartcl}
\usepackage[usenames,dvipsnames,svgnames]{xcolor}
\usepackage[shortlabels]{enumitem}
\usepackage[framemethod=TikZ]{mdframed}
\usepackage{amsmath,amssymb,amsthm}
\usepackage{epigraph}
\usepackage[colorlinks]{hyperref}
\usepackage{microtype}
\usepackage{mathtools}
\usepackage[headsepline]{scrlayer-scrpage}
\usepackage{thmtools}
\usepackage{listings}
\usepackage{derivative}
\renewcommand{\epigraphsize}{\scriptsize}
\renewcommand{\epigraphwidth}{60ex}
\addtolength{\textheight}{3.14cm}
\ihead{\footnotesize\textbf{DCW-hall}}
\ohead{\footnotesize Updated Wed 14 Aug 2024 23:43:07 UTC}
\providecommand{\clubs}[1]{$#1\clubsuit$}
\providecommand{\clubg}[1]{\bgroup\color{green!40!black}[$#1\clubsuit$]\egroup}

\providecommand{\ol}{\overline}
\providecommand{\eps}{\varepsilon}
\providecommand{\half}{\frac{1}{2}}
\providecommand{\dang}{\measuredangle} %% Directed angle
\providecommand{\CC}{\mathbb C}
\providecommand{\FF}{\mathbb F}
\providecommand{\NN}{\mathbb N}
\providecommand{\QQ}{\mathbb Q}
\providecommand{\RR}{\mathbb R}
\providecommand{\ZZ}{\mathbb Z}
\providecommand{\dg}{^\circ}
\providecommand{\ii}{\item}
\providecommand{\alert}{\textbf}
\providecommand{\opname}{\operatorname}
\providecommand{\ts}{\textsuperscript}
% hacks for arc
\providecommand{\tarc}{\mbox{\large$\frown$}}
\providecommand{\arc}[1]{\stackrel{\tarc}{#1}}
\reversemarginpar
\providecommand{\printpuid}[1]{\marginpar{\href{https://otis.evanchen.cc/arch/#1}{\ttfamily\footnotesize\color{green!40!black}#1}}}

\mdfdefinestyle{mdbluebox}{roundcorner=10pt,innerbottommargin=9pt,
    linecolor=blue,backgroundcolor=TealBlue!5,}
\declaretheoremstyle[headfont=\sffamily\bfseries\color{MidnightBlue},
    mdframed={style=mdbluebox},]{thmbluebox}
\mdfdefinestyle{mdredbox}{frametitlefont=\bfseries,innerbottommargin=8pt,
    nobreak=true,backgroundcolor=Salmon!5,linecolor=RawSienna,}
\declaretheoremstyle[headfont=\bfseries\color{RawSienna},
    mdframed={style=mdredbox},headpunct={\\[3pt]},postheadspace=0pt,]{thmredbox}
\mdfdefinestyle{mdgreenbox}{linecolor=ForestGreen,backgroundcolor=ForestGreen!5,
    linewidth=2pt,rightline=false,leftline=true,topline=false,bottomline=false,}
\declaretheoremstyle[headfont=\bfseries\sffamily\color{ForestGreen!70!black},
    mdframed={style=mdgreenbox},headpunct={ --- },]{thmgreenbox}
\mdfdefinestyle{mdblackbox}{linecolor=black,backgroundcolor=RedViolet!5!gray!5,
    linewidth=3pt,rightline=false,leftline=true,topline=false,bottomline=false,}
\declaretheoremstyle[mdframed={style=mdblackbox}]{thmblackbox}
\declaretheorem[style=thmredbox,name=Problem]{problem}
\declaretheorem[style=thmredbox,name=Required Problem,sibling=problem]{reqproblem}
\declaretheorem[style=thmbluebox,name=Theorem,numberwithin=problem]{theorem}
\declaretheorem[style=thmbluebox,name=Lemma,sibling=theorem]{lemma}
\declaretheorem[style=thmbluebox,name=Theorem,numbered=no]{theorem*}
\declaretheorem[style=thmbluebox,name=Lemma,numbered=no]{lemma*}
\declaretheorem[style=thmgreenbox,name=Claim,sibling=theorem]{claim}
\declaretheorem[style=thmgreenbox,name=Claim,numbered=no]{claim*}
\declaretheorem[style=thmblackbox,name=Remark,sibling=theorem]{remark}
\declaretheorem[style=thmblackbox,name=Remark,numbered=no]{remark*}
\declaretheorem[style=thmgreenbox,name=Definition,sibling=theorem]{definition}
\declaretheorem[style=thmgreenbox,name=Definition,numbered=no]{definition*}
\declaretheorem[style=thmblackbox,name=Example,sibling=theorem]{example}
\declaretheorem[style=thmblackbox,name=Example,numbered=no]{example*}

\newenvironment{walkthrough}{\noindent\textbf{\color{green!40!black}Walkthrough.}}{}
\newlist{walk}{enumerate}{3}
\setlist[walk]{label=\bfseries (\alph*)}

\usepackage{asymptote}
\begin{asydef}
size(8cm); // set a reasonable default
usepackage("amsmath");
usepackage("amssymb");
settings.tex="pdflatex";
settings.outformat="pdf";
import geometry;
void filldraw(picture pic = currentpicture, conic g, pen fillpen=defaultpen, pen drawpen=defaultpen) { filldraw(pic, (path) g, fillpen, drawpen); }
void fill(picture pic = currentpicture, conic g, pen p=defaultpen) { filldraw(pic, (path) g, p); }
pair foot(pair P, pair A, pair B) { return foot(triangle(A,B,P).VC); }
pair centroid(pair A, pair B, pair C) { return (A+B+C)/3; }
\end{asydef}

\newcommand{\goals}[2]{\bgroup
\sffamily\small \emph{Instructions}: Solve \clubg{#1}.
If you have time, solve \clubg{#2}.\egroup\par}

%% 426c616e6b204c615465587e
\begin{document}
\title{Submission for DCW-HALL}
\subtitle{OTIS (internal use)}
\author{Michael Middlezong}
\date{\today}
\maketitle

\begin{example*}[PUMaC 2016, Alex Song, $0\clubsuit$]
  There are twelve candies arranged in a circle, four of which are rare candies.
  Chad and Eric want to collaborate on a strategy for the following act.
  First, Eric comes and is told which four candies are rare candies,
  then removes four non-rare candies from the circle.
  Then Eric leaves, and Chad comes and must determine which of the
  four candies (of the eight remaining candies) are rare.
  Decide whether this is possible or not.
\end{example*} \printpuid{16PUMACFA1}

\begin{walkthrough}
Let $|S| = 12$.
Most of this problem is trying to think about the problem
in the right way; afterwards, Hall will destroy it.
\begin{walk}
  \ii Phrase the problem as trying to match $\binom{S}{4}$ to itself.
  \ii What are the constraints on this matching?
  \ii Apply regular Hall to finish.
\end{walk}
\end{walkthrough}
%% You're not expected to write up walkthroughs (unless you really want to).
%% The source is just for your reference.

\newpage

% ========================================
\section*{Practice problems}
\goals{24}{32}

\epigraph{I'm looking for something that can deliver a
50-pound payload of snow on a small feminine target.
Can you suggest something? Hello\dots?}
{Calvin on the phone in \emph{Calvin and Hobbes}}

\begin{problem}[Another useful-later special case of Hall, $2\clubsuit$]
  Let $G = A \cup B$ be a bipartite graph on $2n$ vertices
  with minimum degree $n/2$ and $|A|=|B|=n$.
  Show that $G$ has a perfect matching.
\end{problem} \printpuid{ZB9281DC}

%% Type your solution to Another useful-later special case of Hall (\href{https://otis.evanchen.cc/arch/ZB9281DC/}{ZB9281DC}) here ...
If $k \le \frac{n}{2}$, then the desired result is trivial.
Otherwise, we claim that any set of $k > \frac n2$ gifts is compatible with all $n$ boxes.
Assume not; then, some box is only compatible with less than $\frac n2$ gifts, which is a contradiction.
%% --------------------------------------------------

\begin{problem}[Infinite Hall, added by Edward Yu, $2\clubsuit$]
  Show that Hall's theorem does not hold for infinite graphs,
  even if they are countable.
  That is, find an infinite bipartite graph between
  countably infinite sets $A$ and $B$
  for which every finite subset of $S \subseteq A$
  has at least $|S|$ neighbors in $B$,
  but there is no way of matching $A$ to a subset of $B$.
\end{problem} \printpuid{ZB1FC962}

%% Type your solution to Infinite Hall, added by Edward Yu (\href{https://otis.evanchen.cc/arch/ZB1FC962/}{ZB1FC962}) here ...
Let $A = \{0,1,2,\dots\}$ and $B = \{1,2,\dots\}$.
Match $0$ in $A$ with all numbers in $B$, and
for any $i \neq 0 \in A$, match it with only $i$ in $B$.
Clearly, there cannot be a perfect matching, even though the two sets
have the same cardinality.
%% --------------------------------------------------

\begin{problem}[Birkoff von Neumann, $2\clubsuit$]
  Let $A$ be a square $n \times n$ matrix whose entries are nonnegative
  and whose rows and columns have sum $1$ (such a matrix is \emph{doubly stochastic}).
  Show that it's possible to write
  \[ A = c_1 P_1 + \dots + c_m P_m \]
  where each $P_i$ is a permutation matrix, and each $c_i$ is a positive real.
\end{problem} \printpuid{ZFD257C4}

%% Type your solution to Birkoff von Neumann (\href{https://otis.evanchen.cc/arch/ZFD257C4/}{ZFD257C4}) here ...

%% --------------------------------------------------

\begin{problem}[Baltic Way 2013, $2\clubsuit$]
  Santa Claus has some gifts for $n$ children.
  For $1 \le i \le n$, the $i$-th child considers $x_i > 0$ of these items to be desirable.
  Assume that \[  \frac{1}{x_1} + \dots + \frac{1}{x_n} \le 1. \]
  Prove that Santa Claus can give each child a gift that this child likes.
\end{problem} \printpuid{13BWAY}

%% Type your solution to Baltic Way 2013 (\href{https://otis.evanchen.cc/arch/13BWAY/}{13BWAY}) here ...
We claim that in any set of $k$ children, at least one of them likes at least $k$ of Santa's gifts.
This is clearly sufficient to prove Hall's condition, and thus, the desired conclusion.

Suppose not. Then, summing over the $k$ children,
\[ \sum_{i \in S} \frac{1}{x_i} > \sum_{i \in S} \frac{1}{k} = 1, \]
which, when combined with the problem statement's condition, gives us a contradiction.
%% --------------------------------------------------

\begin{reqproblem}[$3\clubsuit$]
  On a $1000 \times 1000$ chessboard, we delete some squares from the board
  so that each row and each column has at most $k$ deleted squares.
  For which values of $k$ is it always still possible to
  place $1000$ non-attacking rooks on the board?
  (Rooks may still attack across deleted squares.)
\end{reqproblem} \printpuid{ZF13143F}

%% Type your solution to \href{https://otis.evanchen.cc/arch/ZF13143F/}{ZF13143F} here ...
We form a relation (bipartite graph) between the columns and the rows.
Specifically, we connect a row with a column if that row has a cell
in that column (i.e. it wasn't deleted).
Also, placing 1000 non-attacking rooks is just
a perfect matching between the rows and columns.

The condition of removing at most $k$ squares from each row and column
is equivalent to saying that each vertex has degree
greater than or equal to $1000-k$. If $1000-k \ge 1000/2 = 500$, meaning $k \le 500$,
then by the result of Problem 1, we are done.

If $k > 500$, then delete the bottom $k$ cells from the $k$ leftmost columns
and delete the leftmost $k$ cells from the bottom $k$ rows.
It can be easily checked that every row and column has either $k$ or $1000-k$ deleted cells,
which satisfies the condition.
Also, the first $k$ columns are only connected with $1000-k < k$ rows,
so Hall's condition is not satisfied.
%% --------------------------------------------------

\begin{problem}[Tuymaada 2018/7, $3\clubsuit$]
  A school has three senior classes of $M$ students each.
  Every student knows at least $\frac34M$ people in each of the other two classes.
  Prove that the school can send $M$ non-intersecting teams
  to the olympiad so that each team consists of $3$ students
  from different classes who know each other.
\end{problem} \printpuid{18TMD7}

%% Type your solution to Tuymaada 2018/7 (\href{https://otis.evanchen.cc/arch/18TMD7/}{18TMD7}) here ...
The solution is to do Hall twice. First, we consider only two of the classes.
By the result of problem 1, we can pair the students into $M$ pairs
where each pair consists of $2$ students from different classes who know each other.

Then, consider the $M$ pairs on one side of a bipartite graph,
and the $M$ students of the third senior class on the other side.
We connect a pair with a student from the third class if the student knows both students in the pair.
Notice that this graph has minimum degree $\frac 12 M$, because otherwise,
some student would not know at least $\frac 34 M$ students from each of the other two classes.
So, we use the result of problem 1 again to create $M$ teams of $3$.
%% --------------------------------------------------

\begin{problem}[\href{https://aops.com/community/p2865753}{Putnam 2012 B3}, $3\clubsuit$]
  A round-robin tournament among $2n$ teams
  lasted for $2n-1$ days, as follows.
  On each day, every team played one game against another team,
  with one team winning and one team losing in each of the $n$ games.
  Over the course of the tournament,
  each team played every other team exactly once.
  Can one necessarily choose one winning team from each day
  without choosing any team more than once?
\end{problem} \printpuid{12PTNMB3}

%% Type your solution to Putnam 2012 B3 (\href{https://otis.evanchen.cc/arch/12PTNMB3/}{12PTNMB3}) here ...

%% --------------------------------------------------

\begin{problem}[Halls Chessboard, $3\clubsuit$]
  An $n \times n$ chessboard has some of its squares painted blue.
  Assume that for every $n$ squares chosen, no two in the same row or column, at least one of the squares is blue.
  Prove that one can find $a$ rows and $b$ columns whose intersection contains only blue squares, so that $a+b \ge n+1$.
\end{problem} \printpuid{Z685102E}

%% Type your solution to Halls Chessboard (\href{https://otis.evanchen.cc/arch/Z685102E/}{Z685102E}) here ...

%% --------------------------------------------------

\begin{problem}[$3\clubsuit$]
  A table has $m$ rows and $n$ columns with $m,n>1$.
  The following permutations of its $mn$ elements are permitted:
  any permutation leaving each element in the same row (a ``horizontal move''),
  and any permutation leaving each element in the same column (a ``vertical move'').
  Find the smallest integer $k$ in terms of $m$ and $n$
  such that any permutation of the $mn$ elements
  can be realized by at most $k$ permitted moves.
\end{problem} \printpuid{WOOTPO46}

%% Type your solution to \href{https://otis.evanchen.cc/arch/WOOTPO46/}{WOOTPO46} here ...

%% --------------------------------------------------

\begin{problem}[$3\clubsuit$]
  Let $G$ be a graph.
  Show that $G$ has an orientation in which every indegree
  is at most
  \[ \left\lceil \max_H \frac{\#E(H)}{\#V(H)} \right\rceil \]
  where the maximum is taken over nonempty subgraphs of $H$.
\end{problem} \printpuid{Z5F11887}

%% Type your solution to \href{https://otis.evanchen.cc/arch/Z5F11887/}{Z5F11887} here ...

%% --------------------------------------------------

\begin{problem}[\href{https://aops.com/community/p21567977}{Russia 2021, added by Rohan Goyal}, $3\clubsuit$]
  Each gopher among $100$ gophers has $100$ balls;
  there are in total $10000$ balls in $100$ colors,
  from each color there are $100$ balls.
  On a move, two gophers can exchange a ball
  (the first gives the second one of her balls, and vice versa).
  The operations can be made in such a way, that in the end,
  each gopher has $100$ balls, colored in the $100$ distinct colors.

  Prove that there is a sequence of operations,
  in which each ball is exchanged no more than $1$ time,
  and at the end, each gopher has $100$ balls,
  colored in the $100$ colors.
\end{problem} \printpuid{21RUS118}

%% Type your solution to Russia 2021, added by Rohan Goyal (\href{https://otis.evanchen.cc/arch/21RUS118/}{21RUS118}) here ...

%% --------------------------------------------------

\begin{problem}[$5\clubsuit$]
  Let $G$ be a bipartite graph on $A \cup B$ with no isolated vertices.
  Assume that for each edge $ab$ with $a \in A$ and $b \in B$,
  we have $\deg a \ge \deg b$.
  Prove that $G$ contains a matching using all vertices in $A$.
\end{problem} \printpuid{Z043CFE1}

%% Type your solution to \href{https://otis.evanchen.cc/arch/Z043CFE1/}{Z043CFE1} here ...

%% --------------------------------------------------

\begin{problem}[\href{https://aops.com/community/c6h42400p7704829}{Vietnam TST 2001/3, added by Milind Pattanik}, $3\clubsuit$]
  A club has $42$ members, each of which is either a boy or girl.
  It’s known that among $31$ arbitrary club members,
  we can find a boy and a girl who know each other.
  Show that there exist $12$ disjoint pairs of club members,
  each of which is a boy and a girl that know each other.
\end{problem} \printpuid{01VNMTST3}

%% Type your solution to Vietnam TST 2001/3, added by Milind Pattanik (\href{https://otis.evanchen.cc/arch/01VNMTST3/}{01VNMTST3}) here ...

%% --------------------------------------------------

\begin{problem}[\href{https://aops.com/community/p875004}{Shortlist 2006 C6}, $9\clubsuit$]
  An upward equilateral triangle of side length $n$
  is divided into $n^2$ cells which are equilateral triangles of unit length.
  A \emph{holey triangle} is such a triangle
  with $n$ upward unit triangular holes cut out along gridlines.
  A diamond is a $60^\circ-120^\circ$ unit rhombus.
  Prove that a holey triangle $T$ can be tiled with diamonds
  if and only if the following condition holds:
  Every upward equilateral triangle of side length $k$ in $T$
  contains at most $k$ holes, for $1 \le k \le n$.
\end{problem} \printpuid{06SLC6}

%% Type your solution to Shortlist 2006 C6 (\href{https://otis.evanchen.cc/arch/06SLC6/}{06SLC6}) here ...

%% --------------------------------------------------

\begin{problem}[Bulgaria 1997, $3\clubsuit$]
  Let $n \ge 2$ be a positive integer,
  and consider ordered $n$-tuples of
  distinct integers in the set $\{1, \dots, n+1\}$.
  Two such tuples $(a_1, \dots, a_n)$ and $(b_1, \dots, b_n)$
  are called \emph{disjoint} if
  there exists $1 \le i,j \le n$
  such that $i \neq j$ and $a_i = b_j$.
  What is the maximum possible number of pairwise disjoint $n$-tuples?
\end{problem} \printpuid{97BGR}

%% Type your solution to Bulgaria 1997 (\href{https://otis.evanchen.cc/arch/97BGR/}{97BGR}) here ...

%% --------------------------------------------------

\begin{reqproblem}[\href{https://aops.com/community/p2362296}{Shortlist 2010 C2}, $5\clubsuit$]
  Let $n \ge 4$ be an integer.
  A \emph{flag} is a binary string of length $n$.
  We say that a set of $n$ flags is \emph{diverse} if these flags
  can be the rows of an $n \times n$ binary matrix
  with the entries in its main diagonal all equal.
  Determine the smallest positive integer $M$ such that among any $M$ distinct flags,
  there exist $n$ flags forming a diverse set.
\end{reqproblem} \printpuid{10SLC2}

%% Type your solution to Shortlist 2010 C2 (\href{https://otis.evanchen.cc/arch/10SLC2/}{10SLC2}) here ...
We claim the answer is $M = 2^{n-2} + 1$. First, $2^{n-2}$ doesn't work;
just consider all the $2^{n-2}$ flags starting with $01$.

Now, we show that $M = 2^{n-2} + 1$ works.
Let $D$ be either $0$ or $1$, and construct a bipartite graph between
the set of $M$ flags (arranged into an $M$ by $n$ grid) and the $n$ columns.
We connect a row with column $i$ if that row has $D$ in column $i$.
It suffices to find a perfect matching from the columns to the rows (flags).

We claim that this graph satisfies Hall's condition for some choice of $D$, first starting with $k=1$.
Hall's condition for $k=1$ is only false for some $D$ if a column
has the binary digit other than $D$ in every row.
But this can't be true for both choices of $D$, as that
would restrict the number of unique flags to $2^{n-2} < M$.
So, at least one choice of $D$ satisfies Hall's condition for $k=1$.

We do casework on how many choices of $D$ satisfy Hall's condition for $k=1$.
If both choices of $D$ satisfy it, then
we claim that Hall's condition for $k=2$ must be true for some choice of $D$.
Suppose not. Then, $M-1$ rows must have $0$'s in two fixed columns,
and $M-1$ rows must have $1$'s in some two other fixed columns.
These conditions overlap in at least $M-2$ rows,
but the number of distinct flags that satisfy both is $2^{n-4} < M-2$, giving us a contradiction.

If only one choice of $D$ satisfies Hall's condition for $k-1$, then
WLOG let that choice be $D = 0$.
We claim that Hall's condition for $k=2$ must be true for $D = 0$.
Suppose not. Then, at least $M-1$ rows have $1$'s in two fixed columns.
These rows also have $0$ in another fixed column, since $D=1$ fails Hall's condition for $k=1$.
The number of distinct flags satisfying these conditions is $2^{n-3} < M-1$,
so we have a contradiction.
Therefore, Hall's condition must be true for $k=1,2$, for some choice of $D$.

Again WLOG that choice to be $D=0$, and in one final sweep,
we show Hall's condition for $k \ge 3$. Again, suppose not.
Then, at least $M-k+1$ flags have $1$ in some $k$ fixed columns.
But $2^{n-k}$ distinct flags satisfy that condition,
and $k \ge 3 \implies 2^{n-k} < M-k+1$, so we are done.

%% --------------------------------------------------

\begin{problem}[\href{https://aops.com/community/p7743073}{RMM 2017/5}, $5\clubsuit$]
  Fix an integer $n \geq 2$. An $n\times n$ sieve is an $n\times n$
  array with $n$ cells removed so that exactly one cell
  is removed from every row and every column.
  A stick is a $1\times k$ or $k\times 1$ array for any integer $k \ge 1$.
  For any sieve $A$, let $m(A)$ be the minimal number
  of sticks required to partition $A$.
  Find all possible values of $m(A)$,
  as $A$ varies over all possible $n\times n$ sieves.
\end{problem} \printpuid{17RMM5}

%% Type your solution to RMM 2017/5 (\href{https://otis.evanchen.cc/arch/17RMM5/}{17RMM5}), proposed by Palmer Mebane, Nikolai Beluhov here ...

%% --------------------------------------------------

\begin{problem}[\href{https://aops.com/community/p3160571}{Shortlist 2012 C5}, $5\clubsuit$]
  The columns and the rows of a $3n \times 3n$
  square board are numbered $1,2,\dots,3n$.
  Every square $(x,y)$ with $1 \leq x,y \leq 3n$ is colored
  asparagus, byzantium or citrine according as the modulo $3$ remainder
  of $x+y$ is $0$, $1$ or $2$ respectively.
  One token colored asparagus, byzantium or citrine is placed on each square,
  so that there are $3n^2$ tokens of each color.

  Suppose that one can permute the tokens so that each token
  is moved to a distance of at most $d$ from its original position,
  each asparagus token replaces a byzantium token,
  each byzantium token replaces a citrine token,
  and each citrine token replaces an asparagus token.
  Prove that it is possible to permute the tokens so that each token
  is moved to a distance of at most $d+2$ from its original position,
  and each square contains a token with the same color as the square.
\end{problem} \printpuid{12SLC5}

%% Type your solution to Shortlist 2012 C5 (\href{https://otis.evanchen.cc/arch/12SLC5/}{12SLC5}) here ...

%% --------------------------------------------------

\begin{problem}[Harder 2012 C5 by Ji Min Kim, $5\clubsuit$]
  The columns and the rows of a $3n \times 3n$
  square board are numbered $1,2,\dots,3n$.
  Every square $(x,y)$ with $1 \leq x,y \leq 3n$ is colored
  asparagus, byzantium or citrine according as the modulo $3$ remainder
  of $x+y$ is $0$, $1$ or $2$ respectively.
  One token colored asparagus, byzantium or citrine is placed on each square,
  so that there are $3n^2$ tokens of each color.

  Suppose that one can permute the tokens so that each token
  is moved to a distance of at most $d$ from its original position,
  each asparagus token replaces a byzantium token,
  each byzantium token replaces a citrine token,
  and each citrine token replaces an asparagus token.
  Prove that it is possible to permute the tokens so that each token
  is moved to a distance of at most $d+1$ from its original position,
  and each square contains a token with the same color as the square.
\end{problem} \printpuid{12SLC5GEN}

%% Type your solution to Harder 2012 C5 by Ji Min Kim (\href{https://otis.evanchen.cc/arch/12SLC5GEN/}{12SLC5GEN}), proposed by Ji Min Kim (KOR) here ...

%% --------------------------------------------------

\begin{remark*}
  The ``Harder 2012 C5'' is 2012 C5 with $d+2$ improved to $d+1$.
  Solving this problem also gives you the \clubs{5} for the original.
\end{remark*}

\begin{reqproblem}[\href{https://aops.com/community/p26628764}{Math Prize 2022/4}, $9\clubsuit$]
  Let $n > 1$ be an integer.
  Let $A$ denote the set of divisors of $n$ that are less than $\sqrt n$.
  Let $B$ denote the set of divisors of $n$ that are greater than $\sqrt n$.
  Prove that there exists a bijective function $f \colon A \to B$
  such that $a$ divides $f(a)$ for all $a \in A$.
\end{reqproblem} \printpuid{22MPO4}

%% Type your solution to Math Prize 2022/4 (\href{https://otis.evanchen.cc/arch/22MPO4/}{22MPO4}) here ...
Split the divisors of $n$ into equivalence classes based on the following relation:
divisors $a$ and $b$ are related if for every prime $p$ dividing $n$,
either $\nu_p(a) + \nu_p(b) = \nu_p(n)$ or $\nu_p(a) = \nu_p(b)$.
This relation is partially motivated by the representation of the
divisors of $n$ as lattice points in a hypercuboid.

We claim that in each equivalence class $C$, we can
construct a bijection $f$: $A \cap C \to B \cap C$
satisfying $d \mid f(d)$ for all $d$.

Notice that within each equivalence class $C$ and for a fixed prime $p \mid n$,
there are at most two possible values of $\nu_p(d)$ for $d \in C$. If
a prime has only one possible value of $\nu_p(d)$ for all $d \in C$,
then the divisibility condition is trivial for that prime (i.e. we might as well divide
all numbers in $C$ by that prime raised to the maximal exponent).
So, call a prime $p$ \textit{significant} when there are two possible values of $\nu_p(d)$,
and let $e_p$ be the smaller value and $E_p$ be the greater value.

Our bijection $f$ can be reframed as a bijection $g$ from $A \cap C$ to itself,
by reframing our condition $d \mid f(d)$ as
\[ \min\{\nu_p(d), \nu_p(g(d))\} = e_p \]
for all significant primes $p$.
From this, we can construct $f$ as $f(d) = \frac{n}{g(d)}$.

Now, we reframe the problem again. Let $X$ be the set of prime divisors of $n$,
and associate each element $d \in A \cap C$ with the set containing
all significant primes $p$ such that $d$ is divisible by $p^{E_p}$.
It is not hard to check that no two elements can be associated with the same set.

Consider the collection $\mathcal F$ of sets associated with elements
in $d$. This collection satisfies the property that for all $A \in \mathcal F$,
all subsets of $A$ are also in $\mathcal F$ (i.e. $\mathcal F$ is closed downward).
The central claim of the problem is that there exists a bijection
$\sigma: \mathcal F \to \mathcal F$ such that $\sigma(A) \cap A = \varnothing$
for every set $A \in \mathcal F$. This is indeed equivalent to the condition
for the bijection $g$ that we stated earlier. It is also a problem
from an earlier Iran olympiad: \url{https://aops.com/community/p20560}.

We proceed by induction on $\left\lvert X \right\rvert$.
The base case $\left\lvert X \right\rvert = 0$ is trivial.
For the inductive step, fix an element $x \in X$.
We divide the sets in $\mathcal F$ into three categories:
\begin{itemize}
  \item \textbf{Type 1 sets:} Sets that contain $x$.
  \item \textbf{Type 2 sets:} Sets that don't contain $x$ but
  would become a type 1 set if $x$ were added to it.
  \item \textbf{Type 3 sets:} All other sets (not containing $x$).
\end{itemize}
Notice that the set of type 2 sets and the set of type 3 sets are each closed downward.
We construct $\sigma$ as follows:
\begin{itemize}
  \item Let $\sigma_1$ be a bijection from the set of type 2 sets to itself,
  satisfying the condition $\sigma_1(A) \cap A = \varnothing$. It exists
  because of the inductive hypothesis.
  \item Similarly, let $\sigma_2$ be a bijection from the set of type 3 sets to itself,
  satisfying the condition.
  \item If $A$ is a type 1 set, then $\sigma$ takes $A$ to $\sigma_1(A \setminus \{x\})$.
  \item If $A$ is a type 2 set, then $\sigma$ takes $A$ to $\sigma_1(A) \cup \{x\}$.
  \item If $A$ is a type 3 set, then $\sigma(A) = \sigma_2(A)$.
\end{itemize}
One can check that this is indeed a bijection that satisfies our criterion.

%% --------------------------------------------------

\end{document}
