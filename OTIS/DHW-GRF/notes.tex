\documentclass{scrartcl}
\usepackage{graphicx} % Required for inserting images
\usepackage{amsmath}
\usepackage{listings}
\usepackage[sexy]{evan}

\title{Groups, Rings, and Fields}
\subtitle{DHW-GRF Expository Notes}
\author{Michael Middlezong}

\begin{document}
\maketitle

\section{Groups}

Groups are a fundamental structure in abstract algebra, and they can be thought of as representing
the structure of the symmetries of an object.
\begin{example}
    For example, the rotational and reflectional symmetries of a square form the group called $D_4$.
\end{example}
Groups have certain properties that make them so fundamental, leading us to the definition of a group.
\begin{definition}
    A \textbf{group} is a set $G$ with a binary operation $\star$: $G \times G \to G$, satisfying
    the following properties:
    \begin{enumerate}
        \item \textbf{Closure:} For all $a,b \in G$, $a \star b \in G$.
        This is implied by the term "binary operation."
        \item \textbf{Associativity:} For all $a,b,c \in G$,
        \[ (a \star b) \star c = a \star (b \star c). \]
        This is another property of the operation itself.
        \item \textbf{Existence of an identity:} There exists an element $e \in G$ such that
        for all $g \in G$:
        \[ e \star g = g \star e = g. \]
        \item \textbf{Existence of inverses:} Every element must have an inverse.
        Precisely, for every element $g \in G$, there exists
        an element $h \in G$, which we call the inverse of $g$, satisfying
        \[ h \star g = g \star h = e, \]
        where $e$ is an identity element.
    \end{enumerate}
\end{definition}
Sometimes, a group is referred to with the operation attached to it, like $(\mathbb{Z}, +)$,
but when the operation is obvious, we just refer to the group by the set, like $\mathbb{Z}$.

Here is an example:
\begin{example}[Additive integers]
    The set of all integers $\mathbb{Z}$ with the binary operation addition ($+$) form a group.
    This is because adding two integers gives another integer,
    addition is associative,
    $0$ is an identity element,
    and for any element $x$, the element $-x$ is its inverse.
\end{example}
\begin{example}[Symmetries of a square]
    Consider all the actions we can do on a square that preserve its position in space.
    These are:
    \begin{enumerate}
        \item Do nothing.
        \item Rotate by $90^\circ$ (counter-clockwise).
        \item Rotate by $180^\circ$.
        \item Rotate by $270^\circ$.
        \item Reflect about the vertical axis.
        \item Reflect about the horizontal axis.
        \item Reflect about a diagonal axis (say, $y=x$).
        \item Reflect about the other diagonal axis (say, $y=-x$).
    \end{enumerate}
    We can check that these form a group under the operation of "doing one thing after another."
    Notice how different this is from a group of just numbers!
    Also, this group has a certain structure which we call $D_4$.
\end{example}

\end{document}
