\documentclass[11pt]{scrartcl}
\usepackage[usenames,dvipsnames,svgnames]{xcolor}
\usepackage[shortlabels]{enumitem}
\usepackage[framemethod=TikZ]{mdframed}
\usepackage{amsmath,amssymb,amsthm}
\usepackage{epigraph}
\usepackage[colorlinks]{hyperref}
\usepackage{microtype}
\usepackage{mathtools}
\usepackage[headsepline]{scrlayer-scrpage}
\usepackage{thmtools}
\usepackage{listings}
\usepackage{derivative}
\renewcommand{\epigraphsize}{\scriptsize}
\renewcommand{\epigraphwidth}{60ex}
\addtolength{\textheight}{3.14cm}
\ihead{\footnotesize\textbf{DHW-grf}}
\ohead{\footnotesize Updated Sun 17 Nov 2024 00:12:08 UTC}
\providecommand{\clubs}[1]{$#1\clubsuit$}
\providecommand{\clubg}[1]{\bgroup\color{green!40!black}[$#1\clubsuit$]\egroup}

\providecommand{\ol}{\overline}
\providecommand{\eps}{\varepsilon}
\providecommand{\half}{\frac{1}{2}}
\providecommand{\dang}{\measuredangle} %% Directed angle
\providecommand{\CC}{\mathbb C}
\providecommand{\FF}{\mathbb F}
\providecommand{\NN}{\mathbb N}
\providecommand{\QQ}{\mathbb Q}
\providecommand{\RR}{\mathbb R}
\providecommand{\ZZ}{\mathbb Z}
\providecommand{\dg}{^\circ}
\providecommand{\ii}{\item}
\providecommand{\alert}{\textbf}
\providecommand{\opname}{\operatorname}
\providecommand{\ts}{\textsuperscript}
% hacks for arc
\providecommand{\tarc}{\mbox{\large$\frown$}}
\providecommand{\arc}[1]{\stackrel{\tarc}{#1}}
\reversemarginpar
\providecommand{\printpuid}[1]{\marginpar{\href{https://otis.evanchen.cc/arch/#1}{\ttfamily\footnotesize\color{green!40!black}#1}}}

\mdfdefinestyle{mdbluebox}{roundcorner=10pt,innerbottommargin=9pt,
    linecolor=blue,backgroundcolor=TealBlue!5,}
\declaretheoremstyle[headfont=\sffamily\bfseries\color{MidnightBlue},
    mdframed={style=mdbluebox},]{thmbluebox}
\mdfdefinestyle{mdredbox}{frametitlefont=\bfseries,innerbottommargin=8pt,
    nobreak=true,backgroundcolor=Salmon!5,linecolor=RawSienna,}
\declaretheoremstyle[headfont=\bfseries\color{RawSienna},
    mdframed={style=mdredbox},headpunct={\\[3pt]},postheadspace=0pt,]{thmredbox}
\mdfdefinestyle{mdgreenbox}{linecolor=ForestGreen,backgroundcolor=ForestGreen!5,
    linewidth=2pt,rightline=false,leftline=true,topline=false,bottomline=false,}
\declaretheoremstyle[headfont=\bfseries\sffamily\color{ForestGreen!70!black},
    mdframed={style=mdgreenbox},headpunct={ --- },]{thmgreenbox}
\mdfdefinestyle{mdblackbox}{linecolor=black,backgroundcolor=RedViolet!5!gray!5,
    linewidth=3pt,rightline=false,leftline=true,topline=false,bottomline=false,}
\declaretheoremstyle[mdframed={style=mdblackbox}]{thmblackbox}
\declaretheorem[style=thmredbox,name=Problem]{problem}
\declaretheorem[style=thmredbox,name=Required Problem,sibling=problem]{reqproblem}
\declaretheorem[style=thmbluebox,name=Theorem,numberwithin=problem]{theorem}
\declaretheorem[style=thmbluebox,name=Lemma,sibling=theorem]{lemma}
\declaretheorem[style=thmbluebox,name=Theorem,numbered=no]{theorem*}
\declaretheorem[style=thmbluebox,name=Lemma,numbered=no]{lemma*}
\declaretheorem[style=thmgreenbox,name=Claim,sibling=theorem]{claim}
\declaretheorem[style=thmgreenbox,name=Claim,numbered=no]{claim*}
\declaretheorem[style=thmblackbox,name=Remark,sibling=theorem]{remark}
\declaretheorem[style=thmblackbox,name=Remark,numbered=no]{remark*}
\declaretheorem[style=thmgreenbox,name=Definition,sibling=theorem]{definition}
\declaretheorem[style=thmgreenbox,name=Definition,numbered=no]{definition*}
\declaretheorem[style=thmblackbox,name=Example,sibling=theorem]{example}
\declaretheorem[style=thmblackbox,name=Example,numbered=no]{example*}

\newenvironment{walkthrough}{\noindent\textbf{\color{green!40!black}Walkthrough.}}{}
\newlist{walk}{enumerate}{3}
\setlist[walk]{label=\bfseries (\alph*)}

\usepackage{asymptote}
\begin{asydef}
size(8cm); // set a reasonable default
usepackage("amsmath");
usepackage("amssymb");
settings.tex="pdflatex";
settings.outformat="pdf";
import geometry;
void filldraw(picture pic = currentpicture, conic g, pen fillpen=defaultpen, pen drawpen=defaultpen) { filldraw(pic, (path) g, fillpen, drawpen); }
void fill(picture pic = currentpicture, conic g, pen p=defaultpen) { filldraw(pic, (path) g, p); }
pair foot(pair P, pair A, pair B) { return foot(triangle(A,B,P).VC); }
pair centroid(pair A, pair B, pair C) { return (A+B+C)/3; }
\end{asydef}

\newcommand{\goals}[2]{\bgroup
\sffamily\small \emph{Instructions}: Solve \clubg{#1}.
If you have time, solve \clubg{#2}.\egroup\par}

%% 426c616e6b204c615465587e
\begin{document}
\title{Submission for DHW-GRF}
\subtitle{OTIS (internal use)}
\author{YOUR NAME HERE}
\date{\today}
\maketitle

\begin{example*}[\href{https://aops.com/community/p4701758}{Romania TST 2015/2/1}, $0\clubsuit$]
  Let $a$ and $n$ be integers with $n \ge 1$. Show that $n$ divides
  \[ \sum_{k=1}^{n} a^{\gcd(k,n)}. \]
\end{example*} \printpuid{15ROUTST21}

\begin{walkthrough}
Assume $a > 0$.
Consider the following question:
\begin{quote}
  Given a necklace of $n$ beads, and each could be one of $a$ colors,
  what is the number of distinct necklaces (up to rotation)
  that can be formed?
\end{quote}
We will apply Burnside lemma to compute this number.
\begin{walk}
  \ii Describe a group action of $G = \ZZ / n \ZZ$ on the set of $a^n$ necklaces
  (before modding out by rotation),
  so the question above asks for the number of orbits of $G$
  on the set of $a^n$ necklaces.

  \ii Fix $k \in G = \ZZ / n \ZZ$.
  Find, as a function of $a$, $k$, $n$, the size of the stabilizer of $k$.

  \ii Use Burnside lemma to show the answer to the counting question is
  \[  \frac 1n \sum_{k} a^{\gcd(k,n)}. \]

  \ii Conclude the desired result.
\end{walk}
\end{walkthrough}
%% You're not expected to write up walkthroughs (unless you really want to).
%% The source is just for your reference.

\begin{example*}[HMMT 2017 A9, $0\clubsuit$]
  What is the period of the Fibonacci sequence modulo $127$?
\end{example*} \printpuid{17HMMTA9}

\begin{walkthrough}
In what follows let $p = 127$.
Our hope is to use the Fibonacci formula
\[ F_n = \frac{\alpha^n - \beta^n}{\alpha - \beta} \]
where $\alpha = \frac{1 + \sqrt5}{2}$ and $\beta = \frac{1 - \sqrt5}{2}$.
Unfortunately:
\begin{walk}
  \ii Show that $(5/p) = -1$.
  Thus we cannot find $\alpha, \beta \in \FF_p$.
\end{walk}
The idea is that we need to instead work in
the so-called finite field
\[ \FF_{p^2} \coloneq \FF_p[\sqrt5]
  = \left\{ a + b \sqrt 5 \mid a,b \in \FF_p \right\}. \]
I'll explain the reason for the notation $\FF_{p^2}$ later,
but for not, just think of it as that field $a+b\sqrt5$.
Note that this field $\FF_{p^2}$ still has a copy of $\FF_p$
inside it, namely the numbers of the form $a = a+0\sqrt5$.
They still behave as we expect,
for example, $a^p \equiv a \pmod p$
since $p \mid a^p-a$.
\begin{walk}[resume]
  \ii Convince yourself this really is a field.
  How many elements does it have?
  \ii Show that $x^{p^2-1} = 1$
  for any nonzero $x \in \FF_{p^2}$.
  (This is the analog of Fermat's little theorem.)
\end{walk}
Now we take advantage of the fact that $\alpha, \beta \in \FF_{p^2}$.
\begin{walk}[resume]
  \ii Consider the so-called Frobenius endomorphism
  \[ \sigma \colon \FF_{p^2} \to \FF_{p^2}
    \qquad \sigma(t) = t^p. \]
  Determine its fixed points.
  (Hint: you should already know the roots of $X^p - X$.)
  \ii Deduce that $\sigma(\alpha) = \alpha^p \neq \alpha$.
  \ii Prove that $\sigma(x+y) = \sigma(x) + \sigma(y)$
  by using the binomial theorem,
  and also that $\sigma(xy) = \sigma(x) \sigma(y)$.
  Thus the Frobenius endomorphism deserves its name.
  \ii Consider the polynomial
  \[ P(X) = X^2 - X - 1. \]
  Prove that $P(\sigma(x)) = \sigma(P(x))$
  for any $x \in \FF_{p^2}$.
  (Actually, this holds for any $P \in \FF_p[X]$.)
  \ii Put together (e) and (g) to deduce that $\sigma(\alpha) = \alpha^p = \beta$,
  by showing $\sigma(\alpha)$ is a root of $P$ not equal to $\alpha$.
  \ii Find $\sigma(\beta)$.
\end{walk}
Now we are in business:
we can compute Fibonacci numbers using
\[ F_n = \frac{\alpha^n - \beta^n}{\alpha - \beta}. \]
\begin{walk}[resume]
  \ii Using (h) and (i), show that $F_p \equiv -1 \pmod p$,
  $F_{p+1} \equiv 0 \pmod p$.
  \ii Prove that $F_{2p+1} \equiv 1 \pmod p$
  and $F_{2p+2} \equiv 0 \pmod p$.
  \ii Conclude that the period of the Fibonacci sequence modulo $p$
  divides $2p+2= 256$ but not $p+1 = 128$.
  Deduce the answer $256$.
\end{walk}
The solution notes contain some extended comments
about the higher algebra that this problem introduces.
\end{walkthrough}
%% You're not expected to write up walkthroughs (unless you really want to).
%% The source is just for your reference.

\newpage

% ========================================
\section*{Practice problems}
\goals{24}{30}

\epigraph{I forbid any of you from studying for your midterms! \par
  It's not a punishment; there are simply other things you should be studying first.
  I'm responsible, too, for having forgotten to teach them.}
  {Koro-sensei in \emph{Assassination Classroom}, Season 2, Episode 6}
\begin{reqproblem}
[9$\clubsuit$]  Write the blog post described in the previous section.
  Also specify whether you would want your post to be added to the running list above.
\end{reqproblem} \marginpar{\emph{\footnotesize No PUID}}
%% Type your solution here ...

%% --------------------------------------------------

\begin{problem}[Napkin 4B, $2\clubsuit$]
  Prove that the rings $\CC[x] / (x^2-x)$ and $\CC \times \CC$ are isomorphic.
\end{problem} \printpuid{NAP4B}

%% Type your solution to Napkin 4B (\href{https://otis.evanchen.cc/arch/NAP4B/}{NAP4B}) here ...
Let $I = x^2-x$ in $\CC[x]$.
The cosets of $I$ are given by $(ax + b)I$ where $a,b \in \CC$.

%% --------------------------------------------------

\begin{reqproblem}[Napkin 5D, $3\clubsuit$]
  Let $R$ be an integral domain with finitely many elements.
  Prove that $R$ is a field.
\end{reqproblem} \printpuid{NAP5D}

%% Type your solution to Napkin 5D (\href{https://otis.evanchen.cc/arch/NAP5D/}{NAP5D}) here ...
Let $r \in R$ be nonzero.
Consider the map $\phi: R \to R$ given by $x \to rx$.

We claim this map is injective.
Indeed, if $rx = ry$, then $r(x-y) = 0$, and since $r \ne 0$, we have $x = y$.

Since $R$ is finite, this map must also be bijective.
So, there is some $x$ such that $rx = 1$.
This is the multiplicative inverse of $r$.

Since every nonzero element has a multiplicative inverse,
$R$ is a field.
%% --------------------------------------------------

\begin{problem}[Burnside's lemma, $3\clubsuit$]
  Let $G$ be a finite group which acts on a finite set $X$.
  For each $g \in G$, let $\opname{FixPt} g$ denote
  the set of $x \in X$ for which $g \cdot x = x$.
  Prove that the number of orbits of this action is
  given exactly by $\frac{1}{|G|} \sum_{g \in G} \opname{FixPt} g$.
\end{problem} \printpuid{Z33FA3B5}

%% Type your solution to Burnside's lemma (\href{https://otis.evanchen.cc/arch/Z33FA3B5/}{Z33FA3B5}) here ...
We count ordered pairs $(g,x)$ such that $g \cdot x = x$ in two ways.
First, it is equal to $\sum_{g \in G} \opname{FixPt} g$.
It is also the sum of the sizes of the stabilizers, which we can write as
\[ \sum_{x \in X} \lvert \opname{stab}(x) \rvert = \sum_{x \in X} \frac{|G|}{\lvert \opname{orb}(x) \rvert} = |G| \sum_{x \in X} \frac{1}{\lvert \opname{orb}(x) \rvert}. \]
But, notice that $\sum_{x \in X} \frac{1}{\lvert \opname{orb}(x) \rvert}$ just counts the number
of orbits, giving us the result.
%% --------------------------------------------------

\begin{problem}[\href{https://aops.com/community/p5679392}{Napkin 4G; also USA TST 2016/3}, $3\clubsuit$]
  Define $\Psi \colon \FF_p[x] \to \FF_p[x]$ by
  \[ \Psi\left( \sum_{i=0}^n a_i x^i \right) = \sum_{i=0}^n a_i x^{p^i}. \]
  Let $S$ denote the image of $\Psi$.
  \begin{enumerate}[(a)]
  \ii Show that $S$ is a ring with addition given by polynomial addition,
  and multiplication given by \emph{function composition}.
  \ii Prove that $\Psi \colon \FF_p[x] \to S$ is then a ring isomorphism.
  \end{enumerate}
\end{problem} \printpuid{NAP4G}

%% Type your solution to Napkin 4G; also USA TST 2016/3 (\href{https://otis.evanchen.cc/arch/NAP4G/}{NAP4G}), proposed by Mark Sellke here ...

%% --------------------------------------------------

\begin{problem}[Napkin 5G, $3\clubsuit$]
  How many prime ideals of the ring $R = \ZZ[\sqrt{2017}]$ are \emph{not} maximal ideals?
\end{problem} \printpuid{NAP5G}

%% Type your solution to Napkin 5G (\href{https://otis.evanchen.cc/arch/NAP5G/}{NAP5G}) here ...

%% --------------------------------------------------

\begin{problem}[Napkin 1G, $5\clubsuit$]
  Find the smallest integer $n$
  such that the symmetric group $S_n$ has a subgroup
  isomorphic to the dihedral group $D_{2018}$ of order $2018$.
\end{problem} \printpuid{NAP1G}

%% Type your solution to Napkin 1G (\href{https://otis.evanchen.cc/arch/NAP1G/}{NAP1G}) here ...
Note that $D_{2018}$ has an element of order $1009$, namely, the smallest rotation.
Since $1009$ is prime, a permutation has order $1009$ only when it
contains cycle of length $1009$.
So, $n \ge 1009$.

However, $n=\boxed{1009}$ works if we consider the natural action of $D_{2018}$
on a $1009$-gon.
%% --------------------------------------------------

\begin{problem}[\href{https://aops.com/community/p1703561}{Putnam 2009 A5}, $5\clubsuit$]
  Is there a finite abelian group $G$ such that the product of all the orders
  of its elements is $2^{2009}$?
\end{problem} \printpuid{09PTNMA5}

%% Type your solution to Putnam 2009 A5 (\href{https://otis.evanchen.cc/arch/09PTNMA5/}{09PTNMA5}) here ...

%% --------------------------------------------------

\begin{problem}[\href{https://aops.com/community/p978389}{Putnam 2007 A5}, $5\clubsuit$]
  Let $p$ be a prime.
  Let $G$ be a finite group, and suppose there are exactly $n$ elements of order $p$.
  Prove that either $n = 0$ or $p$ divides $n+1$.
\end{problem} \printpuid{07PTNMA5}

%% Type your solution to Putnam 2007 A5 (\href{https://otis.evanchen.cc/arch/07PTNMA5/}{07PTNMA5}) here ...

%% --------------------------------------------------

\begin{problem}[Napkin 16F, $5\clubsuit$]
  Does there exist a faithful transitive action of $S_5$ on a six-element set?
\end{problem} \printpuid{NAP16F}

%% Type your solution to Napkin 16F (\href{https://otis.evanchen.cc/arch/NAP16F/}{NAP16F}) here ...

%% --------------------------------------------------

\begin{problem}[\href{https://aops.com/community/p525984}{Shortlist 2005 C5}, $3\clubsuit$]
  There are $n$ markers, each with one side white and the other side black.
  In the beginning, these $n$ markers are aligned in a row so that their white sides are all up.
  In each step, if possible, we choose a marker whose white side is up
  (but not one of the outermost markers), remove it,
  and reverse the closest marker to the left of it
  and also reverse the closest marker to the right of it.
  Prove that, by a finite sequence of such steps,
  one can achieve a state with only two markers remaining
  if and only if $n - 1$ is not divisible by $3$.
\end{problem} \printpuid{05SLC5}

%% Type your solution to Shortlist 2005 C5 (\href{https://otis.evanchen.cc/arch/05SLC5/}{05SLC5}) here ...

%% --------------------------------------------------

\begin{problem}[PRIMES 2018 M5, $3\clubsuit$]
  Let $G$ be a group with presentation given by
  \[ G = \left< a,b,c \mid
  ab = c^2a^4, \; bc = ca^6, \; ac = ca^8, \;
  c^{2018} = b^{2019}
  \right>. \]
  Determine the order of $G$.
\end{problem} \printpuid{18PRIMESM5}

%% Type your solution to PRIMES 2018 M5 (\href{https://otis.evanchen.cc/arch/18PRIMESM5/}{18PRIMESM5}) here ...

%% --------------------------------------------------

\begin{problem}[PRIMES 2021 M5, $9\clubsuit$]
  Suppose $G$ is a nonabelian finite group and $\varphi \colon G \to G$ a homomorphism.
  Denote by $0 \le k \le 1$ the fraction of elements $g \in G$ which satisfy
  \[ \varphi(g) = g^2. \]
  Find the largest possible value of $k$ across all nonabelian finite groups $G$.
\end{problem} \printpuid{21PRIMESM5}

%% Type your solution to PRIMES 2021 M5 (\href{https://otis.evanchen.cc/arch/21PRIMESM5/}{21PRIMESM5}) here ...

%% --------------------------------------------------

\end{document}
