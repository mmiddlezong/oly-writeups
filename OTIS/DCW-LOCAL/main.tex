\documentclass[11pt]{scrartcl}
\usepackage[usenames,dvipsnames,svgnames]{xcolor}
\usepackage[shortlabels]{enumitem}
\usepackage[framemethod=TikZ]{mdframed}
\usepackage{amsmath,amssymb,amsthm}
\usepackage{epigraph}
\usepackage[colorlinks]{hyperref}
\usepackage{microtype}
\usepackage{mathtools}
\usepackage[headsepline]{scrlayer-scrpage}
\usepackage{thmtools}
\usepackage{listings}
\usepackage{derivative}
\renewcommand{\epigraphsize}{\scriptsize}
\renewcommand{\epigraphwidth}{60ex}
\addtolength{\textheight}{3.14cm}
\ihead{\footnotesize\textbf{DCW-local}}
\ohead{\footnotesize Updated Wed 20 Sep 2023 15:41:36 UTC}
\providecommand{\clubs}[1]{$#1\clubsuit$}
\providecommand{\clubg}[1]{\bgroup\color{green!40!black}[$#1\clubsuit$]\egroup}

\providecommand{\ol}{\overline}
\providecommand{\eps}{\varepsilon}
\providecommand{\half}{\frac{1}{2}}
\providecommand{\dang}{\measuredangle} %% Directed angle
\providecommand{\CC}{\mathbb C}
\providecommand{\FF}{\mathbb F}
\providecommand{\NN}{\mathbb N}
\providecommand{\QQ}{\mathbb Q}
\providecommand{\RR}{\mathbb R}
\providecommand{\ZZ}{\mathbb Z}
\providecommand{\dg}{^\circ}
\providecommand{\ii}{\item}
\providecommand{\alert}{\textbf}
\providecommand{\opname}{\operatorname}
\providecommand{\ts}{\textsuperscript}
% hacks for arc
\providecommand{\tarc}{\mbox{\large$\frown$}}
\providecommand{\arc}[1]{\stackrel{\tarc}{#1}}
\reversemarginpar
\providecommand{\printpuid}[1]{\marginpar{\href{https://otis.evanchen.cc/arch/#1}{\ttfamily\footnotesize\color{green!40!black}#1}}}

\mdfdefinestyle{mdbluebox}{roundcorner=10pt,innerbottommargin=9pt,
    linecolor=blue,backgroundcolor=TealBlue!5,}
\declaretheoremstyle[headfont=\sffamily\bfseries\color{MidnightBlue},
    mdframed={style=mdbluebox},]{thmbluebox}
\mdfdefinestyle{mdredbox}{frametitlefont=\bfseries,innerbottommargin=8pt,
    nobreak=true,backgroundcolor=Salmon!5,linecolor=RawSienna,}
\declaretheoremstyle[headfont=\bfseries\color{RawSienna},
    mdframed={style=mdredbox},headpunct={\\[3pt]},postheadspace=0pt,]{thmredbox}
\mdfdefinestyle{mdgreenbox}{linecolor=ForestGreen,backgroundcolor=ForestGreen!5,
    linewidth=2pt,rightline=false,leftline=true,topline=false,bottomline=false,}
\declaretheoremstyle[headfont=\bfseries\sffamily\color{ForestGreen!70!black},
    mdframed={style=mdgreenbox},headpunct={ --- },]{thmgreenbox}
\mdfdefinestyle{mdblackbox}{linecolor=black,backgroundcolor=RedViolet!5!gray!5,
    linewidth=3pt,rightline=false,leftline=true,topline=false,bottomline=false,}
\declaretheoremstyle[mdframed={style=mdblackbox}]{thmblackbox}
\declaretheorem[style=thmredbox,name=Problem]{problem}
\declaretheorem[style=thmredbox,name=Required Problem,sibling=problem]{reqproblem}
\declaretheorem[style=thmbluebox,name=Theorem,numberwithin=problem]{theorem}
\declaretheorem[style=thmbluebox,name=Lemma,sibling=theorem]{lemma}
\declaretheorem[style=thmbluebox,name=Theorem,numbered=no]{theorem*}
\declaretheorem[style=thmbluebox,name=Lemma,numbered=no]{lemma*}
\declaretheorem[style=thmgreenbox,name=Claim,sibling=theorem]{claim}
\declaretheorem[style=thmgreenbox,name=Claim,numbered=no]{claim*}
\declaretheorem[style=thmblackbox,name=Remark,sibling=theorem]{remark}
\declaretheorem[style=thmblackbox,name=Remark,numbered=no]{remark*}
\declaretheorem[style=thmgreenbox,name=Definition,sibling=theorem]{definition}
\declaretheorem[style=thmgreenbox,name=Definition,numbered=no]{definition*}
\declaretheorem[style=thmblackbox,name=Example,sibling=theorem]{example}
\declaretheorem[style=thmblackbox,name=Example,numbered=no]{example*}

\newenvironment{walkthrough}{\noindent\textbf{\color{green!40!black}Walkthrough.}}{}
\newlist{walk}{enumerate}{3}
\setlist[walk]{label=\bfseries (\alph*)}

\usepackage{asymptote}
\begin{asydef}
size(8cm); // set a reasonable default
usepackage("amsmath");
usepackage("amssymb");
settings.tex="pdflatex";
settings.outformat="pdf";
import geometry;
void filldraw(picture pic = currentpicture, conic g, pen fillpen=defaultpen, pen drawpen=defaultpen) { filldraw(pic, (path) g, fillpen, drawpen); }
void fill(picture pic = currentpicture, conic g, pen p=defaultpen) { filldraw(pic, (path) g, p); }
pair foot(pair P, pair A, pair B) { return foot(triangle(A,B,P).VC); }
pair centroid(pair A, pair B, pair C) { return (A+B+C)/3; }
\end{asydef}

\newcommand{\goals}[2]{\bgroup
\sffamily\small \emph{Instructions}: Solve \clubg{#1}.
If you have time, solve \clubg{#2}.\egroup\par}

%% 426c616e6b204c615465587e
\begin{document}
\title{Submission for DCW-LOCAL}
\subtitle{OTIS (internal use)}
\author{Michael Middlezong}
\date{\today}
\maketitle

\begin{example*}[$0\clubsuit$]
  Suppose $4951$ distinct points in the plane are given
  such that no four points are collinear.
  Show that it is possible to select $100$ of the points
  for which no three points are collinear.
\end{example*} \printpuid{ZEBE7889}

\begin{walkthrough}
Following RUST, keep grabbing points until we cannot take any more.
Suppose at this point we have $n$ points.
\begin{walk}
  \ii Show that $4951-n \le \binom n2$.
  \ii Prove that $n \ge 100$.
\end{walk}
So this is an example of a greedy algorithm
of the most direct sort.
\end{walkthrough}
%% You're not expected to write up walkthroughs (unless you really want to).
%% The source is just for your reference.

\begin{example*}[Putnam 1979, $0\clubsuit$]
  Given $n$ red points and $n$ blue points in the plane,
  no three collinear, prove that we can draw $n$ segments,
  each joining a red point to a blue point,
  such that no segments intersect.
\end{example*} \printpuid{79PTNM}

\begin{walkthrough}
Starting from an arbitrary configuration,
we will use the algorithm ``given a crossing, un-cross it''.
This is a very natural algorithm to come up with,
and playing with some simple examples one finds that it always work.
So we just have to prove that.
\begin{walk}
  \ii Show that a step of this algorithm does
  \emph{not} necessarily decrease the total number of intersections.
  (But this is the first thing we \emph{should} try,
  given that our goal is to get zero intersections at the end.)

  \ii Find a different monovariant $M$ which \emph{does} decrease
  at each step of the algorithm.

  \ii Remark on the finiteness of the configuration space,
  and complete the problem using (b).
\end{walk}

I want to say a few words about why I chose this example.
This problem is touted in olympiad cultures as an example
of ``extremal principle'', with ``choose the minimal $M$''
as the poster description.
In my humble opinion, I think this is hogwash.
The motivation should be the natural algorithm we used;
the monovariant comes after the fact.

Indeed, the fact that the natural guess of the monovariant
in (a) fails is what makes this problem a little interesting
(and not completely standard).
However, it doesn't change the fact that the algorithm
comes before the monovariant in our thought process.
\end{walkthrough}
%% You're not expected to write up walkthroughs (unless you really want to).
%% The source is just for your reference.

\begin{example*}[\href{https://aops.com/community/p63591}{USAMO 1999/4}, $0\clubsuit$]
  Let $a_1$, $a_2$, \dots, $a_n$ be a sequence of $n > 3$ real numbers
  such that
  \[ a_1 + \dots + a_n \ge n \quad\text{and}\quad
  a_1^2 + \dots + a_n^2 \ge n^2. \]
  Prove that $\max(a_1, \dots, a_n) \ge 2$.
\end{example*} \printpuid{99AMO4}

\begin{walkthrough}
We will proceed by contradiction,
and assume that there exists \emph{some} sequence of $(a_i)$
satisfying all three of the following conditions:
\begin{enumerate}[(1)]
  \ii $\sum a_i \ge n$,
  \ii $\sum a_i^2 \ge n^2$,
  \ii but also $a_i < 2$ for all $i$.
\end{enumerate}
Our game plan is then to take this starting sequence,
and smooth the variables to obtain a new simpler sequence
satisfying (1)-(3).
We'll then derive a contradiction in the simplified setting.
\begin{walk}
  \ii If $a_i \le 0$ and $a_j \le 0$,
  how can we smooth these two terms while preserving (1), (2), and (3)?
  \ii If $a_i \ge 0$, how can we smooth
  the $a_i$ while preserving (1), (2) and (3)?
  \ii Combining (a) and (b),
  reduce the case where one term is negative and the rest are $2-\eps$
  for some small $\eps$.
  \ii Solve the problem in the situation of (c).
\end{walk}
One can avoid the issue of $\eps$ in (c) as well
if one notes that increasing $2-\eps$ to $2$
makes the inequalities (1) and (2) strict
(while relaxing (3) to be non-strict).
This makes the calculations a little less technical.
\end{walkthrough}
%% You're not expected to write up walkthroughs (unless you really want to).
%% The source is just for your reference.

\begin{example*}[\href{https://aops.com/community/p9513105}{USA TST 2018/3}, $0\clubsuit$]
  At a university dinner,
  there are $2017$ mathematicians who each order two distinct entr\'ees,
  with no two mathematicians ordering the same pair of entr\'{e}es.
  The cost of each entr\'ee is equal
  to the number of mathematicians who ordered it,
  and the university pays for each mathematician's
  less expensive entr\'ee (ties broken arbitrarily).
  Over all possible sets of orders,
  what is the maximum total amount the university could have paid?
\end{example*} \printpuid{18USATST3}

\begin{walkthrough}
%%fakesection Walk
This is a rather detail-oriented problem which requires a lot of care
in order to get all the arguments to be correct.
In my opinion, being able to work out a flawless solution
in the time limit of the exam is an impressive feat.

The original proposed solution is pretty combinatorial,
and the walkthrough for that is below;
it uses a \emph{local} smoothing idea in the first half
and then a \emph{global} idea in the second part.
(There is also an algebraic solution using
inequality smoothing instead.)

Viewing the problem in graph theory terms,
let \[ S(G) \coloneqq \sum_{e = vw} \min \left( \deg v, \deg w \right) \]
and thus we want to maximize $S(G)$ over simple graphs $G$ with $2017$ edges.

Let's get started.
\begin{walk}
  \ii Find an example of a graph achieving $S(G) = 127009$.
  We will prove this is the maximum value.

  \ii Make a conjecture what the answer is if $2017$
  is replaced by $\binom k2+1$ where $k \ge 4$.

  \ii Show that the natural extension of (b) does \emph{not} hold when $k = 3$.
  This edge case will haunt us briefly later on again.
\end{walk}
Here is the local half:
\begin{walk}[resume]
  \ii Suppose $G$ is a graph with $\binom k2$ edges for $k \ge 3$.
  Assume $G$ has no universal vertex,
  and let $w$ denote the vertex of minimal degree.

  Prove that one can delete $w$, and rewire the edges originally adjacent to $w$,
  to obtain a graph $G^\ast$ with fewer vertices and $S(G^\ast) \ge S(G)$.

  \ii Repeat part (d) where $G$ has $\binom k2 + 1$ edges
  and $k \ge 4$, and $\#V(G) > k+1$.
  Then check the existence of $G^\ast$ by hand for $\#V(G) = k+1$.

  \ii Show that (e) is false if $k=3$.
\end{walk}
So, we can reduce to the case where $G$ has a universal vertex.
We will now prove by induction on $k \ge 3$ the following claim:
\begin{quote}
  For any graph $G$,
  \begin{itemize}
    \ii If $G$ has at most $\binom k2$ edges for $k \ge 3$,
    then $S(G) \le \binom k2 \cdot (k-1)$.
    \ii If $G$ has at most $\binom k2 + 1$ edges for $k \ge 4$,
    then $S(G) \le \binom k2 \cdot (k-1) + 1$.
  \end{itemize}
\end{quote}
This is the global half.
\begin{walk}[resume]
  \ii Check the base case $k=3$.
  (We won't need a base case for the second bullet;
  it will reduce to the first one.)
  \ii WLOG, let $v$ be the universal vertex of $G$.
  Show that if $v$ is deleted,
  the resulting graph $H$ has at most $\binom{k-1}{2}$ edges
  (in both cases) and thus we can apply
  the induction hypothesis to the resulting graph $H$.
  \ii Prove that $S(G) = S(H) + 3\#E(H) + \#V(H)$.
  \ii Combine the previous two parts to complete the problem.
\end{walk}
\end{walkthrough}
%% You're not expected to write up walkthroughs (unless you really want to).
%% The source is just for your reference.

\newpage

% ========================================
\section*{Practice problems}
\goals{36}{50}

\epigraph{Mario, I will follow you to the end, I swear it!
I feel bad for the princess, but that queen must fall before us!
And when she does, you and I can\dots Well, anyway, let's take this fight to her!}
{Vivian in the Shadow Queen battle, from \\ \emph{Paper Mario: The Thousand Year Door}}
%%fakesubsection*{Local problems}

\begin{reqproblem}[\href{https://aops.com/community/p3286455}{PUMaC 2013}, $2\clubsuit$]
  Let $G$ be a graph and let $k$ be a positive integer.
  A $k$-\emph{star} is a set of $k$ edges with a common endpoint
  and a $k$-\emph{matching} is a set of $k$ edges such that
  no two have a common endpoint.
  Prove that if $G$ has more than $2(k-1)^2$ edges then
  it either has a $k$-star or a $k$-matching.
\end{reqproblem} \printpuid{13PUMACFA3}

%% Type your solution to PUMaC 2013 (\href{https://otis.evanchen.cc/arch/13PUMACFA3/}{13PUMACFA3}) here ...
Assume there is no $k$-star. This means every vertex has a degree which is at most $k-1$.
Then, for the sake of contradiction, take a maximal matching of size $m < k$.
Every edge in the graph must either be in the matching or
share an endpoint with an edge in the matching.
The maximum number of edges not in the matching is then
\[ 2m(k-2). \]
Counting up the total number of edges in the graph, we get
\[ 2m(k-2) + m \le 2(k-1)(k-2) + (k-1) = 2k^2 - 5k + 3 \le 2(k-1)^2, \]
contradicting the assumption that $G$ has more than $2(k-1)^2$ edges.
%% --------------------------------------------------

\begin{problem}[$3\clubsuit$]
  Prove that if $n$ is a sufficiently large positive integer
  then it is possible to find a collection $\mathcal F$
  of subsets of $\{1, \dots, n\}$,
  such that $|\mathcal F| > 1.001^n$
  and whenever $X$ and $Y$ are distinct elements of $\mathcal F$,
  we have
  \[ \left\lvert X \cup Y \right\rvert
  - \left\lvert X \cap Y \right\rvert
  \ge \frac n3. \]
\end{problem} \printpuid{GOWERSE3}

%% Type your solution to \href{https://otis.evanchen.cc/arch/GOWERSE3/}{GOWERSE3} here ...

%% --------------------------------------------------

\begin{reqproblem}[\href{https://aops.com/community/p261}{IMO 2003/1}, $3\clubsuit$]
  Let $A$ be a $101$-element subset of $S=\{1,2,\dots,10^6\}$.
  Prove that there exist numbers $t_1$, $t_2, \dots, t_{100}$ in $S$ such that the sets
  \[ A_j=\{x+t_j\mid x\in A\},\qquad j=1,2,\dots,100  \]
  are pairwise disjoint.
\end{reqproblem} \printpuid{03IMO1}

%% Type your solution to IMO 2003/1 (\href{https://otis.evanchen.cc/arch/03IMO1/}{03IMO1}), proposed by Carlos Gustavo Moreira (BRA) here ...

%% --------------------------------------------------

\begin{problem}[\href{https://aops.com/community/p8651644}{BAMO 2017/3}, $2\clubsuit$]
  Let $n$ be a positive integer and consider an
  $n \times n$ multiplication table as shown below,
  where the entry in the $i$th column and the $j$th row is $ij$.
  \[
  \begin{bmatrix}
  1 & 2 & 3 & \dots & n-1 & n \\
  2 & 4 & 6 & \dots & 2n-2 & 2n \\
  3 & 6 & 9 & \dots & 3n-3 & 3n \\
  \vdots & \vdots & \vdots & \ddots & \vdots & \vdots \\
  n & 2n & 3n & \dots & n^2-n & n^2
  \end{bmatrix}
  \]
  An ant starts on the cell labeled $1$,
  and walks towards the cell labeled $n^2$
  by taking $2n-2$ steps each either right or down.
  Determine the minimum and maximum possible sum
  of the labels in such a path, as a function of $n$.
\end{problem} \printpuid{17BAMO3}

%% Type your solution to BAMO 2017/3 (\href{https://otis.evanchen.cc/arch/17BAMO3/}{17BAMO3}) here ...
Consider the set of elements $ij$ for $i+j=k$; call this ``diagonal $k$.'' Clearly,
in any path the ant takes, the number of the diagonal it is in increments by $1$ with each step.
Thus, if we could get the ant to always choose the maximum number in each diagonal,
that would produce the maximum sum (and vice versa for minimum).

It is not hard to show that the element(s) in the middle of each diagonal are the greatest
and the elements on the ends of each diagonal are the smallest.
This means that it is possible to always choose the maximum/minimum element in each diagonal.
The maximum sum is achieved by taking a zigzag path down the main diagonal of the matrix,
and the minimum sum is achieved by going right $n-1$ times and then going down $n-1$ times.

The minimum sum is then
\begin{align*}
    (1+2+\dots + (n-1)) + (n + 2n + \dots + n^2) &= \frac{n(n-1)}{2} + \frac{n^2(n+1)}{2} \\
    &= \frac{n(n^2 + 2n - 1)}{2}.
\end{align*}

Finally, the maximum sum is
\begin{align*}
    \sum_{i=1}^n (i^2 + i(i-1)) &= \sum_{i=1}^n (2i^2 - i) \\
    &= 2\frac{n(n+1)(2n+1)}{6} - \frac{n(n+1)}{2} \\
    &= \frac{n(n+1)(4n-1)}{6}.
\end{align*}

%% --------------------------------------------------

\begin{problem}[\href{https://aops.com/community/p7357500}{Putnam 2016 A5}, $5\clubsuit$]
  Suppose that $G$ is a finite group generated
  by the two elements $g$ and $h$, where the order of $g$ is odd.
  Show that every element of $G$ can be written in the form
  \[ g^{m_1} h^{n_1} g^{m_2} h^{n_2} \dots g^{m_r} h^{n_r} \]
  with $1\le r\le |G|$ and $m_1,n_1,m_2,n_2,\dots,m_r,n_r\in\{1,-1\}$.
  (Here $|G|$ is the number of elements of $G$.)
\end{problem} \printpuid{16PTNMA5}

%% Type your solution to Putnam 2016 A5 (\href{https://otis.evanchen.cc/arch/16PTNMA5/}{16PTNMA5}) here ...

%% --------------------------------------------------

\begin{problem}[\href{https://aops.com/community/p2114068}{MP4G 2010}, $3\clubsuit$]
  Let $S$ be a set of $n$ points in the coordinate plane.
  Say that a pair of points is \emph{aligned} if the two points
  have the same $x$-coordinate or $y$-coordinate.
  Prove that $S$ can be partitioned into disjoint subsets such that
  (a) each of these subsets is a collinear set of points, and
  (b) at most $n^{3/2}$ unordered pairs of distinct points in $S$
  are aligned but not in the same subset.
\end{problem} \printpuid{10MPO4}

%% Type your solution to MP4G 2010 (\href{https://otis.evanchen.cc/arch/10MPO4/}{10MPO4}) here ...

%% --------------------------------------------------

\begin{problem}[\href{https://aops.com/community/p17852}{Russia 2004}, $5\clubsuit$]
  A country has $1001$ cities,
  and each two cities are connected by a one-way street.
  From each city exactly $500$ roads begin,
  and in each city $500$ roads end.
  Now an independent republic splits itself off the country,
  which contains $668$ of the $1001$ cities.
  Prove that one can reach every other city of the republic
  from each city of this republic
  without being forced to leave the republic.
\end{problem} \printpuid{04RUS106}

%% Type your solution to Russia 2004 (\href{https://otis.evanchen.cc/arch/04RUS106/}{04RUS106}) here ...

%% --------------------------------------------------

\begin{problem}[\href{https://aops.com/community/p17829157}{Shortlist 2019 C2}, $3\clubsuit$]
  You're given $n$ blocks each with weight at least $1$ and total weight $2n$.
  Prove that for every real number $r \in [0, 2n-2]$ there is a subset of
  the blocks whose total weight is between $r$ and $r+2$ inclusive.
\end{problem} \printpuid{19SLC2}

%% Type your solution to Shortlist 2019 C2 (\href{https://otis.evanchen.cc/arch/19SLC2/}{19SLC2}) here ...

%% --------------------------------------------------

\begin{problem}[\href{https://aops.com/community/p119173}{Shortlist 2001 C1, added by Pranav Choudhary}, $3\clubsuit$]
  Let $A = (a_1, a_2, \dots, a_{2001})$ be a sequence of positive integers.
  Let $m$ be the number of 3-element subsequences $(a_i,a_j,a_k)$
  with $1 \leq i < j < k \leq 2001$, such that $a_j = a_i + 1$ and $a_k = a_j + 1$.
  Considering all such sequences $A$, find the greatest value of $m$.
\end{problem} \printpuid{01SLC1}

%% Type your solution to Shortlist 2001 C1, added by Pranav Choudhary (\href{https://otis.evanchen.cc/arch/01SLC1/}{01SLC1}) here ...

%% --------------------------------------------------

\begin{reqproblem}[\href{https://aops.com/community/p3543144}{IMO 2014/5}, $9\clubsuit$]
  For every positive integer $n$,
  the Bank of Cape Town issues coins of denomination $\frac 1n$.
  Given a finite collection of such coins (of not necessarily different denominations)
  with total value at most $99 + \frac12$, prove that it is possible to split
  this collection into $100$ or fewer groups, such that each group has total value at most $1$.
\end{reqproblem} \printpuid{14IMO5}

%% Type your solution to IMO 2014/5 (\href{https://otis.evanchen.cc/arch/14IMO5/}{14IMO5}), proposed by Gerhard Woeginger (LUX) here ...

%% --------------------------------------------------

\begin{problem}[\href{https://aops.com/community/p24775345}{USAMO 2022/2}, $5\clubsuit$]
  Let $b\geq 2$ and $w\geq 2$ be fixed integers, and $n=b+w$.
  Given are $2b$ identical black rods and $2w$ identical white rods,
  each of side length $1$.

  We assemble a regular $2n$-gon using these rods
  so that parallel sides are the same color.
  Then, a convex $2b$-gon $B$ is formed by translating the black rods,
  and a convex $2w$-gon $W$ is formed by translating the white rods.
  An example of one way of doing the assembly when $b=3$ and $w=2$ is shown below,
  as well as the resulting polygons $B$ and $W$.
  \begin{center}
  \begin{asy}
  size(10cm);
  real w = 2*Sin(18);
  real h = 0.10 * w;
  real d = 0.33 * h;
  picture wht;
  picture blk;

  draw(wht, (0,0)--(w,0)--(w+d,h)--(-d,h)--cycle);
  fill(blk, (0,0)--(w,0)--(w+d,h)--(-d,h)--cycle, black);

  // draw(unitcircle, blue+dotted);

  // Original polygon
  add(shift(dir(108))*blk);
  add(shift(dir(72))*rotate(324)*blk);
  add(shift(dir(36))*rotate(288)*wht);
  add(shift(dir(0))*rotate(252)*blk);
  add(shift(dir(324))*rotate(216)*wht);

  add(shift(dir(288))*rotate(180)*blk);
  add(shift(dir(252))*rotate(144)*blk);
  add(shift(dir(216))*rotate(108)*wht);
  add(shift(dir(180))*rotate(72)*blk);
  add(shift(dir(144))*rotate(36)*wht);

  // White shifted
  real Wk = 1.2;
  pair W1 = (1.8,0.1);
  pair W2 = W1 + w*dir(36);
  pair W3 = W2 + w*dir(108);
  pair W4 = W3 + w*dir(216);
  path Wgon = W1--W2--W3--W4--cycle;
  draw(Wgon);
  pair WO = (W1+W3)/2;
  transform Wt = shift(WO)*scale(Wk)*shift(-WO);
  draw(Wt * Wgon);
  label("$W$", WO);
  /*
  draw(W1--Wt*W1);
  draw(W2--Wt*W2);
  draw(W3--Wt*W3);
  draw(W4--Wt*W4);
  */

  // Black shifted
  real Bk = 1.10;
  pair B1 = (1.5,-0.1);
  pair B2 = B1 + w*dir(0);
  pair B3 = B2 + w*dir(324);
  pair B4 = B3 + w*dir(252);
  pair B5 = B4 + w*dir(180);
  pair B6 = B5 + w*dir(144);
  path Bgon = B1--B2--B3--B4--B5--B6--cycle;
  pair BO = (B1+B4)/2;
  transform Bt = shift(BO)*scale(Bk)*shift(-BO);
  fill(Bt * Bgon, black);
  fill(Bgon, white);
  label("$B$", BO);
  \end{asy}
  \end{center}
  Prove that the difference of the areas of $B$ and $W$
  depends only on the numbers $b$ and $w$,
  and not on how the $2n$-gon was assembled.
\end{problem} \printpuid{22AMO2}

%% Type your solution to USAMO 2022/2 (\href{https://otis.evanchen.cc/arch/22AMO2/}{22AMO2}), proposed by Ankan Bhattacharya here ...

%% --------------------------------------------------

\begin{problem}[\href{https://aops.com/community/p131846}{IMO 1999/2}, $3\clubsuit$]
  Find the least constant $C$ such that for any integer $n > 1$ the inequality
  \[\sum_{1 \le i < j \le n} x_i x_j (x_i^2 + x_j^2)
  \le C \left( \sum_{1 \le i \le n} x_i \right)^4\]
  holds for all real numbers $x_1, \dots, x_n \ge 0$.
  Determine the cases of equality.
\end{problem} \printpuid{99IMO2}

%% Type your solution to IMO 1999/2 (\href{https://otis.evanchen.cc/arch/99IMO2/}{99IMO2}) here ...

%% --------------------------------------------------

\begin{reqproblem}[\href{https://aops.com/community/p9853492}{HMMT 2018 T6}, $5\clubsuit$]
  Let $n \ge 2$ be a positive integer.
  A subset of positive integers $S$ is said to be
  \emph{comprehensive} if for every integer $0 \le x < n$,
  there is a subset of $S$ (possibly empty)
  whose sum has remainder $x$ when divided by $n$.
  Show that if a set $S$ is comprehensive,
  then there is some (not necessarily proper) subset of $S$
  with at most $n-1$ elements which is also comprehensive.
\end{reqproblem} \printpuid{18HMMTT6}

%% Type your solution to HMMT 2018 T6 (\href{https://otis.evanchen.cc/arch/18HMMTT6/}{18HMMTT6}), proposed by Allen Liu here ...

%% --------------------------------------------------

\begin{problem}[$3\clubsuit$]
  Prove that in any tournament, either
  \begin{itemize}
  \ii there exists a directed Hamiltonian cycle, or
  \ii there exists a nontrivial partition $A \sqcup B$
  of the vertices such that $A$ and $B$ are nonempty and
  whenever $a \in A$ and $b \in B$,
  there is an edge from $a$ to $b$.
  \end{itemize}
\end{problem} \printpuid{GRTH6}

%% Type your solution to \href{https://otis.evanchen.cc/arch/GRTH6/}{GRTH6} here ...

%% --------------------------------------------------

\begin{problem}[\href{https://aops.com/community/p7389115}{USA TST 2017/1}, $5\clubsuit$]
  In a sports league, each team uses a set of at most $t$ signature colors.
  A set $S$ of teams is \emph{color-identifiable} if one can assign
  each team in $S$ one of their signature colors,
  such that no team in $S$ is assigned
  \emph{any} signature color of a different team in $S$.
  For all positive integers $n$ and $t$,
  determine the maximum integer $g(n,t)$ such that:
  In any sports league with exactly $n$ distinct colors
  present over all teams, one can always
  find a color-identifiable set of size at least $g(n,t)$.
\end{problem} \printpuid{17USATST1}

%% Type your solution to USA TST 2017/1 (\href{https://otis.evanchen.cc/arch/17USATST1/}{17USATST1}), proposed by Po-Shen Loh here ...

%% --------------------------------------------------

(This TST1 is easy to misread and quite easy to mess up, so be careful.)

\begin{problem}[\href{https://aops.com/community/p21040403}{Moscow 1957, added by Tilek Askerbekov}, $3\clubsuit$]
  Let $1=a_1 \le a_2 \le \dots \le a_n$ be integers.
  Assume $a_{i+1} \leq 2a_i$ for all $i=1,\dots,n-1$
  and $a_1+\dots+a_n$ is even.
  Prove that the $n$ integers can be split into two piles with equal sum.
\end{problem} \printpuid{57MOSCOW}

%% Type your solution to Moscow 1957, added by Tilek Askerbekov (\href{https://otis.evanchen.cc/arch/57MOSCOW/}{57MOSCOW}) here ...

%% --------------------------------------------------

\begin{problem}[\href{https://aops.com/community/p27258650}{Canada 2023, added by Haozhe Yang}, $9\clubsuit$]
  Let $n \ge 1$ and $k \ge 1$ be fixed positive integers.
  A simple graph on $n$ vertices has the property that whenever
  it its vertices are partitioned into two sets,
  then at most $kn$ edges have endpoints in different sets.
  What is the largest integer $m$ (in terms of $n$ and $k$) such that
  there is guaranteed to be a set of $m$ vertices with no edge between them?
\end{problem} \printpuid{23CAN5}

%% Type your solution to Canada 2023, added by Haozhe Yang (\href{https://otis.evanchen.cc/arch/23CAN5/}{23CAN5}) here ...

%% --------------------------------------------------

%%fakesubsection*{A couple extra problems on extremal principle}
\begin{remark*}
It's debatable whether these problems really count as ``local'':
some find them suitable, while others find them unconnected.
In my opinion, these feel different because they are not algorithmic.
\end{remark*}

\begin{problem}[\href{https://aops.com/community/p7357521}{Korea 1995, also Putnam 2016 B3}, $3\clubsuit$]
  Consider finitely many points in the plane such that,
  if we choose any three points $A$, $B$, $C$ among them,
  the area of triangle $ABC$ is always less than $1$.
  Show that all these points lie within the interior
  of some triangle with area $4$.
\end{problem} \printpuid{16PTNMB3}

%% Type your solution to Korea 1995, also Putnam 2016 B3 (\href{https://otis.evanchen.cc/arch/16PTNMB3/}{16PTNMB3}) here ...

%% --------------------------------------------------

\begin{problem}[\href{https://aops.com/community/p1871433}{Iran TST 2010/7}, $3\clubsuit$]
  Let $\mathcal P = P_1P_2 \dots P_{2n}$ be a polygon,
  possibly concave but not self-intersecting.
  Suppose there is a line $\ell$ such that for each $k=1,2,\dots,n$,
  ray $P_{2k-1} P_{2k}$ intersects $\ell$ and is perpendicular to it.
  Prove that $\ell$ intersects $\mathcal P$.
\end{problem} \printpuid{10IRNTST7}

%% Type your solution to Iran TST 2010/7 (\href{https://otis.evanchen.cc/arch/10IRNTST7/}{10IRNTST7}) here ...

%% --------------------------------------------------

\begin{problem}[Sylvester-Gallai, $3\clubsuit$]
  Finitely many points in the plane are given, not all collinear.
  Prove that there exists a line passing through exactly two of them.
\end{problem} \printpuid{ZC62C56E}

%% Type your solution to Sylvester-Gallai (\href{https://otis.evanchen.cc/arch/ZC62C56E/}{ZC62C56E}) here ...

%% --------------------------------------------------

\end{document}
