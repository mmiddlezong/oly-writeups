\documentclass[11pt]{scrartcl}
\usepackage[usenames,dvipsnames,svgnames]{xcolor}
\usepackage[shortlabels]{enumitem}
\usepackage[framemethod=TikZ]{mdframed}
\usepackage{amsmath,amssymb,amsthm}
\usepackage{epigraph}
\usepackage[colorlinks]{hyperref}
\usepackage{microtype}
\usepackage{mathtools}
\usepackage[headsepline]{scrlayer-scrpage}
\usepackage{thmtools}
\usepackage{listings}
\usepackage{derivative}
\renewcommand{\epigraphsize}{\scriptsize}
\renewcommand{\epigraphwidth}{60ex}
\addtolength{\textheight}{3.14cm}
\ihead{\footnotesize\textbf{DCW-global}}
\ohead{\footnotesize Updated Tue 13 Aug 2024 22:16:50 UTC}
\providecommand{\clubs}[1]{$#1\clubsuit$}
\providecommand{\clubg}[1]{\bgroup\color{green!40!black}[$#1\clubsuit$]\egroup}

\providecommand{\ol}{\overline}
\providecommand{\eps}{\varepsilon}
\providecommand{\half}{\frac{1}{2}}
\providecommand{\dang}{\measuredangle} %% Directed angle
\providecommand{\CC}{\mathbb C}
\providecommand{\FF}{\mathbb F}
\providecommand{\NN}{\mathbb N}
\providecommand{\QQ}{\mathbb Q}
\providecommand{\RR}{\mathbb R}
\providecommand{\ZZ}{\mathbb Z}
\providecommand{\dg}{^\circ}
\providecommand{\ii}{\item}
\providecommand{\alert}{\textbf}
\providecommand{\opname}{\operatorname}
\providecommand{\ts}{\textsuperscript}
% hacks for arc
\providecommand{\tarc}{\mbox{\large$\frown$}}
\providecommand{\arc}[1]{\stackrel{\tarc}{#1}}
\reversemarginpar
\providecommand{\printpuid}[1]{\marginpar{\href{https://otis.evanchen.cc/arch/#1}{\ttfamily\footnotesize\color{green!40!black}#1}}}

\mdfdefinestyle{mdbluebox}{roundcorner=10pt,innerbottommargin=9pt,
    linecolor=blue,backgroundcolor=TealBlue!5,}
\declaretheoremstyle[headfont=\sffamily\bfseries\color{MidnightBlue},
    mdframed={style=mdbluebox},]{thmbluebox}
\mdfdefinestyle{mdredbox}{frametitlefont=\bfseries,innerbottommargin=8pt,
    nobreak=true,backgroundcolor=Salmon!5,linecolor=RawSienna,}
\declaretheoremstyle[headfont=\bfseries\color{RawSienna},
    mdframed={style=mdredbox},headpunct={\\[3pt]},postheadspace=0pt,]{thmredbox}
\mdfdefinestyle{mdgreenbox}{linecolor=ForestGreen,backgroundcolor=ForestGreen!5,
    linewidth=2pt,rightline=false,leftline=true,topline=false,bottomline=false,}
\declaretheoremstyle[headfont=\bfseries\sffamily\color{ForestGreen!70!black},
    mdframed={style=mdgreenbox},headpunct={ --- },]{thmgreenbox}
\mdfdefinestyle{mdblackbox}{linecolor=black,backgroundcolor=RedViolet!5!gray!5,
    linewidth=3pt,rightline=false,leftline=true,topline=false,bottomline=false,}
\declaretheoremstyle[mdframed={style=mdblackbox}]{thmblackbox}
\declaretheorem[style=thmredbox,name=Problem]{problem}
\declaretheorem[style=thmredbox,name=Required Problem,sibling=problem]{reqproblem}
\declaretheorem[style=thmbluebox,name=Theorem,numberwithin=problem]{theorem}
\declaretheorem[style=thmbluebox,name=Lemma,sibling=theorem]{lemma}
\declaretheorem[style=thmbluebox,name=Theorem,numbered=no]{theorem*}
\declaretheorem[style=thmbluebox,name=Lemma,numbered=no]{lemma*}
\declaretheorem[style=thmgreenbox,name=Claim,sibling=theorem]{claim}
\declaretheorem[style=thmgreenbox,name=Claim,numbered=no]{claim*}
\declaretheorem[style=thmblackbox,name=Remark,sibling=theorem]{remark}
\declaretheorem[style=thmblackbox,name=Remark,numbered=no]{remark*}
\declaretheorem[style=thmgreenbox,name=Definition,sibling=theorem]{definition}
\declaretheorem[style=thmgreenbox,name=Definition,numbered=no]{definition*}
\declaretheorem[style=thmblackbox,name=Example,sibling=theorem]{example}
\declaretheorem[style=thmblackbox,name=Example,numbered=no]{example*}

\newenvironment{walkthrough}{\noindent\textbf{\color{green!40!black}Walkthrough.}}{}
\newlist{walk}{enumerate}{3}
\setlist[walk]{label=\bfseries (\alph*)}

\usepackage{asymptote}
\begin{asydef}
size(8cm); // set a reasonable default
usepackage("amsmath");
usepackage("amssymb");
settings.tex="pdflatex";
settings.outformat="pdf";
import geometry;
void filldraw(picture pic = currentpicture, conic g, pen fillpen=defaultpen, pen drawpen=defaultpen) { filldraw(pic, (path) g, fillpen, drawpen); }
void fill(picture pic = currentpicture, conic g, pen p=defaultpen) { filldraw(pic, (path) g, p); }
pair foot(pair P, pair A, pair B) { return foot(triangle(A,B,P).VC); }
pair centroid(pair A, pair B, pair C) { return (A+B+C)/3; }
\end{asydef}

\newcommand{\goals}[2]{\bgroup
\sffamily\small \emph{Instructions}: Solve \clubg{#1}.
If you have time, solve \clubg{#2}.\egroup\par}

%% 426c616e6b204c615465587e
\begin{document}
\title{Submission for DCW-GLOBAL}
\subtitle{OTIS (internal use)}
\author{Michael Middlezong}
\date{\today}
\maketitle

\begin{example*}[\href{https://aops.com/community/p2260935}{Canada 2009/2}, $0\clubsuit$]
  Two circles of different radii are cut out of cardboard.
  Each circle is subdivided into $200$ equal sectors.
  On each circle $100$ sectors are painted white
  and the other $100$ are painted black.
  The smaller circle is then placed on top of the larger circle,
  so that their centers coincide.
  Show that one can rotate the small circle so
  that the sectors on the two circles line up
  and at least $100$ sectors on the small circle
  lie over sectors of the same color on the big circle.
\end{example*} \printpuid{09CAN2}

\begin{walkthrough}
\begin{walk}
  \ii Solve the problem by linearity of expectation.
  \ii Write the proof in a ``low-tech'' way that doesn't
  quote the linearity of expectation,
  by considering all $200$ rotations at once.
  This gives a ``politically correct solution''.
\end{walk}
\end{walkthrough}
%% You're not expected to write up walkthroughs (unless you really want to).
%% The source is just for your reference.

\begin{example*}[\href{https://aops.com/community/p29651001}{OTIS Mock AIME 2024, by Joshua Liu}, $0\clubsuit$]
  Perry the Panda is eating some bamboo
  over a five-day period from Monday to Friday (inclusive).
  On Monday, he eats $14$ pieces of bamboo.
  Each following day, Perry eats either one less than three times the previous day
  or one more than the previous day, with equal probability.
  Compute the expected number of pieces of bamboo Perry has eaten throughout the week
  after the end of Friday.
\end{example*} \printpuid{24OIME3}

\begin{walkthrough}
This is another easy problem with linearity of expectation.
Again, the intention of this walkthrough is to make you work through
both the short technical solution (using linearity of expectation)
as well as the longer elementary solution.
The lesson from this walkthrough is again to see that these two solutions are equivalent
so that you understand how the expected value solution is capturing
the ``work'' of the problem for you.

First, here's the linearity solution.
Let $X_i$ denote the random variable for the number of pieces of bamboo
on the $i$th day, for $i = 1, \dots, 5$.
\begin{walk}
  \ii Show that $\mathbb E[X_i] = 7 \cdot 2^i$.
  \ii Extract the answer.
\end{walk}
Now, if we wanted to show this to a small toddler who would cry foul
if you used the words ``random variable'', what should you do instead?
Well, let's imagine we drew the following tree to represent all outcomes.
\begin{center}
\begin{asy}
  size(13cm);

  for (int i=0; i<=3; ++i) {
    for (int j=0; j<2**i; ++j) {
      transform sh = shift(-(16-2**(3-i)), 3*(3-i));
      path cherry = (0,0)--(2**(3-i),3)--(2**(4-i),0);
      draw( shift(j*2**(5-i),0) * sh * cherry, grey );
    }
  }

  dot("$14$", (0,12), dir(90));

  dot("$15$", (-8,9), dir(90));
  dot("$41$", (8,9), dir(90));

  dot( "$16$", (-12,6), dir(90));
  dot( "$44$", (-4,6), dir(90));
  dot( "$42$", (4,6), dir(90));
  dot("$122$", (12,6), dir(90));

  dot( "$17$", (-14,3), dir(90));
  dot( "$47$", (-10,3), dir(90));
  dot( "$45$", (-6,3), dir(90));
  dot("$131$", (-2,3), dir(90));
  dot( "$43$", (2,3), dir(90));
  dot("$125$", (6,3), dir(90));
  dot("$123$", (10,3), dir(90));
  dot("$365$", (14,3), dir(90));

  dot( "$18$", (-15,0), dir(-90));
  dot( "$50$", (-13,0), dir(-90));
  dot( "$48$", (-11,0), dir(-90));
  dot("$140$", (-9,0), dir(-90));
  dot( "$46$", (-7,0), dir(-90));
  dot("$134$", (-5,0), dir(-90));
  dot("$132$", (-3,0), dir(-90));
  dot("$392$", (-1,0), dir(-90));
  dot(" $44$", (1,0), dir(-90));
  dot("$128$", (3,0), dir(-90));
  dot("$126$", (5,0), dir(-90));
  dot("$374$", (7,0), dir(-90));
  dot("$124$", (9,0), dir(-90));
  dot("$368$", (11,0), dir(-90));
  dot("$366$", (13,0), dir(-90));
  dot("$1094$", (15,0), dir(-90));

  for (int i=0; i<16; ++i) {
    label("$s_{" + (string)(i+1) + "}$", (2*i-15,-2), red);
  }
\end{asy}
\end{center}
Label the bottom entries of the tree by $1$, $2$, \dots, $16$
from left to right and let $s_i$ denote the sum of the entries from the root the
$i$\ts{th} leaf.
(For example, $s_1 = 14 + 15 + 16 + 17 + 18$ while $s_6 = 14 + 15 + 44 + 45 + 134$)
The problem asks us to compute the value of
\[ A = \frac{s_1 + s_2 + \dots + s_{16}}{16}. \]
However, for this large sum it's easier to sum across the rows of the tree instead.
\begin{walk}[resume]
  \ii The numerator of $A$ contains $5 \cdot 16 = 80$ terms total
  when you expand out the definition of each $s_i$.
  For each node in the tree, how many times does it appear
  among these $80$ terms?

  \ii Show that moreover, the sum of each row in the tree
  is four times the sum of the row above it.

  \ii Using these two key observations, show that
  \[ A = \frac{2^4 \cdot 14 + 2^3 \cdot 4^1 \cdot 14
  + 2^2 \cdot 4^2 \cdot 14 + 2^1 \cdot 4^3 \cdot 14 + 2^0 \cdot 4^4 \cdot 14}{2^4}. \]

  \ii Extract the final answer.
\end{walk}
When we compare the two solutions, we see that the elementary solution had two
main ``insights'': first to sum across the rows instead of by the branches,
and then that the average value of each row is doubled compared to the preceding one.
The advanced solution shows that \emph{both} insights are actually
each a special case of linearity of expectation.
This demonstrates how powerful the linearity of expectation theory is:
it ``automagically'' erases both main hurdles of the elementary solution.
\end{walkthrough}
%% You're not expected to write up walkthroughs (unless you really want to).
%% The source is just for your reference.

\begin{example*}[\href{https://aops.com/community/p3104304}{ELMO 2013/1}, $0\clubsuit$]
  Let $a_1,a_2,\dots,a_9$ be nine real numbers,
  not necessarily distinct, with average $m$.
  Let $A$ denote the number of triples $1 \le i < j < k \le 9$
  for which $a_i + a_j + a_k \ge 3m$.
  What is the minimum possible value of $A$?
\end{example*} \printpuid{13ELMO1}

\begin{walkthrough}
We say a triple $t = (a_i, a_j, a_k)$
is \emph{large} if $a_i + a_j + a_k \ge 3m$.
\begin{walk}
  \ii Show that among any three disjoint triples,
  at least one triple is large.
  Give a heuristic argument why we expect $A \ge 28$
  as a result.
  \ii Give a construction for $A = 28$.
  (Try making one element large.)
  \ii We now proceed to the ``global'' idea
  of looking at every possible partition in (a) at once.
  Show that there are
  \[ C = \frac{1}{3!} \binom{9}{3,3,3} = 280 \]
  ways to partition the $9$ elements into three disjoint triples.
  \ii How many of the $C$ partitions does each triple $t$ appear in?
  \ii Use your answer to (d) to prove $A \ge 28$, thereby solving the problem.
  \ii Optionally, for an alternate solution,
  explicitly construct a partition of the $\binom 93 = 84$ triples
  into $28$ disjoint triples.
  This would give another proof that $A \ge 28$.
\end{walk}
When doing this calculation for the first time,
you might be surprised that the division of seemingly
random constants ends up with $28$ in the end.
It's important to recognize that the argument in (e)
is ``guaranteed'' to work in a sense.

To elaborate: we constructed in (b)
an example of an equality case, and every estimate we used was sharp.
At the end of (e) we get some number again.
The existence of the equality case means
that this number \emph{must} match the corresponding
constant in (a), namely $28$.
This point is one of the key ideas in the Equality unit;
the so-called ``Sharpness Principle''.
\end{walkthrough}
%% You're not expected to write up walkthroughs (unless you really want to).
%% The source is just for your reference.

\begin{example*}[\href{https://aops.com/community/p107689}{Romania 2004}, $0\clubsuit$]
  Prove that for any complex numbers $z_1$, $z_2$, \dots, $z_n$, satisfying
  $\left\lvert z_1 \right\rvert^2 + \left\lvert z_2 \right\rvert^2
  + \dots + \left\lvert z_n \right\rvert^2 = 1$,
  one can select $\eps_1, \eps_2, \dots, \eps_n \in \{-1,1\}$ such that
  \[ \left\lvert \sum_{k=1}^n \eps_k z_k \right\rvert \le 1. \]
\end{example*} \printpuid{04ROU}

\begin{walkthrough}
\begin{walk}
  \ii Homogenize the inequality in the $z_i$'s
  to eliminate the given condition.
  (You will need a square root.)
  \ii Square both sides to eliminate the square root,
  and expand the left-hand side in a way
  that eliminates the absolute value.
  \ii Show that the desired conclusion can now be rewritten as
  \[ \sum_{i \neq j} \eps_i \eps_j z_i \ol{z_j} \le 0. \]
  \ii Intuitively we don't expect any reason
  for the left-hand side to be positive versus negative
  for a random choice of $\eps_i$'s.
  Use this idea to show that there exists a choice of $\eps_i$'s
  fulfilling the inequality.
  \ii Is it okay to use these inequalities
  even though we are working over complex numbers?
  (For example, why is the left-hand side of (c) even real?)
\end{walk}
\end{walkthrough}
%% You're not expected to write up walkthroughs (unless you really want to).
%% The source is just for your reference.

\begin{example*}[\href{https://aops.com/community/p3261440}{Online Math Open, Ray Li}, $0\clubsuit$]
  Kevin has $2^n-1$ cookies, each labeled with a
  unique nonempty subset of $\left\{ 1,2,\dots,n \right\}$.
  Each day, he chooses one cookie uniformly
  at random out of the cookies not yet eaten.
  Then, he eats that cookie, and all remaining cookies
  that are labeled with a subset of that cookie.
  Determine the expected value of the number of days
  that Kevin eats a cookie before all cookies are gone.
\end{example*} \printpuid{13OMOF29}

\begin{walkthrough}
\begin{walk}
  \ii How does the answer to the problem change
  if the cookie $\varnothing$ is added?
  This shows the omission of the cookie $\varnothing$
  is sort of a red herring.
  \ii What is the probability that the cookie
  $\{1,\dots,n\}$ is \emph{chosen} on some day?
  How about $\{1, \dots, n-1\}$?
  \ii Solve (b) for a general set $S$.
  Your answer should only depend on $|S|$.
  \ii How is the work in (c) related to the
  expected number of days?
  \ii Using linearity of expectation,
  show that the answer is $(3/2)^n-(1/2)^n$.
\end{walk}
\end{walkthrough}
%% You're not expected to write up walkthroughs (unless you really want to).
%% The source is just for your reference.

\begin{example*}[Buffon's needle/noodle, $0\clubsuit$]
  You are about to drop a needle of length $1$ onto a lined floor;
  the lines of the floor are spaced $1$ unit apart.
  Show that the needle hits a line with probability $\frac{2}{\pi}$.
\end{example*} \printpuid{BUFFON}

\begin{walkthrough}
This walkthrough is a bit unconventional,
but I couldn't resist including it because it's so slick
and yet not really known at all.
I was first shown this argument by Qiaochu Yuan ---
it is more elegant, more general, and doesn't use calculus.

Rather than solving Buffon's needle,
we will solve the general \textbf{Buffon's noodle} instead,
where we drop a noodle of arbitrary shape.
\begin{quote}
Given a noodle $\Gamma$ of length $\ell$,
we let $E(\Gamma)$ denote the
expected \emph{number} of intersections of $\Gamma$ with the lined floor.
Determine $E(\Gamma)$.
\end{quote}
We won't worry exactly how to define the ``length'' of a noodle,
since arc length is a bit finnicky to define.
So you have permission to handwave a bit on this walkthrough.

\begin{walk}
  \ii Show that $E(\Gamma) = c \cdot \ell$ for some absolute constant $c$,
  by approximating $\Gamma$ by a bunch of tiny equal straight line segments
  and using linearity of expectation.
  (You can ignore subtle technical issues with this ``approximating''.)

  \ii Consider the special case where
  $\Gamma$ is a circle of diameter $1$.
  Check that $E(\Gamma) = 2$.

  \ii Use the special case in (b) to extract the value of $c$.
  This solves Buffon's noodle.

  \ii Use the fact that a needle of length $1$
  intersects the floor's ruling at most once (with probability $1$)
  in order to extract the answer for Buffon's needle.
\end{walk}
\end{walkthrough}
%% You're not expected to write up walkthroughs (unless you really want to).
%% The source is just for your reference.

\newpage

% ========================================
\section*{Practice problems}
\goals{40}{54}

\epigraph{By this construction, Yahweh's work was indicated,
and Yahweh's work was concealed. Thus would men know their place.}
{Ted Chiang in \emph{Tower of Babylon}}

\begin{problem}[\href{https://aops.com/community/p26822858}{HMMT 2013}, $2\clubsuit$]
  Values $a_1, \dots , a_{2013}$ are chosen independently
  and at random from the set $\{ 1, \dots , 2013\}$.
  What is the expected number of distinct values in the set $\{a_1, \dots, a_{2013}\}$?
\end{problem} \printpuid{13HMMTC6}

%% Type your solution to HMMT 2013 (\href{https://otis.evanchen.cc/arch/13HMMTC6/}{13HMMTC6}) here ...
Let's count the expected number of values that don't appear,
since the answer is $2023$ minus that.
For a given value, the chance it doesn't appear is $\left(\frac{2012}{2013}\right)^{2013}$.
Linearity of expectation gives us that the expected number of values
that don't appear is just $2013 \cdot \left(\frac{2012}{2013}\right)^{2013}$.
Therefore, the answer is
\[ 2013 - 2013 \cdot \left(\frac{2012}{2013}\right)^{2013}. \]
%% --------------------------------------------------

\begin{reqproblem}[\href{https://aops.com/community/p25410776}{Russia 1996}, $3\clubsuit$]
  In the Duma there are $1600$ delegates,
  who have formed $16000$ committees of $80$ people each.
  Prove that one can find two committees
  having no fewer than four common members.
\end{reqproblem} \printpuid{96RUS94}

%% Type your solution to Russia 1996 (\href{https://otis.evanchen.cc/arch/96RUS94/}{96RUS94}) here ...
First, notice that the average number of committees a delegate is in is $800$ by simply double counting.

Pick two committees at random. Then, if $D$ is the set of delegates, and $n_d$
represents the number of committees delegate $d$ is in, we have
\begin{align*}
    \mathbb{E}(\text{\# of people in both committees}) &= \sum_{d\in D} \mathbb{P}(\text{$d$ is in both}) \\
    &= \sum_{d \in D} \frac{\binom{n_d}{2}}{\binom{16000}{2}} \\
    &\ge \sum_{d \in D} \frac{\binom{800}{2}}{\binom{16000}{2}} & \text{(by Jensen's inequality)} \\
    &= 1600 \cdot \frac{800 \cdot 799}{16000 \cdot 15999} \\
    &> 3.
\end{align*}
There must then be a set of two committees who have at least $4$
delegates in common, so we're done.
%% --------------------------------------------------

\begin{problem}[\href{https://aops.com/community/p12146863}{EGMO 2019/5}, $5\clubsuit$]
  Let $n\ge 2$ be an integer,
  and let $a_1, a_2, \dots , a_n$ be positive integers.
  Show that there exist positive integers $b_1, b_2, \dots, b_n$
  satisfying the following three conditions:
  \begin{enumerate}[(a)]
  \ii $a_i\le b_i$ for $i=1, 2, \dots, n$;
  \ii the remainders of $b_1$, $b_2$, \dots, $b_n$
  on division by $n$ are pairwise different,
  \ii $b_1 + \dots + b_n \le n \left( \frac{n-1}{2}
  + \left\lfloor \frac{a_1 + \dots + a_n}{n} \right\rfloor \right)$.
  \end{enumerate}
\end{problem} \printpuid{19EGMO5}

%% Type your solution to EGMO 2019/5 (\href{https://otis.evanchen.cc/arch/19EGMO5/}{19EGMO5}), proposed by Merlijn Staps (NLD) here ...

%% --------------------------------------------------

\begin{problem}[\href{https://aops.com/community/p8651645}{BAMO 2017/4}, $3\clubsuit$]
  Let $\mathcal P$ be a convex $n$-gon, and let $h > 0$ be a real number.
  On each of the $n$ sides of $\mathcal P$ we
  erect internally a rectangle of height $h$
  (meaning the rectangle shares a side with $\mathcal P$
  and moreover the interiors overlap).
  Prove that it's possible to pick a $h$ such that the $n$ rectangles together
  cover the interior of $\mathcal P$, and moreover the sum of their areas
  is at most twice the area of $\mathcal P$.
\end{problem} \printpuid{17BAMO4}

%% Type your solution to BAMO 2017/4 (\href{https://otis.evanchen.cc/arch/17BAMO4/}{17BAMO4}) here ...
Choose $h$ so that the sum of the areas is exactly twice the area of $\mathcal P$,
which we denote by $A$.
Then, we claim every point is covered.

Suppose FTSOC that some point $X$ is not covered.
In this case, we claim that the distance from $X$ to any side is greater than $h$.
Consider a side $s_i$ which is closer (or as close) to $X$ than any other side,
and suppose it is a distance $h_i$ away from $X$.
The foot of the altitude from $X$ to $s_i$ must be contained in $s_i$,
because otherwise, the altitude must pass through a side closer to $X$ than $s_i$ by convexity.
So, if $h_i \le h$,
the rectangle on $\mathcal S$ would contain it,
so we must have $h_i > h$.

However, by splitting the polygon into triangles and adding up their areas,
\[ 2A = \sum_{i=1}^n s_ih < \sum_{i=1}^n s_ih_i = 2A, \]
where $s_i$ denotes the $i$th side, and $h_i$ denotes
the distance from $X$ to $s_i$. This gives us a contradiction.
%% --------------------------------------------------

\begin{problem}[\href{https://aops.com/community/p5021975}{ELMO 2015/2}, $3\clubsuit$]
  Let $m$, $n$, and $x$ be positive integers.
  Prove that
  \[
  \sum_{i=1}^n \min\left( \left\lfloor \frac xi \right\rfloor, m \right)
  = \sum_{i=1}^m \min\left( \left\lfloor \frac xi \right\rfloor, n
  \right).
  \]
\end{problem} \printpuid{15ELMO2}

%% Type your solution to ELMO 2015/2 (\href{https://otis.evanchen.cc/arch/15ELMO2/}{15ELMO2}) here ...

%% --------------------------------------------------

\begin{problem}[\href{https://aops.com/community/p5913929}{BAMO 2014/5}, $3\clubsuit$]
  Let $n$ be a positive integer.
  There are $2n+1$ ranked chess players in a tournament,
  and each player played every other player exactly once, with no ties.
  It turns out that in exactly $k$ games,
  the lower-rated player beat the higher-rated player.
  Prove that some player won between $n-\sqrt{2k}$ and $n+\sqrt{2k}$ games (inclusive).
\end{problem} \printpuid{14BAMO5}

%% Type your solution to BAMO 2014/5 (\href{https://otis.evanchen.cc/arch/14BAMO5/}{14BAMO5}) here ...

%% --------------------------------------------------

\begin{problem}[\href{https://aops.com/community/p874750}{JBMO 2007/3}, $2\clubsuit$]
  A set of $50$ points in the plane is given, no three collinear.
  Each point is colored one of four colors.
  Prove that there exists a color for which
  at least $130$ scalene triangles have all three vertices of that color.
\end{problem} \printpuid{07JBMO3}

%% Type your solution to JBMO 2007/3 (\href{https://otis.evanchen.cc/arch/07JBMO3/}{07JBMO3}) here ...
Pigeonhole tells us that there must be a color used for $13$ or more points.
We claim that among any $13$ points with no three collinear, there are
at least $130$ scalene triangles among them.

Take any pair of points. Notice that the line segment formed can be the base
of at most two isosceles triangles, because otherwise, there would be three points collinear.
This means there are at most $2 \binom{13}{2} = 156$ isosceles triangles out of
$\binom{13}{3} = 286$ total triangles, giving us at least $286-156 = 130$ scalene triangles.
%% --------------------------------------------------

\begin{problem}[Anthony Wang, $2\clubsuit$]
  Bob has $n$ stacks of rocks in a row,
  each with heights randomly and
  uniformly selected from the set $\{1,2,3,4,5\}$.
  In each move, he picks a group of one or more consecutive stacks
  with positive heights and removes $1$ rock from each stack.
  Find, in terms of $n$, the expected value of the minimum number of moves
  he must execute to remove all rocks.
\end{problem} \printpuid{20JJCA4}

%% Type your solution to Anthony Wang (\href{https://otis.evanchen.cc/arch/20JJCA4/}{20JJCA4}), proposed by Anthony Wang here ...
Let $a_1,a_2,\dots,a_n$ be the sizes of the stacks, from left to right,
and set $a_0 = a_{n+1} = 0$ for convenience.
Consider the monovariant
\[ \sum_{i=0}^{n}|a_{i+1} - a_i|. \]
This quantity is $0$ if and only if all the rocks are gone.
Also, if $a_i = a_j = 0$, $i<j-1$, and $a_{i+1}, a_{i+2}, \dots, a_{j-1} > 0$, Bob can remove one rock
from the stacks $a_{i+1}, a_{i+2}, \dots, a_{j-1}$,
and one can check that the monovariant decreases by $2$.
At any point in time before all the rocks are gone, a pair of indices $i,j$ exists
(remember how we set $a_0 = a_{n+1} = 0$).
So, Bob can remove all rocks in $\frac12 \sum_{i=0}^{n}|a_{i+1} - a_i|$ moves.

We show this is the minimum. Notice that one move only
can affect the value of $|a_{i+1} - a_i|$ for two indices $i$.
And at each of those indices, the value cannot be changed by more than $1$.
So, the monovariant cannot decrease by more than $2$ in one move.

The answer extraction involves finding the expected value of $\frac12 \sum_{i=0}^{n}|a_{i+1} - a_i|$.
By linearity of expectation, this is just
\[ \frac12 (3 + 3 + \frac85 (n-1)) = \boxed{\frac{4n + 11}{5}}. \]
%% --------------------------------------------------

\begin{problem}[\href{https://aops.com/community/p1341792}{Putnam 2008 B3}, $3\clubsuit$]
  Let $n \ge 2$ be a positive integer.
  What is the largest possible radius of a circle inside a
  $n$-dimensional hypercube of side length $2$?
\end{problem} \printpuid{08PTNMB3}

%% Type your solution to Putnam 2008 B3 (\href{https://otis.evanchen.cc/arch/08PTNMB3/}{08PTNMB3}) here ...

%% --------------------------------------------------

\begin{problem}[\href{https://aops.com/community/p26373095}{USEMO 2022/1}, $3\clubsuit$]
  A \emph{stick} is defined as a $1 \times k$ or $k\times 1$ rectangle
  for any integer $k \ge 1$.
  We wish to partition the cells of a $2022 \times 2022$ chessboard
  into $m$ non-overlapping sticks, such that any two of these $m$ sticks
  share at most one unit of perimeter.
  Determine the smallest $m$ for which this is possible.
\end{problem} \printpuid{22USEMO1}

%% Type your solution to USEMO 2022/1 (\href{https://otis.evanchen.cc/arch/22USEMO1/}{22USEMO1}), proposed by Holden Mui here ...

%% --------------------------------------------------

\begin{reqproblem}[\href{https://aops.com/community/p28015691}{TSTST 2023/4}, $3\clubsuit$]
  Let $n \ge 3$ be an integer and let $K_n$ be the complete graph on $n$ vertices.
  Each edge of $K_n$ is colored either red, green, or blue.
  Let $A$ denote the number of triangles in $K_n$
  with all edges of the same color, and
  let $B$ denote the number of triangles in $K_n$
  with all edges of different colors.
  Prove that
  \[ B \le 2A + \frac{n(n-1)}3. \]
\end{reqproblem} \printpuid{23TSTST4}

%% Type your solution to TSTST 2023/4 (\href{https://otis.evanchen.cc/arch/23TSTST4/}{23TSTST4}), proposed by Ankan Bhattacharya here ...
Let $C$ be the number of triangles with exactly two edges of the same color.
Notice that $A + B + C = \binom n3$.

The inequality can be rewritten as
\[ \binom n3 - \frac{n(n-1)}3 \le 3A + C. \]
Let $r_i$ be the number of red edges emanating from vertex $i$,
and define $g_i$ and $b_i$ similarly.
Then, the key observation is that, by counting per vertex,
\begin{align*}
3A + C &= \sum \binom{r_i}{2} + \sum \binom{g_i}{2} + \sum \binom{b_i}{2} \\
&= \sum \left(\binom{r_i}{2} + \binom{g_i}{2} + \binom{b_i}{2}\right) \\
&\ge 3n \binom{(n-1)/3}{2} & \text{(by Jensen's inequality)} \\
&= \binom n3 - \frac{n(n-1)}3,
\end{align*}
which concludes the proof.
%% --------------------------------------------------

\begin{problem}[\href{https://aops.com/community/p23936540}{St Petersburg 2021/10.5}, $3\clubsuit$]
  The vertices of a convex $2550$-gon are colored black and white as follows:
  black, white, two black, two white, three black, three white, \dots,
  $50$ black, $50$ white. Dania divides the polygon into $1274$ quadrilaterals
  by drawing diagonals that do not intersect inside the polygon.
  Prove that there exists a quadrilateral among these,
  in which two adjacent vertices are black and the other two are white.
\end{problem} \printpuid{21SPBRG105}

%% Type your solution to St Petersburg 2021/10.5 (\href{https://otis.evanchen.cc/arch/21SPBRG105/}{21SPBRG105}) here ...

%% --------------------------------------------------

\begin{problem}[\href{https://aops.com/community/p5079689}{IMO 2015/1}, $3\clubsuit$]
  We say that a finite set $\mathcal{S}$ of points in the plane
  is \emph{balanced} if,
  for any two different points $A$ and $B$ in $\mathcal{S}$,
  there is a point $C$ in $\mathcal{S}$ such that $AC=BC$.
  We say that $\mathcal{S}$ is \emph{centre-free} if for
  any three different points $A$, $B$ and $C$ in $\mathcal{S}$,
  there are no points $P$ in $\mathcal{S}$ such that $PA=PB=PC$.

  \begin{enumerate}
  \item[(a)] Show that for all integers $n\ge 3$,
  there exists a balanced set consisting of $n$ points.
  \item[(b)] Determine all integers $n\ge 3$ for which
  there exists a balanced centre-free set consisting of $n$ points.
  \end{enumerate}
\end{problem} \printpuid{15IMO1}

%% Type your solution to IMO 2015/1 (\href{https://otis.evanchen.cc/arch/15IMO1/}{15IMO1}), proposed by Merlijn Staps (NLD) here ...

%% --------------------------------------------------

\begin{problem}[\href{https://aops.com/community/p2670037}{USAMO 2012/6}, $3\clubsuit$]
  For integer $n \ge 2$,
  let $x_1$, $x_2$, \dots, $x_n$ be real numbers satisfying
  \[ x_1 + x_2 +  \dots + x_n = 0
  \quad\text{and}\quad x_1^2 + x_2^2 + \dots + x_n^2 = 1. \]
  For each subset $A\subseteq\{1, 2, \dots, n\}$, define $S_A=\sum_{i\in A} x_i$.
  (If $A$ is the empty set, then $S_A=0$.)
  Prove that for any positive number $\lambda$,
  the number of sets $A$ satisfying $S_A\geq\lambda$ is at most $2^{n-3}/\lambda^2$.
  For which choices of $x_1$, $x_2$, \dots, $x_n$, $\lambda$ does equality hold?
\end{problem} \printpuid{12AMO6}

%% Type your solution to USAMO 2012/6 (\href{https://otis.evanchen.cc/arch/12AMO6/}{12AMO6}), proposed by Gabriel Carroll here ...

%% --------------------------------------------------

\begin{reqproblem}[\href{https://aops.com/community/p170957}{Russia 1999}, $5\clubsuit$]
  In a certain finite nonempty school, every boy likes at least one girl.
  Prove that we can find a set $S$ of strictly more than half
  the students in the school such that each
  boy in $S$ likes an odd number of girls in $S$.
\end{reqproblem} \printpuid{99RUS}

%% Type your solution to Russia 1999 (\href{https://otis.evanchen.cc/arch/99RUS/}{99RUS}) here ...

%% --------------------------------------------------

\begin{problem}[\href{https://aops.com/community/p131872}{Shortlist 1999 C4}, $2\clubsuit$]
  Let $n$ be a positive integer
  and let $Z = \ZZ / n^2 \ZZ$ denote the set of integers modulo $n^2$.
  Suppose $A \subseteq Z$ with $|A| = n$.
  Prove that there exists $B \subseteq Z$
  with $|B| = n$ such that $|A+B| \ge \half n^2$.
\end{problem} \printpuid{99SLC4}

%% Type your solution to Shortlist 1999 C4 (\href{https://otis.evanchen.cc/arch/99SLC4/}{99SLC4}) here ...
Choose $B$ randomly out of all $n$ element subsets of $Z$.
The probability that any given number $i$ appears in $A+B$
is the probability that $i-a \in B$ for some $a \in A$.
This is equivalent to $B$ overlapping with
an arbitrary subset of size $n$, which has probability
\[1 - \frac{\binom{n^2-n}{n}}{\binom{n^2}{n}}. \]
Thus,
\[ \mathbb{E}[|A+B|] = n^2\left(1 - \frac{\binom{n^2-n}{n}}{\binom{n^2}{n}}\right). \]
Since
\begin{align*}
    \frac{\binom{n^2-n}{n}}{\binom{n^2}{n}} &\le \left(\frac{n-1}{n}\right)^n \\
    &< \frac 1e < \frac 12,
\end{align*}
we have $\mathbb{E}[|A+B|] \ge \frac12 n^2$, so there must exist
some $B$ satisfying $|A+B| \ge \frac12 n^2$.
%% --------------------------------------------------

\begin{problem}[\href{https://aops.com/community/p1544473}{Ireland 1994/10}, $3\clubsuit$]
  Fix an integer $n \ge 1$.
  A square is partitioned into $n$ convex polygons,
  A line segment which joins two vertices of polygons in the dissection,
  and does not contain any other vertices of the polygons in its interior,
  is called a \emph{basic} segment.

  Determine the maximum number of basic segments
  which could be present in such a dissection, in terms of $n$.
\end{problem} \printpuid{94IRL10}

%% Type your solution to Ireland 1994/10 (\href{https://otis.evanchen.cc/arch/94IRL10/}{94IRL10}) here ...

%% --------------------------------------------------

\begin{problem}[\href{https://aops.com/community/p105128}{Extension of IMO 1970, due to Ravi Boppana}, $9\clubsuit$]
  Prove that for all sufficiently large positive integers $n$,
  within any $n$ points on the plane in general position,
  at most $66.67\%$ of the triangles with vertices
  among the points are acute.
\end{problem} \printpuid{70IMO6}

%% Type your solution to Extension of IMO 1970, due to Ravi Boppana (\href{https://otis.evanchen.cc/arch/70IMO6/}{70IMO6}) here ...

%% --------------------------------------------------

\begin{problem}[\href{https://aops.com/community/p1136962}{Iran TST 2008/6}, $3\clubsuit$]
  Suppose $799$ teams participate in a round-robin tournament.
  Prove that one can find two disjoint groups $A$ and $B$ of seven teams each
  such that all teams in $A$ defeated all teams in $B$.
\end{problem} \printpuid{08IRNTST6}

%% Type your solution to Iran TST 2008/6 (\href{https://otis.evanchen.cc/arch/08IRNTST6/}{08IRNTST6}) here ...

%% --------------------------------------------------

\begin{reqproblem}[\href{https://aops.com/community/p2362302}{Shortlist 2010 C5}, $5\clubsuit$]
  Suppose $n \geq 4$ players participate in a tennis tournament.
  Any two players have played exactly one game,
  and there was no tie game.
  We call a company of four players \emph{bad} if one player
  was defeated by the other three players, and each of these
  three players won a game and lost another game among themselves.
  Suppose that there is no bad company in this tournament.

  Let $w_i$ and $\ell_i$ be respectively
  the number of wins and losses of the $i$-th player.
  Prove that
  \[\sum^n_{i=1} \left(w_i - \ell_i\right)^3 \geq 0. \]
\end{reqproblem} \printpuid{10SLC5}

%% Type your solution to Shortlist 2010 C5 (\href{https://otis.evanchen.cc/arch/10SLC5/}{10SLC5}) here ...

%% --------------------------------------------------

\begin{problem}[\href{https://aops.com/community/p15952773}{USAMO 2020/2}, $5\clubsuit$]
  An empty $2020 \times 2020 \times 2020$ cube is given,
  and a $2020 \times 2020$ grid of square unit cells is drawn on each of its six faces.
  A \emph{beam} is a $1 \times 1 \times 2020$ rectangular prism.
  Several beams are placed inside the cube subject to the following conditions:
  \begin{itemize}
  \item The two $1 \times 1$ faces of each beam coincide
  with unit cells lying on opposite faces of the cube.
  (Hence, there are $3 \cdot 2020^2$ possible positions for a beam.)
  \item No two beams have intersecting interiors.
  \item The interiors of each of the four $1 \times 2020$ faces of each beam touch
  either a face of the cube or the interior of the face of another beam.
  \end{itemize}
  What is the smallest positive number of beams that can be placed to satisfy these conditions?
\end{problem} \printpuid{20AMO2}

%% Type your solution to USAMO 2020/2 (\href{https://otis.evanchen.cc/arch/20AMO2/}{20AMO2}), proposed by Alex Zhai here ...

%% --------------------------------------------------

\begin{reqproblem}[\href{https://aops.com/community/p1860794}{USAMO 2010/6}, $9\clubsuit$]
  There are $68$ ordered pairs (not necessarily distinct)
  of nonzero integers on a blackboard.
  It's known that for no integer $k$ does both $(k,k)$ and $(-k,-k)$ appear.
  A student erases some of the $136$ integers such that
  no two erased integers have sum zero, and scores one point
  for each ordered pair with at least one erased integer.
  What is the maximum possible score the student can guarantee?
\end{reqproblem} \printpuid{10AMO6}

%% Type your solution to USAMO 2010/6 (\href{https://otis.evanchen.cc/arch/10AMO6/}{10AMO6}), proposed by Gerhard Woeginger here ...

%% --------------------------------------------------

\begin{problem}[\href{https://aops.com/community/p7743073}{RMM 2017/5}, $9\clubsuit$]
  Fix an integer $n \geq 2$. An $n\times n$ sieve is an $n\times n$
  array with $n$ cells removed so that exactly one cell
  is removed from every row and every column.
  A stick is a $1\times k$ or $k\times 1$ array for any integer $k \ge 1$.
  For any sieve $A$, let $m(A)$ be the minimal number
  of sticks required to partition $A$.
  Find all possible values of $m(A)$,
  as $A$ varies over all possible $n\times n$ sieves.
\end{problem} \printpuid{17RMM5}

%% Type your solution to RMM 2017/5 (\href{https://otis.evanchen.cc/arch/17RMM5/}{17RMM5}), proposed by Palmer Mebane, Nikolai Beluhov here ...

%% --------------------------------------------------

\end{document}
