\documentclass[11pt]{scrartcl}
\usepackage[usenames,dvipsnames,svgnames]{xcolor}
\usepackage[shortlabels]{enumitem}
\usepackage[framemethod=TikZ]{mdframed}
\usepackage{amsmath,amssymb,amsthm}
\usepackage{epigraph}
\usepackage[colorlinks]{hyperref}
\usepackage{microtype}
\usepackage{mathtools}
\usepackage[headsepline]{scrlayer-scrpage}
\usepackage{thmtools}
\usepackage{listings}
\usepackage{derivative}
\renewcommand{\epigraphsize}{\scriptsize}
\renewcommand{\epigraphwidth}{60ex}
\addtolength{\textheight}{3.14cm}
\ihead{\footnotesize\textbf{BCX-graph}}
\ohead{\footnotesize Updated Mon 29 Apr 2024 16:04:36 UTC}
\providecommand{\clubs}[1]{$#1\clubsuit$}
\providecommand{\clubg}[1]{\bgroup\color{green!40!black}[$#1\clubsuit$]\egroup}

\providecommand{\ol}{\overline}
\providecommand{\eps}{\varepsilon}
\providecommand{\half}{\frac{1}{2}}
\providecommand{\dang}{\measuredangle} %% Directed angle
\providecommand{\CC}{\mathbb C}
\providecommand{\FF}{\mathbb F}
\providecommand{\NN}{\mathbb N}
\providecommand{\QQ}{\mathbb Q}
\providecommand{\RR}{\mathbb R}
\providecommand{\ZZ}{\mathbb Z}
\providecommand{\dg}{^\circ}
\providecommand{\ii}{\item}
\providecommand{\alert}{\textbf}
\providecommand{\opname}{\operatorname}
\providecommand{\ts}{\textsuperscript}
% hacks for arc
\providecommand{\tarc}{\mbox{\large$\frown$}}
\providecommand{\arc}[1]{\stackrel{\tarc}{#1}}
\reversemarginpar
\providecommand{\printpuid}[1]{\marginpar{\href{https://otis.evanchen.cc/arch/#1}{\ttfamily\footnotesize\color{green!40!black}#1}}}

\mdfdefinestyle{mdbluebox}{roundcorner=10pt,innerbottommargin=9pt,
    linecolor=blue,backgroundcolor=TealBlue!5,}
\declaretheoremstyle[headfont=\sffamily\bfseries\color{MidnightBlue},
    mdframed={style=mdbluebox},]{thmbluebox}
\mdfdefinestyle{mdredbox}{frametitlefont=\bfseries,innerbottommargin=8pt,
    nobreak=true,backgroundcolor=Salmon!5,linecolor=RawSienna,}
\declaretheoremstyle[headfont=\bfseries\color{RawSienna},
    mdframed={style=mdredbox},headpunct={\\[3pt]},postheadspace=0pt,]{thmredbox}
\mdfdefinestyle{mdgreenbox}{linecolor=ForestGreen,backgroundcolor=ForestGreen!5,
    linewidth=2pt,rightline=false,leftline=true,topline=false,bottomline=false,}
\declaretheoremstyle[headfont=\bfseries\sffamily\color{ForestGreen!70!black},
    mdframed={style=mdgreenbox},headpunct={ --- },]{thmgreenbox}
\mdfdefinestyle{mdblackbox}{linecolor=black,backgroundcolor=RedViolet!5!gray!5,
    linewidth=3pt,rightline=false,leftline=true,topline=false,bottomline=false,}
\declaretheoremstyle[mdframed={style=mdblackbox}]{thmblackbox}
\declaretheorem[style=thmredbox,name=Problem]{problem}
\declaretheorem[style=thmredbox,name=Required Problem,sibling=problem]{reqproblem}
\declaretheorem[style=thmbluebox,name=Theorem,numberwithin=problem]{theorem}
\declaretheorem[style=thmbluebox,name=Lemma,sibling=theorem]{lemma}
\declaretheorem[style=thmbluebox,name=Theorem,numbered=no]{theorem*}
\declaretheorem[style=thmbluebox,name=Lemma,numbered=no]{lemma*}
\declaretheorem[style=thmgreenbox,name=Claim,sibling=theorem]{claim}
\declaretheorem[style=thmgreenbox,name=Claim,numbered=no]{claim*}
\declaretheorem[style=thmblackbox,name=Remark,sibling=theorem]{remark}
\declaretheorem[style=thmblackbox,name=Remark,numbered=no]{remark*}
\declaretheorem[style=thmgreenbox,name=Definition,sibling=theorem]{definition}
\declaretheorem[style=thmgreenbox,name=Definition,numbered=no]{definition*}
\declaretheorem[style=thmblackbox,name=Example,sibling=theorem]{example}
\declaretheorem[style=thmblackbox,name=Example,numbered=no]{example*}

\newenvironment{walkthrough}{\noindent\textbf{\color{green!40!black}Walkthrough.}}{}
\newlist{walk}{enumerate}{3}
\setlist[walk]{label=\bfseries (\alph*)}

\usepackage{asymptote}
\begin{asydef}
size(8cm); // set a reasonable default
usepackage("amsmath");
usepackage("amssymb");
settings.tex="pdflatex";
settings.outformat="pdf";
import geometry;
void filldraw(picture pic = currentpicture, conic g, pen fillpen=defaultpen, pen drawpen=defaultpen) { filldraw(pic, (path) g, fillpen, drawpen); }
void fill(picture pic = currentpicture, conic g, pen p=defaultpen) { filldraw(pic, (path) g, p); }
pair foot(pair P, pair A, pair B) { return foot(triangle(A,B,P).VC); }
pair centroid(pair A, pair B, pair C) { return (A+B+C)/3; }
\end{asydef}

\newcommand{\goals}[2]{\bgroup
\sffamily\small \emph{Instructions}: Solve \clubg{#1}.
If you have time, solve \clubg{#2}.\egroup\par}

%% 426c616e6b204c615465587e
\begin{document}
\title{Submission for BCX-GRAPH}
\subtitle{OTIS (internal use)}
\author{Michael Middlezong}
\date{\today}
\maketitle

\begin{example*}[\href{https://aops.com/community/p740398}{Canada 2006/4}, $0\clubsuit$]
  Consider a round-robin tournament with $2k+1$ teams,
  where each team plays each other team exactly once.
  We say that three teams $X$, $Y$ and $Z$,
  form a \emph{cyclic triplet} if $X$ beats $Y$, $Y$ beats $Z$ and $Z$ beats $X$.
  There are no ties.
  Find the minimum and maximum possible number of cyclic triplets.
\end{example*} \printpuid{06CAN4}

\begin{walkthrough}
The minimum bound is not that interesting.
\begin{walk}
  \ii Give an example of a tournament with no cyclic triplet.
  This finds the minimum.
\end{walk}
It's the maximum that we'll be most interested in.
For a team $v$, let $\opname{outdeg} v$ denote the number of teams beaten by $v$.
(This notation is the standard graph-theoretic one.)
In order to count it, it will actually be parametrize
our target in terms of degrees.
\begin{walk}[resume]
  \ii Rephrase the ``maximum'' problem in terms
  of the number of \emph{non-cyclic} triplets.
  \ii By double-counting, find an expression for the number of non-cyclic
  triplets in terms of the outdegrees of the vertices.
  (Possible hint: every non-cyclic triplet can be labeled $vwx$
  with $v \to w$, $v \to x$.)
\end{walk}
Thus we are reduced to an algebraic calculation.
\begin{walk}[resume]
  \ii Use Jensen's inequality to show there are at least
  $(2k+1) \binom k2$ non-cyclic triplets.
  (If you've never seen Jensen's inequality before,
  you can read about it in Section 2.5.1 of the OTIS Excerpts.)
  \ii Give an example where equality holds;
  thus the maximum is $\binom{2k+1}{3} - (2k+1) \binom k2$.
\end{walk}
\end{walkthrough}
%% You're not expected to write up walkthroughs (unless you really want to).
%% The source is just for your reference.

\begin{example*}[\href{https://aops.com/community/p13030771}{BAMO 2005/4}, $0\clubsuit$]
  Let $G$ be a connected graph on $n$ vertices.
  Cynthia wishes to select some subset $S$ of the edges
  so that every vertex is an endpoint of an odd number of edges in $S$.
  \begin{enumerate}
  \ii[(a)] Prove that Cynthia can never fulfill her wish if $n$ is odd.
  \ii[(b)] Prove that Cynthia can always fulfill her wish if $n$ is even.
  \end{enumerate}
\end{example*} \printpuid{05BAMO4}

\begin{walkthrough}
The first part (a) is not so hard (parity arguments)
so this walkthrough focuses on (b).
\begin{walk}
  \ii Why can't we just use induction by removing vertices?
  \ii Figure out under what circumstances you can
  remove an edge, though.
  \ii Reduce the case to where $G$ is a tree.
  \ii When $G$ is a tree, certain moves are now ``forced'';
  what are they? (Possible hint: look at leaves.)
  \ii By ``repeating'' (d),
  describe an algorithm such that, for any tree $G$,
  produces a set $S$ with the desired property.
  Prove that it exists.
  \ii Optionally, show that $S$ is unique (when $G$ is a tree).
  There is a good chance this follows from your solution to (d) and (e)
  and doesn't require any additional insights.
  \ii Where did we use the hypothesis that $G$ is connected?
\end{walk}
\end{walkthrough}
%% You're not expected to write up walkthroughs (unless you really want to).
%% The source is just for your reference.

\newpage

% ========================================
\section*{Practice problems}
\goals{30}{40}

\epigraph{Induction is the coordinate bashing of graph theory.}{A MOP 2012 grader}

\begin{problem}[Chartrand-Zhang 2.14, $2\clubsuit$]
  Let $G$ be a finite graph such that every edge of $G$
  has exactly one endpoint of even degree.
  Show that $G$ has an even number of edges.
\end{problem} \printpuid{CZHGT214}

%% Type your solution to Chartrand-Zhang 2.14 (\href{https://otis.evanchen.cc/arch/CZHGT214/}{CZHGT214}) here ...
We use a double counting approach.
Let an edge sum be the sum of the degrees of the two endpoints of an edge.
Consider the total edge sum over all edges in the graph.
Since each individual edge sum must be odd, it suffices to show that this total is even.

It's not hard to see that this total should equal
\[ \sum_v (\deg v)^2. \]
Finally, this quantity is even because it has the same parity as $\sum_v \deg v$, so we are done.
%% --------------------------------------------------

\begin{reqproblem}[$2\clubsuit$]
  Prove:
  \begin{enumerate}[(a)]
  \ii Let $G$ be a simple undirected graph. Show that
  \[ \sum_v \deg v = 2\#E. \]
  \ii Let $G$ be a directed graph. We say the \emph{indegree} of a vertex $v$
  (denoted $\opname{indeg} v$) is the number of incoming edges,
  while the \emph{outdegree} (denoted $\opname{outdeg} v$) is the number of outgoing edges.
  Show that
  \[ \sum_v \opname{indeg} v = \sum_v \opname{outdeg} v = \#E. \]
  \ii Continuing the notation of (b), suppose further that $G$ is a tournament,
  i.e.\ every pair of vertices has exactly one edge between them
  oriented in one of the two directions. Prove that
  \[ \sum_v (\opname{indeg} v)^2 = \sum_v (\opname{outdeg} v)^2. \]
  \end{enumerate}
\end{reqproblem} \printpuid{GRTH5}

%% Type your solution to \href{https://otis.evanchen.cc/arch/GRTH5/}{GRTH5} here ...
\begin{enumerate}[(a)]
    \ii Each edge contributes $2$ to the total degree, one for each vertex it touches. (Formally we could use induction.)
    \ii Every directed edge contributes $1$ to the total indegree and $1$ to the total outdegree.
    \ii In a graph with $n$ vertices, we have
    \begin{align*}
        \sum_v (\opname{indeg} v)^2 - \sum_v (\opname{outdeg} v)^2 &= \sum_v (\opname{indeg} v)^2 - (\opname{outdeg} v)^2 \\
        &= \sum_v (\opname{indeg} v + \opname{outdeg} v)(\opname{indeg} v - \opname{outdeg} v) \\
        &= (n-1) \sum_v \opname{indeg} v - \opname{outdeg} v \\
        &= 0,
    \end{align*}
    with the last step using the result of part (b). This concludes the proof.
\end{enumerate}
%% --------------------------------------------------

\begin{problem}[Chartrand-Zhang 2.36, $2\clubsuit$]
  Find the smallest positive integer $k$ for which there exists a simple graph
  on $3k$ vertices, for which exactly $k$ vertices have degree $2$,
  exactly $k$ vertices have degree $6$, and exactly $k$ vertices have degree $7$.
\end{problem} \printpuid{CZHGT236}

%% Type your solution to Chartrand-Zhang 2.36 (\href{https://otis.evanchen.cc/arch/CZHGT236/}{CZHGT236}) here ...
The answer is $k=4$.
In order for a vertex to have degree $7$, we must have $k \geq 3$.
We also know that $k=3$ is impossible, because the total degree must be even.

A construction for $k=4$ is to first split the vertices into three groups of $4$.
Then, each vertex in group $1$ is connected with all four vertices in group $2$ and the other three vertices in group $1$.
Next, each vertex in group $2$ is connected with all four vertices in group $1$ and two vertices in group $3$, in such a way that each vertex in group $3$ is hit twice.
In this construction, the vertices in group $1$ have degree $7$, the vertices in group $2$ have degree $6$, and the vertices in group $3$ have degree $2$.
%% --------------------------------------------------

\begin{problem}[$2\clubsuit$]
  Let $G$ be a graph on $n \ge 2$ vertices.
  Prove that either $G$ or its complement is a connected graph.
  (The \emph{complement} of a graph, denoted $\ol G$, is the one
  which has the same vertices as $G$ but has an edge between two vertices
  if and only if the corresponding edge $G$ is \emph{not} present.)
\end{problem} \printpuid{GRTH4}

%% Type your solution to \href{https://otis.evanchen.cc/arch/GRTH4/}{GRTH4} here ...
Assume $G$ is not a connected graph.
Then, it has at least two connected components.

We aim to show that there is a path between any two vertices in $\overline{G}$.
Let $a$ and $b$ be two vertices, and let $A$ and $B$ be their respective connected components in $G$.

If $A$ and $B$ are distinct connected components, then notice that $a$ and $b$ must be directly connected by an edge in $\overline{G}$.

Otherwise, let $C$ be a connected component distinct from $A$ and $B$, and let $c \in C$.
Our desired path is $a \to c \to b$.
%% --------------------------------------------------

\begin{problem}[\href{https://aops.com/community/p2769878}{CGMO 2012/6}, $3\clubsuit$]
  There are $n$ cities, $2$ airline companies in a country.
  Between any two cities, there is exactly one $2$-way flight connecting them
  which is operated by one of the two companies.
  A female mathematician plans a travel route,
  so that it starts and ends at the same city,
  passes through at least two other cities,
  and each city in the route is visited once.
  She finds out that wherever she starts and whatever route she chooses,
  she must take flights of both companies.
  Find the maximum value of $n$.
\end{problem} \printpuid{12CGMO6}

%% Type your solution to CGMO 2012/6 (\href{https://otis.evanchen.cc/arch/12CGMO6/}{12CGMO6}) here ...
The answer is $n=4$.
The problem statement's condition is equivalent to saying that there are no monochromatic cycles in some red-blue edge labeling of $K_n$.
It is not hard to construct such an edge coloring for $K_4$ that works by manually checking.

It is well known that when $n \geq 6$, any red-blue edge coloring of $K_n$ will contain a monochromatic triangle (proof is by Pigeonhole).
In order to not create a monochromatic triangle when $n=5$, each vertex must have exactly $2$ red edges and $2$ blue edges emanating from it.
(Otherwise, in an attempt to not create monochromatic triangles, the endpoints of any $3$ vertices of the same color will form a monochromatic triangle.)
This information is enough to make a constructive argument for why $n=5$ is impossible, using casework at any point where a non-obvious decision must be made.
%% --------------------------------------------------

\begin{problem}[\href{https://aops.com/community/p2374667}{USAMO 1981/2}, $3\clubsuit$]
  Every pair of communities in a county are linked directly
  by one mode of transportation: bus, train, or airplane.
  All three methods of transportation are used in the county
  with no community being serviced by all three modes
  and no three communities being linked pairwise by the same mode.
  Determine the largest number of communities in this county.
\end{problem} \printpuid{81AMO2}

%% Type your solution to USAMO 1981/2 (\href{https://otis.evanchen.cc/arch/81AMO2/}{81AMO2}) here ...
The answer to $n=4$.
A construction for $n=4$ is to make a square with its sides all one mode of transportation, and the diagonals each a different mode of transportation from the sides and other diagonal.

It remains to show $n \geq 5$ is impossible, and it suffices to show $n=5$ is impossible (since we are working with complete graphs).
This problem admits certain not-so-elegant brute force solutions.
The key to a more elegant solution is to partition the communities into three groups.

Let group $X$ contain communities that allow only bus and train.
Let group $Y$ contain communities that allow only train and airplane.
Let group $Z$ contain communities that allow only bus and airplane.

If a town only allows one mode of transportation, it can be put in any of the two groups that allow that mode.
In this partition, at least two of the groups must be non-empty, based on the condition that all three modes of transportation are used at least once.

Suppose exactly two of the groups are non-empty, and WLOG let them be groups $X$ and $Y$.
We claim that neither group can have more than two communities.
For contradiction, assume WLOG that group $X$ has any three communities $x_1, x_2, x_3$, and that $y$ is any community in group $Y$.
Then, the link between $x_i$ and $y$ for each $i$ must be by train.
This means that in order to not form triangles of the same mode, the link between $x_i$ and $x_j$ for $i \neq j$ must be by bus.
However, this means that we have created a triangle anyway.
Thus, when exactly two of the groups are non-empty, the maximum number of communities is $2 \cdot 2 = 4$.

Next, suppose that all three groups are non-empty.
We claim that each group can only have one community.
Suppose for contradiction that a group, WLOG $X$, has any two communities $x_1$ and $x_2$, and let $y$ and $z$ be communities from their respective groups.
Then, $x_1y$ and $x_2y$ are connected by train, $yz$ is connected by airplane, and $x_1z$ and $x_2z$ are connected by bus.
Now, no matter what, the connection of $x_1x_2$ will produce a contradiction.
Thus, in this case, we cannot have more than $3$ communities, concluding the proof.

%% --------------------------------------------------

\begin{problem}[$3\clubsuit$]
  One day, some students living in the University of Nebraska-Lincoln
  are friends with each other.
  The flu affects each student for one day;
  then the student is immune the next day.
  On any day $n$, a healthy student cares for all their sick friends,
  and falls ill on day $n+1$ if they were not immune on day $n$.

  \begin{enumerate}[(a)]
  \ii Prove that eventually the epidemic will die out.
  \ii Suppose that immediately before the epidemic some students
  got vaccinated and thus had an additional immunity on day 1.
  Give an example to show the epidemic may continue indefinitely
  (for some choice of initial students and knowledge relation).
  \end{enumerate}
\end{problem} \printpuid{SICKNESS}

%% Type your solution to \href{https://otis.evanchen.cc/arch/SICKNESS/}{SICKNESS} here ...
\begin{enumerate}[(a)]
  \ii Let $G$ be a connected graph where the vertices represent people and the edges represent friendship.
  Let $S_0$ be the set of vertices initially sick on day $0$.
  For each vertex $v \in G$, we label it with a number;
  this number equals the shortest distance from $v$ to any vertex in $S_0$
  (the distance from a vertex to itself is $0$).
  
  We claim that the set of people sick on day $k$ is precisely the set of vertices labeled $k$,
  which we will denote by $S_k$.
  We will prove this using induction.
  The base case $k=0$ is true by definition.
  Then, suppose it is true for $k$; we aim to show it is true for $k+1$.

  Consider a vertex $v$ which is sick on day $k+1$.
  Then, by our inductive hypothesis, it must neighbor a vertex labeled $k$.
  From this, it follows from the definition of our label that $v$ has label $k-1$ or $k+1$.
  It also must not have been sick on day $k-1$,
  because otherwise, it would be immune on day $k$.
  Thus, $v$ must be labeled $k+1$.
  It remains to show that all vertices labeled $k+1$ neighbor a vertex labeled $k$,
  but this is trivial by moving along the shortest path from the vertex to a vertex in $S_0$.

  \ii Consider three people who are all friends with each other.
  Call them $A$, $B$, and $C$.
  Suppose $A$ is sick and $C$ is immune on day $1$.
  Then, $B$ is sick on day $2$, $C$ is sick on day $3$, $A$ is sick on day $4$, and so on.
  In this way, the epidemic continues indefinitely.
\end{enumerate}
%% --------------------------------------------------

\begin{problem}[$5\clubsuit$]
  Let $n \ge 3$ be an integer.
  Prove that if the edges of $K_n$ are colored with $n$ colors,
  with each color used at least once,
  then some triangle has its three edges of different colors.
\end{problem} \printpuid{H2024647}

%% Type your solution to \href{https://otis.evanchen.cc/arch/H2024647/}{H2024647} here ...
We use induction.
The base case $n=3$ is obvious.
Assuming the statement is true for $n$, we try to prove it for $n+1$.

Take $G \cong K_n$ inside $K_{n+1}$ and let $V$ be the vertex in $K_{n+1}$ but not in $G$.
Then, by the inductive hypothesis, $c \le n-1$ colors are used to color the edges of $G$.
The remaining edges in $K_{n+1}$ must be colored with $n+1-c \ge 2$ unused colors,
and they are the edges connecting the $n$ vertices in $G$ with $V$.

Then, let $VA$ and $VB$ be two differently colored edges,
both colored with a color not used to color the edges of $G$.
Then, $\triangle VAB$ is a triangle with three edges colored differently, so we are done.

%% --------------------------------------------------

\begin{problem}[Chartrand-Zhang 1.17, $3\clubsuit$]
  Let $G$ be a connected graph on finitely many vertices.
  Suppose that the longest path in $G$ has length $\ell$.
  Prove that any two paths of length $\ell$ share a vertex.
\end{problem} \printpuid{CZHGT117}

%% Type your solution to Chartrand-Zhang 1.17 (\href{https://otis.evanchen.cc/arch/CZHGT117/}{CZHGT117}) here ...
Suppose for contradiction that two paths of length $l$ do not share a vertex.
Let $a_1, a_2, \dots, a_{l+1}$ and $b_1, b_2, \dots, b_{l+1}$ denote the vertex sequences of two paths of length $l$ which do not share a vertex.

Since $G$ is connected, there must exist vertices $a_i$ and $b_j$ such that there exists a path from $a_i$ to $b_j$ which is disjoint from the two original paths.
(Proof sketch: consider a path from $a_1$ to $b_1$ and "remove" the two original paths from it.)
We claim that either the path from $a_1 \to a_i \to b_j \to b_1$ or the path from $a_{l+1} \to a_i \to b_j \to b_{l+1}$ must have length greater than $l$.

Let $m$ be the length of the path from $a_i$ to $b_j$ which is disjoint from the two original paths.
Then, the length of the path from $a_1 \to a_i \to b_j \to b_1$ is
\[ i + m + j, \]
and the length of the path from $a_{l+1} \to a_i \to b_j \to b_{l+1}$ is
\[ (l - i) + m + (l - j). \]
These two lengths add up to $2l + 2m > 2l$, so clearly, one of them exceeds $l$.
%% --------------------------------------------------

\begin{problem}[Chartrand-Zhang 3.16, $3\clubsuit$]
  Up to isomorphism, how many graphs on $9$ vertices have every vertex of degree $6$?
  (Two graphs are isomorphic if one can relabel the vertices of one to get the other.)
\end{problem} \printpuid{CZHGT316}

%% Type your solution to Chartrand-Zhang 3.16 (\href{https://otis.evanchen.cc/arch/CZHGT316/}{CZHGT316}) here ...

%% --------------------------------------------------

\begin{problem}[\href{https://aops.com/community/p119177}{Shortlist 2001 C3}, $3\clubsuit$]
  Let $G$ be a finite simple graph with no $K_5$,
  and for which any two triangles share at least one common vertex.
  Prove that one can delete two vertices from $G$
  and obtain a graph with no triangles.
\end{problem} \printpuid{01SLC3}

%% Type your solution to Shortlist 2001 C3 (\href{https://otis.evanchen.cc/arch/01SLC3/}{01SLC3}) here ...

%% --------------------------------------------------

\begin{reqproblem}[\href{https://aops.com/community/p23950776}{PUMaC Finals 2019 A1}, $5\clubsuit$]
  A finite graph $G$ is drawn on a blackboard.
  The following operation is permitted: pick any cycle $C$ of $G$, draw a new vertex $v$,
  connect it to all vertices of $C$, and finally erase all the edges of $C$.
  Prove that this operation can only be done a finite number of times.
\end{reqproblem} \printpuid{19PUMACFA1}

%% Type your solution to PUMaC Finals 2019 A1 (\href{https://otis.evanchen.cc/arch/19PUMACFA1/}{19PUMACFA1}) here ...
Consider $K$, a connected component of $G$, and let $v$ and $e$ be the number of vertices and edges in $K$ at any given point in time.
Note that initially, $e \geq v - 1$.

If $K$ has no cycles (i.e. $e = v-1$), we are done.
Otherwise, each move adds one vertex to $K$ while keeping the number of edges the same.
Also, $K$ remains connected after every move.
So, eventually, we will have $e = v-1$, and thus, $K$ will have no cycles and the process terminates.
Since each connected component of $G$ is independent, this concludes the proof.
%% --------------------------------------------------

\begin{problem}[\href{https://aops.com/community/p6696508}{Tuymaada 2016/8}, $5\clubsuit$]
  Let $G$ be a finite connected graph.
  Prove that one can color the vertices blue and green,
  and the edges black and white,
  such that the following three conditions are fulfilled:
  \begin{itemize}
  \ii Black edges join vertices of different colors.
  \ii White edges have at least one blue endpoint.
  \ii Any two vertices are connected by a black path.
  \end{itemize}
\end{problem} \printpuid{16TMD8}

%% Type your solution to Tuymaada 2016/8 (\href{https://otis.evanchen.cc/arch/16TMD8/}{16TMD8}) here ...

%% --------------------------------------------------

\begin{problem}[\href{https://aops.com/community/p10632393}{Shortlist 2017 C2}, $3\clubsuit$]
  Let $n$ be a positive integer.
  A \emph{chameleon} is a sequence of $3n$ letters
  with exactly $n$ occurrences of each of the letters $a$, $b$, $c$.
  Define a \emph{swap} to be the transposition of two adjacent
  letters in the chameleon.
  Prove that for any $X$ there exists a chameleon $Y$
  such that $X$ cannot be changed to $Y$ using fewer than $3n^2/2$ swaps.
\end{problem} \printpuid{17SLC2}

%% Type your solution to Shortlist 2017 C2 (\href{https://otis.evanchen.cc/arch/17SLC2/}{17SLC2}) here ...

%% --------------------------------------------------

\begin{problem}[\href{https://aops.com/community/p3120007}{Balkan 2013/4}, $9\clubsuit$]
  Let $G$ be a graph with the property that for any chordless cycle $C$ and
  any vertex $v$ not in $C$, at most one vertex of $C$ is adjacent to $v$.
  Prove that $G$ is three-colorable.

  (A \emph{chordless cycle} is a cycle where no two vertices are adjacent
  unless they are consecutive in the cycle.)
\end{problem} \printpuid{13BALK4}

%% Type your solution to Balkan 2013/4 (\href{https://otis.evanchen.cc/arch/13BALK4/}{13BALK4}) here ...

%% --------------------------------------------------

\end{document}
