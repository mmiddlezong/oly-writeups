\documentclass[11pt]{scrartcl}
\usepackage[usenames,dvipsnames,svgnames]{xcolor}
\usepackage[shortlabels]{enumitem}
\usepackage[framemethod=TikZ]{mdframed}
\usepackage{amsmath,amssymb,amsthm}
\usepackage{epigraph}
\usepackage[colorlinks]{hyperref}
\usepackage{microtype}
\usepackage{mathtools}
\usepackage[headsepline]{scrlayer-scrpage}
\usepackage{thmtools}
\usepackage{listings}
\usepackage{derivative}
\renewcommand{\epigraphsize}{\scriptsize}
\renewcommand{\epigraphwidth}{60ex}
\addtolength{\textheight}{3.14cm}
\ihead{\footnotesize\textbf{DCW-arml}}
\ohead{\footnotesize Updated Sun 22 Sep 2024 20:03:02 UTC}
\providecommand{\clubs}[1]{$#1\clubsuit$}
\providecommand{\clubg}[1]{\bgroup\color{green!40!black}[$#1\clubsuit$]\egroup}

\providecommand{\ol}{\overline}
\providecommand{\eps}{\varepsilon}
\providecommand{\half}{\frac{1}{2}}
\providecommand{\dang}{\measuredangle} %% Directed angle
\providecommand{\CC}{\mathbb C}
\providecommand{\FF}{\mathbb F}
\providecommand{\NN}{\mathbb N}
\providecommand{\QQ}{\mathbb Q}
\providecommand{\RR}{\mathbb R}
\providecommand{\ZZ}{\mathbb Z}
\providecommand{\dg}{^\circ}
\providecommand{\ii}{\item}
\providecommand{\alert}{\textbf}
\providecommand{\opname}{\operatorname}
\providecommand{\ts}{\textsuperscript}
% hacks for arc
\providecommand{\tarc}{\mbox{\large$\frown$}}
\providecommand{\arc}[1]{\stackrel{\tarc}{#1}}
\reversemarginpar
\providecommand{\printpuid}[1]{\marginpar{\href{https://otis.evanchen.cc/arch/#1}{\ttfamily\footnotesize\color{green!40!black}#1}}}

\mdfdefinestyle{mdbluebox}{roundcorner=10pt,innerbottommargin=9pt,
    linecolor=blue,backgroundcolor=TealBlue!5,}
\declaretheoremstyle[headfont=\sffamily\bfseries\color{MidnightBlue},
    mdframed={style=mdbluebox},]{thmbluebox}
\mdfdefinestyle{mdredbox}{frametitlefont=\bfseries,innerbottommargin=8pt,
    nobreak=true,backgroundcolor=Salmon!5,linecolor=RawSienna,}
\declaretheoremstyle[headfont=\bfseries\color{RawSienna},
    mdframed={style=mdredbox},headpunct={\\[3pt]},postheadspace=0pt,]{thmredbox}
\mdfdefinestyle{mdgreenbox}{linecolor=ForestGreen,backgroundcolor=ForestGreen!5,
    linewidth=2pt,rightline=false,leftline=true,topline=false,bottomline=false,}
\declaretheoremstyle[headfont=\bfseries\sffamily\color{ForestGreen!70!black},
    mdframed={style=mdgreenbox},headpunct={ --- },]{thmgreenbox}
\mdfdefinestyle{mdblackbox}{linecolor=black,backgroundcolor=RedViolet!5!gray!5,
    linewidth=3pt,rightline=false,leftline=true,topline=false,bottomline=false,}
\declaretheoremstyle[mdframed={style=mdblackbox}]{thmblackbox}
\declaretheorem[style=thmredbox,name=Problem]{problem}
\declaretheorem[style=thmredbox,name=Required Problem,sibling=problem]{reqproblem}
\declaretheorem[style=thmbluebox,name=Theorem,numberwithin=problem]{theorem}
\declaretheorem[style=thmbluebox,name=Lemma,sibling=theorem]{lemma}
\declaretheorem[style=thmbluebox,name=Theorem,numbered=no]{theorem*}
\declaretheorem[style=thmbluebox,name=Lemma,numbered=no]{lemma*}
\declaretheorem[style=thmgreenbox,name=Claim,sibling=theorem]{claim}
\declaretheorem[style=thmgreenbox,name=Claim,numbered=no]{claim*}
\declaretheorem[style=thmblackbox,name=Remark,sibling=theorem]{remark}
\declaretheorem[style=thmblackbox,name=Remark,numbered=no]{remark*}
\declaretheorem[style=thmgreenbox,name=Definition,sibling=theorem]{definition}
\declaretheorem[style=thmgreenbox,name=Definition,numbered=no]{definition*}
\declaretheorem[style=thmblackbox,name=Example,sibling=theorem]{example}
\declaretheorem[style=thmblackbox,name=Example,numbered=no]{example*}

\newenvironment{walkthrough}{\noindent\textbf{\color{green!40!black}Walkthrough.}}{}
\newlist{walk}{enumerate}{3}
\setlist[walk]{label=\bfseries (\alph*)}

\usepackage{asymptote}
\begin{asydef}
size(8cm); // set a reasonable default
usepackage("amsmath");
usepackage("amssymb");
settings.tex="pdflatex";
settings.outformat="pdf";
import geometry;
void filldraw(picture pic = currentpicture, conic g, pen fillpen=defaultpen, pen drawpen=defaultpen) { filldraw(pic, (path) g, fillpen, drawpen); }
void fill(picture pic = currentpicture, conic g, pen p=defaultpen) { filldraw(pic, (path) g, p); }
pair foot(pair P, pair A, pair B) { return foot(triangle(A,B,P).VC); }
pair centroid(pair A, pair B, pair C) { return (A+B+C)/3; }
\end{asydef}

\newcommand{\goals}[2]{\bgroup
\sffamily\small \emph{Instructions}: Solve \clubg{#1}.
If you have time, solve \clubg{#2}.\egroup\par}

%% 426c616e6b204c615465587e
\begin{document}
\title{Submission for DCW-ARML}
\subtitle{OTIS (internal use)}
\author{Michael Middlezong}
\date{\today}
\maketitle

\begin{example*}[\href{https://aops.com/community/p3041823}{USAMO 2013/2}, $0\clubsuit$]
  For a positive integer $n\geq 3$ plot $n$
  equally spaced points around a circle.
  Label one of them $A$, and place a marker at $A$.
  One may move the marker forward in a clockwise direction
  to either the next point or the point after that.
  Hence there are a total of $2n$ distinct moves available;
  two from each point.
  Let $a_n$ count the number of ways to advance
  around the circle exactly twice,
  beginning and ending at $A$, without repeating a move.
  Prove that $a_{n-1}+a_n=2^n$ for all $n\geq 4$.
\end{example*} \printpuid{13AMO2}

\begin{walkthrough}
Imagine the counter is moving along the set $S = \{0, 1, \dots, 2n\}$ instead,
starting at $0$ and ending at $2n$, in jumps of length $1$ and $2$.
We can then record the sequence of moves as a matrix of the form
\[
  \begin{bmatrix}
    p_0 & p_1 & p_2 & \dots & p_{n-1} & p_n \\
    p_n & p_{n+1} & p_{n+2} & \dots & p_{2n-1} & p_{2n}
  \end{bmatrix}
\]
where $p_i = 1$ if the point $i$ was visited by the counter,
and $p_i = 0$ if the point was not visited by the counter.
Note that $p_0 = p_{2n} = 1$ and the upper-right and lower-left entries are equal.

\begin{walk}
  \ii Show that we reduce to counting the number of matrices
  satisfying the above boundary conditions
  which avoid the continuous submatrices
  \[
    \begin{bmatrix} 0 & 0 \end{bmatrix}
    \qquad
    \begin{bmatrix} 0 \\ 0 \end{bmatrix}
    \qquad
    \begin{bmatrix} 1 & 1 \\ 1 & 1 \end{bmatrix}.
  \]
  This is really nice, because it sort of eliminates the circle
  and opens the path for induction.

  \ii We can make the setup further symmetric.
  Let
  \[
  \mathbf u \coloneq \begin{bmatrix} 1 \\ 0 \end{bmatrix}, \qquad
  \mathbf v \coloneq \begin{bmatrix} 0 \\ 1 \end{bmatrix}, \qquad
  \mathbf w \coloneq \begin{bmatrix} 1 \\ 1 \end{bmatrix}
  \]
  Show that we're trying to count sequences of $(n+1)$ such column vectors,
  no two adjacent which coincide,
  and which either start at $\mathbf u$ and end with $\mathbf v$,
  or start at $\mathbf w$ and end with $\mathbf w$.

  \ii Let $x_n$ and $y_n$ denote the number of $2 \times (n+1)$ matrices
  of the two types we just described.
  Use the symmetry to show the recursions
  \begin{align*}
    x_{n+1} &= x_n + y_n \\
    y_{n+1} &= 2x_n.
  \end{align*}

  \ii Prove that \[ 2x_n + y_n = 2^n. \]
  You may either do this by just solving the earlier recursions
  (with $x_1 = 1$ and $y_1 = 0$)
  or by giving a combinatorial interpretation of the left-hand side
  (this does not need the recursions from the previous part).

  \ii From $a_n = x_n + y_n$, conclude.
\end{walk}
\end{walkthrough}
%% You're not expected to write up walkthroughs (unless you really want to).
%% The source is just for your reference.

\begin{example*}[$0\clubsuit$]
  Let $G$ be a simple graph with $k$ connected components,
  which have $a_1$, \dots, $a_k$ vertices, respectively.
  Count the number of ways to add $k-1$ edges
  to $G$ to form a connected graph.
\end{example*} \printpuid{Z6F1A1FD}

\begin{walkthrough}
We let $g(a_1, \dots, a_k)$ denote the answer.
(Cayley's formula is then the assertion that $g(1, \dots, 1) = k^{k-2}$.)
\begin{walk}
  \ii Compute $g(a, b)$.
  \ii Show that $g(a, b, c) = abc(a+b+c)$.
  \ii Prove that
  \[ g(a, b, c, d) = \sum_{\text{sym}}
    \left[  \frac16 a^3bcd + \frac12 a^2b^2cd  \right].  \]
  Factor the resulting expression.
  \ii Come up with a conjecture
  for $g(a_1, \dots, a_k)$ (keeping in mind
  the desired answer $k^{k-2}$ for $g(1, \dots, 1)$).
\end{walk}
For the induction, it's actually easier to
do the casework if you assume the edges have some order
(hence multiplying by a factor of $(k-1)!$).
So let $f(a_1, \dots, a_k)$ be the number of ways
to add $k-1$ edges, \emph{in order}.
\begin{walk}[resume]
  \ii Prove considering where to place the first edge
  gives you the recursion
  \[ f(a_1, \dots, a_k) = \sum_{1 \le i < j \le k}
    a_i a_j f(a_i+a_j, \underbrace{a_1, \dots, a_k}
    _{\text{missing $a_i$ and $a_j$}}) \]
  \ii Now just power through the calculation
  to complete the induction.
\end{walk}
\end{walkthrough}
%% You're not expected to write up walkthroughs (unless you really want to).
%% The source is just for your reference.

\begin{example*}[$0\clubsuit$]
  Let $d_1$, \dots, $d_n$ be positive integers with $\sum_i d_i = 2n-2$.
  Find the number of labeled trees on $n$ vertices
  such that the degree of the $i$th vertex is $d_i$.
\end{example*} \printpuid{Z03DB838}

\begin{walkthrough}
I'll even tell you what the answer is this time:
\[ \frac{(n-2)!}{(d_1-1)! \dots (d_n-1)!} = \binom{n-2}{d_1-1, \dots, d_n-1}. \]
Prove it by induction, in the same way as the previous example.
\end{walkthrough}
%% You're not expected to write up walkthroughs (unless you really want to).
%% The source is just for your reference.

\begin{example*}[\href{https://w.wiki/8CGg}{Catalan}, $0\clubsuit$]
  Fix an integer $n \ge 1$.
  Show that the number of lattice paths from $(0,0)$ to $(n,n)$
  moving only right and up and staying below the line $y=x$
  is given exactly by the $n$\ts{th} Catalan number
  \[ \frac{1}{n+1} \binom{2n}{n}. \]
\end{example*} \printpuid{CATALAN}

\begin{walkthrough}
This walkthrough shows two approaches.
The first is a bijective approach using the so-called \emph{cyclic shift method}.

(To explain the name: a \emph{cyclic shift} refers of a word of length $n$
refers to the $n$ words obtained by moving the first $k$ letters to the end
for $k = 0, \dots, n-1$. For example, the cyclic shifts of $ABCD$
are $ABCD$, $BCDA$, $CDAB$, $DABC$.
The ``cyclic shift method'' means to consider the cyclic shifts.
There's no theorem it promises or anything;
it's up to you to decide whether the cyclic shifts are meaningful in some way
in the context of a specific problem.)

Consider instead paths from $(0,0)$ to $(n,n+1)$
which stay below the ``slightly raised'' diagonal joining $(0,0)$ to $(n,n+1)$.
Obviously there are the same number of each.
\begin{walk}
  \ii Show that these are in bijection with what we want to count.
\end{walk}
So every valid path can be interpreted as a sequence of length $2n+1$
consisting of $n$ right-moves and $n+1$ up-moves.
For example, for $n=10$ the path
\[ \texttt{RRURURURRUURUURRUURUU} \]
is shown below.
\begin{center}
\begin{asy}
  size(10cm);
  draw( (11,0)--(0,0)--(0,11), Arrows);
  for (int i=1; i<=10; ++i) {
    label("$"+(string)i+"$", (i,0), dir(-90));
    label("$"+(string)i+"$", (0,i), dir(180));
    dot( (i,0) );
    dot( (0,i) );
  }
  label("$(0,0)$", (0,0), dir(225));
  label("$\mathtt{RRURURURRUURUURRUURUU}$ ", (3.5,10), dir(90), deepgreen);
  draw((0,0)--(10,10), red);
  draw(
    (0,0)--(2,0)--(2,1)--(3,1)--(3,2)--(4,2)--(4,3)--(6,3)--
    (6,5)--(7,5)--(7,7)--(9,7)--(9,9)--(10,9)--(10,10),
    deepgreen+1.5
  );
  dot( (0,0) );
  draw((10,10)--(10,11), lightcyan+1.5);
  dot("$(n,n)$", (10,10), dir(0));
  dot("$(n,n+1)$", (10,11), dir(90), deepcyan);
  draw((0,0)--(10,11), deepcyan);
\end{asy}
\end{center}

\begin{walk}[resume]
  \ii Given a sequence of $n+1$ up's and $n$ right's,
  consider the $2n+1$ \emph{cyclic shifts} of this path.
  How many of them correspond to paths which stay below the slightly raised diagonal?

  \ii Extract the answer.
\end{walk}
For the recursive approach, let $C_n$ be the answer.
\begin{walk}[resume]
  \ii For each $k \ge 1$, show the number of paths from $(0,0)$ to $(k, k)$
  that stay \emph{strictly} below $y=x$ except at endpoints is $C_{k-1}$.

  \ii By casework on the smallest $k \ge 1$ such that the path touches $k$,
  derive a recursion of the form
  \[ C_n = \sum_{k = 1}^n C_{\text{?}} C_{\text{?}} \]
  valid for every positive integer $n$.

  \ii Define the generating function $F(X) = \sum_{n \ge 0} C_n X^n$.
  Use the previous part to derive the relation
  \[ XF(X)^2 + 1 = F(X). \]

  \ii Solve for $F(X)$ using the quadratic formula.

  \ii Use the extended binomial theorem to get an ugly expression for $C_n$.

  \ii Simplify the ugly expression to the desired one.
\end{walk}
\end{walkthrough}
%% You're not expected to write up walkthroughs (unless you really want to).
%% The source is just for your reference.

\newpage

% ========================================
\section*{Practice problems}
\goals{45}{56}

\epigraph{Every human has regrets, has things they'd like to go back and change.
But I don't! 'Cause I'm a bear!}
{Monokuma in \emph{Danganropa: Trigger Happy Havoc}}
%%fakesubsection{Warm-up}

\begin{problem}[\href{https://aops.com/community/p7710709}{HMMT 2017 C7}, $3\clubsuit$]
  There are $2017$ frogs and $2017$ toads in a room.
  Each frog is friends with exactly $2$ distinct toads.
  Let $N$ be the number of ways to pair every frog with a toad who is its friend,
  so that no toad is paired with more than one frog.
  Find all possible values of $N$.
\end{problem} \printpuid{17HMMTC7}

%% Type your solution to HMMT 2017 C7 (\href{https://otis.evanchen.cc/arch/17HMMTC7/}{17HMMTC7}), proposed by Yang Liu here ...
Form a bipartite graph with frogs on one side
and toads on the other side, with edges in between frogs and toads
if and only if they are friends.

Now, consider the connected components of this graph.
Clearly, pairings must be contained within each connected component,
so we can consider them separately and multiply together the number
of ways to arrange into pairs each connected component of frogs and toads.

If a component does not have the same number of frogs as toads,
then it is evidently impossible, and we have $N=0$.
This is achievable by, for example, having all frogs
be friends with two fixed toads.

Assuming each component has the same number of frogs as toads,
we proceed with induction on $f$, the number of frogs.
Our claim is that the number of ways to arrange each component is $2$.
The smallest possible component has $f=2$,
and in this case, there are clearly two ways to create a matching.

In the general case, suppose all the toads have exactly two frog friends.
Then, each vertex in the graph has degree $2$,
so the graph is just a cycle.
So, there are $2$ ways to pair off this graph.

However, if not all the toads have exactly two friends,
then some toads must have exactly one friend.
These toads can only be paired in one way, so
we remove them from consideration.
The remaining graph is connected and still satisfies the conditions of the problem,
giving us $2$ ways to pair it off.

Thus, if $N$ is nonzero, then we must have $N = 2^k$ where $k$ is the number of
connected components.
It is not hard to come up with a construction for $k = 1,2,\dots, 1008$,
which is all that we can possibly do, since each connected component
must have at least $2$ frogs and $2$ toads.
So, the answer is $\boxed{N = 0,2,2^2,2^3, \dots, 2^{1008}}$.

%% --------------------------------------------------

\begin{problem}[\href{https://aops.com/community/p26536193}{AMC 12B 2022/17}, $2\clubsuit$]
  How many $6 \times 6$ matrices with digits $0$ or $1$
  have the property that the row sums are $\{1,2,\dots,6\}$ in some order,
  as are the column sums?
\end{problem} \printpuid{2212B17}

%% Type your solution to AMC 12B 2022/17 (\href{https://otis.evanchen.cc/arch/2212B17/}{2212B17}) here ...
By permuting the rows and/or columns of a matrix that works,
we can get every possible permutation of the row sums and column sums in a working matrix.
Also, any permutation of the row sums and column sums leads to only one
working matrix.
So, we have formed a bijection and our answer is
\[ (6!)^2 = \boxed{518400}. \]
%% --------------------------------------------------

\begin{problem}[\href{https://aops.com/community/p213012}{USAMO 2005/4}, $2\clubsuit$]
  Legs $L_1$, $L_2$, $L_3$, $L_4$ of a square table each have length $n$,
  where $n$ is a positive integer.
  For how many ordered $4$-tuples $(k_1, k_2, k_3, k_4)$ of nonnegative integers
  can we cut a piece of length $k_i$ from the end of leg $L_i$
  and still have a stable table?

  (The table is \emph{stable} if it can be placed
  so that all four of the leg ends touch the floor.
  Note that a cut leg of length $0$ is permitted.)
\end{problem} \printpuid{05AMO4}

%% Type your solution to USAMO 2005/4 (\href{https://otis.evanchen.cc/arch/05AMO4/}{05AMO4}), proposed by Elgin Johnston here ...
Note that stability is equivalent to the condition $k_1 + k_3 = k_2 + k_4$.
Doing casework on this common sum, we end up with
\begin{align*}
    1^2 + 2^2 + \cdots + n^2 + (n+1)^2 + n^2 + \cdots + 2^2 + 1^2 &= \frac{n(n+1)(2n+1) + (n+1)(n+2)(2n+3)}{6} \\
    &= \boxed{\frac{(n+1)(2n^2+4n+3)}{3}}.
\end{align*}
%% --------------------------------------------------

%%fakesubsection{Permutation problems, in the style of Richard Stanley}

\begin{problem}[Enumerative Combinatorics Volume 1, Exercise 1.40, $2\clubsuit$]
  Fix a positive integer $n \ge 3$ and consider a regular $n$-gon.
  Find, in terms of $n$, the number of subsets of the $n$ vertices
  for which no two vertices are adjacent.
\end{problem} \printpuid{ECV1140}

%% Type your solution to Enumerative Combinatorics Volume 1, Exercise 1.40 (\href{https://otis.evanchen.cc/arch/ECV1140/}{ECV1140}) here ...
Let $l_n$ be the number of subsets of $n$ things arranged in a line
such that no two of them are adjacent.
Then, by casework on whether the first thing is in the subset,
we get $l_n = l_{n-1} + l_{n-2}$.
If we define the Fibonacci numbers as $F_0 = 0$,
$F_1 = 1$, and $F_n = F_{n-1} + F_{n-2}$ for $n \ge 2$,
we find (after doing base cases) that $l_n = F_{n+2}$ for all $n$.

Now, we return to the original problem.
Let $c_n$ be the quantity we are looking for.
By casework on whether one fixed vertex is in the subset,
we get $c_n = l_{n-3} + l_{n-1} = \boxed{F_{n-1} + F_{n+1}}$.
%% --------------------------------------------------

\begin{problem}[Enumerative Combinatorics Volume 1, Exercise 1.120, $3\clubsuit$]
  Let $n \ge k \ge 1$ be integers. In a randomly selected permutation
  of $\{1,\dots,n\}$ (among all $n!$ permutations),
  find the expected value of the number of cycles of length $k$.
\end{problem} \printpuid{ECV11120}

%% Type your solution to Enumerative Combinatorics Volume 1, Exercise 1.120 (\href{https://otis.evanchen.cc/arch/ECV11120/}{ECV11120}) here ...
We will count the total number of cycles of length $k$
when summed across all $n!$ permutations and then divide by $n!$.

Consider any fixed cycle of length $k$ (independent of any specific permutation).
The number of permutations with this cycle is $(n-k)!$, because
$k$ numbers are fixed and the rest can do whatever.
The total number of cycles of length $k$
is $\binom{n}{k} (k-1)!$, since we choose the stuff to be in the cycle
and then arrange the cycle itself.
So, our answer is
\[ \frac{\dbinom{n}{k} (k-1)! (n-k)!}{n!} = \boxed{\frac 1k}.\]
%% --------------------------------------------------

\begin{reqproblem}[\href{https://math.stackexchange.com/q/155867/229197}{$321$-avoiding permutations}, $9\clubsuit$]
  There are $n \ge 3$ penguins of different heights.
  How many ways are there to order the penguins in a line, left to right,
  so that we cannot find any three (not necessarily consecutive)
  that are arranged tallest to shortest (in left to right order)?

  For $n=6$, one example of such a placement is shown below.
  \begin{center}
  \begin{asy}
  size(12cm);
  picture body;
  pen border = black+1.8;
  path outline = (0,0)..(0.5,-0.1)..(0.9,0)..(0.9,0)..(1,0.7)
  ..(0.85,1.3)..(0.4,1.32)..(0.13,1.1)..(0.12,1.08)--(0.12,1.08)..(0.03,0.4)
  ..(0,0.1)--(-0.1,0.02)--(0,0)..cycle;
  filldraw(body, (0.8,0.7)--(1.15,0.35)--(0.85,0.4)--cycle, rgb("#90a0b0"), border);
  filldraw(body, outline, rgb("#90a0b0"), border);
  filldraw(body, subpath(outline, 0.2, 6.6)
  --(0.25,0.9)..(0.25,0.7)--(0.44,0.52)--(0.27,0.45)
  ..(0.2,0.22)..(0.15,0.1)..cycle,
  rgb("#f4f4f4"), border);
  filldraw(body, ellipse((0.32,-0.075), 0.12, 0.07), orange, border);
  filldraw(body, ellipse((0.72,-0.07), 0.12, 0.07), orange, border);
  draw(body, (0.51,1.07)..(0.59,1.09)..(0.66,1.07), black+2);
  draw(body, (0.78,1.07)..(0.86,1.09)..(0.94,1.07), black+2);
  filldraw(body, (0.7,0.98)--(0.89,0.92)--(0.68,0.89)--cycle, yellow, border);
  real r = 1.4;
  add(shift(  0,0)*yscale(1.0)*body);
  add(shift(1.2,0)*yscale(0.7)*xscale(0.8)*body);
  add(shift(2.2,0)*yscale(1.3)*xscale(1.1)*body);
  add(shift(3.5,0)*yscale(1.5)*xscale(1.2)*body);
  add(shift(4.8,0)*yscale(0.85)*xscale(0.9)*body);
  add(shift(5.8,0)*yscale(1.1)*body);
  real y = -0.5;
  label("$3$", (0.5,y), fontsize(19pt));
  label("$1$", (1.5,y), fontsize(19pt));
  label("$5$", (2.7,y), fontsize(19pt));
  label("$6$", (4.1,y), fontsize(19pt));
  label("$2$", (5.3,y), fontsize(19pt));
  label("$4$", (6.3,y), fontsize(19pt));
  \end{asy}
  \end{center}
\end{reqproblem} \printpuid{321AVOID}

%% Type your solution to $321$-avoiding permutations (\href{https://otis.evanchen.cc/arch/321AVOID/}{321AVOID}) here ...

%% --------------------------------------------------

\begin{problem}[$9\clubsuit$]
  Let $n$ and $r$ be fixed positive integers.
  A deck of $n$ cards is numbered $1$, \dots, $n$ in that order with $1$ on top.
  A \emph{top-to-random shuffle} takes the top card of the deck
  and inserts it randomly among in among the remaining $n-1$ cards,
  with each of the $n$ possible insertions being equally likely.

  Prove that if we apply $r$ top-to-random shuffles, then the probability
  the resulting permutation of the deck has exactly one cycle is $\frac 1n$.
\end{problem} \printpuid{Z629E88F}

%% Type your solution to \href{https://otis.evanchen.cc/arch/Z629E88F/}{Z629E88F} here ...

%% --------------------------------------------------

%%fakesubsection{Contest practice}

\begin{reqproblem}[ARML 2024 I-6, $2\clubsuit$]
  In the left figure below, the cell in the fourth row and third column
  of a $9 \times 9$ grid is deleted to obtain a set $S$ of $80$ cells.
  For each $n = 1, \dots, 8$, an L-shaped tile $L_n$
  consisting of $2n+1$ cells is given, as shown on the right figure.
  \begin{center}
  \begin{asy}
  size(15cm);
  picture Ln(int n) {
  picture p;
  for (int i=0; i<=n; ++i) {
  filldraw(p, shift(i,0)*unitsquare, lightgrey, black);
  filldraw(p, shift(0,i)*unitsquare, lightgrey, black);
  }
  label(p, "$L_{" + ((string) n) + "}$", ((n+1)/2,0), dir(-90));
  draw(p, (0,0)--(0,n+1)--(1,n+1)--(1,1)--(n+1,1)--(n+1,0)--cycle, black+1.4);
  return p;
  }
  add(shift(0,11)*Ln(1));
  add(shift(3,11)*Ln(2));
  add(shift(7,11)*Ln(3));
  add(shift(12,11)*Ln(4));
  add(shift(18,11)*Ln(5));

  add(shift(0,0)*Ln(6));
  add(shift(8,0)*Ln(7));
  add(shift(17,0)*Ln(8));

  int xs = -14;
  int ys = 4;
  for (int x = xs; x <= xs+9; ++x) {
  draw((x,ys)--(x,ys+9), grey);
  }

  for (int y = ys; y <= ys+9; ++y) {
  draw((xs,y)--(xs+9,y), grey);
  }

  draw(box((xs,ys),(xs+9,ys+9)), black+1.5);
  fill(shift(xs+2,ys+5)*unitsquare, black);
  label("$S$", (xs+4.5,ys), dir(-90));
  \end{asy}
  \end{center}
  Compute the number of ways to tile $S$ using these eight L-shaped tiles,
  each exactly once, with rotations of tiles allowed.
\end{reqproblem} \printpuid{24ARMLI6}

%% Type your solution to ARML 2024 I-6 (\href{https://otis.evanchen.cc/arch/24ARMLI6/}{24ARMLI6}), proposed by Andy Neidermairer here ...
Note that given a fixed rotation of each tile,
there is only one way to place them.

Each tile can be rotated four ways.
In two of those ways, the corner of the tile is on the left,
and in the other two, the corner is on the right.
Based on the location of the deleted square,
exactly two of the placed tiles must have the corner on the left,
while the other six must have the corner on the right.

Similarly, three tiles must have the corner on the top,
while the other five must have the corner on the bottom.

There are $\binom82$ ways to arrange the right-leftness of the tiles,
and $\binom83$ ways to arrange the top-bottomness,
giving us a total of
\[ \dbinom82 \dbinom83 = \boxed{1568}. \]
%% --------------------------------------------------

\begin{problem}[\href{https://aops.com/community/p27102000}{AIME II 2023/10}, $5\clubsuit$]
  Determine the number of ways to place the integers $1$ through $12$
  in the $12$ cells of a $2 \times 6$ grid so that for any two cells sharing a side,
  the difference between the numbers in those cells is not divisible by $3$.
\end{problem} \printpuid{23AIMEII10}

%% Type your solution to AIME II 2023/10 (\href{https://otis.evanchen.cc/arch/23AIMEII10/}{23AIMEII10}) here ...
We reduce mod $3$, such that we need to place four $0$'s, four $1$'s, and four $2$'s.
We just have to remember to multiply by $(4!)^3$ at the end.

Notice that each column in this reduced configuration
must have two distinct numbers in it.
Furthermore, there are three types of columns which must each appear twice:
$\{0,1\}$ columns, $\{1,2\}$ columns, and $\{0,2\}$ columns.

There are $\frac{6!}{2!2!2!}$ ways to arrange the types of columns.
For a given arrangement of types of columns, each column has two choices
for how to arrange the two numbers within the column.
However, given one column's arrangement, there is only one way to arrange the
others satisfying the conditions.
So, for each arrangement of types of columns, there are $2$ possible arrangements.

This gives us a final count of
\[ \frac{6!}{2!2!2!} \cdot 2 \cdot (4!)^3 = 2^{11}3^55^1. \]
So, the answer is $12 \cdot 6 \cdot 2 = \boxed{144}$.
%% --------------------------------------------------

\begin{problem}[\href{https://aops.com/community/p22246365}{ARML 2021 I-8}, $3\clubsuit$]
  Of the $10^{10}$ functions $g$ from $\{1,2,\dots,10\}$ to itself,
  how many satisfy \[ g^{g(n)}(n) = n \] for all $1 \le n \le 10$?
  Here $g^k$ means $g$ applied $k$ times.
\end{problem} \printpuid{21ARMLI8}

%% Type your solution to ARML 2021 I-8 (\href{https://otis.evanchen.cc/arch/21ARMLI8/}{21ARMLI8}), proposed by Andy Neidermaier here ...
The condition implies that $g$ is surjective,
so $g$ is a permutation.
A permutation can be broken up into disjoint cycles, which we will consider.

Fix any $m$, and consider $x$ such that $g(x) = m$.
If $x = m$, then $m$ is in its own cycle of length $1$.
Otherwise, $g^{g(x)}(x) = x$, or $g^m(x) = x$.
So, the length of the cycle containing $x$, which is also the cycle containing $m$,
must divide $m$. Since $1$ divides $m$, we can say in general that
the length of the cycle containing $m$ divides $m$.

By extension, the length of any cycle must divide the GCD of its elements.
One can see that this is a sufficient condition.
Therefore, there can't be any cycles of length greater than $3$, because there aren't enough
multiples of higher lengths.

We can also glean other restrictions.
In general, $g(1) = 1$, $g(5) = 5$, and $g(7) = 7$.

Now, we do casework on whether there is a $3$-cycle.
If there is one, its elements must be $\{3,6,9\}$, and there are $2$ ways to order
this cycle.
The only elements remaining to decide are $2$, $4$, $8$, and $10$.
They can optionally enter into $2$-cycles, since they are all even.
Doing casework on the number of $2$-cycles formed, we get $1 + 6 + 3 = 10$ cases.
So, in total, we have $10 \cdot 2 = 20$ cases.

If there is no $3$ cycle,
then we must have $g(3) = 3$ and $g(9) = 9$.
The elements remaining are the even numbers,
and doing casework on the number of $2$-cycles formed,
we get $1 + 10 + 15 = 26$ cases.

In total, we get an answer of $20 + 26 = \boxed{46}$.

%% --------------------------------------------------

\begin{reqproblem}[\href{https://aops.com/community/p22245925}{ARML 2021 T-9}, $5\clubsuit$]
  Nine distinct circles are drawn in the plane.
  Of the $\binom92=36$ pairs of circles,
  at least half of the pairs of circles do not intersect.
  The circles cut the plane into several regions;
  $N$ of these regions have finite area.
  Compute the maximum possible value of $N$.
\end{reqproblem} \printpuid{21ARMLT9}

%% Type your solution to ARML 2021 T-9 (\href{https://otis.evanchen.cc/arch/21ARMLT9/}{21ARMLT9}), proposed by Evan Chen here ...
Note that there is no reason to have two circles be tangent or have three circles
intersect at one point, since by expanding one circle, we can increase the number of regions
without violating any conditions.

Consider the ``connected components'' formed by the circles.
Assuming a component has at least one intersection point,
consider the planar graph formed with vertices at intersection points
and edges as the arcs connecting vertices.
This connected planar graph satisfies
\[ V - E + F = 2, \]
where $F = N + 1$.
So, we can write $N = 1 + E - V$.

Since every vertex must have degree $4$,
we have
\[ 2E = \sum_{v \in V} \deg(v) = 4V \implies E = 2V. \]
So, $N = 1 + V$.
This seals the deal for components with at least one vertex,
but if there are no vertices, there must only be one circle.
This circle will create one additional region of area no matter where it is,
so it also satisfies $N = 1+V$.

Summing across all connected components, we get
\[ N = (\text{\# of components}) + (\text{\# of intersection points}). \]

Now, assume there are $4$ (or more) components.
Then, note that the maximum number of intersection points in a component with $k$ circles
is $k(k-1)$, a convex function.
So, the maximum number of intersection points with $4$ components is
\[ 6(6-1) + 1(1-0) + 1(1-0) + 1(1-0) = 30. \]
With more components, this maximum only decreases.
There can be a maximum of $9$ components, giving us $N \le 30 + 9 = 39$.

If there are $3$ or fewer components,
since the maximum number of intersection points is $36$,
we have $N \le 39$.

Now, $N=\boxed{39}$ is achievable with one component of $7$ circles
and two components of $1$ circle each.
Importantly, the last circle placed in the component of $7$ circles can be adjusted
to get exactly $36$ intersection points.
%% --------------------------------------------------

\begin{problem}[\href{https://aops.com/community/p11796049}{HMMT 2019 C9, added by Neil Kolekar}, $5\clubsuit$]
  How many ways can you fill a $3 \times 3$ square grid
  with nonnegative integers such that no nonzero integer appears more than once
  in the same row or column and the sum of the numbers in every row and column equals $7$?
\end{problem} \printpuid{19HMMTC9}

%% Type your solution to HMMT 2019 C9, added by Neil Kolekar (\href{https://otis.evanchen.cc/arch/19HMMTC9/}{19HMMTC9}), proposed by Sam Korsky here ...

%% --------------------------------------------------

\begin{problem}[\href{https://aops.com/community/p26535320}{AMC 10B 2022/18}, $3\clubsuit$]
  Find the number of ordered $9$-tuples
  \[ (a_1, b_1, c_1, a_2, b_2, c_2, a_3, b_3, c_3) \in \{0,1\}^9 \]
  such that the system of three linear equations
  \begin{align*}
  a_1 x + b_1 y + c_1 z = 0 \\
  a_2 x + b_2 y + c_2 z = 0 \\
  a_3 x + b_3 y + c_3 z = 0
  \end{align*}
  with unknowns $x$, $y$, and $z$ has a solution other than $x = y = z = 0$.
\end{problem} \printpuid{2210B18}

%% Type your solution to AMC 10B 2022/18 (\href{https://otis.evanchen.cc/arch/2210B18/}{2210B18}) here ...

%% --------------------------------------------------

\begin{problem}[\href{https://aops.com/community/p1191679}{IMO 2008/5}, $2\clubsuit$]
  Let $n$ and $k$ be positive integers with $k \geq n$ and $k - n$ an even number.
  There are $2n$ lamps labelled $1$, $2$, \dots, $2n$ each of which can be either on or off.
  Initially all the lamps are off.
  We consider sequences of steps: at each step one of the lamps is switched
  (from on to off or from off to on).
  Let $N$ be the number of such sequences consisting of $k$ steps
  and resulting in the state where lamps $1$ through $n$ are all on,
  and lamps $n + 1$ through $2n$ are all off.
  Let $M$ be number of such sequences consisting of $k$ steps,
  resulting in the state where lamps $1$ through $n$ are all on,
  and lamps $n + 1$ through $2n$ are all off,
  but where none of the lamps $n + 1$ through $2n$ is ever switched on.
  Determine $\frac{N}{M}$.
\end{problem} \printpuid{08IMO5}

%% Type your solution to IMO 2008/5 (\href{https://otis.evanchen.cc/arch/08IMO5/}{08IMO5}), proposed by Bruno Le Floch and Ilia Smilga (FRA) here ...
For a sequence, let $a$ be the number of moves on lamps with labels between $n+1$ and $2n$, inclusive.
Note that $a$ must be even.



%% --------------------------------------------------

\begin{problem}[\href{https://aops.com/community/p356696}{IMO 1997/1}, $3\clubsuit$]
  In the plane there is an infinite chessboard.
  For any pair of positive integers $m$ and $n$,
  consider a right-angled triangle with vertices at lattice points
  and whose legs, of lengths $m$ and $n$, lie along edges of the squares.
  Let $S_1$ be the total area of the black part of the triangle
  and $S_2$ be the total area of the white part.
  Let $f(m,n) = | S_1 - S_2 |$.

  \begin{enumerate}[(a)]
  \ii Calculate $f(m,n)$ for all positive integers $m$ and $n$
  which are either both even or both odd.
  \ii Prove that $f(m,n) \leq \frac 12 \max \{m,n\}$ for all $m$ and $n$.
  \ii Show that there is no constant $C$
  such that $f(m,n) < C$ for all $ m$ and $ n$.
  \end{enumerate}
\end{problem} \printpuid{97IMO1}

%% Type your solution to IMO 1997/1 (\href{https://otis.evanchen.cc/arch/97IMO1/}{97IMO1}) here ...

%% --------------------------------------------------

\begin{problem}[\href{https://aops.com/community/p27071286}{The AmongUs mock AIME}, $3\clubsuit$]
  Find the number of functions $f \colon \ZZ \to \ZZ$ such that:
  \begin{itemize}
  \ii $f(x-14f(x)+1) = f(x)$ for all integers $x$;
  \ii the smallest root of $f$ is $35$;
  \ii $f(x) \le 14$ for all $x$.
  \end{itemize}
\end{problem} \printpuid{Z3FDA038}

%% Type your solution to The AmongUs mock AIME (\href{https://otis.evanchen.cc/arch/Z3FDA038/}{Z3FDA038}) here ...
One can obtain $f(35) = f(36) = f(37) = \cdots = 0$.
Also, if $f(x) = k$, then $f(x-dn) = k$ for all $n$ where $d = 14k - 1$.
As a result, $f(x)$ must be positive for all $x < 35$.

% TODO FIINISH

Writing out the possible values of $d$, we see that the shared prime factors are $3$, $5$, and $13$.

%% --------------------------------------------------

\begin{problem}[\href{https://aops.com/community/p29651028}{OTIS Mock AIME 2024/7, by Atticus Stewart}, $3\clubsuit$]
  Compute the number of $9$-tuples $(a_0, a_1, \dots, a_8)$ of integers
  such that $a_i \in \{-1, 0, 1\}$ for $i = 0, 1, \dots, 8$ and the polynomial
  \[ a_8x^8 + a_7x^7 + \dots + a_1x + a_0 \]
  is divisible by $x^2 + x + 1$.
\end{problem} \printpuid{24OIME7}

%% Type your solution to OTIS Mock AIME 2024/7, by Atticus Stewart (\href{https://otis.evanchen.cc/arch/24OIME7/}{24OIME7}), proposed by Atticus Stewart here ...
We want each root of $x^2 + x + 1$ to also be a root of $a_8x^8 + a_7x^7 + \dots + a_1x + a_0$.
Note that if $x$ is a root of $x^2 + x + 1$, then $x^3 = 1$.
So, we can say
\[ a_8x^8 + a_7x^7 + \dots + a_1x + a_0 = (a_8 + a_5 + a_2)x^2 + (a_7 + a_4 + a_1)x + (a_6 + a_3 + a_0). \]
This polynomial is divisible by $x^2 + x + 1$ only if its coefficients are all equal.
Doing casework on this common coefficient,
we get an answer of $7^3 + 6^3 + 6^3 + 3^3 + 3^3 + 1^3 + 1^3 = \boxed{831}$.
%% --------------------------------------------------

\begin{reqproblem}[\href{https://aops.com/community/p29651088}{OTIS Mock AIME 2024/12, by Ethan Lee}, $5\clubsuit$]
  Let $\mathcal G_n$ denote a triangular grid of side length $n$
  consisting of $\frac{(n+1)(n+2)}{2}$ pegs.
  Charles the Otter wishes to place some rubber bands along the pegs of $\mathcal G_n$ such that
  every edge of the grid is covered by exactly one rubber band
  (and no rubber band traverses an edge twice).
  He considers two placements to be different if the sets of edges
  covered by the rubber bands are different
  or if any rubber band traverses its edges in a different order.
  The ordering of which bands are over and under does not matter.

  For example, Charles finds there are exactly $10$ different ways to cover
  $\mathcal G_2$ using exactly two rubber bands; the full list is shown below,
  with one rubber band in orange and the other in blue.
  \begin{center}
  \begin{asy}
  import roundedpath;
  size(12cm);
  pair[][] points;
  transform sh = shift(-1, -sqrt(3)/3);
  path t = sh*((0,0) -- (1,0) -- (1/2, sqrt(3)/2) -- cycle);
  picture base;
  for (int i = 0; i < 3; ++i) {
  points.push(new pair[]);
  for (int j = 0; j < 3 - i; ++j) {
  points[i].push(sh*(j + i/2, i * sqrt(3)/2));
  }
  for (int j = 0; j < 2 - i; ++j) {
  draw(base, shift((j + i/2, i * sqrt(3)/2)) * t, linewidth(10) + grey);
  }
  }

  void band(picture pic, int[][] p, pen c) {
  int n = p.length;
  for (int i = 0; i < n; ++i) {
  pair A = points[p[i][0]][p[i][1]], B = points[p[(i + 1) % n][0]][p[(i + 1) % n][1]], C =  points[p[(i + 2) % n][0]][p[(i + 2) % n][1]];
  draw(pic, roundedpath((A + B) / 2 -- B -- (B + C) / 2, 0.15), white + linewidth(8));
  draw(pic, roundedpath((A * 2 + B) / 3 -- B -- (B + C * 2) / 3, 0.15), c + linewidth(4));
  }
  }

  picture case1;
  add(case1, base);
  band(case1, new int[][]{{0,0},{0,2},{2,0}}, orange);
  band(case1, new int[][]{{1,0},{1,1},{0,1}}, blue);
  add(case1);

  picture case2;
  add(case2, base);
  band(case2, new int[][]{{1,0},{1,1},{2,0}}, orange);
  band(case2, new int[][]{{0,0},{0,1},{1,1},{0,2},{0,1},{1,0}}, blue);
  add(shift(2.5,0)*case2);
  add(shift(5.0,0)*rotate(120)*case2);
  add(shift(7.5,0)*rotate(240)*case2);

  picture case3;
  add(case3, base);
  band(case3, new int[][]{{1,0},{1,1},{2,0}}, orange);
  band(case3, new int[][]{{0,0},{1,0},{0,1},{1,1},{0,2},{0,1}}, blue);
  add(shift(10.0,0)*case3);
  add(shift(0,-2.5)*rotate(120)*case3);
  add(shift(2.5,-2.5)*rotate(240)*case3);

  picture case4;
  add(case4, base);
  band(case4, new int[][]{{0,2},{0,1},{1,0},{2,0}}, orange);
  band(case4, new int[][]{{0,0},{0,1},{1,1},{1,0}}, blue);
  add(shift(5.0,-2.5)*case4);
  add(shift(7.5,-2.5)*rotate(120)*case4);
  add(shift(10.0,-2.5)*rotate(240)*case4);
  \end{asy}
  \end{center}

  Let $N$ denote the total number of ways to cover $\mathcal G_4$
  with \emph{any number} of rubber bands.
  Compute the remainder when $N$ is divided by $1000$.
\end{reqproblem} \printpuid{24OIME12}

%% Type your solution to OTIS Mock AIME 2024/12, by Ethan Lee (\href{https://otis.evanchen.cc/arch/24OIME12/}{24OIME12}), proposed by Ethan Lee here ...

%% --------------------------------------------------

\begin{problem}[\href{https://www.math.ucdavis.edu/~gravner/MAT135A/resources/chpr.pdf}{P.\ Winkler, added by Krishna Pothapragada}, $5\clubsuit$]
  Let $n \ge 2$ be a fixed positive integer.
  Create $n$ random and independently generated intervals in $[0,1]$,
  where each interval is created by choosing $2$ points
  uniformly and independently to be the endpoints.
  Find the probability that there is some interval
  which intersects all of the other intervals.
\end{problem} \printpuid{PWINKLER}

%% Type your solution to P.\ Winkler, added by Krishna Pothapragada (\href{https://otis.evanchen.cc/arch/PWINKLER/}{PWINKLER}) here ...

%% --------------------------------------------------

\begin{reqproblem}[\href{https://aops.com/community/p26495778}{AMC 12A 2022/24}, $9\clubsuit$]
  Find the number of strings of length $10$ with digits $0$ through $9$
  such that for every $j \ge 1$ there are at least $j$
  digits which are strictly less than $j$.
\end{reqproblem} \printpuid{2212A24}

%% Type your solution to AMC 12A 2022/24 (\href{https://otis.evanchen.cc/arch/2212A24/}{2212A24}) here ...

%% --------------------------------------------------

\begin{problem}[\href{https://aops.com/community/p29651079}{OTIS Mock AIME 2024/11, by Neil Kolekar}, $9\clubsuit$]
  Compute the number of ordered triples of positive integers $(a,b,n)$
  satisfying $\max(a,b) \leq \min(\sqrt n, 60)$ and
  \[ \arcsin\left( \frac{a}{\sqrt n} \right)
  + \arcsin\left( \frac{b}{\sqrt n} \right) = \frac{2\pi}{3}. \]
\end{problem} \printpuid{24OIME11}

%% Type your solution to OTIS Mock AIME 2024/11, by Neil Kolekar (\href{https://otis.evanchen.cc/arch/24OIME11/}{24OIME11}), proposed by Neil Kolekar here ...

%% --------------------------------------------------

\begin{remark*}
  The answer to the arcsin problem is supposed to be less than $1000$.
  If you get an answer larger than that, it means you made a mistake.
\end{remark*}
\end{document}
