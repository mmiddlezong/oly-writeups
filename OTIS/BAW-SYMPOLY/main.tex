\documentclass[11pt]{scrartcl}
\usepackage[usenames,dvipsnames,svgnames]{xcolor}
\usepackage[shortlabels]{enumitem}
\usepackage[framemethod=TikZ]{mdframed}
\usepackage{amsmath,amssymb,amsthm}
\usepackage{epigraph}
\usepackage[colorlinks]{hyperref}
\usepackage{microtype}
\usepackage{mathtools}
\usepackage[headsepline]{scrlayer-scrpage}
\usepackage{thmtools}
\usepackage{listings}
\usepackage{derivative}
\renewcommand{\epigraphsize}{\scriptsize}
\renewcommand{\epigraphwidth}{60ex}
\addtolength{\textheight}{3.14cm}
\ihead{\footnotesize\textbf{BAW-sympoly}}
\ohead{\footnotesize Updated Sun 15 Sep 2024 18:35:37 UTC}
\providecommand{\clubs}[1]{$#1\clubsuit$}
\providecommand{\clubg}[1]{\bgroup\color{green!40!black}[$#1\clubsuit$]\egroup}

\providecommand{\ol}{\overline}
\providecommand{\eps}{\varepsilon}
\providecommand{\half}{\frac{1}{2}}
\providecommand{\dang}{\measuredangle} %% Directed angle
\providecommand{\CC}{\mathbb C}
\providecommand{\FF}{\mathbb F}
\providecommand{\NN}{\mathbb N}
\providecommand{\QQ}{\mathbb Q}
\providecommand{\RR}{\mathbb R}
\providecommand{\ZZ}{\mathbb Z}
\providecommand{\dg}{^\circ}
\providecommand{\ii}{\item}
\providecommand{\alert}{\textbf}
\providecommand{\opname}{\operatorname}
\providecommand{\ts}{\textsuperscript}
% hacks for arc
\providecommand{\tarc}{\mbox{\large$\frown$}}
\providecommand{\arc}[1]{\stackrel{\tarc}{#1}}
\reversemarginpar
\providecommand{\printpuid}[1]{\marginpar{\href{https://otis.evanchen.cc/arch/#1}{\ttfamily\footnotesize\color{green!40!black}#1}}}

\mdfdefinestyle{mdbluebox}{roundcorner=10pt,innerbottommargin=9pt,
    linecolor=blue,backgroundcolor=TealBlue!5,}
\declaretheoremstyle[headfont=\sffamily\bfseries\color{MidnightBlue},
    mdframed={style=mdbluebox},]{thmbluebox}
\mdfdefinestyle{mdredbox}{frametitlefont=\bfseries,innerbottommargin=8pt,
    nobreak=true,backgroundcolor=Salmon!5,linecolor=RawSienna,}
\declaretheoremstyle[headfont=\bfseries\color{RawSienna},
    mdframed={style=mdredbox},headpunct={\\[3pt]},postheadspace=0pt,]{thmredbox}
\mdfdefinestyle{mdgreenbox}{linecolor=ForestGreen,backgroundcolor=ForestGreen!5,
    linewidth=2pt,rightline=false,leftline=true,topline=false,bottomline=false,}
\declaretheoremstyle[headfont=\bfseries\sffamily\color{ForestGreen!70!black},
    mdframed={style=mdgreenbox},headpunct={ --- },]{thmgreenbox}
\mdfdefinestyle{mdblackbox}{linecolor=black,backgroundcolor=RedViolet!5!gray!5,
    linewidth=3pt,rightline=false,leftline=true,topline=false,bottomline=false,}
\declaretheoremstyle[mdframed={style=mdblackbox}]{thmblackbox}
\declaretheorem[style=thmredbox,name=Problem]{problem}
\declaretheorem[style=thmredbox,name=Required Problem,sibling=problem]{reqproblem}
\declaretheorem[style=thmbluebox,name=Theorem,numberwithin=problem]{theorem}
\declaretheorem[style=thmbluebox,name=Lemma,sibling=theorem]{lemma}
\declaretheorem[style=thmbluebox,name=Theorem,numbered=no]{theorem*}
\declaretheorem[style=thmbluebox,name=Lemma,numbered=no]{lemma*}
\declaretheorem[style=thmgreenbox,name=Claim,sibling=theorem]{claim}
\declaretheorem[style=thmgreenbox,name=Claim,numbered=no]{claim*}
\declaretheorem[style=thmblackbox,name=Remark,sibling=theorem]{remark}
\declaretheorem[style=thmblackbox,name=Remark,numbered=no]{remark*}
\declaretheorem[style=thmgreenbox,name=Definition,sibling=theorem]{definition}
\declaretheorem[style=thmgreenbox,name=Definition,numbered=no]{definition*}
\declaretheorem[style=thmblackbox,name=Example,sibling=theorem]{example}
\declaretheorem[style=thmblackbox,name=Example,numbered=no]{example*}

\newenvironment{walkthrough}{\noindent\textbf{\color{green!40!black}Walkthrough.}}{}
\newlist{walk}{enumerate}{3}
\setlist[walk]{label=\bfseries (\alph*)}

\usepackage{asymptote}
\begin{asydef}
size(8cm); // set a reasonable default
usepackage("amsmath");
usepackage("amssymb");
settings.tex="pdflatex";
settings.outformat="pdf";
import geometry;
void filldraw(picture pic = currentpicture, conic g, pen fillpen=defaultpen, pen drawpen=defaultpen) { filldraw(pic, (path) g, fillpen, drawpen); }
void fill(picture pic = currentpicture, conic g, pen p=defaultpen) { filldraw(pic, (path) g, p); }
pair foot(pair P, pair A, pair B) { return foot(triangle(A,B,P).VC); }
pair centroid(pair A, pair B, pair C) { return (A+B+C)/3; }
\end{asydef}

\newcommand{\goals}[2]{\bgroup
\sffamily\small \emph{Instructions}: Solve \clubg{#1}.
If you have time, solve \clubg{#2}.\egroup\par}

%% 426c616e6b204c615465587e
\begin{document}
\title{Submission for BAW-SYMPOLY}
\subtitle{OTIS (internal use)}
\author{Michael Middlezong}
\date{\today}
\maketitle

\begin{example*}[$0\clubsuit$]
  Show that $a^4 + b^4 \ge a^3b + ab^3$ for $a,b > 0$.
\end{example*} \printpuid{Z68B4C1C}

\begin{walkthrough}
For the purposes of this example, assume you don't know AM-GM or Muirhead, etc.
The goal is to show how to solve the problem with ``bare hands''.

We will prove $a^4 + b^4 - a^3b - ab^3 \ge 0$.
\begin{walk}
  \ii Noting that equality holds when $a=b$,
  what factor must divide the left-hand side?
  \ii Imagine fixing $b$, and treating the left-hand side
  as a polynomial $P(a)$.
  It has a root at $a = b$.
  If you also know that $P \ge 0$ everywhere,
  what kind of root must that root be?
  \ii Use this to factor the left-hand side completely.
  (There should be three factors.)
\end{walk}
The condition $a,b > 0$ isn't actually used here
but makes things simpler to think about.
\end{walkthrough}
%% You're not expected to write up walkthroughs (unless you really want to).
%% The source is just for your reference.

\begin{example*}[\href{https://aops.com/community/p1813805}{AIME 2010}, $0\clubsuit$]
  Compute the maximum possible value of $a^3+b^3+c^3$
  over all real numbers $(a,b,c)$ satisfying
  \begin{align*}
  a^3 &= abc + 2 \\
  b^3 &= abc + 6 \\
  c^3 &= abc + 20.
  \end{align*}
\end{example*} \printpuid{10AIME9}

\begin{walkthrough}
\begin{walk}
   \ii Express $a^3+b^3+c^3$ in terms of $abc$.
   Thus it suffices to compute $abc$.
   \ii Find a way to get a quadratic equation in $abc$.
   \ii Solve for $abc$ and use it to get the final answer.
\end{walk}
\end{walkthrough}
%% You're not expected to write up walkthroughs (unless you really want to).
%% The source is just for your reference.

\begin{example*}[\href{https://aops.com/community/p5556379}{Evan Chen, Fall 2015}, $0\clubsuit$]
  Let $a$, $b$, $c$ be the distinct roots of the polynomial
  \[ P(x) = x^3 - 10x^2 + x - 2015. \]
  The cubic polynomial $Q(x)$ is monic
  and has distinct roots $bc-a^2$, $ca-b^2$, $ab-c^2$.
  What is the sum of the coefficients of $Q$?
\end{example*} \printpuid{15OMOF12}

\begin{walkthrough}
This can be done with brute force, by actually finding $Q$,
but there is a trick to it.
\begin{walk}
  \ii Show the answer is given by
  $Q(1) = (1-bc+a^2)(1-ca+b^2)(1-ab+c^2)$.
  \ii Show that $ab+bc+ca=1$.
  \ii Prove that $Q(1) = 2015000$.
\end{walk}
\end{walkthrough}
%% You're not expected to write up walkthroughs (unless you really want to).
%% The source is just for your reference.

\begin{example*}[\href{https://aops.com/community/p1810905}{USAMO 1975/3}, $0\clubsuit$]
  If $P(x)$ denotes a polynomial of degree $n$ such that
  $P(k)=\frac{k}{k+1}$ for $k=0,1,2,\dots,n$, determine $P(n+1)$.
\end{example*} \printpuid{75AMO3}

\begin{walkthrough}
The main idea is to define $Q(x) = (x+1)P(x) - x$.
\begin{walk}
  \ii Compute $\deg Q$ (in terms of $n$).
  \ii Determine the roots of $Q$.
  \ii Use (a) and (b) to establish the factorization,
  of $Q$ up to a constant factor.
  \ii Show that $Q(-1) = 1$
  and use this to conclude that the leading
  coefficient of $Q$ is equal to
  \[ c = \frac{(-1)^{n+1}}{(n+1)!}. \]
  \ii Compute $Q(n+1)$.
  \ii Prove that $P(n+1) = \frac{n+1 + (-1)^{n+1}}{n+2}$.
\end{walk}
\end{walkthrough}
%% You're not expected to write up walkthroughs (unless you really want to).
%% The source is just for your reference.

\begin{example*}[\href{https://aops.com/community/p27125715}{HMMT 2023 T2}, $0\clubsuit$]
  Prove there don't exist pairwise distinct complex numbers $a$, $b$, $c$, and $d$
  such that \[ a^3-bcd = b^3-cda = c^3-dab = d^3-abc. \]
\end{example*} \printpuid{23HMMTT2}

\begin{walkthrough}
There is a brute-force approach along the lines of taking
\[ a^3-b^3 = bcd-cda \]
and factoring out the common $a-b$.
\begin{walk}
  \ii Use this idea to show that $a^2+b^2 = c^2+d^2$.
  \ii Similarly, show that $a^2+c^2 = b^2+d^2$, and $a^2+d^2=b^2+c^2$.
\end{walk}
However, I think the following Vieta-based approach is more conceptually nice,
since it does not require any factoring and obviously generalizes to more
variables.
\begin{walk}[resume]
  \ii Let's assume $abcd \neq 0$.
  By scaling, show that we may in fact assume $abcd = 1$.
  \ii Conclude that we may define the number $k$ by
  \[ k \coloneqq a^3 - \frac 1a = \dots = d^3 - \frac 1d. \]

  \ii Let \[ P(X) = (X-a)(X-b)(X-c)(X-d). \]
  Find the coefficients of $P$ in terms of $k$.

  \ii Derive a contradiction by noticing $P(0) = -1$.

  \ii Weed out the edge case $abcd = 0$ we didn't address earlier.
  This gives a complete solution to the problem.
\end{walk}
\end{walkthrough}
%% You're not expected to write up walkthroughs (unless you really want to).
%% The source is just for your reference.

\newpage

% ========================================
\section*{Practice problems}
\goals{40}{52}

\epigraph{The Law speaks: you are cast out.
You are un-dwarf. I AM A WITNESS!}
{Angarthing in \emph{The Hammer of Thursagan}, \\
from \emph{The Battle for Wesnoth}}

\begin{problem}[\href{https://aops.com/community/p1088467}{Added by Eric Wang}, $2\clubsuit$]
  Let $r$, $s$, and $t$ be the three roots of the equation
  \[ 8x^3 + 1001x + 2008 = 0. \]
  Compute $(r + s)^3 + (s + t)^3 + (t + r)^3$.
\end{problem} \printpuid{08AIME7}

%% Type your solution to Added by Eric Wang (\href{https://otis.evanchen.cc/arch/08AIME7/}{08AIME7}) here ...
The sum of the roots is $0$. The desired quantity is then
\[ -t^3 - r^3 - s^3, \]
which is $753$ by Newton sums.
%% --------------------------------------------------

\begin{problem}[\href{https://aops.com/community/p1799885}{USAMO 1973/4}, $2\clubsuit$]
  Determine all triples $(x,y,z)$ of complex numbers satisfying
  \begin{align*}
  x+y+z &= 3, \\
  x^2+y^2+z^2 &= 3, \\
  x^3+y^3+z^3 &= 3.
  \end{align*}
\end{problem} \printpuid{73AMO4}

%% Type your solution to USAMO 1973/4 (\href{https://otis.evanchen.cc/arch/73AMO4/}{73AMO4}) here ...
Consider the cubic polynomial $(t - x)(t - y)(t - z)$.
From Newton's sums and Vieta's, this cubic polynomial
must equal $t^3 - 3t^2 + 3t - 1$.
The only factorization of this is $(t - 1)^3$,
so the only solution must be $(x,y,z)=(1,1,1)$.
%% --------------------------------------------------

\begin{problem}[\href{https://aops.com/community/p445333}{Canada 1996, added by Haozhe Yang}, $2\clubsuit$]
  If $\alpha$, $\beta$, and $\gamma$ are the roots of $x^3 - x - 1 = 0$,
  compute $\frac{1+\alpha}{1-\alpha} + \frac{1+\beta}{1-\beta} + \frac{1+\gamma}{1-\gamma}$.
\end{problem} \printpuid{96CAN1}

%% Type your solution to Canada 1996, added by Haozhe Yang (\href{https://otis.evanchen.cc/arch/96CAN1/}{96CAN1}) here ...
It is easy to show that $1-\alpha$, $1-\beta$, and $1-\gamma$
are the roots of the polynomial
\[ x^3 - 3x^2 + 2x + 1 \]
using Vieta's.

Then, we can easily calculate the desired expression using Vieta's as well. The answer is $-7$.
%% --------------------------------------------------

\begin{problem}[HMMT November 2016 Guts, added by Rohan Bodke, $2\clubsuit$]
  Let $r_1$, $r_2$, $r_3$, $r_4$ be
  the complex roots of the polynomial $x^4-4x^3+8x^2-7x+3$.
  Calculate
  \[ \frac{r_1^2}{r_2^2+r_3^2+r_4^2}+\frac{r_2^2}{r_1^2+r_3^2+r_4^2}
  +\frac{r_3^2}{r_1^2+r_2^2+r_4^2}+\frac{r_4^2}{r_1^2+r_2^2+r_3^2}. \]
\end{problem} \printpuid{16HMNTGUTS27}

%% Type your solution to HMMT November 2016 Guts, added by Rohan Bodke (\href{https://otis.evanchen.cc/arch/16HMNTGUTS27/}{16HMNTGUTS27}) here ...
By Newton's sums, the sum of the squares of the roots is $0$.
This means each term in the requested expression is $-1$, giving us a total answer of $-4$.
%% --------------------------------------------------

\begin{problem}[\href{https://aops.com/community/p3228708}{NIMO \#8}, $2\clubsuit$]
  Let $x$, $y$, $z$ be complex numbers satisfying
  \begin{align*}
  x^2 + 5y &= 10x \\
  y^2 + 5z &= 10y \\
  z^2 + 5x &= 10z.
  \end{align*}
  Find the sum of all possible values of $z$.
\end{problem} \printpuid{NIMO85}

%% Type your solution to NIMO \#8 (\href{https://otis.evanchen.cc/arch/NIMO85/}{NIMO85}), proposed by Evan Chen here ...

%% --------------------------------------------------

\begin{problem}[\href{https://aops.com/community/p1595153}{USAMO 1984/1}, $2\clubsuit$]
  The product of two of the four roots of the quartic
  equation $x^4-18x^3+kx^2+200x-1984=0$ is $-32$.
  Determine $k$.
\end{problem} \printpuid{84AMO1}

%% Type your solution to USAMO 1984/1 (\href{https://otis.evanchen.cc/arch/84AMO1/}{84AMO1}) here ...
Notice that we can write the polynomial as
\[ (x^2 + ax - 32)(x^2 + bx + 62) \]
for constants $a$ and $b$. Expanding this and matching coefficients, we get the system of equations
\begin{align*}
    a + b &= -18 \\
    62a - 32b &= 200.
\end{align*}
We can solve this system to get $a = -4$, $b = -14$.
We also know $k = 30 + ab$ from the earlier expansion, so $k = 86$.
%% --------------------------------------------------

\begin{reqproblem}[$3\clubsuit$]
  The cubic $x^3 - 7x^2 + 3x + 2$ has irrational roots $r > s > t$.
  There exists a unique set of rational numbers $A$, $B$, and $C$,
  such that the cubic $x^3 + Ax^2 + Bx + C$ has $r+s$ as a root.
  What is $A+B+C$?
\end{reqproblem} \printpuid{JASONMAO}

%% Type your solution to \href{https://otis.evanchen.cc/arch/JASONMAO/}{JASONMAO} here ...
Consider the polynomial with roots $r + s$, $s + t$, and $r + t$. We will find its coefficients and show that it is the desired polynomial.
Using Vieta's, we can see that
\[ A = -2(r + s + t) = -14. \]
We can also see that
\[ B = (r + s)(s + t) + (s + t)(r + t) + (r + s)(r + t). \]
Expanding and simplifying with Vieta's and Newton sums, we get $B = 52$.

The $C$ term is slightly more involved, but we can use a combination of Newton sums, Vieta's, and grouping of terms to get $C = -23$.

All these terms are rational, so overall, our answer is $A + B + C = -14 + 52 - 23 = 15$.
%% --------------------------------------------------

\begin{problem}[\href{https://aops.com/community/p15681035}{AIME II 2020, added by Benjamin Wang-Tie}, $3\clubsuit$]
  Let $P(x) = x^2 - 3x - 7$, and let $Q(x)$ and $R(x)$ be two quadratic
  polynomials also with the coefficient of $x^2$ equal to $1$.
  David computes each of the three sums $P + Q$, $P + R$, and $Q + R$
  and is surprised to find that each pair of these sums has a common root,
  and these three common roots are distinct.
  If $Q(0) = 2$, compute $R(0)$.
\end{problem} \printpuid{20AIMEII11}

%% Type your solution to AIME II 2020, added by Benjamin Wang-Tie (\href{https://otis.evanchen.cc/arch/20AIMEII11/}{20AIMEII11}) here ...

%% --------------------------------------------------

\begin{problem}[\href{https://aops.com/community/p11972643}{AIME I 2019, added by Joshua Im}, $2\clubsuit$]
  For distinct complex numbers $z_1$, $z_2$, \ldots, $z_{673}$, the polynomial
  \[ (x-z_1)^3(x-z_2)^3 \cdots (x-z_{673})^3 \]
  can be expressed as $x^{2019}+20x^{2018}+19x^{2017}+g(x)$,
  where $g(x)$ is a polynomial with complex coefficients and with degree at most $2016$.
  Compute \[  \sum_{1\leq j<k\leq 673} z_jz_k. \]
\end{problem} \printpuid{19AIME10}

%% Type your solution to AIME I 2019, added by Joshua Im (\href{https://otis.evanchen.cc/arch/19AIME10/}{19AIME10}) here ...

%% --------------------------------------------------

\begin{problem}[\href{https://aops.com/community/p6412925}{Austria 2016/6, added by Abdullahil Kafi}, $3\clubsuit$]
  Let $a$, $b$, $c$ be three integers for which the sum
  \[ \frac{ab}{c}+ \frac{ac}{b}+ \frac{bc}{a}\]
  is an integer.
  Prove that each of the three numbers
  \[ \frac{ab}{c}, \quad \frac{ac}{b}, \quad \frac{bc}{a}\]
  is an integer.
\end{problem} \printpuid{16AUT6}

%% Type your solution to Austria 2016/6, added by Abdullahil Kafi (\href{https://otis.evanchen.cc/arch/16AUT6/}{16AUT6}) here ...

%% --------------------------------------------------

\begin{problem}[$2\clubsuit$]
  Factor the polynomial \[ a(b-c)^3 + b(c-a)^3 + c(a-b)^3. \]
\end{problem} \printpuid{ZEAC3666}

%% Type your solution to \href{https://otis.evanchen.cc/arch/ZEAC3666/}{ZEAC3666} here ...
We notice that the polynomial vanishes whenever $a=b$, $a=c$, or $b=c$.
So, the polynomial is divisible by $(a-b)(a-c)(b-c)$.
We know the last factor must be a multiple of $a+b+c$.
We can match the coefficient of $ab^3$ to get that the factored form is
\[ (a-b)(a-c)(b-c)(-a-b-c). \]
%% --------------------------------------------------

\begin{problem}[$3\clubsuit$]
  Let $a$, $b$, $c$ be real numbers.
  Prove that
  \[ a^3+b^3+c^3 = (a+b+c)^3 \quad\text{if and only if}\quad
  a^5+b^5+c^5 = (a+b+c)^5.  \]
\end{problem} \printpuid{H1883820}

%% Type your solution to \href{https://otis.evanchen.cc/arch/H1883820/}{H1883820} here ...

%% --------------------------------------------------

\begin{problem}[$2\clubsuit$]
  Let $a$, $b$, $c$, $d$ be distinct real numbers such that
  \[
  a + b + c + d = 0
  \quad \text{and} \quad
  \frac{1}{a} + \frac{1}{b} + \frac{1}{c} + \frac{1}{d} = 0.
  \]
  Prove that two of the four numbers have sum zero.
\end{problem} \printpuid{BMCQ54}

%% Type your solution to \href{https://otis.evanchen.cc/arch/BMCQ54/}{BMCQ54} here ...
We notice that quartic polynomial with roots $a,b,c,d$ is even, and the result follows.
%% --------------------------------------------------

\begin{problem}[\href{https://aops.com/community/p713214}{AIME II 2003, added by Lincoln Liu}, $3\clubsuit$]
  Consider the polynomials $P(x) = x^{6} - x^{5} - x^{3} - x^{2} - x$
  and $Q(x) = x^{4} - x^{3} - x^{2} - 1$.
  Given that $z_1$, $z_2$, $z_3$, and $z_4$ are the roots of $Q(x) = 0$,
  find $P(z_1) + P(z_2) + P(z_3) + P(z_4)$.
\end{problem} \printpuid{03AIMEII9}

%% Type your solution to AIME II 2003, added by Lincoln Liu (\href{https://otis.evanchen.cc/arch/03AIMEII9/}{03AIMEII9}) here ...
Write
\[ P(x) = (x^2 + x)Q(x) + x^2 - x + 1. \]
Then,
\begin{align*}
    \sum_{i=1}^4 P(z_i) &= \sum_{i=1}^4 ((x^2 + x)Q(x) + x^2 - x + 1) \\
    &= \sum_{i=1}^4 (x^2 - x + 1) \\
    &= \sum_{i=1}^4 x^2 - \sum_{i=1}^4 x + \sum_{i=1}^4 1\\
    &= 3 - 1 + 4 = \boxed{6}.
\end{align*}
%% --------------------------------------------------

\begin{problem}[\href{https://aops.com/community/p9768199}{CMIMC 2018 A9}, $9\clubsuit$]
  Given the polynomial identity
  \[ (x^2-3x+1)^{1009} = \sum_{k=0}^{2018} a_k x^k \]
  calculate the remainder when
  $a_0^2 + a_1^2 + \dots + a_{2018}^2$ is divided by $2017$.
\end{problem} \printpuid{18CMIMCA9}

%% Type your solution to CMIMC 2018 A9 (\href{https://otis.evanchen.cc/arch/18CMIMCA9/}{18CMIMCA9}), proposed by David Altizio here ...

%% --------------------------------------------------

\begin{reqproblem}[\href{https://aops.com/community/p28369648}{Stanford Math Tournament 2011}, $3\clubsuit$]
  Let $P(x)$ be a polynomial of degree $2011$ such that
  $P(1) = 0$, $P(2) = 1$, $P(4) = 2$, \dots, and $P(2^{2011}) = 2011$.
  Find the coefficient of $x^1$ in $P$.
\end{reqproblem} \printpuid{11SMTA7}

%% Type your solution to Stanford Math Tournament 2011 (\href{https://otis.evanchen.cc/arch/11SMTA7/}{11SMTA7}) here ...
We can notice that the polynomial $P(2x) - P(x) - 1$ has roots $x = 2^i$ for $0 \leq i \leq 2010$.
Thus, we can write
\[
    P(2x) - P(x) - 1 = c(x - 2^0)(x - 2^1)\cdots(x - 2^{2010}).
\]
Plugging in $x = 0$, we can find $\frac{1}{c} = 1 + 2 + \cdots + 2010$ (denote by $S$ this sum).

Now, let $a$ be the coefficient of the linear term in $P(x)$.
Then, the linear term of $P(2x) - P(x) - 1$ is $2ax - ax = ax$.
So, it suffices to find the linear coefficient of $c(x - 2^0)(x - 2^1)\cdots(x - 2^{2010})$.

For this, we can use Vieta's. We end up with
\[
    a = 2^S + 2^{S-1} + \cdots + 2^{S - 2010}.
\]
We can simplify this to $a = 2 - \frac{1}{2^{2010}}$.
%% --------------------------------------------------

\begin{problem}[$5\clubsuit$]
  Let $a$, $b$, $c$ be integers with $c \neq 0$.
  Suppose the cubic polynomial $x^3 + ax^2 + bx + c$ has roots $r \le s \le t$.
  Show that $\frac rs + \frac st + \frac tr$ can be written as $u \pm \sqrt v$
  for some rational numbers $u$ and $v$.
\end{problem} \printpuid{NOTVIETA}

%% Type your solution to \href{https://otis.evanchen.cc/arch/NOTVIETA/}{NOTVIETA} here ...

%% --------------------------------------------------

\begin{problem}[\href{https://aops.com/community/p29632847}{Mock ARML 2022, added by Shaheem Samsudeen}, $2\clubsuit$]
  Let $a,b,c$ complex numbers with $ab+bc+ca = 61$ such that
  \begin{align*}
  \frac{1}{b+c}+\frac{1}{c+a}+\frac{1}{a+b} &= 5 \\
  \frac{a}{b+c}+\frac{b}{c+a}+\frac{c}{a+b} &= 32.
  \end{align*}
  Find the value of $abc$.
\end{problem} \printpuid{H3233958}

%% Type your solution to Mock ARML 2022, added by Shaheem Samsudeen (\href{https://otis.evanchen.cc/arch/H3233958/}{H3233958}) here ...

%% --------------------------------------------------

\begin{reqproblem}[\href{https://aops.com/community/p28369661}{Stanford Math Tournament 2013}, $3\clubsuit$]
  Let $a = -\sqrt3 + \sqrt5 + \sqrt7$, $b = \sqrt3 - \sqrt5 + \sqrt 7$,
  $c = \sqrt3 + \sqrt5 - \sqrt 7$. Compute
  \[ \frac{a^4}{(a-b)(a-c)} + \frac{b^4}{(b-c)(b-a)} + \frac{c^4}{(c-a)(c-b)}. \]
\end{reqproblem} \printpuid{13SMTA9}

%% Type your solution to Stanford Math Tournament 2013 (\href{https://otis.evanchen.cc/arch/13SMTA9/}{13SMTA9}) here ...
Putting the three terms over a common denominator and factoring the numerator,
we can find that the expression equals
\[
    a^2 + b^2 + c^2 + ab + bc + ca.
\]
We can rewrite this as $(a + b + c)^2 - (ab + bc + ca)$.

Let $x = \sqrt{3}$, $y = \sqrt{5}$, $z = \sqrt{7}$,
and $S = a + b + c = x + y + z$. Then, our desired expression is
\[
    S^2 - [(S - 2x)(S - 2y) + (S - 2y)(S - 2z) + (S - 2z)(S - 2x)].
\]
We can simplify this to get the answer of
\[
    2S^2 - 4(xy + yz + zx) = 30.
\]
%% --------------------------------------------------

\begin{problem}[Ritwin Narra, $5\clubsuit$]
  Fix an integer $n \neq 1$.
  Prove that if real numbers $a$, $b$, $c$, $d$ satisfy
  \[ a+b+c+d = a^n+b^n+c^n+d^n = 0 \]
  then two of $a$, $b$, $c$, $d$ sum to $0$.
\end{problem} \printpuid{Z5394300}

%% Type your solution to Ritwin Narra (\href{https://otis.evanchen.cc/arch/Z5394300/}{Z5394300}) here ...

%% --------------------------------------------------

\begin{reqproblem}[\href{https://aops.com/community/p28369559}{Black MOP 2012}, $5\clubsuit$]
  Let $ABC$ be a triangle and let $h_A$, $h_B$, $h_C$ be the lengths of the altitudes
  from $A$, $B$, and $C$. Let $a = BC$, $b = CA$, $c = AB$.
  Suppose that
  \[ \sqrt{a+h_B} + \sqrt{b+h_C} + \sqrt{c+h_A}
  = \sqrt{a+h_C} + \sqrt{b+h_A} + \sqrt{c+h_B}. \]
  Prove that triangle $ABC$ is isosceles.
\end{reqproblem} \printpuid{12BLACKMOP}

%% Type your solution to Black MOP 2012 (\href{https://otis.evanchen.cc/arch/12BLACKMOP/}{12BLACKMOP}) here ...
Let $\sqrt{a + h_B}$, $\sqrt{b + h_C}$, and $\sqrt{c + h_A}$ be the roots of a polynomial.

Then, we claim this polynomial also has roots $\sqrt{a + h_C}$, $\sqrt{b + h_A}$, and $\sqrt{c + h_B}$.
This can be shown with Vieta's and Newton sums, along with the fact that
\[
    (a + h_B)(b + h_C)(c + h_A) = (a + h_C)(b + h_A)(c + h_B),
\]
which can be shown by expanding and simplifying using the triangle area formula.

Thus, we have three cases:
\begin{enumerate}
    \item $\sqrt{a + h_B} = \sqrt{a + h_C}$.
    Let $A$ be the area of the triangle. Then it is obvious that $b = c$.
    \item $\sqrt{a + h_B} = \sqrt{b + h_A}$.
    We can derive that $a = b$ or $ab = -1$, the latter of which is impossible.
    \item $\sqrt{a + h_B} = \sqrt{c + h_B}$.
    Obviously $a = c$.
\end{enumerate}
In any case, the triangle is isosceles.
%% --------------------------------------------------

\begin{problem}[\href{https://aops.com/community/p26456432}{Added by Jason Lee}, $5\clubsuit$]
  Consider all complex numbers $k$ for which there exist complex numbers
  $a$, $b$, $c$, $d$ and $e$ satisfying
  \begin{align*}
  \frac{a}{b} + \frac{b}{c} &= 1 \\
  \frac{b}{c} + \frac{c}{d} &= 2 \\
  \frac{c}{d} + \frac{d}{e} &= 3 \\
  \frac{d}{e} + \frac{e}{a} &= 4 \\
  \frac{e}{a} + \frac{a}{b} &= k.
  \end{align*}
  Find the sum of all possible values of $k^4$.
\end{problem} \printpuid{H2954369}

%% Type your solution to Added by Jason Lee (\href{https://otis.evanchen.cc/arch/H2954369/}{H2954369}) here ...

%% --------------------------------------------------

\begin{remark*}
  For the previous problem by Jason Lee, avoid using a calculator ---
  you can solve it with rather little arithmetic if you set it up correctly.
\end{remark*}

\begin{problem}[\href{https://w.wiki/Ay2P}{Prove the Newton sums in the reading}, $5\clubsuit$]
  Suppose the complex-coefficient polynomial
  \[ f(x) = a_n x^n + a_{n-1} x^{n-1} + \dots + a_0 \]
  has complex roots $z_1$, $z_2$, \dots, $z_n$.
  For each $d \ge 0$, define $p_d = z_1^d + z_2^d + \dots + z_n^d$.
  Prove that for every integer $k \ge 1$ we have the identity
  \[ a_n p_k + a_{n-1} p_{k-1} + \dots + a_{n-(k-1)} p_1 + k \cdot a_{n-k} = 0. \]
  where we by convention let $a_i = 0$ if $i < 0$.
\end{problem} \printpuid{ZC01F573}

%% Type your solution to Prove the Newton sums in the reading (\href{https://otis.evanchen.cc/arch/ZC01F573/}{ZC01F573}) here ...

%% --------------------------------------------------

\begin{reqproblem}[\href{https://aops.com/community/p2014940}{Longlist 1985/19}, $9\clubsuit$]
  Solve over $\RR$ the system of simultaneous equations
  \begin{align*}
  \sqrt x - \frac 1y - 2w + 3z &= 1, \\
  x + \frac{1}{y^2} - 4w^2 - 9z^2 &= 3, \\
  x \sqrt x - \frac{1}{y^3} - 8w^3 + 27z^3 &= -5, \\
  x^2 + \frac{1}{y^4} - 16w^4 - 81z^4 &= 15.
  \end{align*}
\end{reqproblem} \printpuid{85LL19}

%% Type your solution to Longlist 1985/19 (\href{https://otis.evanchen.cc/arch/85LL19/}{85LL19}) here ...
After the obvious substitution we have
\begin{align*}
    a+b-c-d &= 1\\
    a^2 + b^2 - c^2 - d^2 &= 3 \\
    a^3 + b^3 - c^3 - d^3 &= -5 \\
    a^4 + b^4 - c^4 - d^4 &= 15.
\end{align*}
Guessing small integer solutions, we find that $(a,b,c,d) = (1,-2,-1,-1)$
is a solution. We claim that it is the only solution (up to swapping $a$ and $b$).

Notice that we can write
\[ a^n + b^n + (-1)^n + (-1)^n = c^n + d^n + (-2)^n + 1^n \]
for $n = 1,2,3,4$. This means that the Newton sums of the polynomials
with roots $\{a,b,-1,-1\}$ and $\{c,d,-2,1\}$ are the same.
This uniquely determines the polynomial (up to leading coefficient).
Thus, the multisets of roots must be equal, and we are done.

Finally, our original substitution requires $a$ to be positive,
so the only solution is
\[ (x,y,w,z) = \left(1,\frac12, -\frac12, \frac13 \right). \]
%% --------------------------------------------------

\begin{problem}[\href{https://aops.com/community/p28369832}{{$\ol{\ZZ}$} is a ring}, $9\clubsuit$]
  Suppose that $\alpha$ and $\beta$ are complex numbers
  and monic polynomials $P,Q \in \ZZ[x]$ satisfy $P(\alpha) = Q(\beta) = 0$.
  \begin{enumerate}[(a)]
  \ii Show that there is monic polynomial $R \in \ZZ[x]$
  such that $R(\alpha + \beta) = 0$.
  \ii Show that there is a monic polynomial $S \in \ZZ[x]$
  such that $S(\alpha \beta) = 0$.
  \end{enumerate}
\end{problem} \printpuid{Z40B5559}

%% Type your solution to {$\ol{\ZZ}$} is a ring (\href{https://otis.evanchen.cc/arch/Z40B5559/}{Z40B5559}) here ...

%% --------------------------------------------------

\begin{problem}[\href{https://aops.com/community/p14065082}{HMMT 2020, added by Guanjie Lu}, $9\clubsuit$]
  Let $P(x)=x^{2020}+x+2$.
  Let $Q(x)$ be the monic polynomial of degree $\binom{2020}{2}$
  whose roots are the pairwise products of the roots of $P(x)$.
  Let $\alpha$ satisfy $P(\alpha)=4$.
  Compute the sum of all possible values of $Q(\alpha^2)^2$.
\end{problem} \printpuid{20HMMTA9}

%% Type your solution to HMMT 2020, added by Guanjie Lu (\href{https://otis.evanchen.cc/arch/20HMMTA9/}{20HMMTA9}) here ...

%% --------------------------------------------------

\end{document}
