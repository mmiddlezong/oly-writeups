\documentclass{scrartcl}
\usepackage{graphicx} % Required for inserting images
\usepackage{amsmath}
\usepackage{listings}

\title{EGMO Solutions}
\author{Michael Middlezong}

\begin{document}

\maketitle

\section*{Chapter 4}
\subsection*{Problem 4.48 (Japanese Olympiad 2009)}
Notice $APOQ$ is cyclic. This can be proven using the homothety at $Q$. Then, notice $POQ$ is isosceles and the result shortly follows.
\subsection*{Problem 4.49}
Let ray $AE$ intersect the circumcircle at $W$. Because $\angle BAT = \angle CAE = \angle CAW$, we know arc $BT$ has the same measure as arc $CW$.

Now, extend ray $TD$ to hit the circumcircle at $V$. Line $TV$ is just the reflection of line $WA$ across the perpendicular bisector of $BC$, because of the fact that $BD = CE$ and that arc $BT$ equals arc $CW$.

Thus, arcs $BA$ and $CV$ have the same measure, and the result follows.
\subsection*{Problem 4.50 (Vietnam TST 2003/2)}
Let $I_A, I_B, I_C$ denote the excenters.
We know from a lemma in this chapter that line $A_0D$ is just line $DI_A$, and so forth. Also, we can see that line $DF$ is parallel to line $I_AI_C$. Let $Z$ be the intersection point of lines $DI_A$ and $FI_C$. Then, a homothety at $Z$ takes $F$ to $I_C$ and $D$ to $I_A$. This homothety also takes $E$ to $I_B$ for the same reason. So, lines $DI_A$, $FI_C$, and $EI_B$ concur at $Z$. For the $OI$ part, notice that $O$ is the nine-point center of triangle $I_AI_BI_C$, and Euler line leads to the result.
\subsection*{Problem 4.51 (Sharygin 2013)}
Let $M$ be the midpoint of $AB$. From a previous lemma, we know $CM$, $A'B'$, and $C'I$ are concurrent at a point $X$. Notice that $X$ is also the orthocenter of triangle $CIK$. Thus, line $IX$ is perpendicular to $CK$. However, line $IX$ is also perpendicular to $AB$, so $AB \parallel CK$.
\subsection*{Problem 4.52 (APMO 2012/4)}
Let $H'$ be $H$ reflected over $D$, and $H''$ be $H$ reflected over $M$. It is well known that $H'$ and $H''$ lie on the circumcircle of $ABC$. By PoP, $HE \cdot HH'' = HA \cdot HH'$.
Dividing both sides by two, we obtain the equation $HE \cdot HM = HA \cdot HD$. In other words, $AEDM$ is cyclic.

Now, we claim triangle $ABF$ is similar to triangle $AMC$. We know $\angle ACM = \angle ACB = \angle AFB$.

Also, $\angle AMC = \angle AMD = \angle AED = \angle AEF = \angle ABF$ (using directed angles). Thus, the two triangles are similar, and it follows that $AF$ is a symmedian. 
Finally, the desired result is a well-known consequence of $AF$ being a symmedian.
\subsection*{Problem 4.53 (Shortlist 2002/G7)}
As always, we can remove $M$ from our diagram by noting that line $MK$ is the same as line $KI_A$.
Let $Q$ be the midpoint of $KI_A$. We claim $BNCQ$ is cyclic.
Let $S$ be the midpoint of $NK$. Since $\angle ISI_A = \angle IBI_A = 90$ (well known), we know $S$ lies on the circle containing $B$, $I$, $C$, and $I_A$ (this circle being from a common configuration). By PoP, $KS \cdot KI_A = KB \cdot KC$.
However, we know $KS \cdot KI_A = KN \cdot KQ$. Thus, $BNCQ$ is cyclic.

Let $P$ be the circumcenter of $BCN$. Notice that since $BK = XC$, we have $QB = QC$ and thus $QP$ is the perpendicular bisector of $BC$. In other words, $Q$ is the arc midpoint of arc $BC$ on the circumcircle of $BCN$. Consider a homothety at $N$ that takes $K$ to $Q$. This homothety must also take $I$ to $P$, finishing the proof.
\section*{Chapter 5}
\subsection*{Problem 5.16 (Star Theorem)}
Using the Law of Sines, we write
\[
    \prod_{i = 1}^{5} X_i A_{i + 2} = \prod_{i = 1}^{5} \frac{A_{i + 2}A_{i + 3}}{\sin \angle A_{i+2}X_iA_{i+3}}\sin \angle A_{i+2}A_{i+3}X_i
\]
and
\begin{align*}
    \prod_{i = 1}^{5} X_i A_{i + 3} &= \prod_{i = 1}^{5} \frac{A_{i + 2}A_{i + 3}}{\sin \angle A_{i+2}X_iA_{i+3}}\sin \angle A_{i+3}A_{i+2}X_i \\
    &= \prod_{i = 1}^{5} \frac{A_{i + 2}A_{i + 3}}{\sin \angle A_{i+2}X_iA_{i+3}}\sin \angle A_{i+1}A_{i+2}X_{i-1}.
\end{align*}
Notice that this is the same expression by re-indexing. Thus, we are done.
\subsection*{Problem 5.17}
We know the length of the exradius $r_A$ is $\frac{sr}{s-a}$. Then, simply use Heron's formula and $A = sr$.
\subsection*{Problem 5.18 (APMO 2013/1)}
WLOG we will just prove triangles $ODB$ and $OAE$ have the same area, and then we can get three pairs from symmetry. We note that $OB$ and $OA$ have the same length, so we just need to compare the heights of the altitudes from $D$ and $E$ to their respective sides. So, using some angle chasing and trigonometry, we can reduce what we are trying to prove to
\[
    AE \sin (90 - B) = BD \sin (90 - A).
\]
Then, we notice that $AE = AB \sin (90 - A)$ and $BD = AB \sin (90 - B)$ by drawing altitudes, giving us the result.
\subsection*{Problem 5.19 (EGMO 2013/1)}
Let $a$, $b$, $c$ denote the side lengths of $ABC$ in their usual way. We can compute
\begin{align*}
    AD^2 &= c^2 + 4a^2 - 4ac \cos B \\
    BE^2 &= c^2 + 4b^2 + 4bc \cos A.
\end{align*}
(The $+$ is not a mistake in the second line there!) Equating the two, we get $a^2 - ac \cos B = b^2 + bc \cos A$. Using the Law of Cosines but solving for angles, we get
\begin{align*}
    \cos B &= \frac{a^2 + c^2 - b^2}{2ac} \\
    \cos A &= \frac{b^2 + c^2 - a^2}{2bc}.
\end{align*}
Plugging these back in, we can simplify to get $a^2 = b^2 + c^2$. Thus, triangle $ABC$ is right-angled.
\subsection*{Problem 5.20 (HMMT 2013)}
Let $E$ be the contact point of the incircle with $AB$, and let $M$ be the midpoint of $BC$.
Also, let $a$, $b$, and $c$ mean the usual side lengths. The condition $2a = b + c$ can also be written as $s - a = \frac{a}{2}$, where $s$ is the semiperimeter. Since $AE = s - a$ and $MC = \frac{a}{2}$, we know $AE = MC$.

We also know $\angle DCM = \angle IAE$. So, by AAS congruence, we have that triangle $AIE$ is congruent to triangle $CDM$. Therefore, $DC = AI = DI$ (by another lemma), and we are done.
\subsection*{Problem 5.21 (USAMO 2010/4)}
Notice that $I$ is the incenter.
Law of Cosines tells us
\[
    BC^2 = BI^2 + CI^2 - 2 \cdot BI \cdot CI \cos \angle BIC.
\]
Angle chasing gives us $\angle BIC = 135$. So, we have
\[
    BC^2 = BI^2 + CI^2 + \sqrt{2} \cdot BI \cdot CI.
\]
Assume $BI$ and $CI$ have integer lengths. Then $BC^2 = AB^2 + AC^2$ is not an integer. Thus, the six segments cannot all have integer lengths.
\subsection*{Problem 5.22 (Iran Olympiad 1999)}
We can rewrite the condition as $ID \cdot (\sin B + \sin C) = \frac{1}{2} AD$ (using some angle chasing). Since $ID = BD = CD$, we now use Ptolemy's theorem to get
\[
    (AB + AC) \cdot ID = AD \cdot BC.
\]
However, we know that $ID = \frac{AD}{2(\sin B + \sin C)}$, so we can plug that in and simplify to get
\[
    BC = \frac{AB + AC}{2(\sin B + \sin C)}.
\]
Using the Extended Law of Sines again, we can write $\sin B = \frac{AC}{2R}$ and $\sin C = \frac{AB}{2R}$ where $R$ is the circumradius. Then, the above equation simplifies to
\[
    BC = R.
\]
Using the Extended Law of Sines, this means that $\sin A = \frac{1}{2}$, so $\angle A = 30$ or $\angle A = 150$.
\subsection*{Problem 5.23 (CGMO 2002/4)}
Using the Law of Sines,
\[
    \frac{AH}{HF} = \frac{EA \sin \angle HEA}{EF \sin \angle HEF}.
\]
Note that $EC = EF$ because chord $CF$ is perpendicular to diameter $AB$. So, we rewrite our expression as
\[
    \frac{EA \sin \angle HEA}{EC \sin \angle HEF}.
\]
Simple angle chasing and trig finishes this proof:
\begin{align*}
    \frac{EA \sin \angle HEA}{EC \sin \angle HEF} &= \frac{EA \sin \angle GCB}{EC \sin \angle CBD} \\
    &= \frac{EA \sin (90 - \angle CBD)}{EC \sin \angle CBD} \\
    &= \frac{EA}{EC \tan \angle CBD} \\
    &= \frac{\tan \angle ECA}{\tan \angle CBD} \\
    &= \frac{\tan \angle CBA}{\tan \angle CBD} \\
    &= \frac{AC}{CD}.
\end{align*}
\end{document}
