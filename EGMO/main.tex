\documentclass{scrartcl}
\usepackage{graphicx} % Required for inserting images
\usepackage{amsmath}
\usepackage{listings}
\usepackage{evan}

\title{EGMO Solutions}
\author{Michael Middlezong}

\begin{document}

\maketitle

\section*{Chapter 4}
\subsection*{Problem 4.48 (Japanese Olympiad 2009)}
Notice $APOQ$ is cyclic. This can be proven using the homothety at $Q$. Then, notice $POQ$ is isosceles and the result shortly follows.
\subsection*{Problem 4.49}
Let ray $AE$ intersect the circumcircle at $W$. Because $\angle BAT = \angle CAE = \angle CAW$, we know arc $BT$ has the same measure as arc $CW$.

Now, extend ray $TD$ to hit the circumcircle at $V$. Line $TV$ is just the reflection of line $WA$ across the perpendicular bisector of $BC$, because $BD = CE$ and that arc $BT$ equals arc $CW$.

Thus, arcs $BA$ and $CV$ have the same measure, and the result follows.
\subsection*{Problem 4.50 (Vietnam TST 2003/2)}
Let $I_A, I_B, I_C$ denote the excenters.
We know from a lemma in this chapter that line $A_0D$ is just line $DI_A$, and so forth. Also, we can see that line $DF$ is parallel to line $I_AI_C$. Let $Z$ be the intersection point of lines $DI_A$ and $FI_C$. Then, a homothety at $Z$ takes $F$ to $I_C$ and $D$ to $I_A$. This homothety also takes $E$ to $I_B$ for the same reason. So, lines $DI_A$, $FI_C$, and $EI_B$ concur at $Z$. For the $OI$ part, notice that $O$ is the nine-point center of triangle $I_AI_BI_C$, and Euler line leads to the result.
\subsection*{Problem 4.51 (Sharygin 2013)}
Let $M$ be the midpoint of $AB$. From a previous lemma, we know $CM$, $A'B'$, and $C'I$ are concurrent at a point $X$. Notice that $X$ is also the orthocenter of triangle $CIK$. Thus, line $IX$ is perpendicular to $CK$. However, line $IX$ is also perpendicular to $AB$, so $AB \parallel CK$.
\subsection*{Problem 4.52 (APMO 2012/4)}
Let $H'$ be $H$ reflected over $D$, and $H''$ be $H$ reflected over $M$. It is well known that $H'$ and $H''$ lie on the circumcircle of $ABC$. By PoP, $HE \cdot HH'' = HA \cdot HH'$.
Dividing both sides by two, we obtain the equation $HE \cdot HM = HA \cdot HD$. In other words, $AEDM$ is cyclic.

Now, we claim triangle $ABF$ is similar to triangle $AMC$. We know $\angle ACM = \angle ACB = \angle AFB$.

Also, $\angle AMC = \angle AMD = \angle AED = \angle AEF = \angle ABF$ (using directed angles). Thus, the two triangles are similar, and it follows that $AF$ is a symmedian. 
Finally, the desired result is a well-known consequence of $AF$ being a symmedian.
\subsection*{Problem 4.53 (Shortlist 2002/G7)}
As always, we can remove $M$ from our diagram by noting that line $MK$ is the same as line $KI_A$.
Let $Q$ be the midpoint of $KI_A$. We claim $BNCQ$ is cyclic.
Let $S$ be the midpoint of $NK$. Since $\angle ISI_A = \angle IBI_A = 90$ (well known), we know $S$ lies on the circle containing $B$, $I$, $C$, and $I_A$ (this circle being from a common configuration). By PoP, $KS \cdot KI_A = KB \cdot KC$.
However, we know $KS \cdot KI_A = KN \cdot KQ$. Thus, $BNCQ$ is cyclic.

Let $P$ be the circumcenter of $BCN$. Notice that since $BK = XC$, we have $QB = QC$ and thus $QP$ is the perpendicular bisector of $BC$. In other words, $Q$ is the arc midpoint of arc $BC$ on the circumcircle of $BCN$. Consider a homothety at $N$ that takes $K$ to $Q$. This homothety must also take $I$ to $P$, finishing the proof.
\section*{Chapter 5}
\subsection*{Problem 5.16 (Star Theorem)}
Using the Law of Sines, we write
\[
    \prod_{i = 1}^{5} X_i A_{i + 2} = \prod_{i = 1}^{5} \frac{A_{i + 2}A_{i + 3}}{\sin \angle A_{i+2}X_iA_{i+3}}\sin \angle A_{i+2}A_{i+3}X_i
\]
and
\begin{align*}
    \prod_{i = 1}^{5} X_i A_{i + 3} &= \prod_{i = 1}^{5} \frac{A_{i + 2}A_{i + 3}}{\sin \angle A_{i+2}X_iA_{i+3}}\sin \angle A_{i+3}A_{i+2}X_i \\
    &= \prod_{i = 1}^{5} \frac{A_{i + 2}A_{i + 3}}{\sin \angle A_{i+2}X_iA_{i+3}}\sin \angle A_{i+1}A_{i+2}X_{i-1}.
\end{align*}
Notice that this is the same expression by re-indexing. Thus, we are done.
\subsection*{Problem 5.17}
We know the length of the exradius $r_A$ is $\frac{sr}{s-a}$. Then, simply use Heron's formula and $A = sr$.
\subsection*{Problem 5.18 (APMO 2013/1)}
WLOG we will just prove triangles $ODB$ and $OAE$ have the same area, and then we can get three pairs from symmetry. We note that $OB$ and $OA$ have the same length, so we just need to compare the heights of the altitudes from $D$ and $E$ to their respective sides. So, using some angle chasing and trigonometry, we can reduce what we are trying to prove to
\[
    AE \sin (90 - B) = BD \sin (90 - A).
\]
Then, we notice that $AE = AB \sin (90 - A)$ and $BD = AB \sin (90 - B)$ by drawing altitudes, giving us the result.
\subsection*{Problem 5.19 (EGMO 2013/1)}
Let $a$, $b$, $c$ denote the side lengths of $ABC$ in their usual way. We can compute
\begin{align*}
    AD^2 &= c^2 + 4a^2 - 4ac \cos B \\
    BE^2 &= c^2 + 4b^2 + 4bc \cos A.
\end{align*}
(The $+$ is not a mistake in the second line there!) Equating the two, we get $a^2 - ac \cos B = b^2 + bc \cos A$. Using the Law of Cosines but solving for angles, we get
\begin{align*}
    \cos B &= \frac{a^2 + c^2 - b^2}{2ac} \\
    \cos A &= \frac{b^2 + c^2 - a^2}{2bc}.
\end{align*}
Plugging these back in, we can simplify to get $a^2 = b^2 + c^2$. Thus, triangle $ABC$ is right-angled.
\subsection*{Problem 5.20 (HMMT 2013)}
Let $E$ be the contact point of the incircle with $AB$, and let $M$ be the midpoint of $BC$.
Also, let $a$, $b$, and $c$ mean the usual side lengths. The condition $2a = b + c$ can also be written as $s - a = \frac{a}{2}$, where $s$ is the semiperimeter. Since $AE = s - a$ and $MC = \frac{a}{2}$, we know $AE = MC$.

We also know $\angle DCM = \angle IAE$. So, by AAS congruence, we have that triangle $AIE$ is congruent to triangle $CDM$. Therefore, $DC = AI = DI$ (by another lemma), and we are done.
\subsection*{Problem 5.21 (USAMO 2010/4)}
Notice that $I$ is the incenter.
Law of Cosines tells us
\[
    BC^2 = BI^2 + CI^2 - 2 \cdot BI \cdot CI \cos \angle BIC.
\]
Angle chasing gives us $\angle BIC = 135$. So, we have
\[
    BC^2 = BI^2 + CI^2 + \sqrt{2} \cdot BI \cdot CI.
\]
Assume $BI$ and $CI$ have integer lengths. Then $BC^2 = AB^2 + AC^2$ is not an integer. Thus, the six segments cannot all have integer lengths.
\subsection*{Problem 5.22 (Iran Olympiad 1999)}
We can rewrite the condition as $ID \cdot (\sin B + \sin C) = \frac{1}{2} AD$ (using some angle chasing). Since $ID = BD = CD$, we now use Ptolemy's theorem to get
\[
    (AB + AC) \cdot ID = AD \cdot BC.
\]
However, we know that $ID = \frac{AD}{2(\sin B + \sin C)}$, so we can plug that in and simplify to get
\[
    BC = \frac{AB + AC}{2(\sin B + \sin C)}.
\]
Using the Extended Law of Sines again, we can write $\sin B = \frac{AC}{2R}$ and $\sin C = \frac{AB}{2R}$ where $R$ is the circumradius. Then, the above equation simplifies to
\[
    BC = R.
\]
Using the Extended Law of Sines, this means that $\sin A = \frac{1}{2}$, so $\angle A = 30$ or $\angle A = 150$.
\subsection*{Problem 5.23 (CGMO 2002/4)}
Using the Law of Sines,
\[
    \frac{AH}{HF} = \frac{EA \sin \angle HEA}{EF \sin \angle HEF}.
\]
Note that $EC = EF$ because chord $CF$ is perpendicular to diameter $AB$. So, we rewrite our expression as
\[
    \frac{EA \sin \angle HEA}{EC \sin \angle HEF}.
\]

Simple angle chasing and trig finishes this proof:
\begin{align*}
    \frac{EA \sin \angle HEA}{EC \sin \angle HEF} &= \frac{EA \sin \angle GCB}{EC \sin \angle CBD} \\
    &= \frac{EA \sin (90 - \angle CBD)}{EC \sin \angle CBD} \\
    &= \frac{EA}{EC \tan \angle CBD} \\
    &= \frac{\tan \angle ECA}{\tan \angle CBD} \\
    &= \frac{\tan \angle CBA}{\tan \angle CBD} \\
    &= \frac{AC}{CD}.
\end{align*}

\subsection*{Problem 5.28 (IMO 2001/1)}
Let $M$ be the midpoint of $BC$, and consider right triangle $OMC$.
Since $\angle COM = \angle A$, it suffices to prove that $\angle PCO > \angle COP$, or $CP<PO$.
We claim that $CP < PM$. This is equivalent to $CP < 3PB$, or
\[ c \cos B \geq 3b\cos C. \]
This simplifies to
\[ \tan C \geq 3 \tan B. \]
Using the angle condition and some algebra, we can see that this is true.
Finally, $PM < PO$ is obvious, so we are done.
\section*{Chapter 6}
\subsection*{Problem 6.29}
We scale down to the unit circle and center our arc on the real axis. Let our arc have endpoints at $a$ and $\overline{a} = \frac{1}{a}$, where $a$ is on the unit circle. Let the other point on the circle be $b$, and the center of the circle is obviously $0$.
Then, the inscribed angle theorem is equivalent to
\[
    \arg \left(\frac{a-b}{\frac{1}{a}-b}\right) = \frac{1}{2} \arg \left(\frac{a}{\frac{1}{a}}\right).
\]
Notice that with some manipulation, this is equivalent to proving that $\frac{a-b}{1-ab}$ is real, or equal to its conjugate. Indeed, we have
\begin{align*}
    \overline{\frac{a-b}{1-ab}} &= \frac{\overline{a} - \overline{b}}{1 - \overline{ab}} \\
    &= \frac{\frac{1}{a} - \frac{1}{b}}{1 - \frac{1}{ab}} \\
    &= \frac{\frac{b - a}{ab}}{\frac{ab - 1}{ab}} \\
    &= \frac{b-a}{ab-1} \\
    &= \frac{a-b}{1-ab}.
\end{align*}
So, we are done.
\subsection*{Lemma 6.30}
If $P$ is on chord $AB$, then
\[
    \frac{p-a}{p-b} = \overline{\left(\frac{p-a}{p-b}\right)} = \frac{\overline{p} - \frac{1}{a}}{\overline{p} - \frac{1}{b}}.
\]
With enough algebraic manipulation, we can get to the result.
\subsection*{Problem 6.31}
Let $a$, $b$, $c$, and $d$ be on the unit circle. Then, we have
\begin{align*}
    h_a &= b + c + d \\
    h_b &= a + c + d \\
    h_c &= a + b + d \\
    h_d &= a + b + c.
\end{align*}
We can now see that the point $\frac{1}{2} (a + b + c + d)$ is the midpoint of $AH_A$, $BH_B$, $CH_C$, and $DH_D$, and thus the lines concur at this point.
\subsection*{Problem 6.32}
Let $x$ be the point of tangency of the incircle with $AB$, $y$ be that of $BC$, $z$ be that of $CD$, and $w$ be that of $AD$.
Also, we scale down such that $w$, $x$, $y$, and $z$ are on the unit circle.
Then, using the intersection of tangents formula, we get
\begin{align*}
    a &= \frac{2wx}{w + x} \\
    b &= \frac{2xy}{x + y} \\
    c &= \frac{2yz}{y + z} \\
    d &= \frac{2wz}{w + z}.
\end{align*}
Then, the midpoint of $AC$ is
\[
    m_1 = \frac{wx}{w + x} + \frac{yz}{y + z} = \frac{wxy + wxz + wyz + xyz}{(w + x)(y + z)}.
\]
The midpoint of $BD$ is
\[
    m_2 = \frac{xy}{x + y} + \frac{wz}{w + z} = \frac{wxy + wxz + wyz + xyz}{(x + y)(w + z)}.
\]
Since we have placed $I$ at the origin, we seek to prove $\frac{m_1}{m_2}$ is real. Indeed:
\[
    \frac{m_1}{m_2} = \frac{(x + y)(w + z)}{(w + x)(y + z)}
\]
is equal to its conjugate (through enough algebraic manipulation).
\subsection*{Problem 6.33 (Chinese TST 2011)}
Let $a = A$, $b = B$, and $c = C$ in complex numbers. We can derive
\begin{align*}
    d &= \frac{1}{2} (b + c + p - bc\overline{p}) \\
    e &= \frac{1}{2} (a + c + p - ac\overline{p}) \\
    f &= \frac{1}{2} (a + b + p - ab\overline{p}) \\
    x &= 2d + a \\
    y &= 2e + b \\
    z &= 2f + c. 
\end{align*}
Plugging in the expressions for $d$, $e$, and $f$ into the last three equations and simplifying, we get
\begin{align*}
    x &= a + b + c + p - bc \overline{p} \\
    y &= a + b + c + p - ac \overline{p} \\
    z &= a + b + c + p - ab \overline{p}.
\end{align*}
Then, we have
\begin{align*}
    \frac{z-y}{z-x} &= \frac{ac \overline{p} - ab \overline{p}}{bc \overline{p} - ab \overline{p}} \\
                    &= \frac{ac-ab}{bc-ab} \\
                    &= \frac{\frac{1}{b} - \frac{1}{c}}{\frac{1}{a} - \frac{1}{c}} \\
                    &= \frac{\overline{b}- \overline{c}}{\overline{a} - \overline{c}}.
\end{align*}
Thus, triangles $XYZ$ and $ABC$ are oppositely similar.
\subsection*{Problem 6.34 (Napoleon's Theorem)}

We will compute $o_b$ and then derive the rest using symmetry.
Notice that the magnitude of $o_b - a$ is $\frac{\sqrt{3}}{3}$ times the magnitude of $c - a$. Also, the arguments of $o_b - a$ and $c - a$ differ by $\frac{\pi}{6}$.
Assume WLOG that $A$, $B$, $C$ are arranged in a counterclockwise order (like in the diagram).
Then,
\[
    o_b - a = \left( \frac{\sqrt{3}}{2} + \frac{1}{2}i \right)\left( \frac{\sqrt{3}}{3} \right)(c - a).
\]
We can simplify this to get
\[
    o_b = \left( \frac{1}{2} - \frac{\sqrt{3}}{6}i \right)a + \left( \frac{1}{2} + \frac{\sqrt{3}}{6}i \right)c. 
\]
So by symmetry,
\begin{align*}
    o_c &= \left( \frac{1}{2} - \frac{\sqrt{3}}{6}i \right)b + \left( \frac{1}{2} + \frac{\sqrt{3}}{6}i \right)a \\
    o_a &= \left( \frac{1}{2} - \frac{\sqrt{3}}{6}i \right)c + \left( \frac{1}{2} + \frac{\sqrt{3}}{6}i \right)b.
\end{align*}

Next, we prove this triangle is equilateral. We have
\begin{align*}
    o_b - o_c &= \left( -\frac{\sqrt{3}}{3}i \right)a - \left( \frac{1}{2} - \frac{\sqrt{3}}{6}i \right)b + \left( \frac{1}{2} + \frac{\sqrt{3}}{6}i \right)c \\
    o_b - o_a &= -\left( \frac{1}{2} + \frac{\sqrt{3}}{6}i \right)a + \left( \frac{\sqrt{3}}{3} \right)b + \left( \frac12 - \frac{\sqrt{3}}{6}i \right)c.
\end{align*}
Notice that $\frac{o_b - o_a}{o_b - o_c} = \frac{1}{2} - \frac{\sqrt{3}}{2}i$, which is just a $60^\circ$ rotation. By symmetry, the other angles must also be $60$ degrees. Thus, the triangle is equilateral.
Also,
\[
    \frac{o_a + o_b + o_c}{3} = \frac{a + b + c}{3},
\]
so the center of $O_AO_BO_C$ coincides with the centroid of $ABC$.

\subsection*{Problem 6.35 (USAMO 2015/2)}
The first step is to notice that the center is the midpoint of $AO$, where $O$ is the midpoint of $AB$.
We compute using $a=-1$, $s$, and $t$ as free variables. In our world, the center of the circle on which $M$ travels on is $-\frac{1}{2}$.
We have
\[ x = \frac{1}{2} \left(-1 + s + t + \frac{s}{t}\right). \]
Also, the magnitude we want to compute is
\[ \abs{\frac{s+t}{2} - \left( - \frac{1}{2} \right)} = \frac{1}{2} \abs{s + t + 1}. \]
Notice that
\begin{align*}
    \abs{s + t + 1} ^2 &= (s + t + 1)\overline{(s + t + 1)} \\
    &= 3 + s + t + \frac{1}{s} + \frac{1}{t} + \frac{s}{t} + \frac{t}{s}.
\end{align*}
Computing the real component of $x$, which is $\frac{x + \overline{x}}{2}$, we can see that this only depends on the real
component of $x$, which gives us the result.

\subsection*{Problem 6.36 (MOP 2006)}
I initially solved this problem by encoding the parallel condition as $ad = be = cf$, but a nicer way to solve it is to rotate the diagram such that $d = \overline{a}$, $e = \overline{b}$, and $f = \overline{c}$. This encodes the parallel condition and makes the computation much easier.

\subsection*{Problem 6.37 (USA January TST for IMO 2014)}
Notice that $W$ is the midpoint of $A$ and the orthocenter of triangle $ABD$. Using this, we can compute
\begin{align*}
    w &= a + \frac{b + d}{2} \\
    x &= b + \frac{a + c}{2} \\
    y &= c + \frac{b + d}{2} \\
    z &= d + \frac{a + c}{2}.
\end{align*}
Then, we can also compute the conjugates:
\begin{align*}
    \overline{w} &= \frac{1}{a} + \frac{b + d}{2bd} \\
    \vdots
\end{align*}
Shoelace bash gives us the desired result. (The computation takes around 10 minutes, but be sure to take advantage of cyclic symmetry.)

\section*{Chapter 7}
\subsection*{Problem 7.32}
We have $I = (a:b:c)$ and $G = (1:1:1)$. Then, we compute $N$.
Let $D$ be the contact point of the incircle with $BC$. Then, we know $BD = s - b$ and $CD = s - c$.
Let $D'$ be the contact point of the $A$-excircle with $BC$. We know $D'$ is the reflection of $D$ over
the midpoint of $BC$, so $D' = (0 : s-b : s-c)$.
Similarly, $E' = (s-a : 0 : s-c)$ and $F' = (s-a : s-b : 0)$.
We can now see that $N = (s-a : s-b : s-c)$ falls on all three cevians.
Computing the determinant of the appropriate matrix easily gets us the fact that $I$, $G$, and $N$ are collinear.

Now, we prove $NG = 2GI$. Normalizing coordinates, we have $G = (\frac13, \frac13, \frac13)$, $I = (\frac{a}{2s}, \frac{b}{2s}, \frac{c}{2s})$, and $N = (1 - \frac{a}{s}, 1 - \frac{b}{s}, 1 - \frac{c}{s})$.
We can see that $N = 3G - 2I$, so we are done.
\subsection*{Problem 7.33 (IMO 2014/4)}
We use similar triangles to compute $P$ and $Q$, and then it is quite straightforward to compute the intersection point as $(-a^2:2b^2:2c^2)$ which satisfies the equation of the circumcircle.
\subsection*{Problem 7.34 (EGMO 2013/1)}
The points are easy to compute. Then, use displacement vectors to find
\begin{align*}
    \abs{AD}^2 &= 2a^2 + 2b^2 - c^2, \\
    \abs{BE}^2 &= -2a^2 + 6b^2 + 3c^2.
\end{align*}
Setting them equal, we get $a^2 = b^2 + c^2$, so $ABC$ is a right triangle.
\subsection*{Problem 7.35 (ELMO Shortlist 2013)}
Set $D = (0,m,n)$ where $m + n = 1$. Use the general form of a circle and compute everything. The result is straightforward.
\subsection*{Problem 7.36 (IMO 2012/1)}
The difficulty in this problem mainly lies in algebraic manipulation.

We start by computing $J = (-a : b : c)$ and $M = (0 : s-b : s-c)$.
Notice that $KB = s-c$ and $KA = s$. From this, we can deduce $K = (c-s : s : 0)$.
Similarly, $L = (b-s : 0 : s)$.

Now, we set out to compute $F$. Since $F$ lies on line $BJ$, we know that it can be written in the form $(-a : t : c)$ for some $t$. We also know $F$, $M$, and $L$ are collinear, so we have the equation
\[
\begin{vmatrix}
    -a & t & c \\
    0 & s-b & s-c \\
    b-s & 0 & s
\end{vmatrix}
= 0 \implies t = \frac{-as + c(s-b)}{s-c}.
\]
At this point, continuing with the computation leads to very messy expressions.
We wonder if the expression for $t$ can be simplified. Indeed, after some algebra:
\[ \frac{-as + c(s-b)}{s-c} = -(a+c). \]
So, we have $F = (-a : -(a+c) : c) = (a : a + c : -c)$. Similarly, $G = (a : -b : a + b)$.

Now, we have pretty much finished the problem. Computing $S$ and $T$ and then the midpoint of $ST$ gives $M$, so we are done.
\subsection*{Problem 7.37 (Shortlist 2001/G1)}
Start by taking a homothety so that the squares are outside the triangle. Suppose this homothety takes $A_1$ to $P$. Then, we can compute $P$ using Conway's formula. We end up getting that points on $AP$ can be parametrized as
\[ (t_1 : S_C + S : S_B + S). \]
Similarly, points on $BB_1$ can be written as
\[ (S_C + S : t_2 : S_A + S) \]
and points on $CC_1$ can be written as
\[ (S_B + S : S_A + S : t_3). \]
It is clear that the point of concurrency is
\[ \left(\frac{1}{S_A + S} : \frac{1}{S_B + S} : \frac{1}{S_C + S}\right). \]
\subsection*{Problem 7.38 (USA TST 2008/7)}
We want to prove that the intersection of $(AQR)$ and the isogonal of $AG$ does not depend on the choice of $P$.

Let $P = (0, m, n)$ where $m + n = 1$. Then, it is easy to see that $Q = (m, 0, n)$ and $R = (n, m, 0)$.
Next, we find the equation of $(AQR)$. Using the general form of a circle and plugging in values, we get
that the desired equation is
\[ -a^2yz - b^2zx - c^2xy + (c^2ny + b^2mz)(x + y + z) = 0. \]
Now, we find that the isogonal of $AG$ can be parametrized as $(t : 3b^2 : 3c^2)$ using Lemma 7.6. Plugging this
into the equation for the circle, we notice that $m$ and $n$ cancel out, and the resulting expression does not
depend on the choice of $P$, so we are done.

Note: after the fact, I realized that the isogonal of $AG$ is just the $A$-symmedian, which we already know can be parametrized as $(t : b^2 : c^2)$.
\subsection*{Problem 7.39 (USAMO 2001/2)}
It is well known that $D_2 = (0 : s-b : s-c)$ and $E_2 = (s-a : 0 : s-c)$. We can then deduce that
$P = (s-a : s-b : s-c)$. Also, using a lemma from Chapter 4, we know $QD_1$ is a diameter of the incircle, i.e., $Q$ is the reflection of $D_1$ over $I$.

Since we know $D_1 = (0 : s-c : s-b)$ and $I = (a : b : c)$, we can calculate $Q = \left( \frac{a}{s}, \frac{b}{s} - \frac{s-c}{a}, \frac{c}{s} - \frac{s-b}{a} \right)$ and verify that $\overrightarrow{AQ} = \overrightarrow{PD_2}$, finishing the problem.
\subsection*{Problem 7.40 (USA TSTST 2012/7)}
Through angle chasing, we can reduce this problem to trying to show $\overline{AD}$ is parallel to $\overline{NM}$.
Because $AD$ passes through the incenter $(a : b : c)$, it is easy to see that $D = (0 : b : c)$.
Also, we know $M = (0 : 1 : 1)$. Now, we turn our attention to computing $N$.

Through some work, we find that the equation of $(ADM)$ is
\[ -a^2yz - b^2zx - c^2xy + \left( \frac{a^2c}{2(b+c)}y + \frac{a^2b}{2(b+c)}z \right)(x + y + z) = 0. \]
We can solve for $Q$ and $P$ to get, after a lot of algebra:
\begin{align*}
    Q &= (a^2 : 2c(b+c) - a^2 : 0), \\
    P &= (a^2 : 0 : 2b(b+c) - a^2).
\end{align*}
We can then calculate $N = (a^2(b+c) : 2bc(b+c) - a^2b : 2bc(b+c) - a^2c)$.

The displacement vector $\overrightarrow{NM} = (-a^2(b+c) : a^2b : a^2c)$. Also, the displacement vector $\overrightarrow{AD} = (-(b + c) : b : c)$. It is clear that $\overline{AD}$ and $\overline{NM}$ are parallel, so we are done.
\subsection*{Problem 7.41}
Using properties of angle bisectors, we can compute $P = (a : 0 : b-a) = (\frac{a}{b} : 0 : 1 - \frac{a}{b})$ and
$Q = (a : c-a : 0) = (\frac{a}{c} : 1 - \frac{a}{c} : 0)$. Then, using the theorem for generalized perpendicularity, we can obtain the result.
\subsection*{Lemma 7.42}
Using the mixtilinear incircle configuration, we find that the concurrency point of $AT_A$, $BT_B$, and $CT_C$ is
the isogonal conjugate of the Nagel point. Since the Nagel point has coordinates $(s-a : s-b : s-c)$, the point
of concurrency is $(\frac{a^2}{s-a} : \frac{b^2}{s-b} : \frac{c^2}{s-c}) = (a^2(s-b)(s-c) : b^2(s-a)(s-c) : c^2(s-a)(s-b))$.

All that remains is to show that this point is collinear with $O$ and $I$, which is relatively easy to do by showing the determinant of the matrix is $0$.

\section*{Chapter 8}
\subsection*{Problem 8.23}
Simply invert around $C$. The four points become a rectangle.

\subsection*{Problem 8.24}
Inverting around $A$, we get a structure with two lines and two circles in between. Similar triangles finishes the problem.

\subsection*{Problem 8.25}
Using the inverting the circumcenter lemma, we invert around $P$. The resulting problem is solvable by noticing
the homothety + Simson line, or complex bashing.

\subsection*{Problem 8.26 (BAMO 2008/4)}
Inverting around $D$, the problem becomes equivalent to a simple problem from Chapter 3, which I already solved.

\subsection*{Problem 8.27 (Iran Olympiad 1996)}
Invert around the circle. Then, $AC$ and $BD$ intersect at $K^*$, and we wish to prove that $\angle K^*M^*O = 90$,
where $M^*$ is the second intersection point of $(COD)$ with line $AB$.

Let $M'$ be the phantom point which is the foot of the perpendicular from $K^*$ to line $AB$. We wish to show
$COM'D$ is cyclic. Notice through angle chasing that $C$ is the foot from $B$ to $AK^*$, and similarly for $D$.
Thus, $CM'D$ is the orthic triangle of triangle $K^*AB$, and we have (through angle chasing):
\[ \angle OM'D = \angle AM'D = \angle AKD = \angle AKM' + \angle M'KB = \angle OCB + \angle BCD = \angle OCD. \]
(One of the steps in that equation is significantly more involved than the rest.) Therefore, we are done.

Alternate ending: notice that $(COD)$ is the nine-point circle of triangle $K^*AB$. So, $M^*$ must be the foot of
the altitude.

\subsection*{Problem 8.28 (Shortlist 2003/G4)}
After inverting around $P$, it is a trivial computational problem using the inversive distance formula. Specifically,
begin by noticing that the image of $ABCD$ is a parallelogram.

\subsection*{Problem 8.29}
Inverting about the incircle takes $(ABC)$ to the nine-point circle of $DEF$. Thus, $O$, $I$, and the nine-point
center of $DEF$ are collinear. This means that line $OI$ is the Euler line of triangle $DEF$. Since $G_1$ also
lies on this line, we are done.

\subsection*{Problem 8.30 (NIMO 2014)}
Notice that $Q$ is just the antipode of $A$ on $(ABC)$. Let line $QI$ intersect $(ABC)$ again at $X$. Notice that
$AXFIE$ is cyclic. Now, consider an inversion around the incircle. The circle $(AXFIE)$ is mapped to line $EF$, and
$(ABC)$ is mapped to the nine-point circle of $DEF$. Since $X^*$ must be on line $EF$, it must coincide
with point $P$ since $X$, $P$, and $I$ are collinear. But since $X^*$ also lies on the nine-point circle of $DEF$,
and $X \neq A$, we have that $X^* = P$ must be the foot of the altitude from $D$ to $EF$, and so we are done.
\section*{Chapter 9}
\subsection*{Problem 9.42 (USA TSTST 2012/4)}
Let $H$ be the orthocenter. Brocard's theorem applied to quadrilaterial $BCB_1C_1$ yields that $D$ is the
orthocenter of triangle $AA_2H$, meaning that line $DH$ is perpendicular to line $AA_2$. Similarly, we can see
that all the perpendicular lines pass through $H$, so they are concurrent.
\subsection*{Problem 9.43}
Let $F$ be the reflection of $B$ over $O$. Notice that $ABCF$ is a rectangle, $E$ is the intersection point of lines
$AF$ and $CD$, and lines $BF$ and $AC$ intersect at $O$. Therefore, Pascal's theorem on $BDCAAF$ gives the result.
\subsection*{Problem 9.44 (Canada 1994/5)}
Trivial using the Right Angles and Bisectors lemma (Lemma 9.18).
\subsection*{Problem 9.45 (Bulgarian Olympiad 2001)}
Let $F$ be the midpoint of $AB$, and let $X$ be the intersection point of the tangents to $k$ through $C$ and $E$.
Let $G$ be the intersection point of lines $BD$ and $EC$.

Notice that $BEDC$ is a harmonic quadrilateral, and in particular, $(B,D;G,X)=-1$. Projecting through $C$ onto
line $AB$, we have $(B,A;F,\overline{CX}\cap\overline{AB})=-1$. Since $F$ is the midpoint of $AB$, we must have
that lines $CX$ and $AB$ are parallel, which quickly leads to the desired conclusion.
\subsection*{Problem 9.46 (ELMO Shortlist 2012)}
If $AB = AC$, then we are done by symmetry. Otherwise, let $K$ be the intersection point of lines $IP$ and $BC$.
Notice that $K$ is the inverse of $P$ with respect to the incircle, and thus, $A$ lies on the polar of $K$.
By La Hire's theorem, we know that $K$ lies on the polar of $A$. In other words, if $E$ and $F$ are the contact
points of the incircle with sides $AC$ and $AB$, respectively, then $K$ lies on line $EF$.

It is well known that the cevians $AD$, $BE$, and $CF$ concur, so we can use the concurrent cevians lemma to deduce
that $(E,D;B,C) = -1$. Finally, since $\angle EPD = 90$, the right angles and bisectors lemma tells us that
$\angle BPD = \angle DPC$.

\subsection*{Problem 9.47 (IMO 2014/4)}
Let $X_1$ be the intersection point of $(ABC)$ with line $BM$, and let $X_2$ be that intersection point with line
$CN$. We want to show $X_1 = X_2$.

Some trivial angle chasing reveals that line $OB$ is perpendicular to line $AP$ and line $OC$ is perpendicular to
line $AQ$.
Projecting through $B$, we see
\[ -1 = (A,M;P,P_\infty) = (A,X_1;C,B),\]
since $\overline{BP_\infty}$ is the tangent at $B$.

Similarly, we have $(A,X_2;B,C) = -1$. Thus, $X = X_1 = X_2$ is the point that makes $ABXC$ harmonic, and we are
done.
\subsection*{Problem 9.48 (Shortlist 2004/G8)}
Clearly, $N$ and $M$ must be on opposite sides of chord $AB$. Let $N'$ be the intersection point of line $EF$ and
$(ABM)$ which is not on the same side of chord $AB$ as $M$. Then, we wish to prove $AMBN'$ is harmonic, and by
the uniqueness of harmonic conjugates, we will be done.

Let $P = \overline{EF} \cap \overline{CD}$ and $G = \overline{AB} \cap \overline{CD}$. Using the midpoint lengths lemma and Power of a Point, we see that $P$
lies on $(ABM)$. We also know that $(G,P;C,D) = -1$. Thus,
\[ -1 = (G, P; C, D) =^E (G, \overline{EF} \cap \overline{AB}; B, A) =^P (M, N'; B, A). \]
\subsection*{Problem 9.49 (Sharygin 2013)}
Let $M$ be the midpoint of $AB$, and let $D$ be the foot of the perpendicular from $I$ to $CM$. Notice that since
$K$ lies on the polar of $C$, by La Hire's theorem, $C$ lies on the polar of $K$. In other words, $CM$ is the polar
of $K$. Then, we know
\[ -1 = (B', A'; \overline{A'B'} \cap \overline{CM}, K) =^C (A,B;M,\overline{CK} \cap \overline{AB}). \]
Since $M$ is the midpoint of $AB$, $\overline{CK} \cap \overline{AB}$ must be the point at infinity, and
we are done.

\section*{Chapter 10}
\subsection*{Problem 10.17 (NIMO 2014)}
We will show that $R$, $M$, and $S$ are collinear, from which the result follows easily (perhaps by congruent
triangles).
We know (using directed angles)
\[ \angle SBM = \angle SHM = \angle QHM = \angle QCM = \angle C. \]
Since $\angle MBH = 90 - \angle C$, we know $\angle SBH$ is right. Similarly, $\angle RCH$ is right.
Thus, $\angle SMH = \angle SBH = \angle RCH = \angle RMH$, so we are done.

\subsection*{Problem 10.18 (USAMO 2013/1)}
Draw the Miquel point $M$. Through angle chasing, we get $\triangle MYX \sim \triangle MBP$ and $\triangle MYZ \sim \triangle MBC$ with
the same scale factor $MY/MB$. So, we are done.

\subsection*{Problem 10.19 (Shortlist 1995/G8)}
Let $\omega_{XY}$ denote the circle with diameter $XY$. Notice that the orthocenter of triangle $EAD$ is the
radical center of circles $\omega_{AD}, \omega_{AB},$ and $\omega_{CD}$. Thus, it lies on the radical axis of
$\omega_{AB}$ and $\omega_{CD}$. Notice that the orthocenter of triangle $EBC$ is the radical center of circles
$\omega_{AB}, \omega_{BC},$ and $\omega_{CD}$. Thus, it also lies on the radical axis of $\omega_{AB}$ and $\omega_{CD}$.
Point $F$ also lies on this radical axis because of Power of a Point. So, $F$ and the two orthocenters lie on
the same line.

\subsection*{Problem 10.20 (USA TST 2007/1)}
Take the quadrilateral $APDX$. We know that $Q$ is the Miquel point of this quadrilateral, since $Q$ is the
second intersection of $(BPD)$ and $(CAP)$. Thus, $AXQB$ is cyclic.

Then, we have (using directed angles)
\[ \angle QYP = \angle QAP = \angle QAB = \angle QXB. \]
Since $\overline{XB} \parallel \overline{YP}$, we must have that $Q$, $X$, and $Y$ are collinear.

Similarly,
\[ \angle QZP = \angle QBP = \angle QBA = \angle QXA, \]
so points $Q$, $X$, and $Z$ are collinear.

\subsection*{Problem 10.21 (USAMO 2013/6)}
This problem is 100000 MOHS so I can't really write it up

\subsection*{Problem 10.22 (USA TST 2007/5)}
The length conditions can be interpreted by drawing a circle centered at $T$ passing through $B$.

We show that $A$ is the Miquel point of $BB_1C_1C$. Let $Q = \overline{BB_1} \cap \overline{CC_1}$. Then, by the
three tangents configuration, $Q$ lies on $(ABC)$. Fixing $BB_1C_1C$, only one point $A$ satisfies $\angle BAC$ being acute,
$A$ being on $(QBC)$, and $\angle TAS = 90$. The Miquel point also satisfies these criteria whenever $BB_1C_1C$
is a quadrilateral following from the problem statement (based on various properties listed in EGMO), so $A$
must be the Miquel point.

Note: for me, the motivation for $A$ being a Miquel point came from the fact that $E = \overline{B_1C} \cap \overline{C_1B}$
lies on $(ABC)$ when drawn and looks like the inverse of $A$ (orthogonal circles).

\subsection*{Problem 10.23 (IMO 2005/5)}
Let $M$ be the Miquel point of self-intersecting quadrilateral $BCAD$. In other words, it is the second intersection
of $(PAD)$ and $(PBC)$. We claim that all of the circumcircles of triangles $PQR$ pass through $M$.
Note that $M$ is the center of the spiral similarity taking $A,F,D$ to $C,E,B$.
Then, $M$ is also the center of the spiral similarity taking $AF$ to $CE$, so $M$ is the second intersection of
$(AFR)$ and $(CER)$. This means that $RMEC$ is cyclic. Similarly, $QMBE$ is cyclic. A simple directed angle chase finishes
the proof:
\[ \angle RMQ = \angle RME + \angle EMQ = \angle RCE + \angle EBQ = \angle PCB + \angle CBP = \angle CPB = \angle RPQ. \]

\subsection*{Problem 10.24 (USAMO 2006/6)}
Consider $M$, the center of the spiral similarity taking $AD$ to $BC$. Since $\frac{AE}{ED} = \frac{BF}{FC}$, this
spiral similarity also takes $E$ to $F$. Thus, $M$ is Miquel point of complete quadrilaterals $ABEF$ and $EFDC$,
which all four circles must pass through.

\section*{Chapter 11}
\subsection*{Problem 11.1 (Canada 2000/4)}
Let $\alpha = \angle ADB$ and $\beta = \angle CDB$. Through angle chasing, we compute $\angle BAD = 180 - \alpha - 2\beta$
and $\angle BCD = 180 - 2\alpha - \beta$. Using the law of sines on $\triangle ABD$:
\[ \frac{AB}{\sin \alpha} = \frac{BD}{\sin (180 - 2\beta - \alpha)} = \frac{BD}{\sin (2\beta + \alpha)}. \]
Similarly,
\[ \frac{BC}{\sin \beta} = \frac{BD}{\sin (2\alpha + \beta)}. \]
Since $AB = BC$, we conclude that
\[ \frac{\sin \alpha}{\sin \beta} = \frac{\sin (2\beta + \alpha)}{\sin (2\alpha + \beta)}. \]
Cross multiplying and using product-to-sum, this simplifies to
\[ \cos(3\alpha + \beta) = \cos(3\beta + \alpha). \]
For a non-degenerate quadrilateral, we must have $\alpha, \beta > 0$ and $\alpha + \beta < 90$. So, we see that the only
solution is $\alpha = \beta$, and by symmetry, $AD = CD$.

\subsection*{Problem 11.2 (EGMO 2012/1)}
First, notice through angle chasing that $\angle FDE = 180 - 2\angle A$. Let $I$ be the incenter of triangle $DEF$.
Since it is easy to see that line $ID$ is perpendicular to side $BC$, we wish to show that $K$, $I$, and $D$ are
collinear.

More angle chasing reveals that $AFIE$ is cyclic. Then, by the incenter-excenter lemma on $\triangle DEF$, we are done.

\subsection*{Problem 11.3 (ELMO 2013/4)}
First, we claim that $BE = BC$. Since
\[ \angle AEB = \angle ACE = \angle RCS = \angle RBS = \angle RBA, \]
we know that $S$ lies on the angle bisector of $\angle B$ in isosceles $\triangle BER$. Thus, by symmetry,
\[ \angle BCS = \angle BRS = -\angle BES, \]
so $BE = BC$.

Next, we claim $K$ is the incenter of $\triangle ELD$. It suffices to show that $\angle REL = \angle DER$.
We have
\[ \angle REL = \angle REB + \angle BEL = \angle BRE + \angle ECB = \angle DRE + \angle EDR = \angle DER. \]

Finally,
\[ \angle ELK = \frac12 \angle ELD = \angle BLC, \]
and it is clear that $\triangle BLC \sim \triangle BCD$, so we are done.

\subsection*{Problem 11.4 (Sharygin 2012)}
First, it is easy to see that $C_1$ is the midpoint of $BC$ and $A_1$ is the midpoint of $BA$.

Notice that $\overline{C_1C_2}$ bisects $\angle CC_1A_1$ using basic angle chasing. Similarly,
$\overline{A_1A_2}$ bisects $\angle AA_1C_1$.

Let $X$ be the intersection point of lines $C_1C_2$ and $A_1A_2$. Then, a homothety at $B$ with scale factor $2$
takes $X$ to the $B$-excenter, so the result is obvious.

\subsection*{Problem 11.5 (USAMTS)}
Draw $I$, the incenter of triangle $ABD$. The key step is to notice that $IBCD$ is cyclic; the rest of the problem is easy.

\subsection*{Problem 11.6 (MOP 2012)}
First, we see that $H$ lies on $\gamma$. Then, notice that inverting around $B$, this problem inverts to itself. We see that under this inversion, $P$ is sent to $Q$ and
vice versa. Thus, $B$, $P$, and $Q$ are collinear.

\subsection*{Problem 11.7 (Sharygin 2013)}
Let $K$ be the midpoint of $BC$. We wish to show $DKEN$ is cyclic. Let $A'$ be the reflection of $A$ over $K$. We claim that $DA'EM$ is cyclic.

First, we show $FA'TM$ to be cyclic. Because of the parallel condition and symmetry, $DFA'A$ is an isosceles trapezoid and
\[ \angle FA'A = \angle A'FD = -\angle ADF = -\angle ETF = -\angle MTF = \angle FTM, \]
so $FA'TM$ is cyclic. Then, the radical axis theorem tells us that $DA'EM$ is cyclic.

Finally, we have $AD \cdot AE = AA' \cdot AM = AK \cdot AN$, so we are done.

\subsection*{Problem 11.8 (ELMO 2012/1)}
Let $M$ be the midpoint of $BC$. It is well known that $B$, $F$, $E$, and $C$ lie on a circle centered at $M$ and lines $ME$ and $MF$ are tangent to $\omega$. Thus, line $ME$ is tangent to $w_1$ and line $MF$ is tangent to $w_2$, so $M$ has the same power with respect to $w_1$ and $w_2$. $D$ also has the same power $0$ with respect to both circles, so $\overline{DM} = \overline{BC}$ must be the radical axis of the two circles, making the result obvious.

\subsection*{Problem 11.9 (Sharygin 2013)}
Let $L$ be the midpoint of $BC$ and $P$ be the midpoint of $QR$. Set $AB = 2x$, $CD = 2y$, and $BC = 2l$. Chasing lengths using power of a point, we get
\[ PL = \frac{y^2-x^2}{2l} \]
and $KL = \frac{x+y}{2}$.

Let $E$ be the foot of the perpendicular from $B$ to $DC$. Then, we see by SAS similarity that $\triangle KPL \sim \triangle BEC$. Thus, line $KP$ is perpendicular to line $BC$. Finally, SAS congruence finishes up the problem.
\end{document}
