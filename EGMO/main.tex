\documentclass{scrartcl}
\usepackage{graphicx} % Required for inserting images
\usepackage{amsmath}
\usepackage{listings}
\usepackage{evan}

\title{EGMO Solutions}
\author{Michael Middlezong}

\begin{document}

\maketitle

\section*{Chapter 4}
\subsection*{Problem 4.48 (Japanese Olympiad 2009)}
Notice $APOQ$ is cyclic. This can be proven using the homothety at $Q$. Then, notice $POQ$ is isosceles and the result shortly follows.
\subsection*{Problem 4.49}
Let ray $AE$ intersect the circumcircle at $W$. Because $\angle BAT = \angle CAE = \angle CAW$, we know arc $BT$ has the same measure as arc $CW$.

Now, extend ray $TD$ to hit the circumcircle at $V$. Line $TV$ is just the reflection of line $WA$ across the perpendicular bisector of $BC$, because $BD = CE$ and that arc $BT$ equals arc $CW$.

Thus, arcs $BA$ and $CV$ have the same measure, and the result follows.
\subsection*{Problem 4.50 (Vietnam TST 2003/2)}
Let $I_A, I_B, I_C$ denote the excenters.
We know from a lemma in this chapter that line $A_0D$ is just line $DI_A$, and so forth. Also, we can see that line $DF$ is parallel to line $I_AI_C$. Let $Z$ be the intersection point of lines $DI_A$ and $FI_C$. Then, a homothety at $Z$ takes $F$ to $I_C$ and $D$ to $I_A$. This homothety also takes $E$ to $I_B$ for the same reason. So, lines $DI_A$, $FI_C$, and $EI_B$ concur at $Z$. For the $OI$ part, notice that $O$ is the nine-point center of triangle $I_AI_BI_C$, and Euler line leads to the result.
\subsection*{Problem 4.51 (Sharygin 2013)}
Let $M$ be the midpoint of $AB$. From a previous lemma, we know $CM$, $A'B'$, and $C'I$ are concurrent at a point $X$. Notice that $X$ is also the orthocenter of triangle $CIK$. Thus, line $IX$ is perpendicular to $CK$. However, line $IX$ is also perpendicular to $AB$, so $AB \parallel CK$.
\subsection*{Problem 4.52 (APMO 2012/4)}
Let $H'$ be $H$ reflected over $D$, and $H''$ be $H$ reflected over $M$. It is well known that $H'$ and $H''$ lie on the circumcircle of $ABC$. By PoP, $HE \cdot HH'' = HA \cdot HH'$.
Dividing both sides by two, we obtain the equation $HE \cdot HM = HA \cdot HD$. In other words, $AEDM$ is cyclic.

Now, we claim triangle $ABF$ is similar to triangle $AMC$. We know $\angle ACM = \angle ACB = \angle AFB$.

Also, $\angle AMC = \angle AMD = \angle AED = \angle AEF = \angle ABF$ (using directed angles). Thus, the two triangles are similar, and it follows that $AF$ is a symmedian. 
Finally, the desired result is a well-known consequence of $AF$ being a symmedian.
\subsection*{Problem 4.53 (Shortlist 2002/G7)}
As always, we can remove $M$ from our diagram by noting that line $MK$ is the same as line $KI_A$.
Let $Q$ be the midpoint of $KI_A$. We claim $BNCQ$ is cyclic.
Let $S$ be the midpoint of $NK$. Since $\angle ISI_A = \angle IBI_A = 90$ (well known), we know $S$ lies on the circle containing $B$, $I$, $C$, and $I_A$ (this circle being from a common configuration). By PoP, $KS \cdot KI_A = KB \cdot KC$.
However, we know $KS \cdot KI_A = KN \cdot KQ$. Thus, $BNCQ$ is cyclic.

Let $P$ be the circumcenter of $BCN$. Notice that since $BK = XC$, we have $QB = QC$ and thus $QP$ is the perpendicular bisector of $BC$. In other words, $Q$ is the arc midpoint of arc $BC$ on the circumcircle of $BCN$. Consider a homothety at $N$ that takes $K$ to $Q$. This homothety must also take $I$ to $P$, finishing the proof.
\section*{Chapter 5}
\subsection*{Problem 5.16 (Star Theorem)}
Using the Law of Sines, we write
\[
    \prod_{i = 1}^{5} X_i A_{i + 2} = \prod_{i = 1}^{5} \frac{A_{i + 2}A_{i + 3}}{\sin \angle A_{i+2}X_iA_{i+3}}\sin \angle A_{i+2}A_{i+3}X_i
\]
and
\begin{align*}
    \prod_{i = 1}^{5} X_i A_{i + 3} &= \prod_{i = 1}^{5} \frac{A_{i + 2}A_{i + 3}}{\sin \angle A_{i+2}X_iA_{i+3}}\sin \angle A_{i+3}A_{i+2}X_i \\
    &= \prod_{i = 1}^{5} \frac{A_{i + 2}A_{i + 3}}{\sin \angle A_{i+2}X_iA_{i+3}}\sin \angle A_{i+1}A_{i+2}X_{i-1}.
\end{align*}
Notice that this is the same expression by re-indexing. Thus, we are done.
\subsection*{Problem 5.17}
We know the length of the exradius $r_A$ is $\frac{sr}{s-a}$. Then, simply use Heron's formula and $A = sr$.
\subsection*{Problem 5.18 (APMO 2013/1)}
WLOG we will just prove triangles $ODB$ and $OAE$ have the same area, and then we can get three pairs from symmetry. We note that $OB$ and $OA$ have the same length, so we just need to compare the heights of the altitudes from $D$ and $E$ to their respective sides. So, using some angle chasing and trigonometry, we can reduce what we are trying to prove to
\[
    AE \sin (90 - B) = BD \sin (90 - A).
\]
Then, we notice that $AE = AB \sin (90 - A)$ and $BD = AB \sin (90 - B)$ by drawing altitudes, giving us the result.
\subsection*{Problem 5.19 (EGMO 2013/1)}
Let $a$, $b$, $c$ denote the side lengths of $ABC$ in their usual way. We can compute
\begin{align*}
    AD^2 &= c^2 + 4a^2 - 4ac \cos B \\
    BE^2 &= c^2 + 4b^2 + 4bc \cos A.
\end{align*}
(The $+$ is not a mistake in the second line there!) Equating the two, we get $a^2 - ac \cos B = b^2 + bc \cos A$. Using the Law of Cosines but solving for angles, we get
\begin{align*}
    \cos B &= \frac{a^2 + c^2 - b^2}{2ac} \\
    \cos A &= \frac{b^2 + c^2 - a^2}{2bc}.
\end{align*}
Plugging these back in, we can simplify to get $a^2 = b^2 + c^2$. Thus, triangle $ABC$ is right-angled.
\subsection*{Problem 5.20 (HMMT 2013)}
Let $E$ be the contact point of the incircle with $AB$, and let $M$ be the midpoint of $BC$.
Also, let $a$, $b$, and $c$ mean the usual side lengths. The condition $2a = b + c$ can also be written as $s - a = \frac{a}{2}$, where $s$ is the semiperimeter. Since $AE = s - a$ and $MC = \frac{a}{2}$, we know $AE = MC$.

We also know $\angle DCM = \angle IAE$. So, by AAS congruence, we have that triangle $AIE$ is congruent to triangle $CDM$. Therefore, $DC = AI = DI$ (by another lemma), and we are done.
\subsection*{Problem 5.21 (USAMO 2010/4)}
Notice that $I$ is the incenter.
Law of Cosines tells us
\[
    BC^2 = BI^2 + CI^2 - 2 \cdot BI \cdot CI \cos \angle BIC.
\]
Angle chasing gives us $\angle BIC = 135$. So, we have
\[
    BC^2 = BI^2 + CI^2 + \sqrt{2} \cdot BI \cdot CI.
\]
Assume $BI$ and $CI$ have integer lengths. Then $BC^2 = AB^2 + AC^2$ is not an integer. Thus, the six segments cannot all have integer lengths.
\subsection*{Problem 5.22 (Iran Olympiad 1999)}
We can rewrite the condition as $ID \cdot (\sin B + \sin C) = \frac{1}{2} AD$ (using some angle chasing). Since $ID = BD = CD$, we now use Ptolemy's theorem to get
\[
    (AB + AC) \cdot ID = AD \cdot BC.
\]
However, we know that $ID = \frac{AD}{2(\sin B + \sin C)}$, so we can plug that in and simplify to get
\[
    BC = \frac{AB + AC}{2(\sin B + \sin C)}.
\]
Using the Extended Law of Sines again, we can write $\sin B = \frac{AC}{2R}$ and $\sin C = \frac{AB}{2R}$ where $R$ is the circumradius. Then, the above equation simplifies to
\[
    BC = R.
\]
Using the Extended Law of Sines, this means that $\sin A = \frac{1}{2}$, so $\angle A = 30$ or $\angle A = 150$.
\subsection*{Problem 5.23 (CGMO 2002/4)}
Using the Law of Sines,
\[
    \frac{AH}{HF} = \frac{EA \sin \angle HEA}{EF \sin \angle HEF}.
\]
Note that $EC = EF$ because chord $CF$ is perpendicular to diameter $AB$. So, we rewrite our expression as
\[
    \frac{EA \sin \angle HEA}{EC \sin \angle HEF}.
\]

Simple angle chasing and trig finishes this proof:
\begin{align*}
    \frac{EA \sin \angle HEA}{EC \sin \angle HEF} &= \frac{EA \sin \angle GCB}{EC \sin \angle CBD} \\
    &= \frac{EA \sin (90 - \angle CBD)}{EC \sin \angle CBD} \\
    &= \frac{EA}{EC \tan \angle CBD} \\
    &= \frac{\tan \angle ECA}{\tan \angle CBD} \\
    &= \frac{\tan \angle CBA}{\tan \angle CBD} \\
    &= \frac{AC}{CD}.
\end{align*}
\section*{Chapter 6}
\subsection*{Problem 6.29}
We scale down to the unit circle and center our arc on the real axis. Let our arc have endpoints at $a$ and $\overline{a} = \frac{1}{a}$, where $a$ is on the unit circle. Let the other point on the circle be $b$, and the center of the circle is obviously $0$.
Then, the inscribed angle theorem is equivalent to
\[
    \arg \left(\frac{a-b}{\frac{1}{a}-b}\right) = \frac{1}{2} \arg \left(\frac{a}{\frac{1}{a}}\right).
\]
Notice that with some manipulation, this is equivalent to proving that $\frac{a-b}{1-ab}$ is real, or equal to its conjugate. Indeed, we have
\begin{align*}
    \overline{\frac{a-b}{1-ab}} &= \frac{\overline{a} - \overline{b}}{1 - \overline{ab}} \\
    &= \frac{\frac{1}{a} - \frac{1}{b}}{1 - \frac{1}{ab}} \\
    &= \frac{\frac{b - a}{ab}}{\frac{ab - 1}{ab}} \\
    &= \frac{b-a}{ab-1} \\
    &= \frac{a-b}{1-ab}.
\end{align*}
So, we are done.
\subsection*{Lemma 6.30}
If $P$ is on chord $AB$, then
\[
    \frac{p-a}{p-b} = \overline{\left(\frac{p-a}{p-b}\right)} = \frac{\overline{p} - \frac{1}{a}}{\overline{p} - \frac{1}{b}}.
\]
With enough algebraic manipulation, we can get to the result.
\subsection*{Problem 6.31}
Let $a$, $b$, $c$, and $d$ be on the unit circle. Then, we have
\begin{align*}
    h_a &= b + c + d \\
    h_b &= a + c + d \\
    h_c &= a + b + d \\
    h_d &= a + b + c.
\end{align*}
We can now see that the point $\frac{1}{2} (a + b + c + d)$ is the midpoint of $AH_A$, $BH_B$, $CH_C$, and $DH_D$, and thus the lines concur at this point.
\subsection*{Problem 6.32}
Let $x$ be the point of tangency of the incircle with $AB$, $y$ be that of $BC$, $z$ be that of $CD$, and $w$ be that of $AD$.
Also, we scale down such that $w$, $x$, $y$, and $z$ are on the unit circle.
Then, using the intersection of tangents formula, we get
\begin{align*}
    a &= \frac{2wx}{w + x} \\
    b &= \frac{2xy}{x + y} \\
    c &= \frac{2yz}{y + z} \\
    d &= \frac{2wz}{w + z}.
\end{align*}
Then, the midpoint of $AC$ is
\[
    m_1 = \frac{wx}{w + x} + \frac{yz}{y + z} = \frac{wxy + wxz + wyz + xyz}{(w + x)(y + z)}.
\]
The midpoint of $BD$ is
\[
    m_2 = \frac{xy}{x + y} + \frac{wz}{w + z} = \frac{wxy + wxz + wyz + xyz}{(x + y)(w + z)}.
\]
Since we have placed $I$ at the origin, we seek to prove $\frac{m_1}{m_2}$ is real. Indeed:
\[
    \frac{m_1}{m_2} = \frac{(x + y)(w + z)}{(w + x)(y + z)}
\]
is equal to its conjugate (through enough algebraic manipulation).
\subsection*{Problem 6.33 (Chinese TST 2011)}
Let $a = A$, $b = B$, and $c = C$ in complex numbers. We can derive
\begin{align*}
    d &= \frac{1}{2} (b + c + p - bc\overline{p}) \\
    e &= \frac{1}{2} (a + c + p - ac\overline{p}) \\
    f &= \frac{1}{2} (a + b + p - ab\overline{p}) \\
    x &= 2d + a \\
    y &= 2e + b \\
    z &= 2f + c. 
\end{align*}
Plugging in the expressions for $d$, $e$, and $f$ into the last three equations and simplifying, we get
\begin{align*}
    x &= a + b + c + p - bc \overline{p} \\
    y &= a + b + c + p - ac \overline{p} \\
    z &= a + b + c + p - ab \overline{p}.
\end{align*}
Then, we have
\begin{align*}
    \frac{z-y}{z-x} &= \frac{ac \overline{p} - ab \overline{p}}{bc \overline{p} - ab \overline{p}} \\
                    &= \frac{ac-ab}{bc-ab} \\
                    &= \frac{\frac{1}{b} - \frac{1}{c}}{\frac{1}{a} - \frac{1}{c}} \\
                    &= \frac{\overline{b}- \overline{c}}{\overline{a} - \overline{c}}.
\end{align*}
Thus, triangles $XYZ$ and $ABC$ are oppositely similar.
\subsection*{Problem 6.34 (Napoleon's Theorem)}

We will compute $o_b$ and then derive the rest using symmetry.
Notice that the magnitude of $o_b - a$ is $\frac{\sqrt{3}}{3}$ times the magnitude of $c - a$. Also, the arguments of $o_b - a$ and $c - a$ differ by $\frac{\pi}{6}$.
Assume WLOG that $A$, $B$, $C$ are arranged in a counterclockwise order (like in the diagram).
Then,
\[
    o_b - a = \left( \frac{\sqrt{3}}{2} + \frac{1}{2}i \right)\left( \frac{\sqrt{3}}{3} \right)(c - a).
\]
We can simplify this to get
\[
    o_b = \left( \frac{1}{2} - \frac{\sqrt{3}}{6}i \right)a + \left( \frac{1}{2} + \frac{\sqrt{3}}{6}i \right)c. 
\]
So by symmetry,
\begin{align*}
    o_c &= \left( \frac{1}{2} - \frac{\sqrt{3}}{6}i \right)b + \left( \frac{1}{2} + \frac{\sqrt{3}}{6}i \right)a \\
    o_a &= \left( \frac{1}{2} - \frac{\sqrt{3}}{6}i \right)c + \left( \frac{1}{2} + \frac{\sqrt{3}}{6}i \right)b.
\end{align*}

Next, we prove this triangle is equilateral. We have
\begin{align*}
    o_b - o_c &= \left( -\frac{\sqrt{3}}{3}i \right)a - \left( \frac{1}{2} - \frac{\sqrt{3}}{6}i \right)b + \left( \frac{1}{2} + \frac{\sqrt{3}}{6}i \right)c \\
    o_b - o_a &= -\left( \frac{1}{2} + \frac{\sqrt{3}}{6}i \right)a + \left( \frac{\sqrt{3}}{3} \right)b + \left( \frac12 - \frac{\sqrt{3}}{6}i \right)c.
\end{align*}
Notice that $\frac{o_b - o_a}{o_b - o_c} = \frac{1}{2} - \frac{\sqrt{3}}{2}i$, which is just a $60^\circ$ rotation. By symmetry, the other angles must also be $60$ degrees. Thus, the triangle is equilateral.
Also,
\[
    \frac{o_a + o_b + o_c}{3} = \frac{a + b + c}{3},
\]
so the center of $O_AO_BO_C$ coincides with the centroid of $ABC$.

\subsection*{Problem 6.35 (USAMO 2015/2)}
The first step is to notice that the center is the midpoint of $AO$, where $O$ is the midpoint of $AB$.
We compute using $a=-1$, $s$, and $t$ as free variables. In our world, the center of the circle on which $M$ travels on is $-\frac{1}{2}$.
We have
\[ x = \frac{1}{2} \left(-1 + s + t + \frac{s}{t}\right). \]
Also, the magnitude we want to compute is
\[ \abs{\frac{s+t}{2} - \left( - \frac{1}{2} \right)} = \frac{1}{2} \abs{s + t + 1}. \]
Notice that
\begin{align*}
    \abs{s + t + 1} ^2 &= (s + t + 1)\overline{(s + t + 1)} \\
    &= 3 + s + t + \frac{1}{s} + \frac{1}{t} + \frac{s}{t} + \frac{t}{s}.
\end{align*}
Computing the real component of $x$, which is $\frac{x + \overline{x}}{2}$, we can see that this only depends on the real
component of $x$, which gives us the result.

\subsection*{Problem 6.36 (MOP 2006)}
I initially solved this problem by encoding the parallel condition as $ad = be = cf$, but a nicer way to solve it is to rotate the diagram such that $d = \overline{a}$, $e = \overline{b}$, and $f = \overline{c}$. This encodes the parallel condition and makes the computation much easier.

\subsection*{Problem 6.37 (USA January TST for IMO 2014)}
Notice that $W$ is the midpoint of $A$ and the orthocenter of triangle $ABD$. Using this, we can compute
\begin{align*}
    w &= a + \frac{b + d}{2} \\
    x &= b + \frac{a + c}{2} \\
    y &= c + \frac{b + d}{2} \\
    z &= d + \frac{a + c}{2}.
\end{align*}
Then, we can also compute the conjugates:
\begin{align*}
    \overline{w} &= \frac{1}{a} + \frac{b + d}{2bd} \\
    \vdots
\end{align*}
Shoelace bash gives us the desired result. (The computation takes around 10 minutes, but be sure to take advantage of cyclic symmetry.)

\section*{Chapter 7}
\subsection*{Problem 7.32}
We have $I = (a:b:c)$ and $G = (1:1:1)$. Then, we compute $N$.
Let $D$ be the contact point of the incircle with $BC$. Then, we know $BD = s - b$ and $CD = s - c$.
Let $D'$ be the contact point of the $A$-excircle with $BC$. We know $D'$ is the reflection of $D$ over
the midpoint of $BC$, so $D' = (0 : s-b : s-c)$.
Similarly, $E' = (s-a : 0 : s-c)$ and $F' = (s-a : s-b : 0)$.
We can now see that $N = (s-a : s-b : s-c)$ falls on all three cevians.
Computing the determinant of the appropriate matrix easily gets us the fact that $I$, $G$, and $N$ are collinear.

Now, we prove $NG = 2GI$. Normalizing coordinates, we have $G = (\frac13, \frac13, \frac13)$, $I = (\frac{a}{2s}, \frac{b}{2s}, \frac{c}{2s})$, and $N = (1 - \frac{a}{s}, 1 - \frac{b}{s}, 1 - \frac{c}{s})$.
We can see that $N = 3G - 2I$, so we are done.
\subsection*{Problem 7.33 (IMO 2014/4)}
We use similar triangles to compute $P$ and $Q$, and then it is quite straightforward to compute the intersection point as $(-a^2:2b^2:2c^2)$ which satisfies the equation of the circumcircle.
\subsection*{Problem 7.34 (EGMO 2013/1)}
The points are easy to compute. Then, use displacement vectors to find
\begin{align*}
    \abs{AD}^2 &= 2a^2 + 2b^2 - c^2, \\
    \abs{BE}^2 &= -2a^2 + 6b^2 + 3c^2.
\end{align*}
Setting them equal, we get $a^2 = b^2 + c^2$, so $ABC$ is a right triangle.
\subsection*{Problem 7.35 (ELMO Shortlist 2013)}
Set $D = (0,m,n)$ where $m + n = 1$. Use the general form of a circle and compute everything. The result is straightforward.
\subsection*{Problem 7.36 (IMO 2012/1)}
The difficulty in this problem mainly lies in algebraic manipulation.

We start by computing $J = (-a : b : c)$ and $M = (0 : s-b : s-c)$.
Notice that $KB = s-c$ and $KA = s$. From this, we can deduce $K = (c-s : s : 0)$.
Similarly, $L = (b-s : 0 : s)$.

Now, we set out to compute $F$. Since $F$ lies on line $BJ$, we know that it can be written in the form $(-a : t : c)$ for some $t$. We also know $F$, $M$, and $L$ are collinear, so we have the equation
\[
\begin{vmatrix}
    -a & t & c \\
    0 & s-b & s-c \\
    b-s & 0 & s
\end{vmatrix}
= 0 \implies t = \frac{-as + c(s-b)}{s-c}.
\]
At this point, continuing with the computation leads to very messy expressions.
We wonder if the expression for $t$ can be simplified. Indeed, after some algebra:
\[ \frac{-as + c(s-b)}{s-c} = -(a+c). \]
So, we have $F = (-a : -(a+c) : c) = (a : a + c : -c)$. Similarly, $G = (a : -b : a + b)$.

Now, we have pretty much finished the problem. Computing $S$ and $T$ and then the midpoint of $ST$ gives $M$, so we are done.
\subsection*{Problem 7.37 (Shortlist 2011/G1)}
Start by taking a homothety so that the squares are outside the triangle. Suppose this homothety takes $A_1$ to $P$. Then, we can compute $P$ using Conway's formula. We end up getting that points on $AP$ can be parametrized as
\[ (t_1 : S_C + S : S_B + S). \]
Similarly, points on $BB_1$ can be written as
\[ (S_C + S : t_2 : S_A + S) \]
and points on $CC_1$ can be written as
\[ (S_B + S : S_A + S : t_3). \]
It is clear that the point of concurrency is
\[ \left(\frac{1}{S_A + S} : \frac{1}{S_B + S} : \frac{1}{S_C + S}\right). \]
\subsection*{Problem 7.38 (USA TST 2008/7)}
We want to prove that the intersection of $(AQR)$ and the isogonal of $AG$ does not depend on the choice of $P$.

Let $P = (0, m, n)$ where $m + n = 1$. Then, it is easy to see that $Q = (m, 0, n)$ and $R = (n, m, 0)$.
Next, we find the equation of $(AQR)$. Using the general form of a circle and plugging in values, we get
that the desired equation is
\[ -a^2yz - b^2zx - c^2xy + (c^2ny + b^2mz)(x + y + z) = 0. \]
Now, we find that the isogonal of $AG$ can be parametrized as $(t : 3b^2 : 3c^2)$ using Lemma 7.6. Plugging this
into the equation for the circle, we notice that $m$ and $n$ cancel out, and the resulting expression does not
depend on the choice of $P$, so we are done.
\end{document}
