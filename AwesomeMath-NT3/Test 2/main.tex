\documentclass{scrartcl}
\usepackage{graphicx} % Required for inserting images
\usepackage{amsmath}
\usepackage{listings}
\usepackage{evan}

\title{Test 2 Solutions}
\author{Michael Middlezong}

\begin{document}
\maketitle

\section*{Problem 1}
The main idea behind this solution is that after completing the square, we are left with an expression we have dealt with before.

Completing the square, we get
\[ \sum_{k=0}^{p-1} \left(\dfrac{ak^2 + bk + c}{p}\right) = \sum_{k=0}^{p-1} \left(\dfrac{a(k+\frac{b}{2a})^2 - \frac{b^2 - 4ac}{4a}}{p}\right). \]
Since $\frac{b}{2a}$ is a constant, the expression $k + \frac{b}{2a}$ runs through all residues mod $p$.
Therefore, we can perform the substitution $k' = k + \frac{b}{2a}$ to get
\[ \sum_{k'=0}^{p-1} \left(\dfrac{ak'^2 - \frac{b^2 - 4ac}{4a}}{p}\right). \]
Now, suppose $p \mid b^2 - 4ac$. Then, since $p \nmid 4a$ (assuming $p$ is an odd prime),
\[ \frac{b^2 - 4ac}{4a} \equiv 0 \pmod{p}. \]
In this case, our expression simplifies to
\[ \sum_{k'=0}^{p-1} \left(\dfrac{ak'^2}{p}\right) = \sum_{k'=1}^{p-1} \left(\dfrac{a}{p}\right) = (p-1)\left(\dfrac{a}{p}\right). \]
Next, consider the case where $p \nmid b^2 - 4ac$. Then, writing $C = -\frac{b^2 - 4ac}{4a}$, we are looking for
\[ \sum_{k'=0}^{p-1} \left(\dfrac{ak'^2 + C}{p}\right). \]
If $\left(\dfrac{a}{p}\right) = 1$, then for $m$ satisfying $m^2 \equiv a \pmod{p}$,
\[ \sum_{k'=0}^{p-1} \left(\dfrac{ak'^2 + C}{p}\right) = \sum_{k'=0}^{p-1} \left(\dfrac{(mk')^2 + C}{p}\right) = -1 = -\left(\dfrac{a}{p}\right). \]
Otherwise, the expression $ak'^2$ is always a quadratic nonresidue, and more specifically, it loops through all the quadratic nonresidues twice and equals zero once.

Let $d$ be any nonzero quadratic residue.
From the previous part, we have
\[ \sum_{k'=0}^{p-1} \left(\dfrac{dk'^2 + C}{p}\right) = -1. \]
Also, the expression $dk'^2$ loops through all the nonzero quadratic residues twice and equals zero once.
Together, $ak'^2$ and $dk'^2$ loop through all residues twice.
Since adding a fixed offset $C$ makes no difference, the expression
\[ \sum_{k'=0}^{p-1} \left(\dfrac{ak'^2 + C}{p}\right) + \sum_{k'=0}^{p-1} \left(\dfrac{dk'^2 + C}{p}\right) \]
is then twice the sum of the Legendre symbol for all residues, which we know to be $0$.
This means
\[ \sum_{k'=0}^{p-1} \left(\dfrac{ak'^2 + C}{p}\right) = -\sum_{k'=0}^{p-1} \left(\dfrac{dk'^2 + C}{p}\right) = -(-1) = -\left(\dfrac{a}{p}\right). \]
Therefore, if $p \nmid b^2 - 4ac$, we have
\[ \sum_{k=0}^{p-1} \left(\dfrac{ak^2 + bk + c}{p}\right) = -\left(\dfrac{a}{p}\right) \]
and thus the proof is complete.
\pagebreak
\section*{Problem 2}
\pagebreak
\section*{Problem 3}
\pagebreak
\section*{Problem 4}
First, we show that if $p \equiv -1 \pmod{8}$, then $2$ is a quadratic residue mod $p$.
Using Gauss' lemma, it suffices to count the number of residues among
\[ 2, 4, 6, \ldots, p-1 \]
that are in the interval $[\frac{p+1}{2}, p-1]$.

Since $p \equiv -1 \pmod{8}$, the endpoints of that interval are both even.
Thus, we are looking to count the numbers
\[ \frac{p+1}{2}, \frac{p+1}{2} + 2, \ldots, \frac{p+1}{2} + \frac{p-3}{2} = p-1. \]
There are $s = \frac{p-3}{4} + 1$ of them, and $s$ is even because $p-3 \equiv 4 \pmod{8}$.
Hence, Gauss' lemma yields $\left(\dfrac{2}{p}\right) = (-1)^s = 1$.

For the sake of contradiction, let $p \mid 2^n + 1$ satisfy $p \equiv -1 \pmod{8}$.
Since $2$ is a quadratic residue mod $p$, there exists $m$ such that $m^2 \equiv 2 \pmod{p}$.
Furthermore,
\[ 2^n \equiv -1 \implies m^{2n} \equiv -1 \implies m^{4n} \equiv 1 \pmod{p}. \]
If we let $d = \ord_p(m)$, the above congruences imply $d \mid \gcd(4n, p-1)$ and $d \nmid 2n$.

Since $p-1 \equiv -2 \pmod{8}$, the first condition implies $\nu_2(d) < 2$.
Therefore, $d \mid 4n \implies d \mid 2n$, giving us a contradiction.
Thus, $2^n + 1$ does not have any prime divisors of the form $8k-1$.

\pagebreak
\section*{Problem 5}

\pagebreak
\section*{Problem 6}
When $m$ is odd, $2^m - 1 \equiv 3 \pmod{4}$ and $2^m - 1 \equiv 1 \pmod{3}$, so by CRT,
\[ 2^m - 1 \equiv 7 \pmod{12}. \]
Consider a prime $p$ dividing $2^m - 1$.
Since $p > 3$, the possible residues of $p$ mod $12$ are
\[ p \equiv 1,5,7,11\pmod{12}. \]
We claim that there must exist a prime $p \equiv 5 \pmod{12}$ or $p \equiv 7 \pmod{12}$ dividing $2^m - 1$.
This is because otherwise, all primes dividing $2^m-1$ would be congruent to $1$ or $11 \equiv -1$ mod $12$, and their product could not be $7$ mod $12$.

For the sake of contradiction, assume $2^m - 1 \mid 3^n - 1$.
Then,
\[ 3^n \equiv 1 \pmod{p}. \]
We claim that $3$ is not a quadratic residue mod $p$.
If $p \equiv 5 \pmod{12}$, then quadratic reciprocity gives us
\[ \left(\dfrac{3}{p}\right) = \left(\dfrac{p}{3}\right) = \left(\dfrac{2}{3}\right) = -1. \]
On the other hand, if $p \equiv 7 \pmod{12}$, then quadratic reciprocity gives us
\[ \left(\dfrac{3}{p}\right) = -\left(\dfrac{p}{3}\right) = -\left(\dfrac{1}{3}\right) = -1. \]
In either case, $3$ is not a quadratic residue mod $p$.

Then, let $g$ be a primitive root mod $p$, and let $d$ be the order of $3$ mod $p$.
We can write $3 \equiv g^k \pmod{p}$ where $k$ is odd.
Thus,
\[ 3^d \equiv 1 \implies g^{dk} \equiv 1 \pmod{p}. \]
Since $p-1$ is the order of $g$ mod $p$, we have $p-1 \mid dk \implies 2 \mid dk$.
We know $k$ is odd, so $d$ must be even.
We then know $d \mid n$, but $n$ is odd, so this is a contradiction.
This concludes the proof.


\pagebreak
\section*{Problem 7}

\pagebreak
\section*{Problem 8}

\pagebreak
\section*{Problem 9}

\pagebreak
\end{document}
