\documentclass{scrartcl}
\usepackage{graphicx} % Required for inserting images
\usepackage{amsmath}
\usepackage{listings}

\title{Symmetric Polynomials Solutions}
\author{Michael Middlezong}

\begin{document}

\maketitle
\section*{Problem 1 (USAMO 1973/4)}
Consider the cubic polynomial $(t - x)(t - y)(t - z)$. From Newton's sums and Vieta's, this cubic polynomial must equal $t^3 - 3t^2 + 3t - 1$. The only factorization of this is $(t - 1)^3$, so the only solution must be $(x,y,z)=(1,1,1)$.

\section*{Problem 2 (Canada 1996)}
It is easy to show that $1-\alpha$, $1-\beta$, and $1-\gamma$ are the roots of the polynomial
\[
    x^3 - 3x^2 + 2x + 1
\]
using Vieta's.

Then, we can easily calculate the desired expression using Vieta's as well. The answer is $-7$.
\section*{Problem 3 (HMMT Nov 2016 Guts)}
By Newton's sums, the sum of the squares of the roots is $0$. This means each term in the requested expression is $-1$, giving us a total answer of $-4$.
\section*{Problem 5 (USAMO 1984/1)}
Notice that we can write the polynomial as
\[
    (x^2 + ax - 32)(x^2 + bx + 62)
\]
for constants $a$ and $b$. Expanding this and matching coefficients, we get the system of equations
\begin{align*}
    a + b &= -18 \\
    62a - 32b &= 200.
\end{align*}
We can solve this system to get $a = -4$, $b = -14$. We also know $k = 30 + ab$ from the earlier expansion, so $k = 86$.

\section*{Problem 6}
Consider the polynomial with roots $r + s$, $s + t$, and $r + t$. We will find its coefficients and show that it is the desired polynomial.
Using Vieta's, we can see that
\[
    A = -2(r + s + t) = -14.
\]
We can also see that
\[
    B = (r + s)(s + t) + (s + t)(r + t) + (r + s)(r + t).
\]
Expanding and simplifying with Vieta's and Newton sums, we get $B = 52$.

The $C$ term is slightly more involved, but we can use a combination of Newton sums, Vieta's, and grouping of terms to get $C = -23$.

All these terms are rational, so overall, our answer is $A + B + C = -14 + 52 - 23 = 15$.
\section*{Problem 9}
We notice that the polynomial vanishes whenever $a=b$, $a=c$, or $b=c$. So, the polynomial is divisible by $(a-b)(a-c)(b-c)$. We know the last factor must be a multiple of $a+b+c$. We can match the coefficient of $ab^3$ to get that the factored form is
\[
    (a-b)(a-c)(b-c)(-a-b-c).
\]

\section*{Problem 13 (HMMT 2023/T2)}
We can rearrange the equation $a^3 - bcd = b^3 - cda$ to get
\[
    (a - b)(a^2 + ab + b^2) = cd(b - a).
\]
If we assume to the contrary that $a$, $b$, $c$, and $d$ are pairwise distinct, this means
\[
    a^2 + ab + b^2 = -cd \implies a^2 + ab + b^2 + cd = 0.
\]
Here, the variables $a$ and $b$ can be replaced with any two of $a$, $b$, $c$, or $d$. Thus, we also have:
\[
    c^2 + cd + d^2 + ab = 0.
\]
We can conclude from these two equations that $a^2 + b^2 = c^2 + d^2$.

Notice that there was nothing special about our choices of $a$, $b$, $c$, and $d$. Using symmetry, we can deduce that $a^2 + c^2 = b^2 + d^2$.

Thus, $b^2 = c^2$. Similarly, $a^2 = b^2 = c^2 = d^2$. Therefore, we can see that $a$, $b$, $c$, and $d$ cannot be pairwise distinct.
\section*{Problem 15 (SMT 2011)}
We can notice that the polynomial $P(2x) - P(x) - 1$ has roots $x = 2^i$ for $0 \leq i \leq 2010$.
Thus, we can write
\[
    P(2x) - P(x) - 1 = c(x - 2^0)(x - 2^1)\cdots(x - 2^{2010}).
\]
Plugging in $x = 0$, we can find $\frac{1}{c} = 1 + 2 + \cdots + 2010$ (denote by $S$ this sum).

Now, let $a$ be the coefficient of the linear term in $P(x)$. Then, the linear term of $P(2x) - P(x) - 1$ is $2ax - ax = ax$. So, it suffices to find the linear coefficient of $c(x - 2^0)(x - 2^1)\cdots(x - 2^{2010})$.

For this, we can use Vieta's. We end up with
\[
    a = 2^S + 2^{S-1} + \cdots + 2^{S - 2010}.
\]
We can simplify this to $a = 2 - \frac{1}{2^{2010}}.$
\section*{Problem 18 (SMT 2013)}
Putting the three terms over a common denominator and factoring the numerator, we can find that the expression equals
\[
    a^2 + b^2 + c^2 + ab + bc + ca.
\]
We can rewrite this as $(a + b + c)^2 - (ab + bc + ca)$.

Let $x = \sqrt{3}$, $y = \sqrt{5}$, $z = \sqrt{7}$, and $S = a + b + c = x + y + z$. Then, our desired expression is
\[
    S^2 - [(S - 2x)(S - 2y) + (S - 2y)(S - 2z) + (S - 2z)(S - 2x)].
\]
We can simplify this to get the answer of
\[
    2S^2 - 4(xy + yz + zx) = 30.
\]
\section*{Problem 20 (Black MOP 2012)}
Let $\sqrt{a + h_B}$, $\sqrt{b + h_C}$, and $\sqrt{c + h_A}$ be the roots of a polynomial.

Then, we claim this polynomial also has roots $\sqrt{a + h_C}$, $\sqrt{b + h_A}$, and $\sqrt{c + h_B}$. This can be shown with Vieta's and Newton sums, along with the fact that
\[
    (a + h_B)(b + h_C)(c + h_A) = (a + h_C)(b + h_A)(c + h_B),
\]
which can be shown by expanding and simplifying using the triangle area formula.

Thus, we have three cases:
\begin{enumerate}
    \item $\sqrt{a + h_B} = \sqrt{a + h_C}$. Let $A$ be the area of the triangle. Then it is obvious that $b = c$.
    \item $\sqrt{a + h_B} = \sqrt{b + h_A}$. We can derive that $a = b$ or $ab = -1$, the latter of which is impossible.
    \item $\sqrt{a + h_B} = \sqrt{c + h_B}$. Obviously $a = c$.
\end{enumerate}
In any case, the triangle is isosceles.
\end{document}
% vim: ts=4 sts=4 sw=4 et
