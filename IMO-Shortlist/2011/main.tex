\documentclass{scrartcl}
\usepackage{graphicx} % Required for inserting images
\usepackage{amsmath}
\usepackage{listings}
\usepackage{evan}

\title{Shortlist 2011}
\author{Michael Middlezong}

\begin{document}
\maketitle

\section*{Combinatorics}
\subsection*{C1}
The answer is $(2n-1)!! = (2n-1) \cdot (2n-3) \cdots 3 \cdot 1$.
Let $a_n$ be the answer to the problem when there are $n$ weights.
We will show the recursive formula $a_n = (2n-1)a_{n-1}$, from which the desired result follows.

Consider the placement of the $1$-gram weight.
It cannot be placed on the right pan at step $1$, but it can be placed on either pan in any of the other steps.
There are thus a total of $2n-1$ ways to place the $1$-gram weight.

Then, fixing the placement of the $1$-gram weight, there are $a_{n-1}$ ways to place the remaining weights.
This is because as long as there are other weights on the balance, the $1$-gram weight cannot influence which side is heavier (think about binary representation).
So, it does not affect the placement of the other $n-1$ weights, and we can just scale by a factor of $2$ to get a problem identical to the one with $n-1$ weights.

Thus, multiplying the two, we have $a_n = (2n-1)a_{n-1}$, and we're done.

\section*{Geometry}
\subsection*{G2}
Notice that expression in the denominators is the power of $A_i$ with respect to the circle with center $O_i$, which we will denote by $\text{pow}_{\omega_i}(A_i)$.

We proceed with barycentric coordinates.
Our reference triangle will be $A_2A_3A_4$, and we will let $A_1 = (d,e,f)$, where $d+e+f=1$.
Then, let
\[ P = \text{pow}_{\omega_1}(A_1) = -a^2ef-b^2fd-c^2de. \]
By plugging in points, we find that the equation of $w_2$ is
\[ -a^2yz-b^2zx-c^2xy-\frac{P}{d}x(x+y+z) = 0, \]
and the equations of $w_3$ and $w_4$ can be found by symmetry.

It follows that
\begin{align*}
    \text{pow}_{\omega_2}(A_2) &= -\frac{P}{d}, \\
    \text{pow}_{\omega_3}(A_3) &= -\frac{P}{e}, \\
    \text{pow}_{\omega_4}(A_4) &= -\frac{P}{f},
\end{align*}
and plugging in what we have into the desired equation gives the result.

\section*{Number Theory}
\subsection*{N2}
Assume for some $x$ that $P(x)$ is only divisible by primes less than $20$.
Let $p < 20$ be a prime, and let $M = \prod_{i > j} (d_i - d_j)$.
Then, for distinct indices $i,j$,
\[ \min\{ \nu_p(x + d_i), \nu_p(x + d_j) \} = \nu_p(\gcd(x + d_i, x + d_j)). \]
Since $\gcd(x + d_i, x + d_j) \mid d_i - d_j$, we have
\[ \nu_p(\gcd(x + d_i, x + d_j)) \leq \nu_p(d_i - d_j) \leq \nu_p(M). \]
So,
\[ \min\{ \nu_p(x + d_i), \nu_p(x + d_j) \} \leq \nu_p(M) \]
for all $i \neq j$.
It follows that there is at most one $j$ such that $\nu_p(x + d_j) > \nu_p(M)$.

If we repeat this for all eight of the primes less than $20$, we will find that since there are nine possible indices $j$, at least one index $j$ will be left over.
More precisely, there exists $j$ such that $\nu_p(x + d_j) \leq \nu_p(M)$ for all primes $p < 20$, and by our assumption, this extends to all primes $p$.
So,
\[ x + d_j \mid M \implies x + d_j \leq M \implies x \leq M - d_j, \]
and thus, $x$ is bounded.
The desired result follows.

\end{document}
% vim: ts=4 sts=4 sw=4 et
