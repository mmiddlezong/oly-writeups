\documentclass{scrartcl}
\usepackage{graphicx} % Required for inserting images
\usepackage{amsmath}
\usepackage{listings}
\usepackage{evan}

\title{Shortlist 2012}
\author{Michael Middlezong}

\begin{document}
\maketitle

\section*{Combinatorics}
\subsection*{C1}
Call the act of replacing $(x,y)$ by $(y+1,x)$ a \textit{sum-increasing swap}, and call the act of replacing $(x,y)$ by $(x-1,x)$ a \textit{lifting swap}.

Let $M$ be the maximum number written on the board initially.
We claim that the maximum number written on the board never exceeds $M$.
If it did exceed $M$, then some action must have replaced $(x,y)$ by $(a,x)$, where $a > x > y$.
This is impossible given the two types of actions we have.

Thus, if there are $n$ numbers on the board, the sum of the numbers is bounded above by $nM$.
It follows that there may only be a finite number of sum-increasing swaps.

Assume FTSOC that Alice performs an infinite number of lifting swaps.
If $y = x - 1$ prior to performing a lifting swap on $(x,y)$, then we call it a type 1 lifting swap (or just a swap, as $(x,x-1) \to (x-1,x))$.
Otherwise, we call it a type 2 lifting swap.
Notice that type 2 lifting swaps increase the total sum, so there can be only finitely many of them.
Therefore, Alice must perform infinitely many type 1 lifting swaps.

Now, consider a point in Alice's infinite sequence of actions such that no more sum-increasing swaps or type 2 lifting swaps are to be done.
Such a point must exist because there are finitely many such actions.

From this point onward, every action is a type 1 lifting swap.
However, this cannot continue infinitely, so we are done.
(To show the final statement, consider the monovariant $ia_i$ where $i$ is the number in the $i$th position.)

\subsection*{C2}
The answer is $\left\lfloor \frac{2n-1}{5} \right\rfloor$.
Let $k$ be the number of pairs.
First, we show that $k > \left\lfloor \frac{2n-1}{5} \right\rfloor$ is impossible.

Consider $S$, the sum of the sums of all $k$ pairs.
By double counting, we have that the maximum possible value of $S$ is
\[ n + (n-1) + (n-2) + \dots + (n-(k-1)) \]
and that the minimum possible value of $S$ is
\[ 1 + 2 + \dots + 2k. \]
Thus, we have
\[ n + (n-1) + (n-2) + \dots + (n-(k-1)) \geq 1 + 2 + \dots + 2k. \]
Simplifying this gives $k \leq \frac{2n-1}{5}$, and thus, $k > \left\lfloor \frac{2n-1}{5} \right\rfloor$ is impossible.

Next, we provide a construction for when $k = \left\lfloor \frac{2n-1}{5} \right\rfloor$.
If $n = 5s + 1$ for an integer $s$, then we have $k=2s$, and the desired $k$ pairs are the union of
\[ \{(i, 3s+i+1), (s+i, 2s+i)\} \]
for $i = 1,2,\dots,s$.

If $n = 5s + 2$ for an integer $s$, then we still have $k=2s$ and hence, we can use the same construction as above.

If $n = 5s + 3$ for an integer $s$, then $k = 2s+1$, and the desired $k$ pairs are the union of
\[ \{(i, 3s+i+2), (s+j, 2s+j+1)\} \]
for $i = 1,2,\dots,s$ and $j = 1,2,\dots,s+1$.

Finally, if $n = 5s+4$ or $n = 5s+5$ for an integer $s$, then we still have $k = 2s+1$ and hence, we can use the same construction as above.

These constructions can be easily shown to work, concluding the proof.

\section*{Geometry}
\subsection*{G1}
The difficulty in this problem mainly lies in algebraic manipulation.

We start by computing $J = (-a : b : c)$ and $M = (0 : s-b : s-c)$.
Notice that $KB = s-c$ and $KA = s$. From this, we can deduce $K = (c-s : s : 0)$.
Similarly, $L = (b-s : 0 : s)$.

Now, we set out to compute $F$. Since $F$ lies on line $BJ$, we know that it can be written in the form $(-a : t : c)$ for some $t$. We also know $F$, $M$, and $L$ are collinear, so we have the equation
\[
\begin{vmatrix}
    -a & t & c \\
    0 & s-b & s-c \\
    b-s & 0 & s
\end{vmatrix}
= 0 \implies t = \frac{-as + c(s-b)}{s-c}.
\]
At this point, continuing with the computation leads to very messy expressions.
We wonder if the expression for $t$ can be simplified. Indeed, after some algebra:
\[ \frac{-as + c(s-b)}{s-c} = -(a+c). \]
So, we have $F = (-a : -(a+c) : c) = (a : a + c : -c)$. Similarly, $G = (a : -b : a + b)$.

Now, we have pretty much finished the problem. Computing $S$ and $T$ and then the midpoint of $ST$ gives $M$, so we are done.

\subsection*{G2}
The reflection is arbitrary, so we get rid of point $H$ and instead note that it suffices to show $\angle FED + \angle FGD = 180$.

First, we claim that lines $FE$ and $FG$ are isogonal with respect to $\angle CFD$.
This can be shown using the parallelogram isogonality lemma.
Alternatively, we can prove this by showing $\triangle FAE \sim \triangle FCG$ by SAS similarity.
Indeed, using directed angles,
\[ \angle FAE = \angle DAE = \angle DBC = \angle GCF, \]
and furthermore,
\[ \frac{FA}{FC} = \frac{AB}{CD} = \frac{AE}{ED} = \frac{AE}{CG}, \]
so the claim is proven.

Then, easy angle chasing yields $\angle FEB = \angle DGF$, and the desired result shortly follows.


\section*{Number Theory}
\subsection*{N1}
All nonzero integers $m,n$ such that $\gcd(m,n) = 1$ work.
First, we claim that if integers $a,b$ are in an admissible set and $\abs{a-b} = 1$, then that set must contain all integers.
WLOG let $b = a+1$. Then, since
\[ a^2 - 2a(a+1) + (a+1)^2 = 1, \]
we have that $1$ is also in the set.
From there, it easily follows that the set contains all integers.

Next, suppose nonzero integers $m,n$ with $\gcd(m,n) = 1$ are in the set.
Then, $\gcd(m^2, n^2) = 1$ and by Bezout's lemma,
\[ am^2 - bn^2 = 1 \]
has an integer solution $(a,b)$.
However, $am^2$ and $bn^2$ must be in the set, because by setting $x = y = m$ or $x = y = n$, we can obtain all multiples of $m^2$ and $n^2$ respectively.
So, there are two consecutive integers in the set, and thus, the set contains all integers.

It remains to prove that when $\gcd(m,n) > 1$, the set need not contain all integers.
Consider the set of all multiples of $\gcd(m,n)$.
This set does not contain all integers.
Also, the integers $m$ and $n$ are in this set, and it clearly satisfies the condition for an admissible set, so we are done.

\end{document}
% vim: ts=4 sts=4 sw=4 et
