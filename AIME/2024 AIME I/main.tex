\documentclass{scrartcl}
\usepackage{graphicx} % Required for inserting images
\usepackage{amsmath}
\usepackage{listings}
\usepackage{evan}

\title{2024 AIME I Solutions}
\author{Michael Middlezong}

\begin{document}
\maketitle

\section*{Problem 11}
We do casework on the number of blue vertices.
If there are $0$, $1$, or $2$ blue vertices,
then any arrangement is valid,
giving us $1$, $8$, and $\binom82 = 28$ cases, respectively.

If there are $3$ blue vertices, then suppose no two of them are adjacent.
Then, we can just rotate by one in either direction.
Suppose two of them are adjacent.
Then, it is clear that no matter where the third is,
it will still work out.
So, every arrangement works, giving us $\binom83 = 56$ cases.

The hardest case to deal with is if there are $4$ blue vertices.
For this, we do casework on the maximal contiguous blocks of blue vertices.
\begin{itemize}
\item One block of $4$ yields $8$ cases.
\item A block of $3$ with a separate block of $1$ yields $0$ cases.
\item Two separate blocks of $2$ yields $4$ cases.
\item A block of $2$ and two blocks of $1$ yields $8$ cases.
\item Four blocks of $1$ yields $2$ cases.
\end{itemize}
Adding up yields $22$ cases.

There cannot be more than $4$ blue vertices.
So, the final probability is
\[ \frac{1+8+28+56+22}{256} = \frac{115}{256}. \]
This gives an answer of $115 + 256 = \boxed{371}$.

\begin{remark*}
I sillied in-contest by forgetting to count the case of ``A block of $2$ and two blocks of $1$.''
\end{remark*}

\section*{Problem 12}
After graphing each function, we see that intersections can only happen in the square $[0, 1] \times [0,1]$.
Furthermore, the first graph looks like a line that oscillates, changing direction $16$ times,
while the second graph looks like a line that oscillates $24$ times.
The first graph starts at $(0,1)$ and ends at $(1,1)$,
and the second graph starts at $(1,1)$ and ends at $(1,0)$.
So, there will be an intersection at $(1,1)$, but not in any other corner.

Trying small examples and conjecturing a pattern,
we expect this sort of oscillating line configuration to generate $16 \cdot 24 = 384$
intersection points.
However, we must be careful around the $(1,1)$ corner, since these are not actually lines.
The first graph's derivative is some positive value at $(1,1)$,
and the second graph has a derivative approaching infinity.
Thus, the two graphs must cross twice, once at exactly $(1,1)$ and once right before.
This gives an extra intersection,
so the answer is $384 + 1 = \boxed{385}$.

\section*{Problem 13}
Basic orders yields $8 \mid p-1$, so let's assume $p=17$ for now.
Based on the existence of primitive roots mod $17^2$,
there has to be something of order $8$,
so $p=17$ definitely works.

Using Hensel's lemma, we will lift a root mod $17$ to a root mod $17^2$.
It is not hard to see that the roots mod $17$ of $x^4 + 1$
are $x=\pm2, \pm8$.
Since $4x^3 \ne 0$ in all of those cases, Hensel's will work.

At this point, one can expand $(17k+x)^4$ mod $17^2$ using the binomial
theorem for each of the possible $x$ values, and find
that the minimum is $m = \boxed{110}$.

\section*{Problem 14}
Notice that $4^2 + 5^2 = 41$, $5^2 + 8^2 = 89$, and $4^2 + 8^2 = 80$.
We want to find a way to use this niceness/symmetry to our advantage,
and we can by introducing a coordinate system.

Let $A = (0,0,0)$, $B = (4,5,0)$, $C = (4,0,8)$, and $D = (0,5,8)$.
We want to find the inradius $r$,
so we will prove a formula analogous to $A = sr$ but for tetrahedrons.
Consider splitting the tetrahedron into four mini tetrahedrons
$IABC$, $IBCD$, $IABD$, and $IACD$.
Each of these tetrahedrons has height $r$,
so the sum of the volumes is
\[ V = \frac13 r(b_1 + b_2 + b_3 + b_4) = \frac13 Ar, \]
where $A$ is the surface area.

Thankfully, we can find the surface area and the volume easily,
because of our coordinate system.
The volume of the tetrahedron is $\frac16$ the volume of the
associated parallepiped, or
\[ \frac16 \begin{vmatrix}
    4 & 4 & 0 \\
    5 & 0 & 5 \\
    0 & 8 & 8
\end{vmatrix} = \frac{160}{3}. \]
To find the surface area,
notice that all the sides are congruent.
The area of one of the sides is half the area
of the associated parallelogram, or
\[ \frac12 \abs{\overrightarrow{AB} \times \overrightarrow{AC}} = 2\sqrt{189}. \]
So, the surface area is $8 \sqrt{189}$. Plugging this into the formula, we get
\[ \frac{160}{3} = \frac13 8r\sqrt{189} \implies r = \frac{20\sqrt{21}}{63}. \]
So, our answer is $20 + 21 + 63 = \boxed{104}$.

\section*{Problem 15}
After noticing that two of the sides must be equal,
it becomes a simple algebra exercise.

Otherwise, we proceed with Lagrange multipliers.



\end{document}