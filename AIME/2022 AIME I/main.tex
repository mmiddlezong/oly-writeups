\documentclass{scrartcl}
\usepackage{graphicx} % Required for inserting images
\usepackage{amsmath}
\usepackage{listings}
\usepackage{evan}

\title{2022 AIME I Solutions}
\author{Michael Middlezong}

\begin{document}
\maketitle

\section*{Problem 1}
Let the quadratics be $2x^2 + ax + b$
and $2x^2 + cx + d$.
We want $b+d$.

The sum of the quadratics is $(a+c)x + (b+d)$.
Plugging in $x=16$ and $x=20$, we get
\begin{align*}
16(a+c) + (b+d) &= 108,\\
20(a+c) + (b+d) &= 106.
\end{align*}
Solve to get $b+d = \boxed{116}$.

\section*{Problem 2}
Use mod $9$ to greatly reduce the cases.
Trial and error yields $\boxed{227}$.

\section*{Problem 3}
Angle chasing reveals $\angle APD = 90^\circ$.
Extend ray $AP$ to hit $CD$ at $E$, and let $M$ be the midpoint of $AD$.
Then, $AD = DE$ and $P$ is the midpoint of $AE$.
So, line $MP$ is parallel to line $CD$ and has length $\frac{333}{2}$.

Repeat this on the other side and note by symmetry that line $PQ$ is parallel to line $CD$.
This means that
\[ PQ = 575 - 2 \cdot \frac{333}{2} = \boxed{242}. \]

\section*{Problem 4}
Note that $w = \cis(30\dg)$ and $z = \cis(120\dg)$.
We can take the argument of both sides of the desired equation to get
\[ 90 + 30r \equiv 120s \pmod{360}. \]
Dividing out by $30$ and rearranging yields
\[ r \equiv 4s - 3 \pmod{12}. \]
The RHS cycles every $3$, so we only have to consider $s \pmod{3}$.
We get $9$ values for $r$ whenever $s \equiv 1\pmod{3}$ and $8$ values otherwise.
Adding them up, we get an answer of
\[ 33(9+8+8) + 9 = \boxed{834}. \]

\section*{Problem 5}
Let $t$ be the time that the two swam for.
From the moving water's perspective,
it is as if the two landed $14t$ meters upstream of the point
equidistant from their starting positions; call this new point $P_S$.
In this frame of reference,
Melanie swam for $80t$ meters
and Sherry swam for $60t$ meters.
Setting up a coordinate system where the origin is at the midpoint of their starting positions,
we get the equations
\begin{align*}
    \left(-14t + \frac{D}{2}\right)^2 + 264^2 &= (60t)^2, \\
    \left(-14t - \frac{D}{2}\right)^2 + 264^2 &= (80t)^2.
\end{align*}
Subtracting the first equation from the second,
we get $D = 100t$.

Suppose Sherry started at point $S$ and Melanie started at point $M$.
Then, $P_SSM$ is a right triangle. The altitude to its hypotenuse has length $264$.
This means we can solve for $t$ using the area formula.
We get $t = \frac{11}{2}$, so $D = \boxed{550}$.

\section*{Problem 6}
We use complementary counting.
Consider the arithmetic progression that forms.
It can either involve only $a$ (and three of the constant terms),
involve only $b$ (and three of the constant terms),
or involve both $a$ and $b$.
Technically, these cases are not necessarily disjoint because
multiple arithmetic progressions can form,
but we will easily be able to tell if we overcount.

In the case that it involves only $a$, notice that the first two terms
or the last two terms must consist of only constant terms.
So, the common difference can only take values which can be obtained by
subtracting two of the constant terms.
These values are $1$, $2$, $10$, and a lot of other values which are much too big.
If the common difference is $1$, then this happens when $a=6$.
If the common difference is $2$, there are no cases.
If the common difference is $10$, then this happens when $a=20$.

Similarly, if it involves only $b$,
then there is the case where $b=20$, and nothing else.

If it involves both $a$ and $b$, then
we have to choose two constant terms to include in our sequence.
The two constant terms can either be next to each other in the sequence
or have two terms ($a$ and $b$) in between.
If they are next to each other, then as before,
the common difference must be $1$, $2$, or $10$.
This gives us the case $(a,b)=(6,7)$, which we already included under $a=6$,
the case $(7,9)$, and the case $(10,20)$, which we already included.

If there are two terms in between, then the two constant terms must differ
by a multiple of $3$.
We can enumerate the cases pretty quickly and deduce that $(12,21)$ and $(16,28)$
are the only possibilities.

Summing it up, we want to count the cases
where either $a=6$, $a=20$, or $b=20$ (be careful not to double count).
There are also three extra cases.

It is easy to see that this yields $23 + 9 + 13 + 3 = 48$ cases in the complement,
so our answer is $\binom{24}{2} - 48 = \boxed{228}$.

\begin{remark*}
I initially sillied and got 49 cases in the complement because I overcounted $(6,20)$.
Rookie mistake.
\end{remark*}

\section*{Problem 7}
There are two approaches: maximizing the denominator, and minimizing the numerator.

First, let's try minimizing the numerator.
We can achieve a numerator of $1$ with
\[ \frac{1 \cdot 5 \cdot 7 - 2 \cdot 3 \cdot 6}{4 \cdot 8 \cdot 9} = \frac{1}{288}. \]

Now, the only concern remaining is that something with a numerator greater than $1$ would be better.
However, the maximum possible denominator is $7 \cdot 8 \cdot 9 = 504 < 2 \cdot 288$, so this is
impossible.

So, our minimum is $\frac{1}{288}$, so our answer is $\boxed{289}$.

\begin{remark*}
I sillied because I thought $1$ was impossible,
but I blame this on having seen the problem before and being overconfident.
\end{remark*}

\section*{Problem 8}
Let $X$ be the intersection point opposite $B$,
let $O$ be the center of $\omega$,
and let $O_A$ be the center of $\omega_A$.

It is well known (see: mixtilinear incircle properties)
that $O$ is the midpoint of the tangency points of
$\omega_A$ with lines $AB$ and $AC$, respectively.
From this, we can use $30-60-90$ triangles to deduce
that $OO_A = 6$ and that the radius of $\omega_A$ is $12$.

Now, we can use the law of cosines on triangle $OO_AX$ with known angle $\angle O_AOX = 120\dg$
to get
\[ x^2 + 36 + 6x = 144, \]
where $x = OX$.

We solve to get $x = \sqrt{117} - 3$.
Multiplying this by $\sqrt3$ evidently gives us the side length of the
equilateral triangle formed by the intersection points,
so our answer is $\sqrt{351} - \sqrt{27}$, and we bubble in $351 + 27 = \boxed{378}$.

\section*{Problem 9}
Note that an arrangement is even if and only if
the 1st, 3rd, 5th, 7th, 9th, and 11th slots together contain
all six types colors of blocks.
To prove this, consider an arrangement where this is true.
Then, take any color you want, WLOG red.
One of the red blocks must be in an even numbered position,
and the other one must be in an odd numbered position.
So, the number of blocks in between is even.

For the other direction, suppose that this were not true.
That means some color is repeated in the odd-numbered slots.
This color then violates the even property.

Finally, basic combo gives us a probability of
\[ \frac{(6!)^2 \cdot 2^6}{12!} = \frac{16}{231}. \]
Our answer is $16 + 231 = \boxed{247}$.

\section*{Problem 10}
Let $r$ be the shared radius of the three congruent circles.
Let $h_A$ be the distance from $A$ to the center of the sphere in which it lies,
and define $h_B$ and $h_C$ similarly.
First, get the equations

\begin{align*}
r^2 + h_A^2 &= 11^2,\\
r^2 + h_B^2 &= 13^2,\\
r^2 + h_C^2 &= 19^2.
\end{align*}

Then, take the cross section passing through $A$, $B$, and the center of $A$'s sphere.
This plane also passes through the center of $B$'s sphere,
because the circles line on the same plane, making their normal vectors parallel.
This cross section gives the equation
\[ (h_B - h_A)^2 + 560 = 24^2 \implies h_B - h_A = 4. \]
We can now go back to the first set of equations to get $h_A = 4$ and $h_B = 8$.
This also gives us $r^2 = 105$.
Finally, we take a similar cross section but with $A$ and $C$ instead of $A$ and $B$,
and this gives us the answer
\[ AC^2 = 30^2 - 12^2 = \boxed{756}. \]

\section*{Problem 11}
Let $E$ be the tangency point on $AD$,
let $F$ be the tangency point on $BC$,
and let $G$ be the tangency point on $AB$.
PoP yields $CF = 20$ and $AE = AG = 6$.

Let $H$ be the projection of $B$ onto $AD$.
Pythag on $\triangle AHB$ gives $r^2 = 6x$ after simplification.

Note $\cos(\angle ABC) = -\cos(\angle BAH) = \frac{x-6}{x+6}$.
Now, law of cosines on $\triangle ABC$ leads to a linear equation which yields $x = \frac92$,
so $r = 3\sqrt 3$.
The area is then
\[ (20 + \frac92) \cdot 6\sqrt3 = 147 \sqrt{3}. \]
Our answer is $147 + 3 = \boxed{150}$.

\section*{Problem 12}
Consider a fixed element, WLOG $1$.
We count the number of times it is included in the sum.
This is equivalent to the number of ordered pairs $(A,B)$
such that $1 \in A$ and $1 \in B$.
Doing casework on the size of the set,
we see that this number is
\[ \dbinom{n-1}{0}^2 + \dbinom{n-1}{1}^2 + \cdot + \dbinom{n-1}{n-1}^2. \]
Using Vandermonde's identity, this simplifies to $\binom{2n-2}{n-1}$.
This is how much $1$ contributes to $S_n$, but we have to multiply by $n$
because there are $n$ numbers.
So, we get the formula
\[ S_n = n \dbinom{2n-2}{n-1}. \]
Thus,
\[ \frac{S_{2022}}{S_{2021}} = \frac{2022\binom{4042}{2021}}{2021\binom{4040}{2020}} = \frac{4044 \cdot 4041}{2021 \cdot 2021}. \]
We can see that the numerator and denominator are relatively prime now.
So, the answer is
\[ 44 \cdot 41 + 21 \cdot 21 \equiv \boxed{245} \pmod{1000}. \]

\begin{remark*}
I sillied at the end and thought $\binom{4042}{2021} = \binom{4042}{2}$ which makes no sense but oh well.
\end{remark*}

\section*{Problem 13}
Let $m$ be a possible numerator.
We do casework on whether each of $3$, $11$, or $101$ divides $m$.
For each case,
one obtains a upper bound $b$ on $m$, making it a simple PIE exercise.
The final answer is $3+55+334+6000 \equiv \boxed{392}$.

\section*{Problem 14}
The splitting line through $M$ is parallel to the $C$-angle bisector,
and the line through $N$ is parallel to the $B$-angle bisector.
Angle chasing reveals $\angle A = 120\dg$, so
it suffices to solve $b^2 + bc + c^2 = 219^2$ (by the Law of Cosines).

Bashing mod 3 and mod 9 reveals that $3 \mid b$ and $3 \mid c$,
so it suffices to solve $x^2 + xy + y^2 = 73^2$.
Note that $x^2 + xy + y^2$ is the norm of $x - y\omega$ in $\ZZ[\omega]$,
where $\omega$ is a primitive third root of unity.

So, we can just solve $N(u-v\omega) = 73$ and then square it.
It is easy to guess $N(8-\omega) = 73$,
so then, we have $N((8-\omega)^2) = 73^2$.
Furthermore,
\[ (8 - \omega)^2 = 64 - 16\omega + \omega^2 = 64 - 16\omega + (-1 - \omega) = 63 - 17\omega, \]
giving us the solution $(x,y) = (63,17)$.
We multiply by three to get $(b,c) = (189, 51)$, so the perimeter is $\boxed{459}$.

\section*{Problem 15}
Let $u = 1-x$, $v = 1-y$, and $w = 1-z$.
Substituting yields
\begin{align*}
\sqrt{(1-u)(1+v)} + \sqrt{(1+u)(1-v)} &= 1 \\
\sqrt{(1-v)(1+w)} + \sqrt{(1+v)(1-w)} &= \sqrt 2 \\
\sqrt{(1-w)(1+u)} + \sqrt{(1+w)(1-u)} &= \sqrt 3.
\end{align*}
Squaring both sides and simplifying yields
\begin{align*}
    \sqrt{(1-u^2)(1-v^2)} - uv &= -\frac12 \\
    \sqrt{(1-v^2)(1-w^2)} - vw &= 0 \\
    \sqrt{(1-w^2)(1-u^2)} - wu &= \frac12.
\end{align*}
We are motivated to do trig sub.
Let $u = \sin a$, $v = \sin b$, and $w = \sin c$.
Then, substituting and simplifying yields
\begin{align*}
\cos(a+b) &= -\frac12 \\
\cos(b+c) &= 0 \\
\cos(c+a) &= \frac12.
\end{align*}
Trial and error (while respecting constraints) yields the solution
\[ (a,b,c) = \left(\frac{\pi}{4}, \frac{5\pi}{12}, \frac{\pi}{12}\right) \]
which satisfies the domain constraints of the problem.
The problem is asking for $[(\sin a)(\sin b)(\sin c)]^2$,
which we evaluate to be $\frac{1}{32}$.
We bubble in $1 + 32 = \boxed{33}$.

\begin{remark*}
I sillied because I got $a = \frac{\pi}{2}$ which can't be right because then $x = 0$.
On an unrelated note, in general, we must be careful because squaring both sides is irreversible.
\end{remark*}

\end{document}