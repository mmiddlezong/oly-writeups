\documentclass{scrartcl}
\usepackage{graphicx} % Required for inserting images
\usepackage{amsmath}
\usepackage{listings}
\usepackage{evan}

\title{2022 AIME II Solutions}
\author{Michael Middlezong}

\begin{document}
\maketitle

\section*{Problem 1}
The total number of people before the bus arrived
must be a multiple of $12$ so that the number of adults is an integer.
Let this amount be $12x$ for an integer $x$.

After the bus arrives, the total number of people must be a multiple of $25$.
Call this amonunt $25y$ for an integer $y$.
We have the equation
\[ 12x + 50 = 25y. \]
Taking mod $25$, we see that $25 \mid x$.
So, the least possible value of $x$ is $25$.
This gives $y = 14$, so there are $350$ people in total
after the bus arrived.
This means that the least possible number of adults
is $\frac{11}{25} \cdot 350 = \boxed{154}$.
(Note that minimizing the number of adults is equivalent
to minimizing the total number of people.)

\section*{Problem 2}
There are three cases with equal probability:
A faces C, A faces J, or A faces S.
The last two are symmetric so we only consider two cases.

In the first case, there is a $\frac13$ chance C wins the semifinals
and no matter the outcome of the other semifinal,
there is a $\frac34$ chance C wins the finals, giving us an overall chance of $\frac14$.

In the second case, C will be facing either J or S,
and he has a $\frac34$ chance of winning that match.
A will be facing either J or S, and we must consider two sub-cases.
If A wins his match, which happens with probability $\frac34$, then C will have a $\frac13$
chance of winning the finals.
If A loses, which happens with probability $\frac14$, then C will have a $\frac34$ chance
of winning the finals.
This gives us an overall probability of
\[ \frac34 \cdot \left(\frac34 \cdot \frac13 + \frac14 \cdot \frac34 \right) = \frac{21}{64}. \]

So, our final probability is
\[ \frac13 \cdot \frac14 + \frac23 \cdot \frac{21}{64} = \frac{29}{96}, \]
giving us an answer of $29 + 96 = \boxed{125}$.

\section*{Problem 3}
Use $V = \frac13 bh$ to get $h = \frac92$.
Then, it is obvious by symmetry that the center of the sphere lies on the central altitude.
So, placing it a distance of $r$ away from the apex,
we get the following by the Pythagorean theorem:
\[ \left( \frac92 - r \right)^2 + 18 = r^2 \implies r = \frac{17}{4}. \]
So, our answer is $17 + 4 = \boxed{21}$.

\section*{Problem 4}
Change of base yields
\[ \frac{\log (22x)}{\log (20x)} = \frac{\log (202x)}{\log (2x)}. \]
However, it is a useful fact that whenever $\frac{a}{b} = \frac{c}{d}$, the quantity $\frac{a+c}{b+d}$
is equal to both of them too. Since we have
\[ \frac{\log (22x)}{\log (20x)} = \frac{-\log (202x)}{-\log (2x)} = \frac{\log (22x) - \log (202x)}{\log (20x) - \log (2x)} = \frac{\log\left(\frac{22x}{202x}\right)}{\log 10} = \log\left(\frac{11}{101}\right), \]
we are done and our answer is $11 + 101 = \boxed{112}$.

\section*{Problem 5}
Suppose a triangle is formed with vertices at $a$, $b$, and $c$,
with $a<b<c$.

Then, $b-a=p$ for some prime $p$, $c-b=q$ for some prime $q$, and $c-a=r$ for some prime $r$.
Furthermore, $p+q=r$.

We can easily find all the solutions to $p+q=r$ satisfying $r \le 19$.
Since one of $p$ and $q$ must be $2$, we simply list the twin primes, giving us
$(2,3,5)$, $(2,5,7)$, $(2,11,13)$, $(2,17,19)$, and those but with $p$ and $q$ swapped.

Now, in the case that $(p,q,r) = (2,3,5)$, $a$ can be anything from $1$ to $15$, giving us $15$ cases.
In the other cases, we get $13$, $7$, and $1$ triangles, respectively.

This adds up to $36$ triangles, but we must multiply by two to account for swapping $p$ and $q$.
This gives $36 \cdot 2 = \boxed{72}$ in total.

\section*{Problem 6}
Clearly, there must be some index $i$ such that $x_i$ is the first nonnegative number.
Also, the sum of the absolute values of the negative numbers
is $\frac12$, and the sum of the positive numbers is also $\frac12$.
There are $i-1$ negative numbers and $101-i$ positive numbers.

In order to maximize $x_{76} - x_{16}$, we claim $x_{76}$ must be positive and $x_{16}$ must be negative.
Suppose they were both positive. Then, we could get a bigger value by setting $x_{16}$ and
any positive value below it to $0$ and
compensating in the other direction with $x_{100}$.
A similar argument works in the case that they are both negative.

So, $i$ must be between $17$ and $76$, inclusive.
Consider fixing $i$.
Then, the maximum difference is attained by setting $x_1 = x_2 = \cdots = x_{16}$
and $x_{76} = x_{77} = \cdots = x_{100}$.
Otherwise, one could ``smooth out'' the values and obtain a higher difference.
Furthermore, every other value should be $0$. Otherwise, any negative values could be absorbed by $x_0$
and smoothed out, and any positive values could be absorbed by $x_{100}$ and smoothed out,
creating a larger difference.
Using the sum restriction we got earlier,
we get an answer of
\[ \frac{1}{32} + \frac{1}{50} = \frac{41}{800}, \]
so we bubble in $\boxed{841}$.
\section*{Problem 7}
One-minute solve by similar triangles and PoP. The answer is $\boxed{192}$.

\section*{Problem 8}
It evidently repeats every $60$,
so we just create a table with the first $60$ numbers
and manually count the ones that are uniquely identifiable.
We get a total of $8$ for the first $60$,
so the answer is $10 \cdot 8 = \boxed{80}$.

\section*{Problem 9}
Start with $5$ points on one side and $1$ point on the other.
This gives us $4$ regions.
Now, add another point to the side with $1$ point.
As we connect this new point with the $5$ points on the other side,
we create $1 + 2 + 3 + 4 + 5 = 15$ more regions.
Add yet another point to the side with $2$ points.
We create $1 + 3 + 5 + 7 + 9 = 25$ more regions.
Do it again, and we create $1 + 4 + 7 + 10 + 13 = 35$ more regions.
Continuing the pattern, we get an answer of
\[ 4 + 15 + 25 + 35 + 45 + 55 + 65 = \boxed{244}. \]

\section*{Problem 10}
Algebraic manipulation yields
\[ \dbinom{\binom{n}{2}}{2} = \frac{(n-2)(n-1)(n)(n+1)}{8} = 3 \dbinom{n+1}{4}. \]
So, a simple application of the hockey stick identity yields a final answer of
\[ 3 \dbinom{42}{5} \equiv \boxed{4} \pmod{1000}. \]

\section*{Problem 11}
Extend rays $AB$ and $DC$ to meet at $X$,
and note that the midpoint $M$ of $BC$ is the incenter.

Since $XM$ bisects $\angle X$, we know by the angle bisector theorem
that $XB = XC$. Let this length be $x$.

Now, let $E$, $F$, and $G$ be the tangency points of the
incircle with $AD$, $XA$, and $XB$, respectively.
The semiperimeter $s$ is $6+x$.
We know $AF = s - XD = 3$, so $BF = 1$.
By the Pythagorean theorem, the inradius is $\sqrt{x-1}$.

Now, Heron's formula gives that the area of the triangle is
\[ A = \sqrt{12(x+6)(x-1)}, \]
but this is also equal to $sr$ where $r = \sqrt{x-1}$.
Solving this yields $x = 6$, so the inradius is $\sqrt{5}$.

The rest is easy, and the answer is $(6\sqrt5)^2 = \boxed{180}$.

\section*{Problem 12}
Interpret as two ellipses having an intersection.
One of the ellipses has foci at $A(-4,0)$ and $B(4,0)$, with a semimajor axis $a$.
The other one has foci at $C(20,10)$ and $D(20,12)$, with a semimajor axis $b$.

Consider any point $P$ on both ellipses.
We have
\[ PA + PB = 2a \]
and
\[ PC + PD = 2b. \]
So, we are effectively trying to minimize $a+b = \frac12 (PA + PB + PC + PD)$.
However, because of the triangle inequality,
\[ \frac12 (PA + PB + PC + PD) \le \frac12 (AC + BD) = \frac12 (26 + 20) = \boxed{23}, \]
and this bound is achievable with $P$ as the intersection point of $AC$ and $BD$.
\begin{remark*}
    I sillied because I wrote $PA + PB = a$ instead of $PA + PB = 2a$.
    Remember, $a$ is the semimajor axis!
\end{remark*}

\section*{Problem 13}
Interpret using geometric series formula:
\[ P(x) = (x^{2310} - 1)^2 (1+x^{105} + x^{210} + \cdots + x^{2310-105})(1+x^{70} + \cdots + x^{2310-70})(1+\dots)(1+\dots).\]
For finding the coefficient of $x^{2022}$, the $(x^{2310} - 1)^2$ part doesn't matter.
The rest is counting the number of nonnegative integer solutions to
\[ 105a + 70b + 42c + 30d = 2022. \]
Mod 5 and mod 7 analysis reveals $c \equiv 1\pmod{5}$ and $d \equiv 3 \pmod{7}$.

Now, we can rewrite the LHS to get
\[ 210\left(\frac{a}{2} + \frac{b}{3} + \frac{c}{5} + \frac{d}{7}\right) = 2022. \]

Letting $c' = c-1$ and $d' = d-3$, notice that $c'$ and $d'$ are still guaranteed to be nonnegative.
After simplifying, we get
\[ \frac{a}{2} + \frac{b}{3} + \frac{c'}{5} + \frac{d'}{7} = 9. \]
Our mod analysis from earlier reveals that these are all nonnegative integers.
After a quick check that our steps were reversible,
we get an answer using stars-and-bars of $\binom{12}{3} = \boxed{220}$.
\begin{remark*}
    I sillied by saying $\binom{12}{3} = 660$ for some reason.
\end{remark*}

\section*{Problem 14}
Note that $a$ must be $1$.
It can be shown that
\[ \left\lceil \frac{1000}{c} \right\rceil \le f(a,b,c) \le \left\lfloor \frac{1000}{c} \right\rfloor + \left\lfloor \frac{c}{b} \right\rfloor + (b-1). \]
Bash to get the solutions $(1,7,11)$, $(1,86,88)$, and $(1,86,89)$,
yielding a total of $11 + 88 + 89 = \boxed{188}$.
\begin{remark*}
I sillied because I assumed $n_c = \left\lfloor \frac{1000}{c} \right\rfloor$ in the $c=87$ case.
In fact, $n_c = 10$ is more optimal.
\end{remark*}

\section*{Problem 15}
Reflect $AB$ across the perpendicular bisector of $O_1O_2$
to create symmetry.
Note that this does not change the area of $ABCD$.
Then, $ABCD$ is an isosceles trapezoid.
Also, $AO_1DO_2$ and $BO_1CO_2$ are also isosceles trapezoids,
so $AD = BC = O_1O_2 = 15$.

By dropping an altitude from $A$ to $CD$,
we find that $\cos(\angle ADC) = \frac35$.
Furthermore, the height of trapezoid $ABCD$ is $12$.
So, the area of trapezoid $ABCD$ is
\[ 12 \cdot 9 = 108. \]

It remains to find the area of $\triangle AO_1C \cong \triangle DO_2B$.

This area is equal to $\frac12 r_1r_2 \sin (\angle AO_1C)$.
And, $\sin (\angle AO_1C) = \sin (\angle ADC) = \frac45$,
so we just have to find $r_1r_2$.

For this, we use the law of cosines on $\triangle AO_1C$.
This gives
\[ AC^2 = r_1^2 + r_2^2 - 2r_1r_2 \cos (\angle AO_1C). \]
We know from the altitude from earlier that
\[ AC^2 = 7^2 + 12^2 = 193. \]
We also know
\[ r_1^2 + r_2^2 = (r_1 + r_2)^2 - 2r_1r_2 = 225 - 2r_1r_2. \]
Finally, we know (from cyclic quads)
\[ \cos (\angle AO_1C) = -\cos (\angle ADC) = -\frac35. \]
Putting everything together, we get $r_1r_2 = 40$.
The final area is $\boxed{140}$.


\end{document}