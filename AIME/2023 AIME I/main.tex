\documentclass{scrartcl}
\usepackage{graphicx} % Required for inserting images
\usepackage{amsmath}
\usepackage{listings}
\usepackage{evan}

\title{2023 AIME I Solutions}
\author{Michael Middlezong}

\begin{document}
\maketitle

\section*{Problem 1}
We will only consider where the men are standing, since
the women are guaranteed to fill in the $9$ remaining spots.
To count the number of working cases, we fix the first man
to be somewhere to account for the possible rotations of the circle.
Then we have $12$ choices for the second man, since he cannot
sit diametrically opposite the first man. Similarly,
we have $10$, $8$, and $6$ choices for the third, fourth, and fifth man, respectively.
This gives us $12 \cdot 10 \cdot 8 \cdot 6$ cases that work.
For the total number of cases, we get $13 \cdot 12 \cdot 11 \cdot 10$.
This gives us a final probability of
\[ \frac{12 \cdot 10 \cdot 8 \cdot 6}{13 \cdot 12 \cdot 11 \cdot 10} = \frac{48}{143}. \]
Thus, the answer is $48 + 143 = \boxed{191}$.
\section*{Problem 2}
Let $a = \log_b n$. Our equations become $\sqrt{a} = \frac12 a$ and $ba = 1 + a$, using log rules.
The first equation gives $a=0$ or $a=4$, but $a=0$ yields no solutions to the second equation.
Taking $a=4$, we get $b = \frac54$, and finally, $n = b^4 = \frac{625}{256}$. Our answer is then
$625 + 256 = \boxed{881}$.
\section*{Problem 3}
Two non-parallel lines define an intersection point.
There are $\binom{40}{2}$ of these intersections, but some of them coincide.
From ``there are $3$ points where exactly $3$ lines intersect,''
we get that there are $3$ triplets of coinciding points,
so we must subtract $3 \cdot 3$ from our total count.
Similarly, there are $4$ sets of $\binom42 = 6$ points that coincide,
so we have to subtract $4 \cdot \binom42$.
Continuing on, we get
\[ \binom{40}{2} - 3\cdot 3 - 4\cdot 6 - 5\cdot 10 - 6\cdot 15 = \boxed{607}. \]
\section*{Problem 4}
First, we prime factorize $13! = 2^{10} 3^5 5^2 7^1 11^1 13^1$.
Consider picking the number of each prime in the factorization of $m$
so that $\frac{13!}{m}$ is a perfect square.
For $2$, we can choose an exponent of $0$, $2$, $4$, $6$, $8$, or $10$.
For $3$, we can choose an exponent of $1$, $3$, or $5$.
For $5$, we can choose an exponent of $0$ or $2$.
However, for $7$, $11$, and $13$, we must choose an exponent of $1$ in each case.
Finally, we use the trick of factoring the sum of all possible $m$ to get
a sum of
\[ (2^0 + 2^2 + \cdots + 2^{10})(3^1 + 3^3 + 3^5)(5^0 + 5^2)(7)(11)(13) = 2^13^25^17^311^113^4. \]
Our answer is then $1+2+1+3+1+4 = \boxed{12}$.
\section*{Problem 5}
Let $s = AB$, let $x$ be the height from $P$ to $AC$, and let $y$ be the height from $P$ to $BD$.
Using triangle area, we have
\begin{align*}
PA \cdot PC &= sx \sqrt 2, \\
PB \cdot PD &= sy \sqrt 2.
\end{align*}
Squaring both equations and adding them yields
\[ (PA \cdot PC)^2 + (PB \cdot PD)^2 = (sx \sqrt 2)^2 + (sy \sqrt 2)^2. \]
We can simplify this to get
\[ 56^2 + 90^2 = 2s^2 (x^2 + y^2). \]
Notice that $x^2 + y^2 = r^2$, where $r$ is the radius of the circle.
Since $r = \frac{s}{\sqrt 2}$, we have
\[ 56^2 + 90^2 = s^4. \]
We're looking for $s^2$, which is $\sqrt{56^2 + 90^2} = \boxed{106}$.
\section*{Problem 6}
List out the $\binom63$ orders in which the cards can be revealed,
and calculate the number of correct guesses for each order.
We make the assumption that whenever the number of red and black cards
remaining is equal, Alice will guess black.
We get an expected value of $\frac{41}{10}$, so our answer is $\boxed{51}$.
\section*{Problem 7}
Let's find the possible sets of remainders.
If $n$ is $1 \pmod{2}$, then notice that $n$ physically
cannot be $0$ or $2 \pmod{4}$.
So, to satisfy the conditions, $n$ must be $3 \pmod{4}$.
Similarly, $n$ must be $5 \pmod{6}$.
This means that $n$ is $2 \pmod{3}$,
and the only remaining options are for $n$ to be $0$ or $4 \pmod{5}$.
So, we have
\[ (n \bmod 2, n \bmod 3, n \bmod 4, n \bmod 5, n \bmod 6) = (1,2,3,4,5) \]
or
\[ (n \bmod 2, n \bmod 3, n \bmod 4, n \bmod 5, n \bmod 6) = (1,2,3,0,5). \]

If $n$ is $0 \pmod{2}$,
then we use a similar line of reasoning to get that
the only possible case is
\[ (n \mod 2, n \mod 3, n \mod 4, n \mod 5, n \bmod 6) = (0,1,2,3,4). \]

Now that we only have three possible cases,
note that for each case, a unique residue mod $60$ satisfies all of the congruences
at the same time, because of CRT.
In the $(1,2,3,4,5)$ case, we have $n \equiv 59 \pmod{60}$, giving us $16$ possible values
of $n$ less than $1000$.
In the $(1,2,3,0,5)$ case, we have $n \equiv 35 \pmod{60}$, giving us $17$ possible values
of $n$ less than $1000$.
Finally, in the $(0,1,2,3,4)$ case, we have $n \equiv 58 \pmod{60}$,
giving us $16$ possible values of $n$ less than $1000$.

Adding up, we get a final answer of $16+17+16 = \boxed{49}$.
\section*{Problem 8}
Let $E$ and $F$ be the feet of the altitudes
from $P$ to $AD$ and $AB$, respectively.
Let $G$ and $H$ be the foot of the altitudes
from $I$ to $AD$ and $AB$, respectively.

From the information given, it is not hard to tell
that the radius of the incircle is $\frac{25}{2}$.
Then, a Pythagorean relation can be obtained (say, by drawing a perpendicular from $I$ to line $PE$)
which gives $EG^2 = \left( \frac{25}{2} \right)^2 - \left( \frac{7}{2} \right)^2 = 12^2$.
Thus, $EG = 12$.
We can do a similar thing to get $FH = 10$.
Letting $x = AG = AH$, we have $AE = x - 12$ and $AF = x - 10$.
Notice that $AF^2 + FP^2 = AE^2 + EP^2 = AP^2$. This gives us
\[ (x-10)^2 + 5^2 = (x-12)^2 + 9^2 \implies x = 25. \]
By similar triangles, $AG \cdot GD = IG^2 = \left( \frac{25}{2} \right)^2$,
so $GD = \frac{25}{4}$.
So, the perimeter is $4(25 + \frac{25}{4}) = \boxed{125}$.
\section*{Problem 9}
The given condition can be written as
\[ m^3 + am^2 + bm + c = 8 + 4a + 2b + c. \]
We see that the value of $c$ doesn't matter, so we ignore it and remember
to multiply by $41$ at the end.
Now, let $n = m-2$.
The equation simplifies to
\[ n^3 + (6+a)n^2 + (12 + 4a + b)n = 0. \]
We already know that $n=0$ works,
and indeed, we are trying to find cases where there is another integer root.
So, assuming $n \neq 0$, we can divide by $n$ to get a quadratic:
\[ n^2 + (6+a)n + (12 + 4a + b) = 0. \]
Note that because the coefficients are integers,
the existence of an integer root implies that both roots are integers.
So, we have two cases:

\begin{itemize}
\item \textbf{Case 1: there is a double root and it is not $n=0$.}
In this case, the discriminant must be $0$:
\[ a^2 - 4a - 12 - 4b = 0 \implies (a-6)(a+2) = 4b. \]
Forget about the condition that $n \neq 0$ for now; we will
return to it after handling Case 2.
For this case, the possible values of $(a,b)$ in the desired range are
\[ (-6,12), (-4,5), (-2,0), (0,-3), (2,-4), (4,-3), (6,-0), (8,5), (10,12). \]
\item \textbf{Case 2: there is no double root and one of the roots is $n=0$.}
This means we need $12 + 4a + b = 0$. The ordered pairs in the desired range that satisfy that equation are
\[ (-8, 20), (-7, 16), \dots, (1, -16), (2,-20). \]
But there cannot be a double root. All the ordered pairs that lead to a double root
were listed during Case 1, so we simply need to recognize that $(-6,12)$ appears in both lists
and remove it from both lists.
This simultaneously handles the condition that $n \neq 0$ in Case 1.
\end{itemize}
Putting everything together, we count $18$ cases for $(a,b)$.
Since $c$ can be anything, our final answer is $18 \cdot 41 = \boxed{738}$.

\section*{Problem 10}
Let $S = \sum_{n=1}^{2023} n$ and $S_2 = \sum_{n=1}^{2023} n^2$.
Then
\[ U = \frac15 (S_2 - aS) - E, \]
where $E$ is an error term resulting from the floor operations.
We can bound $E$: the minimum is $0$ and the maximum will result from
a ``loss'' of $\frac45$ with each operation, giving us $E = \frac45 \cdot 2023$.
Since $E \ll S$, we can conclude that the value of $a$ must be the one such that $|S_2 - aS|$ is
minimized.
A quick computation reveals that when $a = 1349$, we have $S_2 = aS$.
So, it remains to find $E$ when $a = 1349$.

For this, we simply count the number of times $n^2 - na$ attains each residue mod $5$.
Simple modular arithmetic reveals that it is $0$ mod $5$ $808$ times,
$1$ mod $5$ 405 times, and $2$ mod $5$ $810$ times.
This gives us $E = 405$, so $U = -405$
Our final answer is $1349 - 405 = \boxed{944}$.

\section*{Problem 11}
Let $F_n$ be the number of subsets of $\{1,2,\dots, n\}$ with no pairs of consecutive integers.
This satisfies the recurrence relation $F_n = F_{n-1} + F_{n-2}$ by casework on whether $n$ is
included in the subset.
So, we get $F_1 = 2$, $F_2 = 3$, $F_3 = 5$, and so on.

Returning to the original problem, we do casework on what the one pair of consecutive integers is.
If it's $\{1,2\}$, for example, then what remains is to select a subset of
$\{4,5,\dots, 10\}$ with no pairs of consecutive integers.
We know this can be done in $F_7 = 34$ ways.
We can do a similar thing for each possible pair of consecutive integers.
For example, if it's $\{4,5\}$,
then we select a subset of $\{1,2\}$ with no consecutive integers
and a subset of $\{7,8,9,10\}$ with no consecutive integers.
There are $F_2 \cdot F_4$ ways to do that.
Doing this for every case, we get a final answer of
\[ 2(F_7 + F_6 + F_5 \cdot F_1 + F_4 \cdot F_2) + F_3 \cdot F_3 = \boxed{235}. \]

\begin{remark*}
I got this wrong in-contest because I thought the base cases were $F_1 = 1$ and $F_2 = 2$.
I had forgotten about the empty set.
\end{remark*}
\section*{Problem 12}
It is well known that $(AEF)$, $(BDF)$, and $(CDE)$ intersect
at the Miquel point, which satisfies the properties that $P$ has to satisfy.
So, $P$ is the Miquel point.
Using the law of cosines, we get $EF = 35$, $DF = 13$, and $DE = 42$.
Let $x = PD$, $y = PE$, and $z = PF$.
We use the law of cosines again on triangles $PEF$, $PDF$, and $PDE$ to get
\begin{align*}
y^2 + yz + z^2 &= 35^2, \\
x^2 + xz + z^2 &= 13^2, \\
x^2 + xy + y^2 &= 42^2.
\end{align*}
The area of the triangle can be calculated in two ways.
One is by adding the areas of triangles $PEF$, $PDF$, and $PDE$,
which can be found using the sin area formula,
to get
\[ A = \frac{\sqrt3}{4} (xy + yz + zx). \]
We can also use Heron's formula with $s = \frac{35 + 13 + 42}{2} = 45$ to obtain
\[ A = \sqrt{45 \cdot 10 \cdot 32 \cdot 3} = 120 \sqrt3. \]
So, $xy + yz + zx = 480$.
Now,
\[ (x+y+z)^2 = x^2 + y^2 + z^2 + 2(xy + yz + zx) = \frac12(35^2 + 13^2 + 42^2 + 3 \cdot 480) = 2299. \]
From this, we get $x+y+z = 11 \sqrt{19}$.
Let $\theta = \angle PDC$. By splitting triangle $ABC$ into many triangles,
we get
\[ \frac{55^2 \sqrt3}{4} = [ABC] = \frac12 \cdot 55 \cdot (x+y+z) \cdot \sin \theta \implies \sin \theta = \frac{5\sqrt 3}{2 \sqrt{19}}. \]
Then,
\[ \cos ^2 \theta = 1 - \sin ^2 \theta = 1 - \frac{75}{76} \implies \cos \theta = \frac{1}{2\sqrt{19}}. \]
Finally, $\tan \theta = 5 \sqrt 3$, so our final answer is $\tan^2 \theta = \boxed{75}$.
\section*{Problem 13}
Take for granted that in one of the parallelepipeds,
there is a vertex whose three edges create three acute angles with each other,
and in the other,
there is a vertex whose three edges create three obtuse angles with each other.
(I'm not sure how to visualize it yet.)

Let $a$, $b$, and $c$ be the vectors corresponding to the edges emanating from this vertex,
such that the angles formed by the vectors are either all acute or all obtuse.
The angle is one of the angles of the rhombus,
and we can calculate the cosine of the two possible cases:
\[ \cos \theta_1 = \frac{5}{26} \qquad \text{and} \qquad \cos \theta_2 = -\frac{5}{26}. \]
Let's focus on the $\theta_1$ case for now.
We want the height of the parallelepiped, which
is just $\sqrt{13} \sin \phi$, where $\phi$ is the angle between the vectors
$a$ and $b+c$.
Now,
\[ \cos \phi = \frac{a \cdot (b+c)}{\|a\| \|b+c\|} = \frac{a \cdot b + a \cdot c}{\sqrt{13} \cdot \sqrt{31}} = \frac{2 \cdot 13 \cdot \frac{5}{26}}{\sqrt{13} \cdot \sqrt{31}} = \frac{5}{\sqrt{403}}. \]
So,
\[ h_1 = \sqrt{13} \sin \phi = \sqrt{13} \sqrt{\frac{378}{403}} = \sqrt{\frac{378}{31}}. \]
Similarly, in the other case,
\[ \cos \phi = \frac{a \cdot (b+c)}{\|a\| \|b+c\|} = \frac{a \cdot b + a \cdot c}{\sqrt{13} \cdot \sqrt{21}} = \frac{2 \cdot 13 \cdot \left(-\frac{5}{26}\right)}{\sqrt{13} \cdot \sqrt{21}} = -\frac{5}{\sqrt{273}}. \]
So,
\[ h_2 = \sqrt{13} \sin \phi = \sqrt{13} \sqrt{\frac{248}{273}} = \sqrt{\frac{248}{21}}. \]
So, the ratio between the heights, which is the ratio between the volumes, is
\[ \frac{h_1}{h_2} = \frac{63}{62}. \]
Our final answer is $63 + 62 = \boxed{125}$.

\section*{Problem 14}
Associate each state with an ordered pair $(i,j)$
with $0 \le i,j \le 11$.
Each action is either adding $(0,1)$ or $(1,0)$, mod 12.
We can plot this on a 12x12 grid of points, and now
our actions are either ``up arrows'' or ``right arrows.''
Since this is mod 12, we have portals on the top side and the right side
that loop us back to the bottom and left side, respectively.

Consider a working sequence of 144 actions.
We claim that it is periodic with period 12.
Notice that from any point, precisely one arrow will emanate from it.
If an up arrow emanates from the point $(x,y)$, for example,
then notice that we are forced to have an up arrow out of the point $(x+1,y-1)$.
Likewise, if a right arrow emanates from the point $(x,y)$,
we need a right arrow out of $(x+1,y-1)$. Thus, the first 12 arrows placed
can together determine the rest of the arrows.

Now, consider the number $m$ of right arrows in the first 12 arrows.
We claim that $m$ must be relatively prime to $12$.
This is because each period of $12$ actions is equivalent to adding $(m,12-m)$,
and we have to add this $12$ times to get back to the origin.
If $m$ shared a factor with $12$, then we would get back to the origin too fast.

As long as $m$ is relatively prime to $12$, any sequence of $m$ right arrows
and $12-m$ up arrows will generate a compatible sequence of $144$ actions (by repeating it $12$ times).
So, the answer is
\[ \dbinom{12}{1} + \dbinom{12}{5} + \dbinom{12}{7} + \dbinom{12}{11} = 1608 \equiv \boxed{608} \pmod{1000}. \]
\section*{Problem 15}
Write $z = a + bi$.
We need $p > \abs{\opname{Re}(z^3) - \opname{Im}(z^3)}$.
This can be rewritten as
\[ a^2 + b^2 > |a+b| |a^2 - 4ab + b^2|. \]
Since the magnitude of $z^3$ is $p^{3/2}$, assuming $p$ is somewhat large,
this means the real and imaginary parts of $z^3$ are close to each other.
Also, both the real and imaginary parts of $z^3$ must be positive as per the given condition.
This means that the argument of $z^3$ should be close to $45$ degrees.

So, the argument of $z$ should be around $15$, $135$, or $255$ degrees.
It can't be around $135$ because that would imply $a \approx -b$, which,
along with the fact that $a^2 \neq b^2$, leads to $|a+b| |a^2 - 4ab + b^2|$
being too big.
The cases where the argument of $z$ is around $15$ or $255$ degrees are symmetric,
so we assume WLOG that $a,b > 0$ and thus, the argument of $z$ is around $15$ degrees.
In other words,
\[ a \approx \tan(75\dg) \approx 3.7. \]

We can now bound $b \le 8$ because $a^2 + b^2 = p < 1000$.
The rest is just trying every single possibility.
We eventually find that $5^2 + 18^2 = \boxed{349}$ works.

\end{document}