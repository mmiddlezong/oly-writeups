\documentclass{scrartcl}
\usepackage{graphicx} % Required for inserting images
\usepackage{amsmath}
\usepackage{listings}
\usepackage{evan}

\title{2023 AIME II Solutions}
\author{Michael Middlezong}

\begin{document}
\maketitle

\section*{Problem 1}
If $a$ is the least number of apples growing on any of the trees,
then we have
\[ \frac55 a + \frac65 a + \frac75 a + \frac85 a + \frac95 a + \frac{10}{5}a = 990 \implies a = 110. \]
Our answer is $2a = \boxed{220}$.
\section*{Problem 2}
Note that $512 = 1000_8$.
The only palindromes in base $8$ between $1000_8$ and $2000_8 > 1000$
are those of the form $1cc1_8$, for some digit $c$.
We have
\[ 1cc1_8 = 513 + 72c, \]
and we can list them out in base $10$:
\[ 513, 585, 657, 729, 801, 873, 945. \]
The largest palindrome is $\boxed{585}$.
\section*{Problem 3}
By angle chasing, $\triangle PAB \sim \triangle PBC$ with $\triangle PBC$ being $\sqrt2$ times bigger.
So, $PB = 10\sqrt2$ and $PC = 20$.
Then, the Pythagorean theorem yields $AC = \sqrt{500}$, so the area is $\boxed{250}$.

\section*{Problem 4}
Subtracting the second equation from the first yields
\[ (y-4)(x-z) = 0. \]
So, either $y=4$ or $x=z$.

If $y=4$, then we have
\begin{align*}
4x + 4z &= 60, \\
xz &= 44.
\end{align*}
It's not hard to guess the solution of $\{x,z\} = \{11,4\}$.

If $x=z$, then subtracting the third equation from the second reveals
that either $z=4$ or $x=y$.
If $z=4$, then $x=4$ and $y=11$.
If $x=y$, then $x=y=z$ and it suffices to solve the quadratic
\[ x^2 + 4x = 60 \implies x = -10, 6. \]
This gives us the solutions $x=y=z=-10$ and $x=y=z=6$.

That's all of the cases.
The system is symmetric in the three variables,
so the possible values of $x$ are
\[ -10, 4, 6, 11. \]
The answer is then
\[ (-10)^2 + 4^2 + 6^2 + 11^2 = \boxed{273}. \]

\section*{Problem 5}
Write $r=\frac{a}{b}$ in lowest terms.
Then, when $\frac{55a}{b}$ is written in lowest terms,
the sum of the numerator and denominator must be $a+b$.
Evidently, we need $q= \gcd(55,b) > 1$.
The cases for $q$ are $5$, $11$, and $55$.

In general, we have $\frac{55a}{q} + \frac{b}{q} = a+b$.
We can rewrite this as
\[ 55a + b = qa + qb \implies (55-q)\frac{a}{b} = \frac{q-1}{55-q}. \]
We see that $q=5$ and $q=11$ yield $\frac{4}{50} = \frac{2}{25}$ and $\frac{10}{44} = \frac{5}{22}$
respectively.
Notice that $q=55$ doesn't work.
We can verify that the two fractions we got indeed do work, so the sum of $S$ is
\[ \frac{2}{25} + \frac{5}{22} = \frac{169}{550}. \]
Our final answer is $169 + 550 = \boxed{719}$.

\section*{Problem 6}
When counting the complement,
we only need to worry about the case where one of $A$ and $B$ is in the top square,
and the other is in the right square.
This happens with probability $\frac{2}{9}$.

Assume WLOG that $A$ is chosen in the top square.
We can model this situation with $A = (a_1, a_2 + 1)$ and $B = (b_1 + 1, b_2)$,
where $a_i$ and $b_i$ are uniformly randomly chosen within $[0,1]$.
The midpoint, $\left(\frac12 (a_1 + b_1 + 1),\frac12 (a_2 + b_2 + 1)\right)$,
is only outside of the region
when both its $x$ and $y$-coordinate are greater than $1$.
This means $a_1 + b_1 > 1$ and $a_2 + b_2 > 1$.
Each of those occurs with probability $\frac12$ by geometric probability,
so the probability of failing is $\frac14$.

Our final probability is then $1 - \frac29 \cdot \frac14 = \frac{17}{18}$.
The answer is $17 + 18 = \boxed{35}$.

\section*{Problem 7}
Let a blue pair be a pair of opposite vertices both colored blue, and define a red pair similarly.
Then, there are four cases for $(\text{\# of blue pairs}, \text{\# of red pairs})$: $(0,0)$, $(0,1)$, $(1,0)$,
and $(1,1)$.
(Notice that the existence of two or more pairs of the same color creates a monochromatic rectangle.)

For the $(0,0)$ case, there are two options for each pair of opposite vertices,
and $6$ such pairs, giving us $2^6 = 64$ colorings.

For the $(0,1)$ case and $(1,0)$ case, we choose one of $6$ pairs to be monochromatic, and
there are $2$ options for $5$ other pairs, giving us $6 \cdot 2^5 = 64 \cdot 3$ colorings.

Finally, for the $(1,1)$ case, we choose a pair to be a blue pair, a pair to be a red pair,
and between $2$ options for the other $4$ pairs, giving us $6 \cdot 5 \cdot 2^4 = 480$ colorings.

Adding them up, we get
\[ 7 \cdot 64 + 480 = \boxed{928}. \]

\section*{Problem 8}
Expand
\[ (\omega^{3k} + \omega^{k} + 1)(\omega^{3(7-k)} + \omega^{7-k} + 1) \]
to see that it equals $2$.
There are three such pairs, and the term for $k=0$ is just $1+1+1=3$,
so the answer is
\[ 3 \cdot 2^3 = \boxed{24}. \]

\section*{Problem 9}
Extend $QP$ to hit $AB$ at $K$.
It can be shown that $XY = 2AB$, so $AB = 12$.
By radical axis, $KA = KB = 6$.
By PoP, $KA^2 = KP \cdot KQ \implies KP = 4$.

Let $D$ be the foot of the perpendicular from $P$ to $AB$.
Then, one can see that $DK = 1$.
So, $PD^2 = 4^2 - 1^2 = 15 \implies PD = \sqrt{15}$.
This is the height of the trapezoid, so the area is
\[ \frac12 \sqrt{15} (12 + 24) = 18\sqrt{15}. \]
Our answer is $18 + 15 = \boxed{33}$.

\section*{Problem 10}
We reduce mod $3$, such that we need to place four $0$'s, four $1$'s, and four $2$'s.
We just have to remember to multiply by $(4!)^3$ at the end.

Notice that each column in this reduced configuration
must have two distinct numbers in it.
Furthermore, there are three types of columns which must each appear twice:
$\{0,1\}$ columns, $\{1,2\}$ columns, and $\{0,2\}$ columns.

There are $\frac{6!}{2!2!2!}$ ways to arrange the types of columns.
For a given arrangement of types of columns, each column has two choices
for how to arrange the two numbers within the column.
However, given one column's arrangement, there is only one way to arrange the
others satisfying the conditions.
So, for each arrangement of types of columns, there are $2$ possible arrangements.

This gives us a final count of
\[ \frac{6!}{2!2!2!} \cdot 2 \cdot (4!)^3 = 2^{11}3^55^1. \]
So, the answer is $12 \cdot 6 \cdot 2 = \boxed{144}$.

\section*{Problem 11}
We do casework on the minimum size of a subset in the collection:
\begin{itemize}
\item Min size: 1. Then, there are five ways to choose the singleton set to include.
Notice that there can only be one such set.
The rest of the sets must be the $15$ other sets containing the element of the singleton set.
This gives us a total of $5$ possibilities.
\item Min size: 2. Generally, if there are $n$ two-element sets in our collection,
then the rest of the sets must be the $10-n$ three-element sets that aren't complements
of any of the two-element sets, and the $6$ sets with four or more elements.
This always works because any two sets with three or more elements each will
necessarily have an element in common, so the only interactions with a chance of failing
are those between 3+ element sets and two-element sets, and those between $2$ two-element sets.
So, it suffices to choose the two-element sets to include in our collection in a way that works.
We do casework on the number of two-element sets in our collection:
\begin{itemize}
\item If there is only one, then there are $\binom52 = 10$ cases.
\item If there are two, then we pick the shared element (5 ways) and the two other elements (6 ways)
for a total of $30$ cases.
\item If there are three, then either we have something like $\{1,2\},\{2,3\},\{1,3\}$
or there is one element which all three sets share.
In the first case, there are 10 ways, and in the second case, there are 20 ways, for a total of $30$.
\item If there are four, then they must all share one element. There are 5 ways to pick that element,
and the rest is fixed.
\end{itemize}
This gives us a grand total of $75$ cases.
\item Min size: 3. There are only $16$ subsets with $3+$ elements, so they must all be part of the
collection. This works because any two sets with three or more elements each will
necessarily have an element in common. For this case, we have a total of just $1$.
\end{itemize}
In total, we get $5+75+1 = \boxed{81}$ possible collections.

\section*{Problem 12}
Notice that $BPQC$ is a parallelogram.
Use Stewart's or Pythagorean theorem to find $AM = \sqrt{148}$.
Then, PoP on $M$ to finish.
We get $AQ = \frac{99}{\sqrt{148}}$, so our answer is $99 + 148 = \boxed{247}$.

\section*{Problem 13}
The desired expression can be rewritten as
\[ x^{n/2} + y^{n/2} \]
where $x,y = \frac{\sqrt{17} \pm 1}{2}$.
It can be shown that $n$ must be a multiple of $4$ for this expression to be an integer.

Let $b_i = x^{2i} + y^{2i}$.
We can manually find $b_1 = 9$ and $b_2 = 49$.
Notice that
\begin{align*}
b_i &= x^{2i} + y^{2i} \\
&= (x^{2i-2} + y^{2i-2})(x^2 + y^2) - x^2y^2(x^{2i-4} + y^{2i-4}) \\
&= 9b_{i-1} - 16b_{i-2}.
\end{align*}
This recurrence relation helps us to calculate the rest of the $b_i$'s.
We eventually find that $b_i$ mod $10$ is periodic with a period of $3$,
so we can calculate our final answer of $83 \cdot 2 + 1 = \boxed{167}$.

\section*{Problem 14}
We want to subtract the volume of the frustum without water
from the total volume of the cube.
Linear algebra and coordinate bash yields the areas of the two bases,
$\frac98$ and $\frac{81}{8}$.
What's remaining is to use the formula for the volume of a frustum (we have $h=6$):
\[ V = \frac13 \cdot 6 \left(\frac98 + \frac{81}{8} + \sqrt{\frac98 \cdot \frac{81}{8}}\right) = \frac{117}{4}. \]
Subtracting this from the total volume, which is $6^3 = 216$,
we get our water volume of $\frac{747}{4}$.
Our answer is thus $747 + 4 = \boxed{751}$.

\section*{Problem 15}
The problem is equivalent to finding the number of $0$s in the first $1001$ digits
of the $2$-adic expansion of $\frac{1}{23}$.
This can be done manually using long multiplication until one finds a pattern.
For a real solution, write
\[ \frac{1}{23} = \frac{89}{2047} = 1 + \frac{2^{11} - 1 - 89}{1 - 2^{11}} = 1 + (11110100110_2)(1 + 2^{11} + 2^{22} + \cdots). \]
Thus, it is evident that it cycles every $11$ digits.
The answer is
\[ 90 \cdot 4 + 3 = \boxed{363}. \]

\end{document}