\documentclass{scrartcl}
\usepackage{graphicx} % Required for inserting images
\usepackage{amsmath}
\usepackage{listings}
\usepackage{evan}

\title{DNW-EXPNT Solutions}
\author{Michael Middlezong}

\begin{document}
\maketitle

\section*{Problem 1 (JMO 2011/1)}
Obviously, $n=1$ is a solution. Then, assume $n\geq 2$.

The expression is $(-1)^n + 1$ mod 3. If $n$ is even, then it will equal $2$ mod 3, which is not a quadratic residue. So, $n$ is odd.

The expression is $(-1)^n$ mod 4. Since $-1$ is not a quadratic residue, $n$ must be even. Thus, the only solution is $n=1$.

\section*{Problem 2 (Putnam 2018 B3)}
We claim that only numbers of the form $n = 2^{2^{2^r}}$ for $r \geq 0$ work.

Obviously, the first condition is true iff $n = 2^m$ for some $m$. Then, the second and third condition are equivalent to
\[ 2^n \equiv 1 \pmod{2^{m} - 1}, \]
\[ 2^{n-1} \equiv 1 \pmod{2^{m-1} - 1}. \]
Looking at the first congruence, the order of $2$, which is $m$, must divide $n$, so $m \mid n$. This means that $m = 2^q$ for some $q$.

Looking at the second congruence, the order of $2$, which is $m-1$, must divide $n-1$, so $m-1 \mid n-1$. This means $n \equiv 1 \pmod{m - 1}$, or
\[ 2^m \equiv 1 \pmod{2^q - 1}. \]
Again, this is satisfied when $q \mid m$, so $q = 2^r$ for some $r$. Note that all of these steps are reversible. Therefore, the condition is necessary and sufficient.

\section*{Problem 3 (USAMO 1987/1)}
First, $n = 0$ obviously works. Then, expand both sides and write it as a quadratic in $n$. The discriminant factors as $m(m-8)(m+1)^2$. Thus, either $m = -1$ or $m(m-8) = k^2$ for some integer $k$. Therefore, we have $(m-4)^2 - k^2 = 16$ and the only $m$ that work are $-1$, $8$, and $9$.

Plugging it back in and checking the solutions gives us
\[ \{(-1,-1),(8,-10),(9,-6),(9,-21)\} \]
as our set of nonzero solutions.

\section*{Problem 4 (Shortlist 2002/N1)}
Taking mod $9$ is enough to prove the answer is at least $4$. Since $10^3 + 10^3 + 1^3 + 1^3 = 2002$, it is not hard to construct a working solution.

\section*{Problem 5 (Pixton)}
The only solution is $(x,y) = (45,4)$. We can manually check for $y<6$ that $y=4$ is the only solution. Assume $y \geq 6$. Then $9 \mid y!$, so we have $y! + 2001 \equiv 3 \pmod{9}$. This means $3$ divides $x^2$ exactly once, which is impossible.

\section*{Problem 6 (USEMO 2019/4)}
We can guess that $2020$ is just an arbitrary number, and try to prove that $f(n) = 1^n + 2^{n-1} + \cdots + n^1$ attains every residue mod $p$. Using the lemma that $1^n + 2^n + \cdots + (p-1)^n = 0$ whenever $p-1 \nmid n$, we can see that $f(p(p-1)) \equiv -1 \pmod{p}$. Moreover, $f(cp(p-1)) \equiv -c \pmod{p}$ because all base-exponent combos reset every $p(p-1)$, so we are done.

\section*{Problem 8 (Brazil 2007/2)}
The problem is just asking us to characterize quadratic residues mod $2^{2007}$.
We claim that mod $2^n$, all residues which are $1$ mod $8$ are precisely the odd quadratic residues.
It is easy to see by mods that all quadratic residues must be $1$ mod $8$.
Next, we claim that if $p$ and $q$ are distinct elements of the set $\{1, 3, \ldots, 2^{n-2} - 1\}$, then $p^2 \not\equiv q^2 \pmod{2^n}$.
This can be shown by analyzing $\nu_2(p^2 - q^2)$.

The above is the majority of the problem.
Notice that even quadratic residues are just $4^k$ times an odd quadratic residue.
Answer extraction is done by casework on the power of $4$.

\section*{Problem 10 (Qiao Zhang)}
The answer is all $n \equiv 1,3 \pmod{8}$.
First, we show it is necessary.
Considering mod $8$, we see that if $n$ is not $1$ or $3$ mod $8$, then expressions of the form $3^k - n$ will never be divisible by $8$.

To show it is sufficient, consider the order of $3$ mod $2^m$ for any positive integer $m \geq 4$.
By lifting the exponent, we see that the order is $2^{m-2}$.
Thus, the orbit of possible values of $3^a$ covers all residues mod $2^m$ which are $1$ or $3$ mod $8$.
This means that eventually, some term of the form $3^k - n$ will be divisible by $2^m$, so the sequence is unbounded.

\section*{Problem 11 (Shortlist 2006/N5)}
We claim there are no solutions.
Suppose $(x,y)$ is an integer solution to the equation.
Then, consider a prime $p$ dividing $\frac{x^7 - 1}{x-1} = y^5 - 1$.
By the divisors of cyclotomic polynomials lemma, $p = 7$ or $p \equiv 1 \pmod{7}$.
This means every factor of $y^5 - 1$ is either $0$ or $1$ mod $7$.

Consider the case where $y^5 - 1 \equiv 0 \pmod{7}$.
Then, we have $y \equiv 1 \pmod{7}$. 
Since $y^4 + y^3 + y^2 + y^1 + 1 \equiv 5 \pmod{7}$ is a factor dividing $y^5 - 1$, we have a contradiction.

If $y^5 - 1 \equiv 1 \pmod{7}$, then we have $y \equiv 4 \pmod{7}$.
But then, $y-1 \equiv 3 \pmod{7}$ is a factor dividing $y^5 - 1$, so we have a contradiction.

Thus, there are no solutions.

\section*{Problem 12 (Shortlist 1998/N5)}

\textbf{Lemma:} $m^2 \equiv -1 \pmod{p^k}$ has a solution mod $p^k$ if and only if $p \equiv 1 \pmod{4}$.
\begin{proof}
    Let $g$ be a primitive root mod $p^k$.
    Then, if $p \equiv 1 \pmod{4}$, we can take $m = g^{\frac{\varphi(p^k)}{4}} = g^{\frac{p^k - p^{k-1}}{4}}$.
    It satisfies
    \[ m^2 \equiv g^{\frac{\varphi(p^k)}{2}} \equiv -1 \pmod{p^k}. \]
    For the other direction, notice that $m$ has order $4$ mod $p^k$, so $4$ divides $\varphi(p^k) = p^k - p^{k-1}$.
    Looking at this expression mod $4$, we see that we must have $p \equiv 1 \pmod{4}$.
\end{proof}

The answer is $n = 2^a$ where $a \geq 0$.

First, let $p \mid 2^n - 1$.
Then, assuming $n$ works, there exists $m$ such that
\[ m^2 \equiv -9 \pmod{p}. \]
Assume $p \neq 3$.
Then, we have (using our lemma)
\[ \left( \frac{m}{3} \right)^2 \equiv -1 \pmod{p} \implies p \equiv 1 \pmod{4}. \]
So, every prime dividing $2^n - 1$ must be either $3$ or congruent to $1$ mod $4$.

We claim that if $n$ is not of the form $2^a$, then $n$ does not work.
If $n$ is not of this form, an odd number $k > 1$ must divide $n$.
Then, we have $2^k - 1 \mid 2^n - 1$.

Notice that $2^k - 1 \equiv 3 \pmod{4}$ and $2^k - 1 \equiv 1 \pmod{3}$.
This implies that there exists a prime $p$ dividing $2^k - 1$ (and thus $2^n - 1$) with $p \neq 3$ and $p \equiv 3 \pmod{4}$, giving us a contradiction.

Next, we claim that all $n = 2^a$ work. Let $p \mid 2^{2^a} - 1$. Then,
\[ 2^{2^a} \equiv -1 \pmod{p} \implies 2^{2^{a+1}} \equiv 1 \pmod{p}. \]
This means the order of $2$ mod $p$ divides $2^{a+1}$.

\begin{itemize}
    \item If this order is $1$, then $p = 1$, contradiction.
    \item If this order is $2$, then $p = 3$.
    \item If this order is greater than $2$, then the order is divisible by $4$.
    Since the order divides $p-1$, we must have $p \equiv 1\pmod{4}$.
\end{itemize}

This means every prime dividing $2^{2^a} - 1$ is either $3$ or congruent to $1$ mod $4$.
In addition, we know $2^{2^a} - 1 \equiv 0 \pmod{3}$, so $3$ must be one of the primes dividing $2^{2^a} - 1$.

So, let $2^{2^a} - 1 = 3p_1^{\alpha_1}p_2^{\alpha_2}\ldots p_k^{\alpha_k}$. We claim there is an $m$ satisfying the following congruences:
\begin{align*}
    m^2 &\equiv -9 \pmod{3}, \\
    m^2 &\equiv -9 \pmod{p_1^{\alpha_1}}, \\
    m^2 &\equiv -9 \pmod{p_2^{\alpha_2}}, \\
    &\vdots \\
    m^2 &\equiv -9 \pmod{p_k^{\alpha_k}}.
\end{align*}
The first congruence is equivalent to $m \equiv 0 \pmod{3}$.
For the other congruences, by our lemma, there exists $l$ such that $l^2 \equiv -1 \pmod{p_i^{\alpha_i}}$.
Then $m = 3l$ is the solution we want.

Therefore, we can use the Chinese Remainder Theorem to guarantee the existence of $m$ satisfying
\[ m^2 \equiv -9 \pmod{2^{2^a} - 1} \implies 2^{2^a} - 1 \mid m^2 + 9. \]

\section*{Problem 19 (IMO 1990/3)}
The answer is $n=1,3$. We show that there are no other solutions. Since $n$ must be odd, assume $n \geq 5$.
Let $p$ be the least prime factor of $n$. We have $p > 2$ since $n$ is odd. Furthermore, $2^n + 1 \equiv 0 \implies 4^n \equiv 1 \pmod{p}$.

We claim the order of $4$ mod $p$ is $1$. This is because the order must divide $\gcd{n, p-1}$, which must equal $1$ otherwise $\gcd{n, p-1}$ must have smaller prime factors which also divide $n$, contradicting $p$'s minimality. Thus, $4 \equiv 1 \pmod{p}$, so $p = 3$.

In order for $n^2$ to divide $2^n + 1$, we must have $\nu_3(n^2) = 2\nu_3(n) \leq \nu_3(2^n + 1)$. Since $3 \mid 2 + 1$, using the lifting the exponent lemma yields $\nu_3(2^n + 1) = 1 + \nu_3(n)$. We then conclude that $\nu_3(n) \leq 1$, but since $3$ is the smallest prime factor of $n$, we must have $\nu_3(n) = 1$.

Now, we write $n = 3k$ for some $k$ not divisible by $2$ or $3$. Since $n \geq 5$, we have $k > 1$ and we can let $p$ be the smallest prime factor of $k$. But similarly to before, we see that $2^{6k} \equiv 1 \pmod{p}$ and thus $64 \equiv 1 \pmod{p}$. This means $p \mid 63$. Since $p \geq 5$, we must have $p = 7$.

This tells us that $n^2$ is divisible by $7$. However, $2^n + 1 \equiv 8^k + 1 \equiv 2 \pmod{7}$, so we are done.

\section*{Problem 20 (IMO 2000/5)}
Define two sequences as follows: $n_0 = 1$; $p_i$ is (a) the smallest prime factor of $2^{n_i} + 1$ that is not a factor of $n_i$, or (b) if that doesn't exist, the smallest prime factor of $2^{n_i} + 1$; and $n_{i+1} = n_ip_i.$

Notice that each $n_i$ is divisible by all of the previous ones, and that all $n_i$ and $p_i$ are odd.

First, we show that all $n_i$ satisfy $n_i \mid 2^{n_i} + 1$. We proceed by induction. We can see that $n_0 = 1$ works, so assume $n_i$ works (and all the ones before it). We want to prove $n_{i+1} \mid 2^{n_{i+1}} + 1$.

Note that if a prime $p$ divides $n_{i+1}$, then $p = p_j$ for some $j$ satisfying $0 \leq j \leq i$.
This also means that $p_j$ is a factor of $2^{n_j} + 1$.
Then, by LTE, we have
\[ \nu_{p_j}(2^{n_{i+1}} + 1) = \nu_{p_j}(2^{n_j} + 1) + \nu_{p_j}\left(\frac{n_{i+1}}{n_j}\right) = \nu_{p_j}(n_{i + 1}) + \nu_{p_j}(2^{n_j}+1) - \nu_{p_j}(n_j). \]
However, the strong inductive hypothesis implies $\nu_{p_j}(2^{n_j}+1) - \nu_{p_j}(n_j) \geq 0$, so we have $\nu_{p_j}(2^{n_{i+1}} + 1) \geq \nu_{p_j}(n_{i+1})$.
As this is true for all $p$, the inductive step is complete.

Next, we claim that eventually, the number of distinct prime factors of $n_i$ is always one more than the number of distinct prime factors of $n_{i-1}$. This is equivalent to showing that eventually, there always exists a prime factor of $2^{n_i} + 1$ that is not a factor of $n_i$. This is essentially Zsigmondy's theorem, so we could be done here. However, I forgot that theorem existed, so here is a size argument:

Suppose, for some $i$, that every prime factor of $2^{n_i} + 1$ is also a factor of $n_i$.
Then, let $p$ be a prime factor of $n_i$, and pick the minimal $j$ such that $p_j = p$. This minimality implies that $\nu_{p_j}(n_j) = 0$.
We have
\[ \nu_{p_j}(2^{n_i} + 1) = \nu_{p_j}(2^{n_j} + 1) + \nu_{p_j}(n_i) - \nu_{p_j}(n_j) = \nu_{p_j}(2^{n_j} + 1) + \nu_{p_j}(n_i). \]
We can raise $p_j$ to the power of both sides to get 
\[ {p_j}^{\nu_{p_j}(2^{n_i} + 1)} = {p_j}^{\nu_{p_j}(2^{n_j} + 1)} {p_j}^{\nu_{p_j}(n_i)}. \]
Doing this for every prime factor $p$ of $n_i$ (notice that $j$ is now a one-to-one function of $p$) and multiplying the resulting equations, we get
\[ 2^{n_i} + 1 = n_i \prod_p p^{\nu_p(2^{n_{j(p)}} + 1)}. \]
For all $p$, we have $p^{\nu_p(2^{n_{j(p)}} + 1)} \leq 2^{n_{j(p)}} + 1$. Thus,
\[ 2^{n_i} + 1 \leq n_i \prod_p (2^{n_{j(p)}} + 1). \]
Since $j$ is one-to-one, every $j(p)$ is unique and in the set $\{0, 1, \ldots, i-1\}$. Therefore, we have the inequality
\[ 2^{n_i} + 1 \leq n_i \prod_{j=0}^{i-1} (2^{n_j} + 1). \]
Taking the log base $2$ of both sides, we have
\[ n_i < \log _2 (2^{n_i} + 1) \leq \log _2 n_i + \sum _{j=0}^{i-1} \log_2(2^{n_j} + 1) < \log_2 n_i + \sum _{j=0}^{i-1} (n_j + 1) = \log_2 n_i + i + \sum_{j=0}^{i-1}n_j. \]
Now, in order to achieve a bound on $n_i$, we notice that $p_i \geq 3$ for all $i$, so therefore, $n_i \geq 3^i$. It is then easy to see that $\sum_{j=0}^{i-1}n_j \leq \frac12 n_i$ for all $i \geq 1$. Then,
\[ n_i < \log_2 n_i + i + \frac12 n_i \leq \log_2 n_i + \log_3 n_i + \frac12 n_i \iff n_i < 2(\log _2 n_i + \log _3 n_i). \]
This inequality obviously cannot be satisfied as $n_i$ grows large. Thus, eventually, we must have that there exists a prime factor of $2^{n_i} + 1$ that is not a factor of $n_i$. Hence, there must eventually exist an $n$ in our sequence with exactly 2000 distinct prime factors.

\section*{Problem 22 (Generalized IMO 1999/4)}
We claim the solutions are $(1,p)$ for all $p$, $(2,2)$, and $(3,3)$. They can be checked to work. Furthermore, if $p=2$, then $x$ must be a divisor of $1^x + 1 = 2$, so $(1,2)$ and $(2,2)$ are the only solutions. Thus, from now on, we can assume $x > 1$ and $p \geq 3$.

We know $x$ has a least prime factor; let that be $q$. Then, we have
\begin{align*}
    (p-1)^x \equiv -1 \pmod{q} \\
    (p-1)^{2x} \equiv 1 \pmod{q}.
\end{align*}
Looking at the order of $(p-1)^2$, it must divide the GCD of $q-1$ and $x$. But $q$ being the least prime factor implies that GCD is $1$, so $(p-1)^2 \equiv 1 \pmod{q}$. This means $q \mid p(p-2)$.

However, $q \mid p-2$ is impossible. To see why, notice that $(p-1)^x + 1 \equiv 1 + 1 \pmod{p-2}$, so $(p-1)^x + 1 \equiv 2 \pmod{q}$. Since $q$ is odd, this means that $(p-1)^x + 1$ is not divisible by $q$, contradicting the fact that $x^{p-1}$ divides $(p-1)^x + 1$.

Thus, we must have $q = p$. Now, notice that since $x$ must be odd, we can use lifting the exponent:
\[ \nu_p((p-1)^x + 1) = 1 + \nu_p(x) \geq \nu_p(x^{p-1}) = (p-1)\nu_p(x). \]
This implies
\[ \frac{1}{p-2} \geq \nu_p(x) \geq 1, \]
so $p \leq 3$.

We only need to consider the case where $p = 3$. However, the result of IMO 1990/3 tells us that $(1,3)$ and $(3,3)$ are the only solutions in this case, so we are done.

\section*{Problem 21 (TSTST 2018/8)}
\textbf{Lemma:} If $a \mid b$ and $a$ and $b$ are odd, then $x^a + 1 \mid x^b + 1$ for all natural numbers $x$.
\begin{proof}
We know $\frac{b}{a}$ is odd, so
\[ \frac{x^b + 1}{x^a + 1} = x^{b-a} - x^{b-2a} + x^{b-3a} - \ldots + 1. \]
\end{proof}

We claim that all $b$ such that $b+1$ is not a power of $2$ work.
We will generate an infinite sequence of $n$'s satisfying the condition, starting with $n_1 = p_1$ where $p_1$ is an odd prime that divides $b+1$.
The sequence will also satisfy the additional condition that $n_i$ can be written in the form $p_1p_2\ldots p_i$ where the $p_j$'s are distinct odd primes.
To establish the base case, notice that since $p_1$ is odd, we can use lifting the exponent:
\[ \nu_{p_1}(b^{p_1} + 1) = \nu_{p_1}(b + 1) + 1 \geq 2. \]
Thus, $n_1^2 \mid b^{n_1} + 1$.

Next, assume $n_i = p_1p_2\ldots p_i$ satisfies the condition that $n_i^2 \mid b^{n_i} + 1$.
Then, let $p_{i+1}$ be a primitive prime divisor of $b^n + 1$.
Its existence is guaranteed by Zsigmondy's theorem.
We claim that $p_{i+1} \neq 2$.
If $b$ is even, then that is obvious.
Otherwise, $2 \mid b+1$, so $2$ would not be a primitive divisor.
Thus, $p_{i+1}$ is an odd prime distinct from any of $p_1,p_2,\ldots,p_i$.

We then claim that if $n_{i+1} = p_1p_2\ldots p_{i+1}$, then $n_{i+1}$ satisfies the condition from the problem statement.
By our lemma, we know that $n_i^2 = (p_1p_2\ldots p_i)^2 \mid b^{n_i} + 1 \mid b^{n_{i+1}} + 1$.
It remains to check $p_{i+1}^2 \mid b^{n_{i+1}} + 1$.
Again, we use lifting the exponent:
\[ \nu_{p_{i+1}}(b^{n_{i+1}} + 1) = \nu_{p_{i+1}}(b^{n_i} + 1) + \nu_{p_{i+1}}(p_{i+1}) = \nu_{p_{i+1}}(b^{n_i} + 1) + 1 \geq 2. \]

Thus, by induction, every $n$ in this infinite sequence works.

Finally, we prove that if $b+1$ is a power of $2$, there are finitely many solutions to $n^2 \mid b^n + 1$.
Suppose $n > 1$ satisfies $n^2 \mid b^n + 1$.
If $n$ were even, then $b^n + 1$ would be either $1$ or $2$ mod $4$, but it also must be $0$ mod $4$ in order to be divisible by $n^2$.
Hence, $n$ must be odd.

Next, let $p$ be the least prime factor of $n$.
We have
\[ b^n \equiv -1 \implies b^{2n} \equiv 1 \pmod{p}, \]
so the order of $b^2$ must divide $\gcd(p-1,n)$.
Since $p$ is the least prime factor of $n$, this GCD must equal $1$; otherwise, $\gcd(p-1,n)$ would have a prime factor less than $p$ which is also a factor of $n$, contradicting $p$'s minimality.
Thus, $b^2 \equiv 1 \pmod{p}$, or in other words, $p \mid (b-1)(b+1)$.
We know $p$ cannot divide $b+1$ since $b+1$ is a power of $2$.

Thus, $p$ must divide $b-1$, and $b \equiv 1 \pmod{p}$.
However, this means $b^n \equiv 1 \pmod{p}$, so $p$ must be $2$, which is impossible since $n$ is odd!
This concludes the proof.

\section*{Problem 26 (USEMO 2021/2)}
The answer is $n \in \{1,2,4,6,8,16,32\}$.
Let $f(n)$ be the number of divisors of $2^n - 1$, and let $d(n)$ be the number of divisors of $n$.
Notice the function $d(n)$ is multiplicative (with relatively prime numbers).
A key element of the solution is to notice that if we write $n = 2^ak$ where $k$ is odd, then
\[ 2^n - 1 = (2^k - 1)(2^k + 1)(2^{2k} + 1)\cdots (2^{2^{a-1}k} + 1). \]
Furthermore, all of these factors are odd and relatively prime.

For all $k$, we have
\[ f(2k) = d(2^{2k} - 1) = d(2^k - 1)d(2^k + 1) = f(k)d(2^k + 1). \]
Therefore, $f(2k) \geq 2f(k)$ with equality iff $2^k + 1$ is prime.
This means that if $f(k) = k$ but $2^k + 1$ is not prime, multiplying $k$ by a power of $2$ is guaranteed to fail.
Along with the fact that $2^{2^5} + 1$ is not prime (look at Fermat primes), this is enough to show that the only powers of $2$ satisfying $f(n) = n$ are $1$, $2$, $4$, $8$, $16$, and $32$.

If $n$ is not a power of $2$, then write $n = 2^ak$ where $k$ is odd and $k \geq 3$.
At this point, we claim that $n$ must be even, so we can make this useful.
If $n$ is odd and $f(n) = n$, then this implies $2^n - 1$ is a perfect square.
However, looking mod $4$, this is only true when $n = 1$.
Thus, $n$ must be even and $a \geq 1$.

Then, looking at the $\nu_2$ of
\[ f(2^ak) = f(k)d(2^k + 1)d(2^{2^k} + 1)\cdots d(2^{2^{a-1}k} + 1), \]
we see that $2^k + 1$ must be a perfect square.
Mihailescu's theorem tells us that the only possible solution to this is $k=3$, but here is another way to see it.

Let $2^k + 1 = m^2$.
Then, we have $2^k = (m-1)(m+1)$.
Thus, $m-1$ and $m+1$ must both be powers of $2$, but the only powers of $2$ that differ by $2$ are $2^1 = 2$ and $2^2 = 4$, so we must have $k = 3$.

Anyways, we can manually verify that $n=6$ does work but $n=12$ does not.
We turn again to the fact that $f(2k) \geq 2f(k)$ with equality iff $2^k + 1$ is prime.
Unfortunately, $2^6 + 1$ is not prime, so we are done.
\end{document}
% vim: ts=4 sts=4 sw=4 et
